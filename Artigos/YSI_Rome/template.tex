\documentclass{article}


\usepackage{arxiv}
\usepackage[english]{babel}
\usepackage[utf8]{inputenc} % allow utf-8 input
\usepackage[T1]{fontenc}    % use 8-bit T1 fonts
\usepackage[hidelinks]{hyperref}       % hyperlinks
\usepackage{url}            % simple URL typesetting
\usepackage{booktabs}       % professional-quality tables
\usepackage{amsfonts}       % blackboard math symbols
\usepackage{nicefrac}       % compact symbols for 1/2, etc.
\usepackage{microtype}      % microtypography
\usepackage{lipsum}

\usepackage{csquotes}
\usepackage[backend=biber,%
	url = false,
	style = apa,
	%extrayear, %
	uniquename=init,% 
	giveninits, %
	%justify, %
	sorting=nyt,
	%repeattitles, %
	maxcitenames=3]{biblatex}
\addbibresource{references.bib}

\title{Residential investment and asset bubbles in a Sraffian Supermultiplier Stock-Flow Consistent model: Towards open economy}


\author{
  Gabriel Petrini da Silveira\\
  Master student at Institute of Economics (Unicamp - Brazil)\\
  \texttt{gpetrinidasilveira@gmail.com} \\
}

\begin{document}
\maketitle

\begin{abstract}
The problem left by \textcite{harrod_essay_1939} questions the conditions that allow a dynamic long-term equilibrium between demand and productive capacity. 
As a result of the Harrodian instability, the conjugation between the multiplier effect and the accelerator principle has been regarded as equally unstable by literature. 
From an heterodox perspective, Cambridge, Oxford, and supermultiplier models emerged to address this issue. 
In this literature (majorly Kaleckain), it was up to the degree of capacity utilization to guarantee the stylized fact reported above so income distribution can remain  exogenous. 
However, such artifice imposes that capacity utilization  stays persistently different from the desired on the long run. 

That said, this in-progress research extends the contribution of \textcite{serrano_sraffian_1995} towards models that validate the principle of effective demand on the long-run which non-capacity creating expenditure leads the long run growth rate and capacity utilization converge to its normal level. 
In addition, the Kaleckian alternative  that includes autonomous expenditures that do not create capacity initiated by \textcite{allain_tackling_2015} and reinforced by \textcite{lavoie_convergence_2016} (among others,  \cites{dutt_growth_2016}{dutt_observations_2018}{hein_autonomous_2018}{nah_long-run_2017})  is critically reviewed. 

From this literature review, we highlight the negligence regarding the treatment given to residential investment. 
The importance of this component of aggregate demand in determining the economic cycle is highlighted by \textcite{leamer_housing_2007} and recently taken up by \textcite{fiebiger_semi-autonomous_2018} in which residential investment anticipates the cycle. Following \textcite{da_silva_teixeira_crescimento_2015}, we include asset bubbles to deal with real estate growth rate ($g_Z$):

\begin{equation}
\label{OWN}
g_Z = \phi_0 - \phi_1\left(\frac{1+r_{MO}}{1+\dot p_h}-1\right)
\end{equation}
In equation \ref{OWN}, $\phi_0$ represents long-term factors of housing market (such as credit constraints and institutional aspects) while $\phi_1$ is a parameter for the own interest rate for houses which captures its demand by speculative reasons.
Inspired by \citeauthor{sraffa1932dr}'s own interest rates (\cite{sraffa1932dr}), \textcite{da_silva_teixeira_crescimento_2015} develops the own interest rate for houses --- defined as the mortgage rate ($r_{MO}$) deflated by house prices ($\dot p_h$) --- and indicates the house prices in house's terms. In other words, this rate express the real cost of buying dwellings which is the relevant cost for households' decisions to undertake in residential investment. In this way, we build a Sraffian supermultiplier model (SSM) to emphasize the importance of household investment using \citeauthor{da_silva_teixeira_crescimento_2015}'s rate.

However, as \textcite{brochier_supermultiplier_2018} point out, such model lacks an adequate treatment of financial relations.
Therefore, the inclusion of asset bubbles allows to add capital gains in the traditional SSM model.  
With this gap in mind, a SSM-Stock-Flow Consistent (thereafter SSM-SFC) is modeled to initiate a research area towards the endogeinization of this autonomous expenditure. 
The importance of this research is the possibility to analyze the cyclicity of the economy in light of the instability of the aggregate demand as \textcite{dejuan_hidden_2017} suggests.  

In short, this article analyses the dynamics of household investment using a    SSM-SFC  model  based  on  the  U.S.  economy (1980-2000). 
The first section presents a review of Kaleckian and sraffian supermultiplier models with autonomous expenditures. 
The second section highlights stylezed facts for the American economy which support the idea that non-capacity generating expenditures, mainly household investment, led the economic growth and determinates the cycle. 
Finally, a SSM-SFC model with two sorts of capital stock (productive capital and real estate) is simulated to analyse the effects of changes in income distribution, interest rates, autonomous component of housing growth rate and residential inflation. 
The results are: 
    (i) changes in income distribuition affect the growth rate only during the transverse; 
    (ii) house stock ratio\footnote{Defined as share of real estate in the total capital stock of the economy.} decreases as a result of the overall increase in productive capacity; 
    (iii) long-run growth rate is affected only by household investment which depends positively on residential inflation ($\dot p_h$) and institutional elements ($\phi_0$) and negativetly on mortgage interest rates.  

Therefore, this base model introduces housing on sraffian supermultiplier agenda and extends the range of autonomous expenditures alternatives.
It worth noting that future (already in progress) versions of this paper will --- not simultaneously --- move in two directions.
The first one (in the finishing phase) analyzes the relationship between residential investment growth and house's own interest rate using a time series model (VEC). 
The second (in early stage) intends to open the economy and evaluate the relationship between exchange rate and income distribution (exogenous in the basic model) and its impacts over balance of payments equilibrium.
\end{abstract}


\keywords{Demand-led Growth \and Sraffian supermultiplier \and Stock flow consistent approach \and Residential investiment \and Own interest rate}

\printbibliography{}

\end{document}
