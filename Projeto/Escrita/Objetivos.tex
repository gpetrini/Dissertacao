\section{Objetivos}\label{OBJ}

\begin{description}
	\item[Objetivo geral] Investigar os arranjos institucionais do mercado imobiliário e de crédito dos países da OCDE e investigar as implicações sobre o investimento residencial e crescimento, contrastando com o caso norte-americano. Com isso, pretende-se construir um modelo SFC com supermultiplicador sraffiano cujo gasto autônomo é o investimento residencial e replicar as referidas características institucionais do mercados de crédito e imobiliário para assim:
	\item[Objetivos específicos] {\color{white} Texto em branco para espaçamento}
	\begin{itemize}
		\item Examinar o processo de ``hipotecarização'' destacado por \textcite{jorda_great_2014} e comparar o caso norte-americano com os demais países da OCDE;
		\item Testar a aplicabilidade da taxa própria de juros dos imóveis desenvolvida por \textcite{teixeira_crescimento_2015} para os países da OCDE; 
		\item Detectar os principais determinantes macroeconômicos do investimento residencial por meio de um modelo de séries temporais com auxílio da base de dados desenvolvida por \textcite{jorda_great_2014};
	\end{itemize}
\end{description}


Para atender estes objetivos, a pesquisa proposta será dividida em três frentes.
A primeira delas trata da inserção e contextualização do investimento residencial em processos mais estruturais como a financeirização em contraposição a ``hipotecarização''. Para isso, serão analisadas as especificidades dos países em questão no que diz respeito ao mercado imobiliário e sua conexão com o mercado de crédito bem como a regulação de preços dos imóveis. Em seguida, será investigado o quão destoante é o caso norte-americano frente aos demais, com especial ênfase ao caso alemão. Compreendidas as especificidades institucionais de cada país, serão analisados os determinantes do investimento residencial por meio de um modelo séries temporais. 

Essas duas partes fornecem o embasamento qualitativo e quantitativo para o modelo SFC que será desenvolvido em seguida. Serão testados os diferentes arranjos institucionais destacados bem como os determinantes do investimento residencial e espera-se replicar: 
	(i) ausência de relação entre crescimento e distribuição no longo prazo; 
	(ii) convergência da taxa de crescimento da economia a taxa de crescimento do investimento residencial e do grau de utilização ao normal; 
	(iii) aumento da participação das hipotecas no balanço patrimonial dos bancos;
	(iv) relação entre inflação de ativos e investimento residencial e; 
	(v) inter-relações entre mercado imobiliário e de crédito. 


%RESULTADOS ESPERADOS