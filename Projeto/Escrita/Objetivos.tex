\section{Objetivos}\label{OBJ}

\begin{description}
	\item[Objetivo geral] Investigar os arranjos institucionais do mercado imobiliário e de crédito dos países da OCDE e investigar as implicações sobre o investimento residencial e contrastá-las com o caso norte-americano
	\item[Objetivos específicos] {\color{white} Texto em branco para espaçamento}
	\begin{itemize}
		\item Examinar o processo de ``hipotecarização'' destacado por \textcite{jorda_great_2014};
		\item Comparar o caso norte-americano com os demais países da OCDE;
		\item Testar a aplicabilidade da taxa própria de juros dos imóveis desenvolvida por \textcite{teixeira_crescimento_2015} para os países da OCDE; 
		\item Detectar os principais determinantes macroeconômicos do investimento residencial por meio de um modelo de dados em painel com auxílio da base de dados desenvolvida por \textcite{jorda_great_2014};
		\item Construir um modelo SFC com supermultiplicador sraffiano cujo gasto autônomo é o investimento residencial e replicar as referidas características institucionais do mercados de crédito e imobiliário.
	\end{itemize}
\end{description}


Para atender estes objetivos, a pesquisa proposta será dividida em três frentes cada qual com seu respectivo capítulo.
A primeira delas trata da inserção e contextualização do investimento residencial em processos mais estruturais como a financeirização em contraposição a ``hipotecarização''. Para isso, serão analisadas as especificidades dos países em questão no que diz respeito ao mercado imobiliário e sua conexão com o mercado de crédito bem como a regulação de preços dos imóveis. Em outras palavras, caberá a este capítulo investigar o quão destoante é o caso norte-americano frente aos demais, com especial ênfase ao caso alemão. Compreendidas as especificidades institucionais de cada país, o capítulo seguinte irá analisar os determinantes do investimento residencial por meio de um modelo de dados em panel. 

Com isso, esses dois capítulos fornecem o embasamento qualitativo e quantitativo para o modelo teórico que será desenvolvido no terceiro capítulo da tese. A partir da metodologia \textit{Stock-Flow Consistent} (SFC), serão testados os diferentes arranjos institucionais destacados no capítulo primeiro bem como os determinantes do investimento residencial reportados no capítulo segundo e, assim, conectar com a literatura do supermultiplicador sraffiano. Desse modo, a partir do modelo construído espera-se replicar: (i) ausência de relação entre crescimento e distribuição no longo prazo; (ii) convergência da taxa de crescimento da economia a taxa de crescimento do investimento residencial e do grau de utilização ao normal; (iii) aumento da participação das hipotecas no balanço patrimonial dos bancos; (iv) relação entre inflação de ativos e investimento residencial e (v) inter-relações entre mercado imobiliário e de crédito. Expostos os objetivos da pesquisa, cabe a seção seguinte esclarecer as etapas da metodologia SFC.



%RESULTADOS ESPERADOS