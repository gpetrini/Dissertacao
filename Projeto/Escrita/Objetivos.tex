\section{Objetivos}\label{OBJ}

%OBJETIVOS ESPECÍFICOS COMO CAPÍTULOS

\begin{description}
	\item[Objetivo geral] Investigar a relação entre investimento residencial e dinâmica macroeconômica tendo em vista os arranjos institucionais do mercado imobiliário e de crédito, contrastando com o caso norte-americano.
	\item[Objetivos específicos] {\color{white}Teste}
	\begin{itemize}
		\item Examinar o processo de ``hipotecarização'' destacado por \textcite{jorda_great_2014} destacando as especificidades institucionais por meio de uma análise comparativa qualitativa (QCA);
		\item  Detectar os principais determinantes macroeconômicos do investimento residencial por meio de um modelo de dados em painel dinâmico;
		\item Construir um modelo SFC com supermultiplicador sraffiano cujo gasto autônomo é o investimento residencial e replicar as referidas características institucionais do mercados de crédito e imobiliário
	\end{itemize}
\end{description}

%DELIMITAR OCDE

Para atender estes objetivos, a pesquisa proposta será dividida em três frentes (capítulos).
A primeira delas trata da inserção e contextualização do investimento residencial em processos mais estruturais como a financeirização em contraposição a ``hipotecarização''. 
Para isso, serão analisadas as relações teóricas entre o mercado de imobiliário e de crédito a luz das especificidades dos países por meio de uma análise qualitativa comparativa. Compreendidos esses elementos, serão analisados os determinantes do investimento residencial por meio de um modelo de dados em painel dinâmico. 

Essas duas partes fornecem o embasamento qualitativo e quantitativo para o modelo SFC que será desenvolvido em seguida. Serão testados diferentes arranjos institucionais bem como os determinantes do investimento residencial e espera-se replicar\footnote{Os resultados espetados (i) a (iii) são resultados do modelo do supermultiplicador srrafiano enquanto os demais dizem respeito às contribuições de \textcite{teixeira_crescimento_2015} e \textcite{jorda_great_2014} respectivamente.}: 
	(i) ausência de relação entre crescimento e distribuição no longo prazo; 
	(ii) convergência da taxa de crescimento da economia a taxa de crescimento do investimento residencial e do grau de utilização ao normal; 
	(iii) aumento da participação das hipotecas no balanço patrimonial dos bancos;
	(iv) relação entre inflação de ativos e investimento residencial e; 
	(v) inter-relações entre mercado imobiliário e de crédito. 


%RESULTADOS ESPERADOS