\section{Metodologia}\label{passos}


A metodologia para a primeira parte consiste em uma ampla revisão de literatura e de identificação de argumentos centrais sobre os determinantes institucionais do investimento residencial enquanto para a segunda será utilizada econometria de séries temporais. 
Como apresentado anteriormente, o modelo SFC é o ponto de chegada da pesquisa em que são reunidos os esforços da análise qualitativa bem como os resultados do modelo empírico. Sucintamente, tal metodologia é composta de três procedimentos: (i) determinação da estrutura contábil; (ii) exposição das equações comportamentais e; (iii) solução/simulação. Dito isso, a figura \ref{Resuminho} resume as etapas mencionadas e explicitar a diferença entre a \textit{metodologia} e um \textit{modelo} SFC. 


\begin{figure}[htb]
	\caption{Resumo esquemático da Metodologia SFC}
	\label{Resuminho}
	\centering
	\begin{tikzpicture}
	[node distance = 1cm, auto,font=\footnotesize,
	% STYLES
	every node/.style={node distance=3cm},
	% The comment style is used to describe the characteristics of each force
	comment/.style={rectangle, inner sep= 5pt, text width=4cm, node distance=0.25cm, font=\scriptsize\sffamily},
	% The force style is used to draw the forces' name
	force/.style={rectangle, draw, fill=black!10, inner sep=5pt, text width=4cm, text badly centered, minimum height=1.2cm, font=\bfseries\footnotesize\sffamily}] 
	
	% Draw forces
	\node [force] (rivalry) {Hipóteses};
	\node [force, above of=rivalry, fill=red!70] (substitutes) {Equações comportamentais};
	\node [force, text width=3cm, dashed, left=1.5cm of substitutes,fill=blue!50] (state) {Metodologia SFC};
	\node [force, left=1cm of rivalry] (suppliers) {Estrutura contábil};
	\node [comment, below=0.25 of suppliers] (comment-suppliers) {Relação entre lado real \\e financeiro};
	\node [force, right=1cm of rivalry] (users) {Solução};
	\node [force, right=1cm of substitutes, dashed, fill=purple!50 ] (PK) {Modelo \\_-SFC};
	%	\node [force, below of=rivalry] (entrants) {Threat of new entrants};
	
	%%%%%%%%%%%%%%%
	% Change data from here
	
	% RIVALRY
	\node [comment, below=0.25 of rivalry] (comment-rivalry) {Cambridge/New Cambridge\\
		Kaleckiano\\
		\textbf{Supermultiplicador Sraffiano}};
	
	
	% SUBSTITUTES
	%\node [comment, right=0.25 of substitutes] {};
	
	% USERS
	\node [comment, below=0.25 of users] {Analítico\\
		\textbf{Simulação}};
	
	% NEW ENTRANTS
	%	\node [comment, right=0.25 of entrants] {(+) EC vs. Microsoft};
	
	% PUBLIC POLICIES
	%	\node [comment, text width=3cm, below=0.25 of state] {(1) Estrutura contábil\\
	%	(2) Equações comportamentais\\
	%	(3) \textit{Closure} do modelo};
	
	%%%%%%%%%%%%%%%%
	
	% Draw the links between forces
	\path[->,thick] 
	(rivalry) edge (substitutes)
	(suppliers) edge (rivalry)
%	(rivalry) edge (users)
	(state) edge (substitutes)
	(state) edge (suppliers)
	(substitutes) edge (PK)
	(PK) edge (users);
	
	%(entrants) edge (comment-rivalry);
	
	\end{tikzpicture} 
	\caption*{Fonte: \textcite[p.~64, adaptado]{da_silveira_politica_2017}}

	
\end{figure}



As etapas contábeis da abordagem SFC constituem em: (i) seleção dos setores institucionais e dos ativos a serem incorporados; (ii) mapeamento das relações dos fluxos entre os mencionados setores por meio da construção da matriz de fluxos; (iii) construção da matriz dos estoques de riqueza (real e financeira) em que são contabilizadas os ativos e passivos  bem como a posição líquida de cada setor; (iv) identificação das formas que os fluxos são financiados e sua respectiva acumulação/alocação dos estoques. 
Como todo modelo macroeconômico, ao partir de um aparato analítico
baseado em identidades contábeis, surgem restrições que precisam ser seguidas mas o que distingue
a metodologia SFC das demais é a conexão do lado real com o financeiro de forma integrada.
Tal procedimento garante que para que um setor acumule riqueza financeira, outro precisa necessariamente liquidá-la de modo que não existam ``buracos negros'' \cite{godley_money_1996}.

Por mais que esta etapa é centrada nas contas nacionais, isso não implica que não possua um componente teórico associado \cite[p.~15--16]{macedo_e_silva_peering_2011}.  No entanto, por se tratar de identidades, nada de causal pode ser extraído delas. As relações de causalidade do modelo (agora modelo e não metodologia) decorrem das equações comportamentais que, respeitando a consistência, podem ser de qualquer linhagem teórica. Feitas essas ressalvas, dada a estrutura contábil e explicitadas as hipóteses (via equações comportamentais), resta seguir para a resolução do modelo. Como pontuam \textcite{caverzasi_stock-flow_2013}, existem três vias: (i) simulação; (ii) analítica e; (iii) descritiva. A primeira delas permite expor as relações entre as variáveis de modelos mais complexos em que a solução analítica não é facilmente encontrada. No entanto, tal caminho fez com que o grau de complexidade dos modelos simulados fosse exponencializada de modo que a intuição econômica torna-se facilmente turva.  
Diante disso, esta pesquisa prioriza a parcimônia de modo que serão incluídos apenas os elementos necessários.
Tal postura permite encontrar soluções analíticas com maior facilidade de modo que são explicitados os parâmetros mais relevantes para as
trajetórias de longo prazo. 
Apesar de parcimoniosidade do modelo, a simulação tem a vantagem de fornecer informações que não se restringem às soluções de equilíbrio e esta forma também será selecionada para resolver o modelo. 
