% Last update/Última versão: 11/Sep/2016
%%%%%%%%%%%%%%%%%%%%%%%%%%%%%%%%%%%%%%%%%%%%%%%%%%%%%%%%%%%%%%%%%%%%%%
%=====================================================================
% 							Pacotes Fundamentais
%=====================================================================
\documentclass[
% -- op\c{c}\~{o}es da classe memoir --
12pt,				% tamanho da fonte
openright,			% cap\'{\i}tulos come\c{c}am em p\'{a}g \'{\i}mpar (insere p\'{a}gina vazia caso preciso)
oneside,			% para impress\~{a}o em verso e anverso. Oposto a oneside
a4paper,		% tamanho do papel.
% -- op\c{c}\~{o}es da classe abntex2 --
%chapter=TITLE,		% t\'{\i}tulos de cap\'{\i}tulos convertidos em letras mai\'{u}sculas
%section=TITLE,		% t\'{\i}tulos de se\c{c}\~{o}es convertidos em letras mai\'{u}sculas
%subsection=TITLE,	% t\'{\i}tulos de subse\c{c}\~{o}es convertidos em letras mai\'{u}sculas
%subsubsection=TITLE,% t\'{\i}tulos de subsubse\c{c}\~{o}es convertidos em letras mai\'{u}sculas
% -- op\c{c}\~{o}es do pacote babel --
english,			% idioma adicional para hifeniza\c{c}\~{a}o
%french,			% idioma adicional para hifeniza\c{c}\~{a}o
%spanish,			% idioma adicional para hifeniza\c{c}\~{a}o
brazil,				% o \'{u}ltimo idioma \'{e} o principal do documento
%sumario=tradicional,
]{article}
\usepackage[a4paper, right = 1.5cm, left = 3.5cm, top = 2cm, bottom = 2cm]{geometry}
\usepackage[utf8]{inputenc}
\usepackage[english,portuguese]{babel}
%\usepackage[myheadings]{fullpage}
\usepackage[T1]{fontenc}
%\usepackage{fancyhdr}
\usepackage{graphicx, setspace}
\usepackage{sectsty}
\usepackage{url}
\usepackage{mathptmx} %% Para times
\usepackage{comment}
\usepackage{multirow}
\usepackage{graphicx}
\usepackage[table,xcdraw]{xcolor}
\usepackage{enumitem}
\usepackage{blindtext}
\usepackage{float}
\usepackage[bottom]{footmisc}
\usepackage{pdfpages}
\usepackage{caption}
\usepackage{csquotes}
\usepackage{footnote}


%=====================================================================
% 							Pacotes Bibliográficos
%=====================================================================

\usepackage[backend=biber,
	style = abnt,%
	noslsn, %
	extrayear, %
	uniquename=init,% 
	giveninits, %
	justify, %
	sccite,% 
	scbib, %
	repeattitles, %
	maxcitenames=3]{biblatex}
\addbibresource{../../Escrita_Dissertacao/Da_Silveira_Dissertacao_Atual/Bibliografia_Dissertacao/Bibliografia_Dissertacao.bib}


%=====================================================================
% 							Comandos específicos da FAPESP
%=====================================================================
\input{comandos}
%=====================================================================
% 							Outros Pacotes
%=====================================================================
\usepackage[colorinlistoftodos]{todonotes}
\usepackage{subfigure}
\usepackage{setspace}
\usepackage{lipsum}  
\usepackage{amsmath} % Para cases
%=====================================================================
% 							Página de título e folha de rosto
%=====================================================================
%-----
% Página de título
% Observação: As definições que aparecem a seguir comporão a
%             página de título e a folha de rosto.
%-----
%% Define o nome da universidade onde o projeto foi desenvolvido.
\universidade{Universidade Estadual de Campinas}
%
%% Define o nome da faculdade onde o projeto foi desenvolvido.
\faculdade{Instituto de Economia}
%
%% Define o título do projeto.
\titulo{Título}
%
%
%% Define o autor do relatório.
\autor{}
\cidade{Campinas}


\begin{document}
%=====================================================================
% 							Numeração pré-textual
%=====================================================================
\pagenumbering{roman}
%=====================================================================
% 							Folha de título
%=====================================================================
%\geraTitulo
%=====================================================================
% 							Folha de rosto
%=====================================================================
% Gera a folha de rosto.
%\folhaDeRosto
%=====================================================================
% 							Título
%=====================================================================
\begin{center}
\rule{\textwidth}{1.2pt}
	Projeto Preliminar de Tese para Doutorado em Ciência Econômica\\
	\textbf{Título}
\rule{\textwidth}{1.2pt}
\end{center}
%=====================================================================
% 							Resumo
%=====================================================================
\Resumo{
\lipsum[57]
\lipsum[57]
\lipsum[57]
\lipsum[57]\\
\textbf{Palavras-chave:} Palavra 1, Palavra 2, Palavra 3, Palavra 4, Palavra 5.
}
%=====================================================================
% 							Abstract
%=====================================================================
\begin{comment}
Habilitar caso seja necessário um abstract em outra página

\Abstract{
teste in english
}
\end{comment}
%=====================================================================
% 							Sumário
%=====================================================================
%\tableofcontents
%\thispagestyle{empty}
%\clearpage
%=====================================================================
% 							Numeração textual
%=====================================================================
\pagenumbering{arabic}

%=====================================================================
% 							Formatação título seção
%=====================================================================
\sectionfont{\scshape}

%=====================================================================
% 							Corpo de texto
%=====================================================================
\doublespacing
\begingroup
\let\clearpage\relax
\section{Introdução e Justificativas}\label{Intro}


%=============================================================================
% 							INTRODUÇÃO
%=============================================================================

%GFC E QUESTIONAMENTO DO INVESTIMENTO COMO CAUSA CAUSANS
PALAVRA INICIAL 
Os impactos sócio-econômicos da Crise Financeira Global (CFG) são imensuráveis, mas algumas das mudanças sobre a teoria econômica já podem ser tateadas. Se, por um lado, abalou a macroeconomia ortodoxa ao ponto da política fiscal estar sendo repensada, por outro, redirecionou algumas pautas na heterodoxia. Distribuição e desigualdade, temas tão caros a esta última tradição, ganharam novo fôlego enquanto parte da literatura passou a destacar o consumo como um dos possíveis motores de crescimento \cite{brochier_macroeconomics_2017}. Desse modo, umas das consequências da CGF é a reavaliação do investimento das firmas enquanto  ``a causa das causas''. Paralelamente, verificou-se um crescente interesse nas implicações macroeconômicas do investimento residencial\footnote{E isso é verificado até na literatura ortodoxa. Inspecionando modelos DSGE que incluem investimento residencial, \textcite{iacoviello_housing_2010} conclui que um melhor entendimento dos impactos deste gasto se faz necessária para a compreensão das flutuações macroeconômicas. } \cite{fiebiger_semi-autonomous_2018}. 

Neste ponto, cabe mencionar o ineditismo de \textcite{green_follow_1997} e \textcite{leamer_housing_2007} --- e revisitado em \textcite{leamer_housing_2015} e por \textcite{fiebiger_trend_2017} --- ao lançar luz sobre a importância do investimento residencial na determinação dos ciclos econômicos antes da crise dos \textit{subprimes}. Ao avaliar o caso norte-americano, \textcite{green_follow_1997} conclui que o investimento residencial possui uma capacidade preditiva maior que o investimento das firmas, mas que isso não implica no estabelecimento de uma relação causal. Na tentativa de compreender tais resultados, afirma:

\begin{quote}
	
	[P]erhaps residential investiment, like stock prices and interest rates, is a good predictor of GDP because it is a series that reflects \textbf{foward looking behavior}. Presumably households will not increase their expenditures on housing unless they expect to prosper in the future. Building a house is a natural mechanism for doing this. Thus, the series can do a good job of predicting GDP without necessarily causing GDP.
	\cite[p.~267, grifos adicionados]{green_follow_1997}
\end{quote}
\textcite{leamer_housing_2007}, por sua vez, avança em direção a relação de causalidade entre este gasto e o PIB. Grosso modo, afirma que a construção de novos imóveis implica em maior consumo de bens duráveis e, portanto, trata-se de um ciclo decorrente do \textit{volume} e não do preço dos imóveis. MAIS REFERÊNCIAS SOBRE EUA.

Além disso, parte da literatura empírica recente também têm lançado luz sobre a importância deste gasto para o ciclo econômico. \textcite{alvarez_does_2010}, por exemplo, concluem que tal tipo de investimento antecede o ciclo econômico para o caso de espanhol e resultados semelhantes podem ser encontrados para França \cite{ferrara_cyclical_2010}, Espanha  e Itália enquanto o caso alemão apresenta uma dinâmica distinta \cite{ferrara_common_2010}. 
Outros estudos, por sua vez, têm enfatizado o efeito riqueza sobre o consumo e indicam tais canais de transmissão são mais incidentes, em ordem, sobre Estados Unidos e Grã Bretanha mas mais brandos no caso francês e alemão \cites{sastre_assessment_2010}{chauvin_wealth_2010}{bassanetti_effects_2010}{arrondel_housing_2010}. Por fim, \textcite{huang_is_2018} testam ambas as hipóteses aventadas por Leamer  (predição e causalidade) e concluem que: (i) o investimento residencial não é um mero canal de transmissão da política monetária; (ii) a construção de novos imóveis tem maior capacidade preditiva que os preços; (iii) o preço dos imóveis tem maior influência no longo prazo e (iv) os resultados sobre a relação de causalidade não são conclusivos para todos os países dada heterogeneidade institucional.

A pluralidade de resultados reportada acima sugere que a especificidade institucional de cada país desempenha um papel central nas implicações macroeconômicas do investimento residencial e, portanto, carece de uma análise mais pormenorizada. A título de exemplo, o caso alemão se destoa dos demais em que \textcite{wijburg_alternative_2017} destacam a estabilidade de longo prazo do mercado imobiliário alemão é um contra ponto ao mercado ameriacano\footnote{
	A metodologia utilizada por \textcite{wijburg_alternative_2017} é a das ondas de financeirização em que a última onda iniciou no fim da CFG. Dito isso, os autores negam a ideia de que o mercado imobiliário alemão não é financeirizado uma vez que a financeirização imobiliária pode assumir várias formas.}. 
Apontam que os preços dos imóveis na Alemanha estagnaram enquanto o resto do mundo presenciou um aumento. No entanto, observa-se um movimento recente de aumento nos preços no país, indicando uma maior relevância do tema em um futuro próximo:

\begin{quote}
\textit{On the one hand, the German housing market was one of the few markets in Western Europe that was not severely affected by the global housing boom of the early 2000s. On the other hand, recent developments suggest that the role of finance in the German housing system is \textbf{changing}, but not in the same way as in other countries} \cite[p.~969, grifos adicionados]{wijburg_alternative_2017}
\end{quote}

Neste ponto, cabe destacar a importância da institucionalidade para a compreensão da distinção entro os países bem com as inter-relações entre o mercado imobiliário com o mercado de crédito.

INVESTIMENTO RESIDENCIAL, INSTITUCIONALIDADE: Van guten e cagnin

Dentre os determinantes institucionais, destaca-se: (i) transferência de riscos, dentre eles a securitização que têm aumentado entre os países europeus \cite{european_central_bank_housing_2010}; (ii) disponibilidade de crédito de longo-prazo para as famílias \cite{schwartz_politics_2009}; (iii) duração das hipotecas e existência de um mercado secundário; (iv) determinação  e tipo da taxa de juros das hipotecas (fixa ou flexível); (v) arranjo regulatório sobre reembolso antecipado (contrato ou legislação) e formas de refinanciamento; acesso a linhas de crédito, ou seja, permissividade da retirada do capital próprio (\textit{equity withdrawal contracts})

Dentre os itens elencados anteriormente, destaca-se o acesso a linhas de crédito através das hipotecas que é relevante para o caso norte-americano \cite{cagnin_o_2009} --- e possuem efeitos significativos sobre o ciclo econômico uma vez que impacta o consumo de bens duráveis \cite{leamer_housing_2007} --- mas são mais incomuns nos países europeus \cite[p.~95]{van_gunten_varieties_2018}.

\begin{table}[htb]
	\centering
	\caption{Características institucionais de alguns países europeus}
	\label{Institucional}
	\resizebox{\textwidth}{!}{%
		\begin{tabular}{l|c|c|c|c|c|c}
			\hline \hline\\
			\textbf{Fatores institucionias}                                                              & \multicolumn{1}{c}{\textbf{França}} & \multicolumn{1}{c}{\textbf{Alemanha}} & \multicolumn{1}{c}{\textbf{Itália}} & \multicolumn{1}{c}{\textbf{Holanda}} & \multicolumn{1}{c}{\textbf{Portugal}} & \multicolumn{1}{c}{\textbf{Espanha}} \\\hline
			Maturidade das hipotecas                                                                       & 19                                  & 25-30                                 & 22                                  & 30                                   & 30-40                                 & 30                                   \\\hline
			Tipo de taxa de juros                                                                        & Fixa                                & Fixa                                  & Variável                            & Fixa                                 & Variável                              & Variável                             \\\hline
			\begin{tabular}[c]{@{}l@{}}Reembolso antecipado:\\ Contrato (C)/ Legislação (L)\end{tabular} & C/L                                 & C/L                                   & L                                   & C                                    & L                                     & C/L                                  \\\hline
			\begin{tabular}[c]{@{}l@{}}Retirada de capital próprio \\ (Permissão)\end{tabular}           & Não                                 & Não                                   & Não                                 & Sim                                  & -                                     & Limitado                             \\\hline
			\begin{tabular}[c]{@{}l@{}}Financiamento pelo \\ mercado de capitais\end{tabular}            & 12\%                                & 14\%                                  & 20\%                                & 25\%                                 & 27\%                                  & 45\%                                 \\\hline
			\begin{tabular}[c]{@{}l@{}}Execução hipotecária (\textit{Foreclosure}): \\ duração (meses)\end{tabular}             & 20                                  & 9                                     & 56                                  & 5                                    & 24                                    & 8 \\ \hline\hline                                 
		\end{tabular}%
	}
\caption*{\textbf{Fonte:}  \textcite[p.~94, adaptado e traduzido]{van_gunten_varieties_2018}}
\end{table}



Pontuada a importância deste gasto, cabe inspecionar a forma com que a heterodoxia tratou do tema. Parte significativa desta literatura  --- emergente no pós-crise imobiliária --- centra esforços na conexão deste tipo de gasto com processos mais gerais como a financeirização \cites{aalbers_financialization_2008}{bibow_financialization_2010} enquanto uma fração minoritária o relaciona com as variabilidades de capitalismo e as relações com o \textit{welfare state} \cite{schwartz_politics_2009}. Uma análise alternativa --- apesar de não heterodoxa --- é da  ``hipotecarização'' desenvolvida por \textcite{jorda_great_2014} que destaca a crescente participação das hipotecas nos balanços patrimoniais dos bancos\footnote{\textcite{jorda_great_2014} também destacam que o crédito hipotecário era concedido fora do sistema bancário até os 1900 e isso dificulta a estimação dos dados.} (ver gráfico \ref{GraficoJorda}): 

\begin{quote}
	\textit{To a large extent the core business model of banks in advanced economies today resembles that of real estate funds: banks are borrowing (short) from the public and capital markets to invest (long) into assets linked to real estate.} [...] \textit{looking more deeply at the composition of bank credit, it becomes clear that the rapid growth of \textbf{mortgage lending} to households has been the \textbf{driving force} behind this remarkable change in the composition of banks’ balance sheets} \cite[p.~2, grifos adicionados]{jorda_great_2014}
\end{quote}

Outra contribuição de \citeauthor*{jorda_great_2014} é o desenvolvimento de uma base de dados que contém os subcomponentes dos empréstimos dos bancos para 17 países da OCDE\footnote{São eles: Austria, Bélgica, Canada, Suiça, Alemanha, Dinamarca, Espanha, Finlândia, França, Reino Unido, Itália, Japão, Holanda, Noruega, Portugal, Suécia, Estados Unidos} para os anos de 1870 a 2016 e, portanto, fornece informações ainda não exploradas sobre o sistema bancário. A partir destes dados destacam que os empréstimos tem aumentado a uma velocidade superior ao valor dos ativos e, portanto, verifica-se uma maior alavancagem das famílias apesar do aumento do preço dos imóveis. Portanto, a compreensão do papel das hipotecas bem como do investimento residencial se justifica pelos impactos reais e financeiros sobre o ciclo econômico.
%Adicionalmente, afirmam que a estrutura de crédito tem implicações sobre o ciclo econômico.

%DESENVOLVER MAIS HIPOTECARIZAÇÃO  E FALAR QUE VAI SER INVESTIGADA.


\begin{figure}
	\centering
	\caption{Participação do empréstimo imobiliário no total do balanço patrimonial dos bancos (1880-2016)}
	\label{GraficoJorda}
	\includegraphics{Jorda_Mean.png}
	\caption*{\textbf{Fonte:}}
\end{figure}

Além disso, a partir da revisão bibliográfica, verificou-se que uma fração pequena da literatura heterodoxa\footnote{
	A título de menção, vale destacar também o trabalho de \textcite{zezza_u.s._2008} em que são investigados os efeitos distributivos sobre o crescimento para a economia norte-americana a partir da metodologia \textit{Stock-Flow Consistent}.}
aborda as relações entre crescimento e investimento residencial que, vale mencionar, não cria capacidade produtiva ao setor privado\footnote{A título de nota, destaca-se que debate ortodoxo sobre desenvolvimento e investimento residencial centrado na década de 60-70 (ver \textcite{arku_housing_2006}) se restringiu em categorizá-lo como um gasto absorvedor de recursos produtivos \cite{solow_importance_1995} e indicava  a possibilidade de um sobreinvestimento residencial \cite{mills_has_1987}. Desse modo, a literatura do supermultiplicador é um contraponto ao \textit{trade-off} apontado por \textcite{solow_importance_1995} uma vez que são os gastos autônomos que lideram o crescimento no longo prazo. 
}. Uma forma de incluir esse gasto nos modelos de crescimento heterodoxos é a de \textcite{da_silveira_investimento_2019} em que os autores utilizam o supermultiplicador sraffiano (SSM em inglês) por estabelecer um papel fundamental aos gastos autônomos que não criam capacidade no crescimento econômico e na acumulação de capital. Na contribuição original de \textcite{serrano_sraffian_1995} e nas apresentações mais recentes \cite{freitas_growth_2015}, o modelo é apresentado de modo bastante parcimonioso para evidenciá-lo como um fechamento alternativo, dentro da tradição da teoria do crescimento liderada pela demanda \cite{serrano_sraffian_2017}. Nesta família de modelos: (i) o grau de utilização converge ao normal no longo prazo; (ii) a distribuição renda tem efeitos de nível apenas e; (iii) a taxa de crescimento converge a taxa de crescimento dos gastos autônomos.

A partir do estabelecimento do SSM, algumas questões são colocadas: quais são esses gastos autônomos e quais seus determinantes? qual o padrão de financiamento e suas consequências? \textcite{pariboni_household_2016} e \textcite{fagundes_role_2017}, por exemplo, avançaram em detalhar o consumo financiado por crédito.  \textcite{brochier_supermultiplier_2018}, por sua vez, incorporam o SSM em uma estrutura contábil mais completa, o arcabouço de consistência entre fluxos e estoques (SFC, na sigla em inglês), para compreender a dinâmica do consumo a partir da riqueza. No entanto, um gasto autônomo tem sido negligenciado: o investimento residencial. 

Uma forma de conectar o investimento residencial com o modelo do supermultiplicador sraffiano é por meio da taxa própria de juros dos imóveis (Taxa Própria) desenvolvida por \textcite{teixeira_crescimento_2015} para avaliar o caso norte americano e é definida como a taxa de juros hipotecária ($r_{mo}$) deflacionada pela inflação dos imóveis ({$\dot p_h$}) de modo que o investimento residencial, autônomo e não criador de capacidade produtiva, cresce a taxa $g_Z$ dada por:
\begin{equation}
g_Z = \phi_0 - \phi_1 \overbrace{\left(\frac{1+r_{mo}}{1+\dot p_h} - 1\right)}^{\text{Taxa Própria}}
\end{equation}
em que os $\phi_i$s são parâmetros e cujo termo em parênteses é a Taxa Própria. O primeiro parâmetro se refere aos determinantes de longo prazo (\textit{e.g.} arranjos institucionais do mercado imobiliários e de crédito) enquanto o segundo capta a demanda por imóveis decorrente das expectativas de ganhos de capital resultantes da especulação com o estoque de imóveis existente e diz respeito ao ciclo econômico.

Em outras palavras, a taxa de juros das hipotecas capta o serviço da dívida para os ``investidores'' (neste caso, famílias) enquanto a variação do preço dos imóveis permite incorporar mudança no patrimonio líquido. Portanto, aufere de modo satisfatório o custo real em imóveis de se comprar imóveis \cite[p.~53]{teixeira_crescimento_2015}. Desse modo, a partir da taxa própria de juros do imóveis é possível revelar importância do investimento residencial para além do ciclo e estendê-la para o longo prazo.  Tal proposta, portanto, lança luz sobre a influência da inflação imobiliária na construção de novos imóveis e, de acordo com o supermultiplicador sraffiano, sobre o produto como um todo. 

Como mencionado anteriormente, a referida taxa própria dos imóveis foi desenvolvida para examinar a bolha de ativos ocorria nos EUA e, portanto, requer uma maior investigação a despeito da aplicabilidade para outros países e este é um dos objetivos desta pesquisa (Ver seção BLA). Além disso, uma vez que a dívida hipotecária é o principal componente do endividamento das famílias, se faz necessária uma melhor compreenssão da conexão entre o investimento residencial com as formas de financiamento e estoques financeiros de forma integrada. Nesses termos, a abordagem \textit{Stock-Flow Consistent} se mostra a mais adequada para este tipo de análise (Ver seção Bla).

SFC E SSM COMO ALTERNATIVA

PERGUNTA

Portanto, a presente investigação estende as contribuições de \textcite{serrano_sraffian_1995} ao incluir o investimento residencial na agenda de pesquisa do supermultiplicador sraffiano tal como em \textcite{da_silveira_investimento_2019}, de \textcite{teixeira_crescimento_2015} ao incorporar o conceito de taxa própria de juros dos imóveis para avaliar a dinâmica de tal gasto autônomo, de \textcite{brochier_supermultiplier_2018} por adicionar um tratamento adequado das relações financeiras no SSM por meio da metodologia SFC 
e a de \textcite{jorda_great_2014} ao lançar luz sobre o processo de ``hipotecarização''. 




\begin{comment}
%=====================================================
%				TEMPORARIAMENTE DESCARTADO
%=====================================================


\end{comment}




\section{Objetivos}\label{OBJ}

Esta seção irá apresentar os objetivos desta pesquisa divididos em dois grupos: geral e específicos. Isto posto, a seção \ref{Just} irá realças as justificativas para esta investigação.\todo{Precisa?}

\begin{description}
	\item[Objetivo geral] Analisar a dinâmica da economia brasileira em termos de crescimento nos anos de 2003-2014 com ênfase nas mudanças redistributivas observadas assim como identificar os fatores que explicam esta trajetória;
	\item[Objetivos específicos] {\color{white}são eles}
% TODO Primeiro item
	\begin{itemize}
		\item Investigar as diferentes teorias de crescimento heterodoxas e suas respectivas relações com distribuição de renda;
		\item Apresentar a teoria monetária da distribuição de \textcite{pivetti_essay_1992} assim como suas limitações e adequar este arcabouço teórico ao Brasil;
		\item Explorar as mudanças na distribuição pessoal e funcional da renda no caso brasileiro;
		\item Dialogar com a literatura assim como expor suas respectivas limitações e  diferenças argumentativas em relação ao objetivo geral apresentado;
		\item Explicitar as políticas econômicas adotadas no período assim como seus impactos à luz da teoria monetária da distribuição, tais como:
		\begin{itemize}
			\item Ampliação do crédito ao consumidor e endividamento das famílias;
			\item Determinação da taxa de juros e distribuição de renda;
			\item Valorização real do salário mínimo e participação dos salários na renda;
		\end{itemize}
		\item Examinar a economia brasileira à luz do modelo do supermultiplicador sraffiano a partir de simulações computacionais.
	\end{itemize}
\end{description}



\section{Metodologia}\label{passos}

%MAIS SOBRE BASE DE DADOS E INSTITUIÇÕES

%BREVE PARÁGRAFO SOBRE OCDE


A metodologia para a primeira parte consiste em uma ampla revisão de literatura e de identificação de argumentos centrais sobre a relação entre investimento residencial e dinâmica macroeconômica. Neste capítulo, em especial, serão investigados as especificidades institucionais das inter-relações entre mercado de crédito e imobiliário. Dentre os principais determinantes institucionais (organizados na tabela \ref{Institucional} para alguns países) a serem analisados, destaca-se: (i) possibilidade de transferência de riscos (\textit{e.g.} securitização\footnote{Para uma descrição do aumento da securitização nos Estados Unidos, ver \textcite{green_american_2005} e \textcite{cagnin_o_2009}. Destaca-se também o aumento desta prática entre os países europeus \cite{european_central_bank_housing_2010}.}); (ii) disponibilidade de crédito de longo-prazo para as famílias \cite{schwartz_politics_2009}; (iii) duração das hipotecas e existência de um mercado secundário \cite{green_american_2005}; (iv) determinação  e tipo da taxa de juros das hipotecas (fixa ou flexível); (v) arranjo regulatório sobre reembolso antecipado (contrato ou legislação) e formas de refinanciamento e; (vi) permissividade da retirada do capital próprio (\textit{equity withdrawal contracts}). Dentre os itens elencados anteriormente, destaca-se o acesso a linhas de crédito através das hipotecas cuja relevância é maior para o caso norte-americano --- pelos efeitos significativos já mencionados sobre o ciclo econômico --- e por serem mais incomuns nos países europeus \cite[p.~95]{van_gunten_varieties_2018}.

Vale ressaltar que uma análise pormenorizada de cada um destes elementos, no entanto, seria uma agenda de pesquisa por si só. Desse modo, uma análise comparativa qualitativa (QCA, desenvolvida por \textcite{ragin_comparative_1989}) é a melhor forma de captar tais elementos sem, ao mesmo tempo, incorrer em imprecisões econométricas decorrentes do número de observações e sem se limitar a um estudo de caso reduzido\footnote{A metodologia que pretendemos usar para dar conta desse objetivo é semelhante a utilizada em outro trabalho \cite{petrini_comparacao_2019} aplicada a outro objeto.}. A luz deste procedimento, o que seria uma análise \textit{cross-section} pouco robusta, se torna um estudo de caso comparado capaz de dar conta dos principais determinantes institucionais responsáveis pela relação entre mercado de crédito, imobiliário e dinâmica macroeconômica para um conjunto considerável de países. Adicionalmente, a escolha desta metodologia se dá por\footnote{Uma das vantagens dessa metodologia é a capacidade --- via operações de lógica booleana --- de se realizar estudos comparativos para um número maior de casos em que o número de observações é insuficiente para uma análise quantitativa do tipo \textit{cross-section}.}: (i) enfatizar as singularidades de cada unidade de investigação; (ii) por tratar os casos holisticamente, ou seja, como unidades integradas por uma complexa combinação de propriedades e; (iii) ser possível destacar quais elementos institucionais são necessários ou suficientes para que mudanças no mercado imobiliário afetem o mercado de crédito.

\begin{table}[htb]
	\centering
	\caption{Características institucionais de alguns países europeus}
	\label{Institucional}
	\resizebox{\textwidth}{!}{%
		\begin{tabular}{l|c|c|c|c|c|c}
			\hline \hline\\
			\textbf{Fatores institucionias}                                                              & \multicolumn{1}{c}{\textbf{França}} & \multicolumn{1}{c}{\textbf{Alemanha}} & \multicolumn{1}{c}{\textbf{Itália}} & \multicolumn{1}{c}{\textbf{Holanda}} & \multicolumn{1}{c}{\textbf{Portugal}} & \multicolumn{1}{c}{\textbf{Espanha}} \\\hline
			Maturidade das hipotecas                                                                       & 19                                  & 25-30                                 & 22                                  & 30                                   & 30-40                                 & 30                                   \\\hline
			Tipo de taxa de juros                                                                        & Fixa                                & Fixa                                  & Variável                            & Fixa                                 & Variável                              & Variável                             \\\hline
			\begin{tabular}[c]{@{}l@{}}Reembolso antecipado:\\ Contrato (C)/ Legislação (L)\end{tabular} & C/L                                 & C/L                                   & L                                   & C                                    & L                                     & C/L                                  \\\hline
			\begin{tabular}[c]{@{}l@{}}Retirada de capital próprio \\ (Permissão)\end{tabular}           & Não                                 & Não                                   & Não                                 & Sim                                  & -                                     & Limitado                             \\\hline
			\begin{tabular}[c]{@{}l@{}}Financiamento pelo \\ mercado de capitais\end{tabular}            & 12\%                                & 14\%                                  & 20\%                                & 25\%                                 & 27\%                                  & 45\%                                 \\\hline
			\begin{tabular}[c]{@{}l@{}}Execução hipotecária (\textit{Foreclosure}): \\ duração (meses)\end{tabular}             & 20                                  & 9                                     & 56                                  & 5                                    & 24                                    & 8 \\ \hline\hline                                 
		\end{tabular}%
	}
	\caption*{\textbf{Fonte:}  \textcite[p.~94, adaptado e traduzido]{van_gunten_varieties_2018}}
\end{table}


Compreendidos os fatores institucionais, a segunda parte desta pesquisa irá analisar os determinantes macroeconômicos do investimento residencial por meio de um modelo de dados em painel dinâmicos por permitir incorporar as defasagens de algumas variáveis e, assim, enriquecer a análise\footnote{Cabe aqui pontuar que \textcite{petrini_investimento_2019} encontrou defasagens estatisticamente significantes entre taxa real de juros dos imóveis e taxa de crescimento dos imóveis para o caso norte-americano por meio de um VEC. A realização de um modelo de dados em painel também é, portanto, uma extensão de \textcite{petrini_demanda_2019}.}.
É importante ressaltar que para manter a comparatibilidade entre esses dois capítulos, serão utilizados os países presentes --- todos eles países-membro da OCDE\footnote{Vale mencionar que por serem países membros da OCDE, possuem um grau maior de comparação entre si por terem características semelhantes e, portanto, destaca-se melhor as especificidades institucionais mencionadas anteriormente.} --- na base de dados desenvolvida por \textcite{jorda_great_2014}, são eles: Alemanha, Áustria, Bélgica, Canadá, Dinamarca, Espanha, Estados Unidos, Finlândia,  França,   Holanda, Itália, Japão,  Noruega, Portugal, Reino Unido, Suécia e Suíça. Vale pontuar que a grande contribuição desta base de dados é reunir os subcomponentes dos empréstimos bancários desde 1870 que abre uma extensa agenda de pesquisa ainda não suficientemente explorada.
Apesar da amplitude temporal desta base, o modelo macroeconométrico se restringirá ao pós década de 80 para captar os efeitos da ``hipotecarização'' e contrastá-los com o modelo qualitativo desenvolvido no capítulo anterior.

Como apresentado anteriormente, o modelo SFC é o ponto de chegada da pesquisa\footnote{Para tanto, será utilizado o pacote \textit{pysolve3} escrito em python 3 e desenvolvido por \textcite{petrini_pysolve3_2019}.} em que são reunidos os esforços da análise qualitativa bem como os resultados do modelo empírico. Sucintamente, tal metodologia é composta de três procedimentos: (i) determinação da estrutura contábil; (ii) construção das equações comportamentais e; (iii) solução/simulação. Dito isso, a figura \ref{Resuminho} resume as etapas mencionadas e explicitar a diferença entre a \textit{metodologia} e um \textit{modelo} SFC. 


\begin{figure}[htb]
	\caption{Resumo esquemático da Metodologia SFC}
	\label{Resuminho}
	\centering
	\begin{tikzpicture}
	[node distance = 1cm, auto,font=\footnotesize,
	% STYLES
	every node/.style={node distance=3cm},
	% The comment style is used to describe the characteristics of each force
	comment/.style={rectangle, inner sep= 5pt, text width=4cm, node distance=0.25cm, font=\scriptsize\sffamily},
	% The force style is used to draw the forces' name
	force/.style={rectangle, draw, fill=black!10, inner sep=5pt, text width=4cm, text badly centered, minimum height=1.2cm, font=\bfseries\footnotesize\sffamily}] 
	
	% Draw forces
	\node [force] (rivalry) {Hipóteses};
	\node [force, above of=rivalry, fill=red!70] (substitutes) {Equações comportamentais};
	\node [force, text width=3cm, dashed, left=1.5cm of substitutes,fill=blue!50] (state) {Metodologia SFC};
	\node [force, left=1cm of rivalry] (suppliers) {Estrutura contábil};
	\node [comment, below=0.25 of suppliers] (comment-suppliers) {Relação entre lado real \\e financeiro};
	\node [force, right=1cm of rivalry] (users) {Solução};
	\node [force, right=1cm of substitutes, dashed, fill=purple!50 ] (PK) {Modelo \\_-SFC};
	%	\node [force, below of=rivalry] (entrants) {Threat of new entrants};
	
	%%%%%%%%%%%%%%%
	% Change data from here
	
	% RIVALRY
	\node [comment, below=0.25 of rivalry] (comment-rivalry) {Cambridge/New Cambridge\\
		Kaleckiano\\
		\textbf{Supermultiplicador Sraffiano}};
	
	
	% SUBSTITUTES
	%\node [comment, right=0.25 of substitutes] {};
	
	% USERS
	\node [comment, below=0.25 of users] {Analítico\\
		\textbf{Simulação}};
	
	% NEW ENTRANTS
	%	\node [comment, right=0.25 of entrants] {(+) EC vs. Microsoft};
	
	% PUBLIC POLICIES
	%	\node [comment, text width=3cm, below=0.25 of state] {(1) Estrutura contábil\\
	%	(2) Equações comportamentais\\
	%	(3) \textit{Closure} do modelo};
	
	%%%%%%%%%%%%%%%%
	
	% Draw the links between forces
	\path[->,thick] 
	(rivalry) edge (substitutes)
	(suppliers) edge (rivalry)
%	(rivalry) edge (users)
	(state) edge (substitutes)
	(state) edge (suppliers)
	(substitutes) edge (PK)
	(PK) edge (users);
	
	%(entrants) edge (comment-rivalry);
	
	\end{tikzpicture} 
	\caption*{Fonte: \textcite[p.~64, adaptado]{da_silveira_politica_2017}}

	
\end{figure}



As etapas contábeis da abordagem SFC constituem em: (i) seleção dos setores institucionais e dos ativos a serem incorporados; (ii) mapeamento das relações dos fluxos entre os mencionados setores por meio da construção da matriz de fluxos; (iii) construção da matriz dos estoques de riqueza (real e financeira) em que são contabilizadas os ativos e passivos  bem como a posição líquida de cada setor; (iv) identificação das formas que os fluxos são financiados e sua respectiva acumulação/alocação dos estoques. 
Como todo modelo macroeconômico, ao partir de um aparato analítico
baseado em identidades contábeis, surgem restrições que precisam ser seguidas mas o que distingue
a metodologia SFC das demais é a conexão do lado real com o financeiro de forma integrada.
Tal procedimento garante que para que um setor acumule riqueza financeira, outro precisa necessariamente liquidá-la de modo que não existam ``buracos negros'' \cite{godley_money_1996}.

Com isso, conclui-se a estrutura contábil que, no entanto, por se basear em identidades, nada de causal poder ser extraído desta etapa. As relações de causalidade do modelo (agora modelo e não metodologia) decorrem das equações comportamentais que, respeitando a consistência, podem ser de qualquer linhagem teórica. Feitas essas ressalvas, dada a estrutura contábil e explicitadas as hipóteses (via equações comportamentais), resta seguir para a resolução do modelo. Como pontuam \textcite{caverzasi_stock-flow_2013}, existem três vias: (i) simulação; (ii) analítica e; (iii) descritiva. A primeira delas permite expor as relações entre as variáveis de modelos mais complexos em que a solução analítica não é facilmente encontrada. No entanto, tal caminho fez com que o grau de complexidade dos modelos simulados fosse exponencializada de modo que a intuição econômica torna-se facilmente turva.  
Diante disso, esta pesquisa prioriza a parcimônia de modo que serão incluídos apenas os elementos necessários dados os objetivos desta pesquisa.
Tal postura permite encontrar soluções analíticas com maior facilidade de modo que são explicitados os parâmetros mais relevantes para as
trajetórias de longo prazo. 
Apesar de parcimônia do modelo, a simulação tem a vantagem de fornecer informações que não se restringem às soluções de equilíbrio e esta forma também será selecionada para resolver o modelo\footnote{Cabe aqui pontuar a realização de modelos de simulação em \textcite{da_silveira_investimento_2019} e \textcite{petrini_demanda_2019}.}. 

\section{Resultados esperados}\label{Result}
Realizada esta pesquisa, esperam-se os seguintes resultados:
\begin{itemize}
	\item As mudanças redistributivas observadas são relevantes para explicar a dinâmica da economia brasileira no período em questão;
	\item O crédito ao consumidor teve efeitos significativos tanto sobre o consumo de bens duráveis quanto no aumento do endividamento das famílias;
	\item O maior acesso ao crédito decorre tanto da maior participação dos salários na renda viabilizada pelas valorizações reais do salário mínimo (aumento do colateral) quanto medidas deliberadas de política econômica;
	\item Encontrar uma taxa de juros relevante ao longo prazo tal como argumentado por \textcite{pivetti_essay_1992};
	\item Espera-se destacar o conflito distributivo por meio de mudanças na taxa de juros mencionada acima para o caso brasileiro;
	\item Os componentes que explicam a dinâmica econômica do Brasil podem ser captados pelo modelo do supermultiplicador sraffiano.
\end{itemize} % Resultados esperados temporariamente desativados
\section{Plano de trabalho e cronograma de atividades}\label{cronograma}

A tabela \ref{crono} apresenta um esboço das atividades a serem desempenhadas ao longo desta pesquisa. 

\begin{table}[H]
	\centering
	\caption{Cronograma de atividades}
	\tiny
	\label{crono}
	\resizebox{\textwidth}{!}{%
	\begin{tabular}{ll|l|l|l|l|ll}
	\hline\hline
\multicolumn{1}{c}{} & \multicolumn{6}{c}{\textbf{Período}} \\ \cline{2-7} 
\multicolumn{1}{c}{\multirow{-2}{*}{\textbf{Atividades}}} & \multicolumn{1}{c|}{\textbf{1º Semestre 2020}} & \multicolumn{1}{c|}{\textbf{2º Semestre  2020}} & \multicolumn{1}{c|}{\textbf{1º Semestre  2021}} & \multicolumn{1}{c|}{\textbf{2º Semestre  2021}} & \multicolumn{1}{c|}{\textbf{2022}} & \multicolumn{1}{c}{\textbf{2023}} \\ \hline

\textbf{1. Fundamentação teórica} & \cellcolor[HTML]{9B9B9B} &\cellcolor[HTML]{9B9B9B}&\cellcolor[HTML]{9B9B9B}&  & &  \\ \hline
1.1. Disciplinas & \cellcolor[HTML]{9B9B9B} &&&&&  \\ \hline
1.2. Revisão bibliográfica & \cellcolor[HTML]{9B9B9B} &\cellcolor[HTML]{9B9B9B}&\cellcolor[HTML]{9B9B9B}&  & &  \\ \hline

\textbf{2. Modelo Qualitativo} &&&\cellcolor[HTML]{9B9B9B}&\cellcolor[HTML]{9B9B9B}&& \\ \hline
2.1. Análise comparativa &&&\cellcolor[HTML]{9B9B9B}&\cellcolor[HTML]{9B9B9B}&& \\ \hline
2.2. Construção e resultados &&&&\cellcolor[HTML]{9B9B9B}&& \\ \hline

\textbf{3. Qualificação} &&&&\cellcolor[HTML]{9B9B9B}&& \\ \hline

\textbf{4. Modelo Quantitativo} &&&&\cellcolor[HTML]{9B9B9B}&\cellcolor[HTML]{9B9B9B}& \\ \hline
4.1. Preparação dos dados &&&&\cellcolor[HTML]{9B9B9B}&& \\ \hline
4.2. Estimação e análise &&&&\cellcolor[HTML]{9B9B9B}&\cellcolor[HTML]{9B9B9B}& \\ \hline

\textbf{5. Preparação sanduíche}\footnotemark &&&&\cellcolor[HTML]{9B9B9B}&\cellcolor[HTML]{9B9B9B}& \\ \hline

\textbf{6. Modelo SFC} &&&&\cellcolor[HTML]{9B9B9B}&\cellcolor[HTML]{9B9B9B}& \\ \hline
6.1. Construção &&&&\cellcolor[HTML]{9B9B9B}&& \\ \hline
6.2. Simulação e análise &&&&&\cellcolor[HTML]{9B9B9B}& \\ \hline

\textbf{7. Conclusão e Defesa} & & &  &  & & \cellcolor[HTML]{9B9B9B} \\ \hline \hline
		
	

\end{tabular}%
	\renewcommand{\arraystretch}{0.4}
	}
\caption*{\textbf{Fonte:} Elaboração própria}
\end{table}
\footnotetext{A depender da disponibilidade de financiamento.}




\begin{comment}

\end{comment}





\endgroup

%=====================================================================
% 							Bibliografia
%=====================================================================
%{\let\clearpage\relax \chapter{Bibliografia}}
%-----

{\let\clearpage\relax\printbibliography}
%=====================================================================
% 							Fim do Documento
%=====================================================================
\end{document}