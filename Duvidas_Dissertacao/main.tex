\documentclass[]{article}
\usepackage{lmodern}
\usepackage{amssymb,amsmath}
\usepackage{ifxetex,ifluatex}
\ifnum 0\ifxetex 1\fi\ifluatex 1\fi=0 % if pdftex
  \usepackage[T1]{fontenc}
  \usepackage[utf8]{inputenc}
\else % if luatex or xelatex
  \ifxetex
    \usepackage{mathspec}
  \else
    \usepackage{fontspec}
  \fi
  \defaultfontfeatures{Ligatures=TeX,Scale=MatchLowercase}
\fi
% use upquote if available, for straight quotes in verbatim environments
\IfFileExists{upquote.sty}{\usepackage{upquote}}{}
% use microtype if available
\IfFileExists{microtype.sty}{%
\usepackage{microtype}
\UseMicrotypeSet[protrusion]{basicmath} % disable protrusion for tt fonts
}{}
\usepackage[margin=1in]{geometry}
\usepackage{hyperref}
\hypersetup{unicode=true,
            pdftitle={Dúvidas e impressões},
            pdfborder={0 0 0},
            breaklinks=true}
\urlstyle{same}  % don't use monospace font for urls
\usepackage{graphicx,grffile}
\makeatletter
\def\maxwidth{\ifdim\Gin@nat@width>\linewidth\linewidth\else\Gin@nat@width\fi}
\def\maxheight{\ifdim\Gin@nat@height>\textheight\textheight\else\Gin@nat@height\fi}
\makeatother
% Scale images if necessary, so that they will not overflow the page
% margins by default, and it is still possible to overwrite the defaults
% using explicit options in \includegraphics[width, height, ...]{}
\setkeys{Gin}{width=\maxwidth,height=\maxheight,keepaspectratio}
\IfFileExists{parskip.sty}{%
\usepackage{parskip}
}{% else
\setlength{\parindent}{0pt}
\setlength{\parskip}{6pt plus 2pt minus 1pt}
}
\setlength{\emergencystretch}{3em}  % prevent overfull lines
\providecommand{\tightlist}{%
  \setlength{\itemsep}{0pt}\setlength{\parskip}{0pt}}
\setcounter{secnumdepth}{5}
% Redefines (sub)paragraphs to behave more like sections
\ifx\paragraph\undefined\else
\let\oldparagraph\paragraph
\renewcommand{\paragraph}[1]{\oldparagraph{#1}\mbox{}}
\fi
\ifx\subparagraph\undefined\else
\let\oldsubparagraph\subparagraph
\renewcommand{\subparagraph}[1]{\oldsubparagraph{#1}\mbox{}}
\fi

%%% Use protect on footnotes to avoid problems with footnotes in titles
\let\rmarkdownfootnote\footnote%
\def\footnote{\protect\rmarkdownfootnote}

%%% Change title format to be more compact
\usepackage{titling}

% Create subtitle command for use in maketitle
\newcommand{\subtitle}[1]{
  \posttitle{
    \begin{center}\large#1\end{center}
    }
}

\setlength{\droptitle}{-2em}

  \title{Dúvidas e impressões}
    \pretitle{\vspace{\droptitle}\centering\huge}
  \posttitle{\par}
    \author{}
    \preauthor{}\postauthor{}
    \date{}
    \predate{}\postdate{}
 
\usepackage{color,soul}
\usepackage[dvipsnames]{xcolor}



\begin{document}
\maketitle
\newcommand{\open}{$\square$}
\newcommand{\done}{$\boxtimes$}

{
\setcounter{tocdepth}{2}
\tableofcontents
}
\section{Teoria Sraffiana}\label{teoria-sraffiana}

\begin{itemize}
\tightlist
\item
  [\done] Seguindo o raciocínio de Amadeo (1986), fora do pleno-emprego,
  a economia não está operando sob a curva salário-lucro. Se isso
  estiver correto, para o modelo do supermultiplicador permitir que a
  distribuição seja determinada exogenamente por uma teoria sraffiana,
  se faz necessário que \(u_n = 1\) ou para \(u_n \neq 1\) (ex:
  \(u_n = 0.8\)) isso ainda é válido?

  \begin{quote}
    \textbf{Lucas:} O Amadeo não está correto nesse ponto. Para
    operar sobre a curva salário-lucro é necessário que o \underline{grau de
    utilização seja normal}. No modelo do supermultiplicador, esse grau
    ocorre independente de haver ou não pleno emprego.
    
    Além disso, para a distribuição (entendida como wage-share e profit-share) ser determinada exogenamente não é necessário estar na curva-salário lucro. A curva salário lucro representa uma relação entre salário real e taxa real de lucro.  Um exemplo disso, é o modelo kaleckiano. As parcelas distributivas são exógenas (determinadas pelo mark-up), mas a economia não opera em cima da curva salário-lucro, pq a \underline{taxa de lucro é endógena} (e qualquer uma, determinada pelos \hl{gastos dos capitalitas}). 
    
    Esse é o tema daquela aula do curso de macro que compara os diferente fechamentos. Uma boa bibliografia para isso é a dissertação de mestrado do Franklin (já te passei?)
    \end{quote}
    \begin{itemize}
     \item[]
        \begin{itemize}
        \item[] \textbf{Gabriel:} Me interessei, irei ler.
        \end{itemize}
    \end{itemize}
\item
  [\done] Pelo pouco que li sobre a teoria de Sraffa, o lucro é a
  variável independente que não é determinada pelos preços relativos nem
  pela esfera da produção, mas sim por outros determinantes tal como a
  taxa de juros. Se isso estiver correto, a negação do paradoxo dos
  custos (entendido como uma redução do \emph{mark-up} gerar aumento na \underline{massa de lucro}) é uma consequência lógica se adotar uma abordagem sraffiana?
  Em outras palavras, se a determinação dos lucros não depender dos
  preços, o paradoxo dos custos deixa de ser paradoxal?
\end{itemize}

\begin{quote}
    \textbf{Lucas:} Essa é uma das interpretações da teoria do Sraffa, seguida pelo Pivetti, Garegnani e Franklin. Existem outras interpretações. 
    
Dito isso, não entendi muito bem o seu ponto. O que vc quer dizer por “aumento nos lucros”? é a \textbf{massa}, a taxa ou a margem de lucro? Além disso, vc está pensando nessa discussão em \textbf{nível} ou em taxa de crescimento?
\end{quote}

\begin{itemize}
 \item[]
    \begin{itemize}
        \item[] \textbf{Gabriel:} Estava pensando em aumentos na \textbf{massa} de lucros para uma economia em nível. Ficou mais claro que o paradoxo dos custos se mantém em nível para ambos. 
    \end{itemize}
\end{itemize}

\begin{quote}
Se estamos falando da economia em nível (determinação do nível de renda, modelo keynesiano simples, por exemplo), tanto para sraffiano como para kaleckianos vale o “paradoxo dos custos”: aumento do salário real (queda do mark-up) geral aumento do nível do produto. Note que se a prdonesão marginal a consumir dos trabalhadores for 1 e a dos capitalistas for zero, a massa e a taxa de lucros não se altera. 

Em uma economia em crescimento, os kaleckianos dizem que existe também esse paradoxo dos custos. Nessa situação, o paradoxo dos custo ocorre quando um aumento do salário real gera aumento da taxa de crescimento da economia. Isso só ocorre nos modelos neo-kaleckianos por conta de uma característica específica da \underline{função investimento} deles, que é dependente do grau de utilização. Se o investimento fosse plenamente exógeno, mudanças distributivas não afetariam a taxa de crescimento. Ou seja, \underline{não precisamos do super para negar o paradoxo dos custos}. 
\end{quote}

\section{Capítulo teórico}\label{capitulo-teorico}

\subsection{Modelos Neo-Kaleckianos}\label{modelos-neo-kaleckianos}

\begin{itemize}
\tightlist
\item
  [\done] No \emph{Steady State} em que a taxa de crescimento do estoque
  de capital e taxa de crescimento do produto se igualam, qual a taxa de
  crescimento do grau de utilização da capacidade?

  \begin{itemize}
  \tightlist
  \item
    [\done] Se nula, este grau de utilização de longo prazo equivale ao
    normal?
  \item[\done] Se não for nulo, a economia \textbf{não} estará no \emph{Steady
    State} (Corrigido)
  \end{itemize}
\item
  [\done] Se esses modelos baseiam a distribuição funcional da renda à
  uma estrutura de mercado oligopolizada (\emph{mark-up} não-nulo)
  então, no limite:

  \begin{itemize}
  \tightlist
  \item
    [\done] Em uma economia concorrencial, a participação dos lucros na
    renda é nula? O conflito distributivo é amenizado na medida em que
    estruturas de mercado não-concorrenciais são reguladas?
  \item
    [\done] O argumento sempre recai sobre o poder
    de barganha ao longo das negociações entre os entes institucionais? Não há margem para o conflito distributivo ser internalizado na
    política econômica?
    \item \textbf{Atenção:} Não confundir economia concorrencial com concorrência perfeita.
  \item
    [\done]  Modelo de Steindl (1979): Se, no curto prazo, a
    participação dos lucros na renda depende do grau de utilização, o
    que a determina no longo prazo?
  \item
    [\done]
  \end{itemize}
\item
  [\done] A taxa de crescimento da economia é determinado pelos gastos
  autônomos, seja o componente autônomo do investimento, sejam gastos
  que não criam capacidade produtiva. Desse modo:
\end{itemize}

\begin{equation}
g = g_I
\end{equation}

\begin{itemize}
\tightlist
\item
  Ao normalizar o investimento pelo estoque de capital, (\(I/K\)),
  obtém-se (por definição!) a taxa de crescimento do estoque de capital,
  logo, fazer:
\end{itemize}

\begin{equation}
\frac{I}{K} = g
\end{equation}

\begin{itemize}
\tightlist
\item
  É incorrer em um erro contábil que implica dizer que a poupança é
  determinada pelo investimento via mudanças na capacidade produtiva
  sendo que só é verdade na presença de gastos autônomos que não criam
  capacidade.
\end{itemize}

\subsection{Supermultiplicador}\label{supermultiplicador}

\begin{itemize}
\item
  [\done] Como a prova da estabilidade do modelo de Harrod implica na
  validade da Lei de Say no longo prazo?
\item
  [\done] \(\Uparrow s \Rightarrow \Uparrow g\)? Isso é consequência da
  Lei de Say?
  \begin{itemize}
      \item \textbf{Não}, e sim a prdoneção à \underline{gastar} ser igual à 1.
  \end{itemize}
\item
  [\open] Como deduzir a Equação (16) \(z=\frac{h}{v}u\) a partir do
  crescimento do investimento?

  \begin{itemize}
  \tightlist
  \item
    Se for a taxa de crescimento do estoque de capital, a dedução é
    direta e o mesmo vale para \(g_I \to g_K\)
  \item
    Os autores queriam dizer a partir do crescimento do estoque de
    capital?
  \end{itemize}
\item
  [\open] Jacobiano para demonstrar estabilidade também é válida para
  tempo discreto?

  \begin{itemize}
  \tightlist
  \item
    [\open] \textbf{Checar:} O ferramental para tempo contínuo é
    aplicável ao tempo discreto, mas o inverso não é válido
  \end{itemize}
\item
  [\done] Como a hipótese de que o investimento é autônomo implicou na
  superação do problema da instabilidade de Harrod nos modelos
  heterodoxos?
\item
  [\done] Como os gastos autônomos, ao crescerem à uma taxa exógena,
  garantem que a taxa de crescimento da capacidade produtiva reagirá
  mais do que a demanda?
  \begin{itemize}
      \item \textbf{Gabriel:} Ficou mais claro. 
  \end{itemize}
\end{itemize}

\subsection{Proposta de apresentação simplificada dos modelos de
crescimento}\label{proposta-de-apresentacao-simplificada-dos-modelos-de-crescimento}

Seja \(u\) o grau de utilização definido como:

\begin{equation}
\label{Comum}
u = \frac{Y}{Y^*}
\end{equation}

em que \(Y\) é o produto e \(Y^*\) o produto potencial utilizado como
\emph{proxy} para a capacidade produtiva. Enquanto \(v\) é a relação
técnica capital produto definida por: \begin{equation}
v = \frac{K}{Y^*}
\end{equation} A equação \ref{Comum} pode ser reescrita como: \begin{equation}
Y = uY^* 
\end{equation} Tomando a diferença: \begin{equation}
\Delta Y = \Delta uY^* + u\Delta Y^*
\end{equation} Dividindo pelo produto: \begin{equation}
g_Y = g_u + g_{Y^*}
\end{equation} Com a equação acima, pretende-se apresentar um denominador comum às
teorias de crescimento heterodoxas em que:

\begin{itemize}
\tightlist
\item
  \textbf{Cambridge:} economia opera no pleno-emprego (Kaldor (1957)) ou
  grau de utilização converge ao normal (Robinson (1962)). Além disso, o
  sistema econômico não é restringido pelo lado da demanda, mas sim pela
  oferta. Assim
\end{itemize}

\begin{equation}
g_Y \stackrel{\leftharpoonup}{=} g_{Y^*}
\end{equation}

\begin{itemize}
\tightlist
\item
  \textbf{Neo-Kaleckiano:} Economia não opera no pleno-emprego e, nos
  modelos convencionais, grau de utilização não converge ao normal e
  acomoda as mudanças no nível de atividade dada mudanças residuais na
  capacidade produtiva:
\end{itemize}

\begin{equation}
g_u \stackrel{\leftharpoonup}{=} g_Y - g_{Y^*}
\end{equation}

\begin{itemize}
\tightlist
\item
  \textbf{Supermultiplicador Sraffiano:} Grau de utilização converge ao
  normal no longo prazo e capacidade produtiva se ajusta à demanda
  efetiva dada a existência de gastos autônomos não criadores de
  capacidade:
\end{itemize}

\begin{equation}
g_{Y^*} \stackrel{\leftharpoonup}{=} g_Y
\end{equation}

\textbf{Dúvida:} Alguma passagem incorreta?

\textbf{Dúvida:} \(v\) precisa ser definido em termos do estoque de
capital com defasagem?

\textbf{Dúvida:} É correto afirmar que nos modelos de crescimento
neo-Kaleckianos a capacidade produtiva tem uma dinâmica residual? 
\begin{itemize}
    \item \textbf{Gabriel:} Vou pensar em como reformular. De todo modo, percebi que estava errado.
\end{itemize}

\section{Capítulo Fatos Estilizados}\label{capitulo-fatos-estilizados}

\section{Capítulo Modelo}\label{capitulo-modelo}

\begin{itemize}
\item
  [\open] Como incorporar investimento residencial no balanço das
  famílias? Semelhante à conta capital das firmas?
\item
  [\open] Robinson (1962) não trata de investimento residencial por se
  tratar de uma concomitância entre poupança e gasto, como tratar
  investimento residencial? O que seria essa concomitância
  (\emph{border-line})?
\item
  Em Serrano et al (2015) em que são apresentadas as condições de
  estabilidade estática e dinâmica do modelo, os autores afirma que se
  próximo de \(u=u_n\) a prdonesão marginal à gastar superar a unidade,
  a economia apresentará uma trajetória cíclica com limites superiores e
  inferiores. Esta pode ser uma forma de incluir não-linearidade ao
  modelo para simular o ciclo econômico. Ao longo da investigação dos
  fatos estilizados, pesquisarei se existem evidências para incorporar a
  não linearidade dessa forma, caso contrário, por ser uma postura
  \emph{ad hoc}, não irei adotá-la.
  \begin{itemize}
      \item \textbf{Gabriel} Lerei Hicks (1950)
  \end{itemize}
\end{itemize}


\end{document}
