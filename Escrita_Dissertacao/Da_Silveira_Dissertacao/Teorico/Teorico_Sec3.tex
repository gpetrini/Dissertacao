

\section{Gastos autônomos não criadores de capacidade e a instabilidade velada}\label{Literatura}

A presente seção tem por objetivo destacar como os modelos Kaleckianos incorporaram os gastos autônomos não criadores de capacidade produtiva ao setor privado ($Z$). Antes de prosseguir, no entanto, cabe destacar que por serem modelos na fronteira da literatura, podem não ser representativos do que se entende por modelo Kaleckiano. Desse modo, ao longo desta seção, estão sendo analisados modelos kaleckianos não-tradicionais\footnote{Manifestação deste não-consenso é a divergência do significado econômico do componente autônomo do investimento.}. Além disso, a inclusão deste componente de gasto revela a resolução parcial da instabilidade harrodiana nos modelos Kaleckianos\footnote{Como visto, nos modelos mais convencionais, portanto, a endogeneidade do grau de utilização é suficiente para contornar esse problema. As complicações mencionadas, decorrem das sofisticações dos modelos Kaleckianos.}. A razão do porquê pode ser explicitada seguindo a exposição de \textcite{hein_harrodian_2012}\footnote{Apresentação semelhante pode ser encontrada em \textcite{allain_macroeconomic_2014}}. 

%INICIO EXPOSIÇÃO
Considerando, como em \textcite{amadeo_role_1986}, que o investimento reaja às expectativas sobre o grau de utilização ($u^e$), a função de acumulação ($g_I$) pode ser reescrita como:

\begin{equation}
\label{Kalecki_Autonomous}
g_I = \gamma + \gamma_u (u^e - u_n)
\end{equation}
em que $\gamma$ corresponde ao componente autônomo do investimento e pode ser traduzido tanto como \textit{animal spirits} quanto expectativa média da taxa de crescimento de longo prazo \cite[p.~4]{allain_macroeconomic_2014}.
No curto prazo, o grau de utilização da capacidade pode ser diferente do grau de utilização de equilíbrio. Se as firmas ajustam o estoque de capital para que o grau de utilização esteja no nível desejado, tais expectativas devem ser revistas:

\begin{equation}
\Delta u^e = \xi (u - u^e), \hspace{3cm} \xi > 0
\end{equation}
Da mesma forma, as expectativas em relação à taxa de crescimento secular ($\gamma$) são corrigidas pelas taxas de crescimento efetivas ($g^*$), ou seja, 
\begin{equation}
    \Delta \gamma = \phi (g^* - \gamma)
\end{equation}
em que $\phi$ indica um fator de correção positivo. Substituindo recursivamente, obtém-se:
\begin{equation}
    \Delta \gamma = \phi \gamma_u (u - u^e) \Leftrightarrow \Delta g_I = \varphi (u^e - u_n), \hspace{2cm} \varphi > 0
\end{equation}

Tal equação implica na instabilidade de Harrod uma vez que há uma sobre/sub-estimação do grau de utilização de equilíbrio que, por sua vez, se afasta cada vez mais do grau de equilíbrio. Essa instabilidade\footnote{Vale destacar que não é necessário recorrer à mudanças nos modelos Kaleckianos para incorrer em instabilidade, como pontua \textcite{dallery_kaleckian_2007}, o que não implica que todas elas são do tipo Harrodiana.}, argumentam \textcite{hein_harrodian_2012}, decorre do coeficiente $\gamma$ da função de investimento que deixa de ser constante na medida que o grau de utilização se afasta do normal. Nesses termos, não é paradoxal um modelo apresentar estabilidade Keynesiana e não resolver a instabilidade Harrodiana\footnote{Para mostrar que este problema não se restringe a uma parcela desses modelos, é apresentado mesmo raciocínio para a equação de acumulação que inclui a sensibilidade do investimento à participação dos lucros na renda nos moldes de \textcite{bhaduri_unemployment_1990}:

\begin{equation}
\label{KaleckiHarrod}
\begin{split}
    \dot \gamma = &\xi (g^* -\gamma), \hspace{3cm} \xi > 0\\
     = &\xi[\gamma_u (u-u_n) + \gamma_\pi\pi]
\end{split}
\end{equation}
A equação acima indica que $\dot \gamma$ é necessariamente positivo, o que faz com que a taxa de crescimento se afaste da taxa de longo prazo. Em outras palavras, a instabilidade harrodiana está presente nos modelos Kaleckianos tradicionais com correção das expectativas no componente autônomo do investimento.}. 

Diante deste problema, foram feitas modificações nos modelos Kaleckianos convencionais para que a estabilidade Harrodiana fosse balizada. Grosso modo, tais mudanças têm a inclusão de gastos autônomos como denominador comum, mas uma mediação se faz necessária. Tal como aponta \textcite[p.~3]{allain_macroeconomic_2014}, os resultados são distintos a depender de quais gastos são considerados autônomos e isso será avaliado adiante. Para manter a comparatividade entre os modelos apresentados, serão realçados os resultados que dizem respeito a efeitos em comum no \textbf{longo prazo}\footnote{Destaca-se um ponto de concordância com \textcite{nikiforos_comments_2018} em que o supermultiplicador sraffiano é um modelo para longo prazo e deve ser avaliado enquanto tal. No entanto, isso não impede de incluir elementos cíclicos no modelo e será objeto do capítulo \ref{CapModelo}. Por enquanto, resta enfatizar a atenção dada aos resultados de longo prazo para torná-los devidamente comparáveis.}, são eles: (i) mudanças na distribuição de renda; (ii) alterações nas propensões à poupar; (ii) efeitos sobre o grau de utilização; (iv) impactos do aumento da taxa de crescimento dos gastos autônomos. Já aqueles resultados que são exclusivos do modelo analisado serão postos em evidência quando necessário.

%MODELO LAVOIE: EXPORTAÇÃO

%MODELO ALLAIN: GASTOS DO GOVERNO

%MODELO HEIN (2018): GASTOS DO GOVERNO E SUSTENTABILIDADE DA DÍVIDA

Por mais que o modelo de \textcite{allain_macroeconomic_2014} inclua os gastos do governo como sendo os gastos autônomos e preserve as características dos modelos Kaleckianos (em nível), \textcite{hein_autonomous_2018} argumenta que não inclui uma discussão sobre a dinâmica do \textit{déficit} e da dívida pública no longo prazo. Os gastos do governo, financiados por crédito e emissão monetária, crescem a uma taxa exógena tal como em \textcite{allain_macroeconomic_2014}. Uma distinção deste modelo é que o autor julga não ser razoável, dada a incerteza keynesiana fundamental, que o grau de utilização convirja ao normal no longo prazo\footnote{Dentre as equações para o equilíbrio de longo prazo, cabe mencionar àquela que diz respeito ao grau de utilização. Adaptando as variáveis,
$$
u = \frac{g_z - \gamma}{\gamma_u}
$$
que indica que o grau de utilização não converge ao nível normal e pode se manter persistentemente em um patamar elevado a depender dos parâmetros. Além disso, se os gastos autônomos crescerem a uma mesma taxa que o valor do \textit{animal spirits}, o grau de utilização será nulo. Em outras palavras, como a estabilidade independe de ($\gamma$), não existem restrições para esse parâmetro de modo que possa zerar o grau de utilização.
}. 
Dito isso, cabe realçar os resultados que tocam os objetivos desta seção: (i) Mudanças na distribuição de renda não afetam a taxa de crescimento de longo prazo; (ii) o mesmo vale para mudanças na propensão marginal a poupar e a consumir a partir da riqueza; (iii) Aumento na taxa de crescimento dos gasto do governo ($g_z$) afetam positivamente o grau de utilização\footnote{Vale mencionar que uma das peculiaridades deste modelo é a endogeinização da distribuição funcional da renda pelo grau de utilização. No entanto, tal resultado pode decorrer da diferenciação feita por \textcite{hein_autonomous_2018} entre renda decorrente da produção e renda financeira.}; (iv) o mesmo vale para a taxa de crescimento de longo prazo. Dentre os resultados restritos a esse modelo, destaca-se:  (a) Mudanças nos \textit{animal spirits} afetam negativamente o grau de utilização mas não possuem efeitos na taxa de crescimento; (b) redução do \textit{déficit} e da dívida do governo em decorrência de: (b.i) aumento nos \textit{animal spirits}; (b.ii) diminuição da propensão marginal e aumento da propensão a consumir a partir da riqueza. 

Por se tratar de um modelo do tipo Stock-Flow Consistent (adiante, SFC), a dívida do governo é tratada como riqueza financeira privada. Nesses termos, um aumento na taxa de juros que incide sobre os títulos do governo reduz os gastos mas aumenta a dívida\footnote{\textcite{hein_autonomous_2018} afirma que este modelo permite incluir o que denomina de paradoxo da dívida, ou seja, redução da dívida pública como resultado de um aumento dos gastos.}. Por fim, o autor concluir, tal como \textcite{arestis_effectiveness_2012}, que uma política fiscal ativa pode atuar para aquecer a economia sem implicar em insustentabilidade da dívida pública.

%MODELO BROCHIER (2018): RIQUEZA FINANCEIRA ACUMULADA
Outro modelo SFC que merece ser pontuado é o de \textcite{brochier_supermultiplier_2018}


%MODELO DUTT: INOVAÇÃO
Apesar dessa variabilidade de modelos, \textcite{dutt_observations_2018} afirma que são incapazes de fazer com que o investimento (criador de capacidade produtiva) como determinante do crescimento no longo prazo tal como Kalecki. Para tanto, inclui-se um componente de crescimento que expressa o progresso tecnológico determinado exogenamente ($\gamma$). No entanto, tal formulação não faz com que o grau de utilização convirja ao normal e que a taxa de crescimento seja determinada pelos gastos autônomos uma vez que essa nova variável afeta a capacidade produtiva no longo prazo. Para garantir as propriedade do supermultiplicador, o progresso técnico é endogeneizado pelos gastos com P\&D ($g_R$) de forma que:
$$
g_I + g_R = g_S
$$
Neste modelo, uma vez cessados os efeitos do progresso tecnológico ($\dot \gamma = 0$): (i); (ii); (iii) grau de utilização converge ao normal; (iv) . Portanto, partindo desta formulação, o progresso tecnológico pode determinar o ritmo de crescimento no longo prazo sem afetar o investimento.

%LAVOIE E FIEBGER: INVESTIMENTO RESIDENCIAL

%CRÍTICAS DE NIKIFOROS
Antes de prosseguir para as conclusões deste capítulo, serão destacadas algumas das críticas de \textcite{nikiforos_comments_2018} aos supermultiplicador. 

%FAGUNDES E FREITAS