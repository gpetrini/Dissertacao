\chapter{Conclusão}

%%%%%%%%%%%%% Introdução %%%%%%%%%%%%%
Ao longo desta dissertação, buscou-se contribuir para a literatura de crescimento liderado pela demanda.
Em particular, procurou-se evidenciar as relações entre investimento residencial, bolha de ativos (\textit{i.e.} imóveis) e macrodinâmica inspirando-se no caso norte-americano pós-desregulamentação financeira.
Por se tratar de uma bolha de ativos dinamizadora e não de uma tendência, esta pesquisa não é centrada somente nas posições plenamente ajustadas mas sim nos ciclos econômicos (médio prazo).
Para tanto, estruturou-se esta investigação em três frentes complementares: (i) teórica; (ii) empírica e; (iii) simulações.



%%%%%%%%%%%%% Capítulo teórico: revisão convencional %%%%%%%%%%
Para atender estes objetivos, iniciou-se, no capítulo \ref{CapTeorico}, uma revisão dos modelos heterodoxos de crescimento a partir da problemática deixada por Harrod.
Sendo assim, avaliou-se criticamente os modelos de Cambridge, Oxford e supermultiplicador sraffiano (SSM).
O princípio da demanda efetiva e alguns fatos estilizados foram utilizados como critérios de seleção do modelo a ser adotado.
Desta primeira discussão, selecionou-se o SSM por incluir os gastos autônomos não criadores de capacidade produtiva em sua formulação original.
Em seguida, foram abordadas questões envolvendo autonomia destes gastos de modo a esclarecer alguns pontos.

%%%%%%%%%%%%% Capítulo teórico: revisão fronteira %%%%%%%%%%
Por mais que o SSM seja um candidato apto a atender os objetivos desta investigação, não é o único caminho existente dentro da literatura heterodoxa.
Desse modo, mapeou-se as alternativas recentes que procuram incluir os gastos ditos improdutivos.
A via kaleckiana, no entanto, foi descartada pela não reprodução da relação positiva entre taxa de investimento e crescimento no médio prazo e subsequente incompatibilidade com os objetivos desta pesquisa.
Além disso, dessa revisão da literatura, identificou-se que poucos modelos incluem o investimento das famílias enquanto gasto autônomo, selecionando a especificação por meio  da taxa própria de juros dos imóveis uma vez que permite incluir inflação de ativos no  SSM.
A despeito dos modelos teóricos terem explorado pouco
esse elemento da demanda, há uma crescente literatura empírica destacando seu papel para a dinâmica macroeconômica e este é o tema do capítulo seguinte.


%%%%%%%%%%%%% Capítulo empírico: modelos e fatos estilizados %%%%%%%%%%
Selecionado o modelo teórico a ser seguido, coube ao capítulo \ref{CapFatos} avançar em direção da discussão empírica.
Sendo assim, fez-se um breve mapeamento dos modelos macroeconométricos que incorporam gastos autônomos, destacando a ausência de trabalhos que analisam o investimento residencial em específico.
Em seguida, foram apresentados alguns fatos estilizados a respeito da economia norte-americana de modo a evidenciar a relevância do investimento residencial para o ciclo econômico.
Adicionalmente, argumentou-se que a influência deste gasto não está limitada a momentos de mudanças na distribuição da renda a favor dos lucros e de ampliação do crédito, sendo um movimento mais geral que foi acompanhando de uma popularização dos imóveis seja entre os percentis de riqueza seja no portfólio das famílias.
Em outras palavras,
concluiu-se que a importância deste componente da demanda agregada não se restringe a crise recente, antecipando as recessões  e liderando as recuperações desde (ao menos) o pós-guerra.


%%%%%%%%%%%%% Capítulo empírico: vec %%%%%%%%%%
Compreendida a relevância do investimento residencial para a dinâmica macroeconômica norte-americana, seguiu-se para a discussão econométrica a respeito do tema.
Para tanto, foi feita uma breve revisão empírica e destacou-se a existência de trabalhos que pontuam a importância deste gasto para além dos Estados Unidos.
Dentre os modelos que tratam do caso norte-americano, evidenciou-se a ausência de trabalhos macroeconométricos que o incorporam na agenda macroeconomia da demanda efetiva.
Sendo assim --- e de modo a dar suporte a discussão teórica --- testou-se se a taxa própria de juros dos imóveis explica a taxa de crescimento do investimento residencial.
Para isso, estimou-se um VECM  e concluiu-se que --- apesar de incluir várias defasagens --- é um modelo bastante parcimonioso em termos de variáveis cujos resultados estão respaldados pela literatura:
(i) taxa própria além de cointegrada com a taxa de crescimento dos imóveis a afeta negativamente; (ii) demanda por imóveis por motivos não-especulativos é estatisticamente significante e; (iii) efeito da taxa de investimento residencial sobre a taxa própria não é estatisticamente significante a 5\%.

%%%%%%%%%%%%% Capítulo sfc: resultados gerais %%%%%%%%%%

Em seguida, no capítulo \ref{CapModelo}, construiu-se um modelo de simulação seguindo a estrutura contábil da metodologia \textit{Stock-Flow Consistent} com supermultiplicador sraffiano inspirado no caso norte-americano reportado anteriormente.
Para tanto, priorizou-se a parcimônia de modo incluir os setores institucionais estritamente necessários para representar a dinâmica do investimento residencial.
Sendo assim, trata-se de uma economia fechada e sem governo com duas classes sociais (trabalhadores e capitalistas) e com dois gastos autônomos (consumo financiado por crédito e investimento residencial).
Com o modelo em mãos, foram realizados choques baseados nos fatos estilizados apresentados no capítulo anterior: (i) aumento da taxa de crescimento do componente autônomo do investimento residencial, representando ampliação da demanda por imóveis por motivos não-especulativos; (ii) inflação de ativos, incorporados pela taxa própria de juros dos imóveis; (iii) redução da participação dos salários na renda, indicando a piora na distribuição (funcional), sobretudo entre as famílias mais pobres e; (iv) redução das taxas de juros, movimento iniciado na década de 80.


Os resultados estão em linha com a literatura do supermultiplicador sraffiano, ou seja, apenas os choques que alteraram a taxa de crescimento dos gastos autônomos (i, ii e iv) modificaram a taxa de crescimento da economia como um todo no longo prazo enquanto a redução da participação dos salários na renda tem efeito transitório apenas.
Em todos os choques, o grau de utilização convergiu ao normal enquanto o crescimento foi liderado pelos gastos autônomos.
O resultado particular do presente modelo é a redução da participação dos imóveis no estoque de capital total da economia decorrente do aumento da taxa de crescimento dos gastos autônomos enquanto a menor participação dos salários na renda implicou no inverso.
Ambos os casos estão respaldados pela literatura do supermultiplicador sraffiano em que o investimento das firmas segue o princípio de ajuste do estoque de capital.


%%%%%%%%%%%%% Capítulo sfc: resultados imputados %%%%%%%%%%
Com o modelo base em mãos, seguiu-se para a introdução dos dados observados que foram utilizados na estimação do modelo econométrico do capítulo anterior.
Apesar de preliminar, tal estratégia teve como objetivo investigar a dinâmica de médio prazo em que a propensão marginal a investir da economia não está plenamente ajustada assim como o grau de utilização gravita em torno do normal.
Ao imputar os dados nas variáveis correspondentes ao investimento residencial --- determinante último da dinâmica do modelo --- foi-se possível reproduzir tanto um comportamento cíclico quanto alguns dos fatos estilizados, dentre eles: (i) maior comprometimento da renda das famílias com pagamento de juros; (ii) gravitação do grau de utilização em torno do desejado e subsequente ajustamento da propensão marginal a investir e; (iii) gastos autônomos liderando a taxa de crescimento da economia.

%%%%%%%%%%%%% Encerammento %%%%%%%%%%

Apesar dos avanços reportados anteriormente, existem outras direções que podem ser melhor exploradas no futuro: (i) análise das relações entre mercado imobiliário e mercado de crédito e subsequente aumento da participação das hipotecas no balanço patrimonial dos bancos; (ii) reprodução da maior volatilidade do investimento residencial em relação aos demais componentes da demanda agregada; (iii) investigação de outros determinantes do investimento residencial que vão além da taxa própria de juros dos imóveis.
%Por fim, cabe uma última reflexão a respeito do tema
%O embricamento dinâmico entre investimento residencial e ciclo econômico norte-americano aqui avaliado decorre de uma configuração institucional que lhe é particular. 
%e que permitiu a ampliação do financiamento do consumo por meio do aumento do colateral associado a bolha dos imóveis.
Além disso, investigações futuras podem expandir a agenda de pesquisa aqui iniciada nas seguintes frentes:
	(i) aprofundar o entendimento da permissividade institucional e suas respectivas consequências sobre a dinâmica do investimento residencial; 
	(ii) avaliar a generalidade da importância do investimento residencial sobre outros países sob a agenda da macroeconomia da demanda efetiva bem como examinar a aplicabilidade da taxa própria de juros dos imóveis para além do caso norte-americano e;
	(iii) compreender porque alguns países não apresentam tal relação entre investimento residencial e dinâmica macroeconômica e quais os mecanismos que emperram tais encadeamentos.
Este, portanto, é apenas o primeiro passo numa agenda de pesquisa mais ampla sobre o papel do investimento residencial no ciclo e no crescimento econômico de modo que os modelos aqui apresentados podem (e devem) ser estendidos e aprimorados.