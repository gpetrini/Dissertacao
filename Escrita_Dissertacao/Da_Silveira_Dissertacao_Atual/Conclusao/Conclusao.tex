\chapter{Conclusão}

%%%%%%%%%%%%% Introdução %%%%%%%%%%%%%

%%%%%%%%%%%%% Capítulo teórico: revisão convencional %%%%%%%%%%

%%%%%%%%%%%%% Capítulo teórico: revisão fronteira %%%%%%%%%%

%%%%%%%%%%%%% Capítulo empírico: modelos e fatos estilizados %%%%%%%%%%

%%%%%%%%%%%%% Capítulo empírico: vec %%%%%%%%%%

%%%%%%%%%%%%% Capítulo sfc: resultados gerais %%%%%%%%%%

Em seguida, no capítulo \ref{CapModelo}, construiu-se um modelo de simulação seguindo a estrutura contábil da metodologia \textit{Stock-Flow Consistent} com supermultiplicador sraffiano inspirado no caso norte-americano reportado anteriormente.
Para tanto, priorizou-se a parcimônia de modo incluir os setores institucionais estritamente necessários para representar a dinâmica do investimento residencial.
Sendo assim, trata-se de uma economia fechada e sem governo com duas classes sociais (trabalhadores e capitalistas) com dois gastos autônomos (consumo financiado por crédito e investimento residencial).
Com o modelo em mãos, foram realizados choques com baseados nos fatos estilizados apresentados no capítulo anterior: (i) aumento da taxa de crescimento do componente autônomo do investimento residencial, representando ampliação da demanda por imóveis por motivos não-especulativos; (ii) inflação de ativos, incorporados pela taxa própria de juros dos imóveis; (iii) redução da participação dos salários na renda, indicando a piora na distribuição funcional da renda, sobretudo entre as famílias mais pobres e; (iv) redução das taxas de juros, movimento iniciado na década de 80.


Os resultados estão em linha com a literatura do supermultiplicador sraffiano, ou seja, apenas os choques que alteraram a taxa de crescimento dos gastos autônomos (i, ii e iv) modificaram a taxa de crescimento da economia como um todo no longo prazo enquanto a redução da participação dos salários na renda teve efeito transitório apenas.
Em todos os choques, o grau de utilização convergiu ao normal enquanto o crescimento foi liderado pelos gastos autônomos.
O resultado particular do presente modelo é a redução da participação dos imóveis no estoque de capital total da economia decorrente do aumento da taxa de crescimento dos gastos autônomos enquanto a menor participação dos salários na renda implicou no inverso.
Ambos os casos estão respaldados pela literatura do supermultiplicador sraffiano em que o investimento das firmas segue o princípio de ajuste do estoque de capital.


%%%%%%%%%%%%% Capítulo sfc: resultados imputados %%%%%%%%%%

Com o modelo base em mãos, seguiu-se para a introdução dos dados observados que foram utilizados na estimação do modelo econométrico do capítulo anterior.
Apesar de preliminar, tal estratégia teve como objetivo investigar a dinâmica de médio prazo em que a propensão marginal a investir da economia não está plenamente ajustada assim como o grau de utilização gravita em torno do normal.
Sendo assim, ao imputar os dados nas variáveis correspondentes ao investimento residencial --- determinante último da dinâmica do modelo --- foi-se possível reproduzir tanto um comportamento cíclico quanto alguns dos fatos estilizados, dentre eles: (i) maior comprometimento da renda das famílias com pagamento de juros; (ii) gravitação do grau de utilização em torno do desejado e subsequente ajustamento da propensão marginal a investir e; (iii) gastos autônomos liderando a taxa de crescimento da economia.

%%%%%%%%%%%%% Encerammento %%%%%%%%%%

Esta pesquisa, portanto, é um primeiro passo em direção a uma agenda de pesquisa mais ampla pertencente a família de modelos de crescimento liderados pela demanda.
Apesar dos avanços reportados anteriormente, existem outras direções que podem ser melhor exploradas no futuro: (i) análise das relações entre mercado imobiliário e mercado de crédito e subsequente aumento da participação das hipotecas no balanço patrimonial dos bancos; (ii) reprodução da maior volatilidade do investimento residencial em relação aos demais componentes da demanda agregada; (iii) investigação de outros determinantes do investimento residencial que vão além da taxa própria de juros dos imóveis.

Por fim, cabe uma última reflexão a respeito do tema.
O embricamento dinâmico entre investimento residencial e ciclo econômico norte-americano aqui avaliado decorre de uma configuração institucional que lhe é particular e que permitiu a ampliação do financiamento do consumo por meio do aumento do colateral associado a bolha dos imóveis.
Investigações futuras podem ir nas seguintes direções:
	(i) aprofundar o entendimento da permissividade institucional e suas respectivas consequências sobre a dinâmica do investimento residencial; 
	(ii) avaliar a generalidade da importância do investimento residencial sobre outros países sob a agenda do princípio da demanda efetiva;
	(iii) compreender o porquê alguns países não apresentam tal relação entre investimento residencial e dinâmica macroeconômica e quais os mecanismos que emperram tais encadeamentos e;
	(iv) examinar a aplicabilidade da taxa própria de juros dos imóveis para além do caso norte-americano.