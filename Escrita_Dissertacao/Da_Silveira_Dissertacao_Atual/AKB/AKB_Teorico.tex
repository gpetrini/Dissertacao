\section{Modelos de crescimento heterodoxos}

\subsection{Análise entre os modelos}


\subsection{Modelos com gastos autônomos}

A presente seção tem por objetivo destacar como os modelos que incorporaram os gastos autônomos não criadores de capacidade produtiva ao setor privado ($Z$). Seguindo a exposição de \textcite{cesaratto_technical_2003} e a categorização de \textcite{serrano_sraffian_1995}, os gastos ditos autônomos são aqueles que independem do nível de renda (efetivo ou esperado) decorrente das decisões de produzir e podem ou não criam capacidade produtiva ao setor privado. Seguindo a literatura do supermultiplicador sraffiano, os componentes efetivamente autônomos da demanda agregada são: (i) Gastos do governo; (ii) Consumo financiado por crédito ou riqueza financeira acumulada; (iii) Investimento residencial; (iv) Gastos com P\&D e; (v) Exportações. Dito isso, a forma com que a literatura tem introduzido esses gastos é discutida a seguir\footnote{As variáveis serão adaptadas de modo que a $\gamma$ é o componente autônomo do investimento,  $z$ é a participação dos gastos autônomos ($Z$) na renda/Capital que crescem a taxa $g_Z$.}.

%MODELO ALLAIN: GASTOS DO GOVERNO
No modelo de \textcite{allain_macroeconomic_2014}, os gastos do governo são autônomos e não criam capacidade ($Z$) e são financiados por impostos que se ajustam endogenamente para manter o saldo primário equilibrado\footnote{
%MODELO HEIN (2018): GASTOS DO GOVERNO E SUSTENTABILIDADE DA DÍVIDA
\textcite{hein_autonomous_2018} argumenta que o modelo de \textcite{allain_macroeconomic_2014} não inclui uma discussão sobre a dinâmica do \textit{déficit} e da dívida pública no longo prazo. Dito isso, desenvolve um modelo SFC em que a dívida do governo é tratada como riqueza financeira privada.
}. No longo prazo\footnote{Dentre os resultados particulares do modelo, \textcite{allain_macroeconomic_2014} pontua os efeitos contra-cíclicos do gasto público sobre o nível de atividade e seu papel enquanto estabilizador automático do crescimento.}: (i) Mudanças na distribuição de renda e na propensão marginal a poupar geram alterações no nível, mas não na taxa de crescimento, eliminado o paradoxo dos custos e da parcimônia; (ii) o grau de utilização converge ao normal no longo prazo e (iii) aumento da taxa de crescimento dos gastos autônomos tem impactos positivos sobre a taxa de acumulação e da taxa de lucro.


%MODELO BROCHIER (2018): RIQUEZA FINANCEIRA ACUMULADA
Um modelo SFC com supermultiplicador que merece ser pontuado é o de \textcite{brochier_supermultiplier_2018}. Os gastos autônomos foram endogeneizados e são determinados pelo consumo a partir da riqueza financeira acumulada em uma economia com governo. Como consequência da inclusão das relações financeiras no supermultiplicador, argumentam, alguns dos resultados apresentados anteriormente se alteram: (i) alterações na distribuição de renda e na propensão marginal a consumir a partir da renda disponível e da riqueza (componente correspondente ao $Z$) impactam a taxa de acumulação no longo prazo; (ii) grau de utilização converge ao normal. Desse modo, este modelo apresenta uma exceção importante em que os paradoxos dos custos e da parcimônia são mantidos inclusive com o grau de utilização convergindo ao desejado, configurando uma exceção ao que foi exposto até então.

Prosseguindo para o consumo financiado por crédito, destaca-se os trabalhos de \textcite{dutt_maturity_2006}, \textcite{palley_inside_2010} e \textcite{hein_finance-dominated_2012} em que o consumo dos capitalistas são os gastos autônomos. Por se tratar de um modelo Kaleckiano, o investimento não é totalmente induzido e, portanto, a estabilidade só é garantida se o estoque de dívida dos capitalistas crescer a mesma taxa que a acumulação. Diante desta limitação, \textcite{pariboni_household_2016} argumenta que os gastos autônomos desempenham um papel passivo e sugere uma alternativa \textit{à la} supermultiplicador sraffiano. Com estas modificações, a causalidade anterior é revertida de modo que a taxa de acumulação converge gradualmente a taxa dos gastos autônomos. Adicionalmente, afirma que o grau de endividamento das famílias não cresce assintoticamente uma vez que os efeitos sobre o nível de atividade fazem com que a renda disponível aumente. 
%TESE MANDARINO: CONSUMO FINANCIADO POR CRÉDITO
Outro modelo em linha com \textcite{brochier_supermultiplier_2018} é o de \textcite{mandarino_financing_2018} em que o consumo dos trabalhadores é financiado por crédito\footnote{% LAVOIE (2016): CONSUMO AUTÔNOMO
Vale a menção ao modelo de \textcite{lavoie_convergence_2016} que obtém resultados semelhantes aos de \textcite{allain_macroeconomic_2014} para o caso do consumo dos capitalistas como gasto autônomo. Outro modelo com consumo a ser destacado é o de \textcite{nah_role_2019} %NAH E LAVOIE (2019): INFLAÇÃO E DISTRIBUIÇÃO ENDÓGENA
que inclui inflação por conflito distributivo. Por mais que tal modelo apresente gastos autônomos como os demais nesta seção, a endogeinização da distribuição de renda elimina uma das hipóteses compartilhadas entre os modelos analisados e, portanto, compromete a comparação e deve ser discutido a parte.} como em \textcite{fagundes_dinamica_2017} mas centrado nas condições de estabilidade do endividamento dos trabalhadores no longo prazo. No que diz respeito às implicações para o longo prazo, obtém resultados semelhantes aos de \textcite{allain_tackling_2015}.

% MODELO NAH E LAVOIE (2019): CONSUMO AUTÔNOMO E INFLAÇÃO POR CONFLITO DISTRIBUTIVO E ENDOGEINIZAÇÃO DA DISTRIBUIÇÃO

%MODELO NAH AND LAVOIE: EXPORTAÇÃO
Já no modelo de \textcite{nah_long-run_2017}, e semelhante ao de \textcite{dejuan_hidden_2017}, as exportações desempenham o papel dos gastos autônomos. Os resultados de longo prazo são iguais aos apresentados anteriormente e por conta disso não serão repetidos. No entanto, este modelo se destaca pelo regime de acumulação pode ser caracterizado como \textit{wage-} ou \textit{profit-led} a depender da sensibilidade da taxa de câmbio real a mudanças na distribuição de renda. 

%MODELO DUTT: INOVAÇÃO
Apesar dessa variabilidade de modelos, \textcite{dutt_observations_2018} afirma que são incapazes de fazer com que o investimento (criador de capacidade produtiva) seja determinante do crescimento no longo prazo. Para tanto, inclui um componente de crescimento que expressa o progresso tecnológico determinado exogenamente ($\gamma$). No entanto, tal formulação não faz com que o grau de utilização convirja ao normal e que a taxa de crescimento seja determinada pelos gastos autônomos uma vez que essa nova variável afeta a capacidade produtiva no longo prazo. Para garantir as propriedades do supermultiplicador, o progresso técnico é endogeneizado pelos gastos com P\&D ($g_R$) e os resultados de \textcite{allain_macroeconomic_2014} são reestabelecidos.

Da discussão anterior, verifica-se que a literatura sobre investimento residencial é bastante escassa nos modelos com gastos autônomos. Diante desta motivação empírica reportada na seção anterior, é desenvolvido um modelo SFC com supermultiplicador sraffiano em que tal investimento são os gastos autônomos e são financiados por hipotecas.