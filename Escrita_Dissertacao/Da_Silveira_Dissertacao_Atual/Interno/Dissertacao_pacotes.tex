%%%%%%%%%%%%%%%%%%%%%%%%%%%%%% Pacotes: básicos %%%%%%%%%%%%%%%%%%%%%%%%%%%%%% 
\usepackage{cmap}
\usepackage[T1]{fontenc}
\usepackage[utf8]{inputenc}
\usepackage{indentfirst}

\usepackage{collectbox}
\usepackage{appendix}
\usepackage{tocloft}
\usepackage{titlesec}


\makeatletter
\newcommand{\mybox}{%
    \collectbox{%
        \setlength{\fboxsep}{1pt}%
        \fbox{\BOXCONTENT}%
    }%
}
\makeatother


\usepackage[top=3cm,bottom=3cm,right=2cm,left=2cm]{geometry}
% Para corrigir 'babel/polyglossia' detected but 'csquotes' missing.
\usepackage{csquotes}
% Para corrigir:  '<namepart>inits' conflicts with 'uniquename=full'. - uniquename=init
\usepackage[backend=biber,%
	style = abnt,%
	noslsn, %
	isbn = false,
	url = false,
	extrayear, %
	uniquename=init,% 
	giveninits, %
	justify, %
	sccite,% 
	scbib, %
	sorting=nyt,
%	mergedate=compact,
	repeattitles, %
	maxcitenames=3]{biblatex}
\addbibresource{Bibliografia_Dissertacao/Bibliografia_Dissertacao.bib}
\addbibresource{Fatos_Estilizados/Biblio.bib}


%%%%%%%%%%%%%%%%%%%%% Inicio Origdate
\DeclareLabeldate{%
	\field{date}
	\field{year}
	\field{eventdate}
	\field{urldate}
	\literal{nodate}
}

\renewbibmacro*{date+extradate}{%
	\iffieldundef{labelyear}
	{}
	{\printtext[parens]{%
			\iffieldundef{origyear}
			{}
			{\printtext[brackets]{\printorigdate}%
				\setunit{\addspace}}%
			\iflabeldateisdate
			{\printdateextra}
			{\printlabeldateextra}}}}

\renewbibmacro*{cite:labeldate+extradate}{%
	\iffieldundef{labelyear}
	{}
	{\printtext[bibhyperref]{%
			\iffieldundef{origyear}
			{}
			{\printtext[brackets]{\printorigdate}%
				\setunit{\addspace}}%
			\printlabeldateextra}}}

%%%%%%%%%%%%%%%%%%%%% Fim Origdate

%%%%%%%%%%%%%%%%%% Usar cites
\makeatletter
\renewbibmacro*{cite:init}{%
  \ifnumless{\value{multicitecount}}{2}%
    {\global\boolfalse{cbx:parens}%
     \global\undef\cbx@lasthash%
     \global\undef\cbx@lastyear}%
    {}}%
\makeatother


%%%%%%%%%%%%%%%%%%%%%%%%%%%%%%% Pacotes: layoyt %%%%%%%%%%%%%%%%%%%%%%%%%%%%%%%
\usepackage{etoolbox}  % É preciso para mudar o layout do frontmatter


%%%%%%%%%%%%%%%%%%%%%%%%%%%%%%% Pacotes: links %%%%%%%%%%%%%%%%%%%%%%%%%%%%%%%
\usepackage{url}
%\usepackage{breakurl} %Pacote incompatível
\usepackage{hyperref}


%%%%%%%%%%%%%%%%%%%%%%%%%%%%%%%% Pacotes: ams %%%%%%%%%%%%%%%%%%%%%%%%%%%%%%%% 
\usepackage{amsmath}
\usepackage{amsfonts}
\usepackage{amssymb}
\usepackage{amsthm}
\usepackage{breqn}


%%%%%%%%%%%%%%%%%%%%%%%%%%%%%% Pacotes: tabelas %%%%%%%%%%%%%%%%%%%%%%%%%%%%%%
\usepackage{multicol}
\usepackage{multirow}
\usepackage{array}
\usepackage{booktabs}


%%%%%%%%%%%%%%%%%%%%%%%%%%%%%% Pacotes: cores %%%%%%%%%%%%%%%%%%%%%%%%%%%%%%%% 
\usepackage[usenames,dvipsnames,svgnames,table]{xcolor}


%%%%%%%%%%%%%%%%%%%%%%%%%%%%%% Pacotes: figuras %%%%%%%%%%%%%%%%%%%%%%%%%%%%%% 
\usepackage{pdfpages}
\usepackage{graphicx}
\usepackage{tikz}
\usetikzlibrary{calc,trees,positioning,arrows,chains,shapes.geometric,%
    decorations.pathreplacing,decorations.pathmorphing,shapes,%
    matrix,shapes.symbols,through}

\usetikzlibrary{fit}
\usepackage{wrapfig}


%%%%%%%%%%%%%%%%%%%%%%%%%%%%% Pacotes: algoritmos %%%%%%%%%%%%%%%%%%%%%%%%%%%%% 
\usepackage{algorithmic}
\usepackage[chapter]{algorithm}
\floatname{algorithm}{Algoritmo}
\renewcommand{\listalgorithmname}{Lista de Algoritmos}


%%%%%%%%%%%%%%%%%%%%%%%%%%%%%% Pacotes: códigos %%%%%%%%%%%%%%%%%%%%%%%%%%%%%% 
\usepackage{textcomp}
\usepackage{listings}
\renewcommand\lstlistingname{Código}
\renewcommand\lstlistlistingname{Lista de Códigos}


%%%%%%%%%%%%%%%%%%%%%%%%%%%%%%% Pacotes: index %%%%%%%%%%%%%%%%%%%%%%%%%%%%%%% 
\usepackage{makeidx}
\makeindex


%%%%%%%%%%%%%%%%%%%%%%%%%%%%%%% Pacotes: fontes %%%%%%%%%%%%%%%%%%%%%%%%%%%%%% 
\usepackage{lmodern} \normalfont
\DeclareFontShape{T1}{lmr}{bx}{sc} { <-> ssub * cmr/bx/sc }{}
\usepackage{mathrsfs}


% TODO Inserir pacotes adicionais aqui.
%==================================================================================================
%                                   OUTROS PACOTES
%====================================================================================
%\usepackage[english, brazil]{babel}
\usepackage{lastpage}			      % Usado pela Ficha catalogr\'{a}fica
\usepackage{color}				      % Controle das cores
\usepackage{epstopdf}           % Pacote que converte as figuras em eps para pdf
\usepackage{lipsum}             % Pacote que gera texto dummy
\usepackage{blindtext}          % Pacote que gera texto dummy
\usepackage{caption}
\usepackage{subcaption}
\usepackage{multirow}
\usepackage{longtable}
\usepackage{lscape}
\usepackage{array}
\usepackage{tikz}
\usepackage{venndiagram}
\usepackage[multiple, bottom]{footmisc} % Para nota de rodapé no fim da página
\usepackage{chngcntr}
\counterwithin{equation}{section}


%customiza\c{c}\~{a}o do negrito em ambientes matem\'{a}ticos
\newcommand{\mb}[1]{\mathbf{#1}}
\newcommand{\abs}[1]{\left|#1\right|}
\newcommand{\norm}[1]{\left\|#1\right\|}
\newcommand{\partialorder}{\cdot \geq}
\newcommand{\h}{\mathrm{h}}
\newcommand{\x}{\mathrm{x}}
\newcommand{\z}{\mathrm{z}}
\newcommand{\entropia}[1]{H{(#1)}}
%customiza\c{c}\~{a}o de teoremas
\newtheorem{mydef}{Defini\c{c}\~{a}o}[chapter]
\newtheorem{lemm}{Lema}[chapter]
\newtheorem{theorem}{Teorema}[chapter]
%Fontes em times
\setlength{\parskip}{\onelineskip}
\usepackage{mathptmx}
\renewcommand{\ABNTEXchapterfont}{\rmfamily\bfseries}
%\usepackage[nottoc]{tocbibind} % Evita auto-referencia do sumário %Solução: \tableofcontents*




%==================================================================================================
%                                   CUSTOMIZAÇÃO UNICAMP
%====================================================================================
\usepackage{unicamp}
% ---
% Espa\c{c}amentos entre linhas e par\'{a}grafos
% ---
\linespread{1.3}

% O tamanho do par\'{a}grafo \'{e} dado por:
\setlength{\parindent}{2.0cm}

% Controle do espa\c{c}amento entre um par\'{a}grafo e outro:
\setlength{\parskip}{0.2cm}  % tente tamb\'{e}m \onelineskip


