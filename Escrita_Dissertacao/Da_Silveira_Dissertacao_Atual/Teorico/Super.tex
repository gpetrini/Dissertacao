\subsubsection{Supermultiplicador Sraffiano}\label{SecSuper}

O Supermultiplicador Sraffiano desenvolvido por \textcite{serrano_sraffian_1995} --- e paralelamente por \textcite{bortis_institutions_1996} --- prossegue a agenda de pesquisa iniciada por \textcite{garegnani_problem_2015} em que o PDE fosse validado no longo prazo. 
Para tanto, lança-se mão das seguintes hipóteses adicionais: 
	(i) livre --- mas não instantânea --- concorrência entre capitais (existentes e novos concorrentes potenciais);
	(ii) o produto se ajusta relativamente rápido à demanda efetiva;
	(iii) os preços de mercado são iguais aos preços normais que auferem uma taxa de lucro também normal (quando utilizada a técnica produtiva dominante operando no grau de utilização normal);
	(iv) distribuição funcional da renda é exógena e determinada por fatores históricos e institucionais;
	(v) existem gastos autônomos não criadores de capacidade produtiva ao setor privado ($Z>0$) de modo que uma parcela da demanda agregada não está relacionada às decisões de produção das firmas e;
	(vi) o investimento das firmas é determinado pelo princípio do ajuste do estoque de capital.

O item (iv) permite expressar o \textit{wage-share} nos seguintes termos\footnote{
	A teoria sraffiana da distribuição aqui adotada é a de \textcite{pivetti_essay_1992} em que a taxa de lucro é determinada pela taxa básica de juros acrescida de um componente associado ao \textit{risk and trouble}.
	Para uma discussão sobre as diferentes determinações da taxa de lucro, ver \textcite{serrano_teoria_1988}.
	Para uma síntese das diferentes vertentes dentro da abordagem sraffiana, ver \textcite{aspromourgos_sraffian_2004}.
}:
$$
\omega = \overline \omega
$$
Já os pontos (i) e (vi) acima implicam a indução do investimento (das \textbf{firmas}\footnote{
	Para manter os modelos comparáveis, adota-se a seguinte notação:
	$$
	I = I_f
	$$
	$$
	Z = \ldots + I_h + \ldots
	$$
	em que $I_f$ é o investimento das firmas enquanto $I_h$ é o investimento residencial (um dos componentes dos gastos autônomos).
}) 
$$
I = h\cdot Y
$$
de modo que a capacidade produtiva se ajusta a demanda através de alterações na propensão marginal a investir ($h$) que se ajusta a desvios do grau de utilização em relação ao normal de forma lenta e gradual como indicado pelo parâmetro de velocidade ($\gamma_u$) positivo e suficientemente pequeno \cite[p.~271]{freitas_growth_2015}:
$$
\frac{\Delta h}{h_{-1}} = \gamma_u (u - u_N)
$$

Apresentado o investimento das firmas, resta discutir os demais componentes da demanda agregada em nível para então explicitá-los em taxas de crescimento.
Considerando que a parcela induzida do consumo é determinada pela participação dos salários  na renda\footnote{Neste caso que existem gastos autônomos não criadores de capacidade, o consumo pode não ser totalmente induzido. Além disso, vale a menção de que o componente autônomo não se restringe ao consumo e pode ser estendido ao investimento residencial cujas implicações são analisadas no capitulo \ref{CapModelo}.}, o produto determinado pela demanda torna-se:
\begin{equation}
\label{PIBSuper}
Y = \omega\cdot Y + h\cdot Y + Z
\end{equation}
o que implica:
\begin{equation}
\label{Supermultiplicador}
Y = \left(\frac{1}{1 - \omega - h}\right)Z
\end{equation}
cujo termo destacado em parênteses é o supermultiplicador sraffiano. Tal como no multiplicador convencional, o produto é determinado pelos gastos autônomos.
A principal diferença, portanto, consiste
na indução do investimento das firmas de modo que são os gastos autônomos e não criadores de capacidade produtiva que determinam o nível do produto da economia.


Como explicitado anteriormente, a existência deste tipo de gasto faz com que propensão marginal e média a poupar sejam distintas. A relevância desta diferença é que a propensão média passa a depender também do nível dos gastos autônomos, preservando a determinação da poupança pelo investimento:


\begin{equation}
\label{frac_h}
\frac{S}{Y} = \overline{s} - \frac{Z}{Y} = f\cdot s \equiv \frac{I}{Y} = h
\end{equation}
Dito isso, cabe pontuar que  existem duas condições necessárias para que o modelo seja estável: (i) propensão marginal a gastar (consumir e investir) é menor que a unidade\footnote{
	Caso esta condição seja violada, obtém-se um modelo que valida a lei de Say uma vez que todo gasto é induzido pela produção \cite[p.~ 75]{serrano_sraffian_1995}.
	Vale mencionar que \textcite{serrano_trouble_2017} argumentam a estabilidade do modelo de Harrod requer a validade da Lei de Say. Uma vez que Harrod segue o PDE, seu modelo incorre na instabilidade fundamental. Partindo da Eq fundamental (\ref{Fundamental}), é possível indicar este raciocínio:
	
	$$
	g = \frac{s}{v} \Leftrightarrow g\cdot v = s \Rightarrow s - g\cdot v = 0
	$$
	
	$$
	\therefore \omega + g\cdot v = 1
	$$
	em que $g\cdot v$ pode ser entendido como propensão marginal a investir que somada à propensão marginal a consumir ($\omega$), obtém-se a propensão marginal a gastar que, como demonstrado, é idêntica à unidade.
} e; (ii) existem gastos autônomos no longo prazo ($Z > 0$).
Sendo assim, para apresentar o fechamento do modelo, resta expô-lo em termos de crescimento.
Tomando a diferença total da equação \ref{PIBSuper}, tem-se:
$$
g = \omega\cdot g + h\cdot g + \Delta h + \frac{Z}{Y}\cdot \overline g_Z
$$
em que $Z/Y$ é o inverso do supermultiplicador como definido em \ref{Supermultiplicador} e $g_Z$ é a taxa de crescimento dos gastos autônomos determinada exogenamente. Rearranjando, obtém-se:
$$
g_Z = \overline g_Z
$$
\begin{equation}
\label{crescimentosuper}
g = \frac{\Delta h}{1 - \omega - h} + \overline g_Z
\end{equation}
e igualando à taxa de crescimento da Eq. \ref{EqGeral}:

\begin{equation}
\label{SuperEtapa}
f\frac{s\cdot u}{v} \equiv g_K = g \equiv \frac{\Delta h}{1 - \omega - h} + \overline g_Z
\end{equation}


A inclusão dos gastos autônomos não criadores de capacidade produtiva permite que o investimento cresça (temporariamente) a uma taxa diferente do produto.
%TODO Non sectur
No entanto, uma vez esgotado o mecanismo de ajuste do estoque de capital, ou seja, quando o grau de utilização é igual ao desejado ($u = u_N$), não há razões para que a propensão marginal a poupar se altere ($\Delta h = 0$). 
Dito isso e rearranjando a equação \ref{SuperEtapa}, obtém-se o fechamento deste modelo:
$$
f\frac{\overline s\cdot \overline u_N}{\overline v} = \overline g_Z
$$

\begin{equation}
\label{FechamentoSuper}
\overline g_Z\frac{\overline v}{\overline s\cdot \overline u_N} = f = \frac{h^*}{s}
\end{equation}
A partir da equação \ref{FechamentoSuper} é possível explicitar tanto o valor da propensão marginal a investir na posição plenamente ajustada ($u = u_N$)
$$
h^* = \frac{v}{u_N}g_Z
$$
quanto que a propensão média a poupar\footnote{A propensão \textbf{marginal} poupar, determinada exogenamente, é tão somente um limite superior que a propensão média pode assumir. \textcite[p.~51--52]{serrano_o_2000} esclarecem a diferença entre essas duas taxas.} (expressa em termos da fração) é determinada pela taxa de investimento 
$$
\frac{S}{Y} = f\cdot s = h^*
$$
Dito isso, resta destacar a importância dos gastos autônomos para o ajustamento da propensão média a poupar.
%TODO Reescrever
Em linhas gerais, a taxa de poupança varia com a participação dos gastos autônomos na renda.
Uma vez que esta relação é o inverso do supermultiplicador (ver Eq. \ref{PIBSuper}), se alterará na medida em que a propensão marginal a investir se ajustar.
Em resumo, é justamente pela presença dos gastos autônomos que a propensão média pode variar por meio da fração\footnote{
	Vale destacar que a existência de gastos autônomos não é condição suficiente para que a propensão média a poupar se torne uma variável endógena, mas sim a combinação desta hipótese com a do acelerador flexível.
}.

Nesse modelo, portanto, a taxa de acumulação responde aos movimentos da demanda efetiva que são determinadas pelos gastos autônomos não criadores de capacidade produtiva. Além disso, a existência de gastos autônomos que crescem a uma taxa exógena e o investimento produtivo induzido garantem a resolução do problema imposto por Harrod. Isso pode ser verificado ao considerar que a taxa de investimento (regida pela propensão marginal a investir) se adapta aos desvios entre a taxa de crescimento efetiva e à taxa garantida na direção correta\footnote{\textcite{cesaratto_neo-kaleckian_2015} chama atenção para a resolução da singularidade da taxa garantida que se ajusta à efetiva tal como nos modelos analisados anteriormente.} e é nesse sentido que o supermultiplicador sraffiano é fundamentalmente estável\footnote{
	Como pontuado anteriormente, no modelo de \textcite{harrod_essay_1939}, quando a taxa de crescimento corrente excede a taxa garantida ($g > g_w$), há sobreutilização da capacidade uma vez que não existem gastos autônomos. 
	No supermultiplicador, por outro lado, quando a taxa de crescimento corrente excede a taxa garantida, haverá 
	%, dado o caráter dual do investimento e a existência de gastos autônomos, 
	um aumento temporário --- mas não simultâneo --- do grau de utilização que será seguido de um aumento na taxa de crescimento do investimento das firmas de modo que o grau de utilização diminua até a convergência ao grau normal.}:

\begin{citacao}
	The crucial point is that the process of growth led by the expansion of autonomous consumption is thus fundamentally or statically stable because the reaction of \textbf{induced investment} to the initial imbalance between capacity and demand has, at some point during the adjustment disequilibrium process, a \textbf{greater impact} on the rate of growth of productive capacity than on the rate of growth of demand. \cite[p.~19, grifos adicionados]{serrano_trouble_2017}
\end{citacao}

Vale ressaltar que apesar do supermultiplicador ser --- nos termos de \textcite{hicks_capital_1965} e em comparação a \textcite{harrod_essay_1939} --- ``fundamentalmente estável''\footnote{
	Ao apresentarem o modelo do Supermultiplicador Sraffiano em comparação ao modelo de \textcite{harrod_essay_1939}, \textcite{serrano_trouble_2017} argumentam que este é estaticamente instável enquanto o modelo do Supermultiplicador Sraffiano é fundamentalmente estável mas dinamicamente instável à depender da intensidade do ajuste da capacidade produtiva decorrente dos parâmetros do modelo.
	Para isso, retomam a definição de instabilidade de \textcite{hicks_contribution_1972} em que considera um modelo estaticamente estável quando não se afasta do equilíbrio enquanto a estabilidade dinâmica depende da intensidade. Destacam ainda que a estabilidade estática (direção) é condição necessária mas não suficiente para gerar estabilidade dinâmica.
}, pode ser dinamicamente instável a depender dos parâmetros que dizem respeito ao ajuste da capacidade produtiva. Desse modo, não é só a existência de gastos autônomos que garante a possibilidade de um regime de crescimento \textit{estável} liderado pela demanda, mas também o ajuste gradual da propensão marginal a investir. Basta que, fora de equilíbrio, a propensão marginal a gastar seja menor que a unidade para que o sistema seja dinamicamente estável\footnote{Tal como mencionado no corpor do texto e como será visto no capítulo \ref{CapModelo}, é preciso que o parâmetro $\gamma_u$ seja suficientemente pequeno.}. Assim, atendidas essas condições, a capacidade produtiva irá se ajustar à demanda:
$$
u \to u_N : g_I \to g_K \to g \to \overline g_Z
$$
$$
\frac{K}{v}\cdot u_N = Y_{FC}\cdot u_N = Y = \left(\frac{1}{1 - \omega - h}\right)Z
$$
A equação acima evidencia que a capacidade produtiva se ajusta à demanda que, como indicado anteriormente, cresce à taxa tendencial dos gastos autônomos. 
Desse modo, conclui-se que o supermultiplicador sraffiano é um canditado ao modelo a ser selecionado por incluir gastos autônomos não criadores de capacidade --- e este é o caso do investimento residencial --- por construção.
Com isso, completam-se os objetivos desta seção, restando à seguinte avançar em direção a fronteira da heterodoxia para avaliar outras alternativas existentes na literatura.





%Uma das implicações é que, na medida que a economia cresce, a participação dos gastos autônomos na renda diminui enquanto a participação do investimento aumenta, gerando o fluxo necessário para determinar a poupança. 

%Isso pode ser expresso em termos da Equação \ref{EqGeral}. 

%Dito isso e reapresentando a taxa de crescimento do estoque de capital ($g_K$)\footnote{Cabe aqui o esclarecimento que esta forma não é exclusiva do supermultiplicador sraffiano, mas sim comum a todos os modelos apresentados. A razão pela qual optou-se expor a taxa de acumulação nestes termos é meramente convencional dado o destaque a taxa de investimento.}:
%$$
%g_K = \frac{I}{K} = \frac{I}{Y}\frac{Y}{Y_{FC}}\frac{Y_{FC}}{K}
%$$
%$$
%\therefore g_K = \frac{h\cdot u}{v}
%$$


\begin{comment}
%Em linhas gerais, tal modelo avança em direção ao ajuste da capacidade produtiva à demanda e não o inverso.
%Partindo do fato estilizado de que, no longo-prazo, demanda agregada e capacidade produtiva estão equilibradas, argumenta-se que, diferentemente da teoria ortodoxa, é possível que a economia seja estritamente \textit{demand-led}. 
%Para tanto, 

Antes de prosseguir para a discussão sobre a convergência do grau de utilização, é necessário pontuar uma qualificação quanto o papel das expectativas no supermultiplicador. \textcite[p.~87]{serrano_long_1995} reconhece que o grau de utilização pode não convergir ao normal, mas tal resultado decorre de formulações \textbf{persistentemente} erradas sobre a evolução da demanda efetiva. Em resposta à esse argumento, \textcites{allain_macroeconomic_2014}{palley_economics_2018} afirmam que a instabilidade harrodiana é eliminada no Supermultiplicador por hipótese.

Vale notar que a exposição anterior permitiu apresentar a resolução desse problema sem recorrer à suposições sobre a formulação das expectativas. Desse modo, dizer que o Supermultiplicador Sraffiano resolve a instabilidade Harrodiana por meio de hipóteses expectacionais não contempla de forma adequada o papel desempenhado pelo investimento induzido e dão muita ênfase à existência de gastos autônomos. Uma implicação dessa incompreensão é o esforço da literatura Kaleckiana em garantir os resultados do modelo canônico na presença de gastos autônomos sem abandonar a ideia de que o investimento produtivo é autônomo no longo prazo. Tal discussão é endereçada parcialmente na seção seguinte.
\end{comment}