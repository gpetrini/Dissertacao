\section*{Próximos passos}


\textcite{harrod_essay_1939} apresenta um aparato teórico que permite analisar modelos em sua forma dinâmica sem precisar recorrer à defasagens entre as variáveis. Apresenta uma equação que engloba tanto o efeito multiplicador quanto o princípio acelerador cuja implicação é que o equilíbrio dinâmico não é estável. Diante desta problemática, surgiram os modelos de Cambridge, kaleckianos e supermultiplicador sraffiano na tentativa para domar tal instabilidade (ver tabela \ref{crescimento}). Na seção \ref{SecHarrod}, foram apresentadas tais alternativas em que o modelo de Cambridge não se mostrou adequado dadas as incompatibilidades com o comportamento das firmas associada a essa teoria. Desse modo, restaram os modelos kaleckianos e o SSM. 

A seção seguinte (\ref{debate}) irá abordar a controvérsia em torno do grau de utilização e sua convergência ao normal no longo prazo e as implicações para os paradoxos dos custos e da parcimônia. Além disso, serão realçadas algumas críticas aos modelos kaleckianos relacionadas a convergência/endogeinização ao/do grau de utilização normal. Argumenta-se que um modelo que privilegia o PDE no longo prazo deve reportar o fato estilizado reportado acima uma vez que implica no equilíbrio (dinâmico) estável entre demanda e capacidade produtiva. Na próxima seção (\ref{Literatura}), será apresentada a resposta kaleckiana a crítica envolvendo o pricípio do ajuste do estoque de capital em que foram incluídos gastos autônomos não criadores de capacidade. Resta, por fim, checar se a alternativa pelo SSM incorre nos mesmos problemas que os modelos kaleckianos híbridos e isso será feito na seção final (sec. \ref{Concl1}).

	%===========================================
\begin{table}[htb]
	\centering
	\caption{Fechamento das principais teorias de crescimento heterodoxas}
	\label{crescimento}
	\resizebox{\textwidth}{!}{%
		\begin{tabular}{|l|ccccl|}
			\hline 
			\textbf{Modelo} & \begin{tabular}[c]{@{}c@{}} \textbf{Regime de} \\\textbf{crescimento} \end{tabular} &  \begin{tabular}[c]{@{}c@{}} \textbf{Distribuição} \\\textbf{de renda} \end{tabular} & \begin{tabular}[c]{@{}c@{}}\textbf{Grau de utilização} \\ \textbf{da capacidade}\end{tabular} & \begin{tabular}[c]{@{}c@{}} \textbf{Capacidade}  \\ \textbf{produtiva} \end{tabular} & \textbf{Fechamento} \\ \hline
			\textbf{Cambridge} & Ausente  & Endógena & \begin{tabular}[c]{@{}c@{}} Exógena \\ \end{tabular} & Exógena & Distribuição de renda\\
			\textbf{Kaleckiano} & Wage/Profit-led &  \begin{tabular}[c]{@{}c@{}} Exógena \\ (\textit{Mark-up}) \end{tabular} & Endógena   & Exógena & Grau de utilização \\ 
			\begin{tabular}[l]{@{}l@{}}\textbf{Supermultiplicador} \\\textbf{Sraffiano} \end{tabular} & Ausente & \begin{tabular}[c]{@{}c@{}} Exógena \\ (Teoria Monetária\\da distribuição)  \end{tabular} & Tende ao normal & Endógena & \begin{tabular}[c]{@{}c@{}} Propensão média \\ a poupar \end{tabular} \\ \hline
		\end{tabular}%
	}
	\caption*{\textbf{Fonte:} Elaboração própria}
\end{table}
