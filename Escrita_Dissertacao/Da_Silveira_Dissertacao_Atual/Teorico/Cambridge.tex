\subsection{Modelo de Cambridge}

%TODO Mais contexto

O modelo de Cambridge\footnote{Para uma análise mais detalhada das origens e extensões do modelo de Cambridge, ver \textcite{baranzini_cambridge_2013}.} tinha entre seus objetivos estender as implicações do princípio da demanda efetiva para o longo prazo sem que, para isso, incorresse em um aparato marginalista\footnote{
	Como destaca \textcite[p.~127--133]{davidson_accumulation_1978}, tais autores estavam preocupados em analisar economias com taxas de crescimento equilibradas --- e não na transição entre diferentes trajetórias de crescimento --- em condições normais (``tranquilas'', sem distúrbios persistentes) na ausência de frustração de expectativas.
} \cites{kaldor_alternative_1955}{kaldor_model_1957}{robinson_model_1962}{pasinetti_rate_1962}. Para tanto, lançam mão das seguintes hipóteses adicionais: (i) os preços são mais flexíveis do que os salários no longo prazo; (ii) economia opera ao nível normal da capacidade; (iii) investimento é autônomo e depende tanto da taxa de lucro (esperada) quanto do \textit{animal spirits}\footnote{Esse componente autônomo do investimento produtivo será levado adiante pelos modelos Kaleckianos.} e (iv) as propensões marginais a poupar das classes sociais são distintas. Dito isso, resta analisar como tais autores lidaram com o problema levantado por Harrod.

Em um primeiro momento, é preciso estabelecer vínculos entre tais modelos e a taxa garantida. \textcite{robinson_model_1962} afirma que quando a composição do estoque de capital está adequada com a taxa de crescimento desejada e quando as expectativas das firmas estão de acordo com o desempenho corrente da economia, então o modelo está sob uma taxa de equilíbrio interna. Já \textcite{kaldor_alternative_1955}\footnote{
	\textcite{kaldor_model_1957}, por outro lado, afirma que a metodologia por ele utilizada se assemelha à de \textcite{harrod_essay_1939}, mas tem diferenças, tais como: (i) Crescimento é limitado pela disponibilidade de recursos e não pela insuficiência de demanda efetiva; (ii) Não distingue mudanças técnicas decorrentes de maior acumulação de capital daquelas resultantes de inovações; (iii) Estoque de capital em termos reais é medido pela quantidade de ferro incorporada; (iv) O crescimento econômico decorre tanto da rapidez na absorção de mudanças tecnológicas quanto da propensão à investir; (v) Autoridade monetária é passiva de modo que a taxa de juros de longo prazo é igual à taxa de lucro.
} supõe que o multiplicador keynesiano determinaria o nível de produto no curto-prazo, quando preços e salários são rígidos já que no longo prazo o nível de produto seria igual ao seu potencial ($Y = Y_{FC}, \,u=1$) enquanto os preços seriam flexíveis\footnote{
	Neste ponto vale lembrar as distinções entre os principais autores desta família de modelos que, por sua vez, não comprometem o grau de generalidade da análise aqui realizada. Enquanto \textcite[5--6]{kaldor_alternative_1955} e \textcite{pasinetti_rate_1962} assumiam que o nível de investimento seria suficiente, no longo prazo, para garantir o pleno-emprego; \textcite{robinson_model_1962} pressupunha --- na ausência de uma barreira inflacionária --- uma política monetária acomodatícia de modo que não existiriam restrições ao financiamento do investimento e, assim, a economia operaria em plena utilização da capacidade.
}. Assim, mudanças na taxa de crescimento do gasto autônomo (investimento) teriam como contrapartida variação do nível de preços e mudanças na distribuição\footnote{Para uma crítica da relação entre grau de utilização e preços normais, ver \textcite{ciccone_2017}.}. 

A primeira diferença em relação ao modelo de Harrod é o investimento tratado como autônomo, com isso eliminando a origem da instabilidade harrodiana --- nesta leitura, decorre do investimento reagir ao grau de utilização da capacidade. Esses autores assumem uma estrutura da economia kaleckiana \cite{kalecki_theory_1954}, explicitando as classes sociais. Os trabalhadores, por hipótese, não poupam, logo toda a poupança é feita pelos capitalistas

$$
S = s_p\cdot FT
$$
em que $s_p$ é a propensão marginal a poupar dos capitalistas a partir dos lucros e; $FT$ são os lucros totais. Alterando a equação \ref{Sintetica}, seguindo \textcite{serrano_trouble_2017}, temos

$$
\frac{I}{K} = \frac{S}{K}\frac{Y}{Y}\frac{Y_{fc}}{Y_{fc}} = s_p\frac{FT}{K}\frac{Y}{Y}\frac{Y_{fc}}{Y_{fc}}
$$

\begin{equation}
\label{LucroCambridge}
r = \frac{(1-\omega)\cdot u}{v}
\end{equation}

\begin{equation}
\label{Sintetica}
g = s_p\cdot \frac{(1-\omega)\cdot u}{v}
\end{equation}

\begin{equation}
\label{EqCambridge}
g = s_p\cdot r
\end{equation}

\begin{equation}
\label{Cambridge}
(1-\omega) = \frac{g\cdot v}{s_p}
\end{equation}
em que $\omega$ é a participação dos salários na renda ($\omega = W/Y$) e $r$ é a taxa efetiva de lucro enquanto a Equação \ref{EqCambridge} é a famosa equação de Cambridge\footnote{
	Para uma exposição de seus antecedentes e implicações, ver \textcite{bortis_notes_1993}.
}.


As equações acima explicitam que neste modelo a distribuição funcional da renda é a variável de fechamento  e apresenta uma relação simultânea com a taxa de lucro. Como destacado anteriormente, o investimento é positivamente relacionado pelos lucros e esse resultado decorre dos microfundamentos relacionados com a teoria gerencialista da firma em que maiores taxas de crescimento requerem maiores taxas de lucro, implicando em maiores \textit{mark-ups} e em uma barreira inflacionária \cite[p.~353]{lavoie_post-keynesian_2015} \footnote{Parte considerável das críticas dizem respeito à função de poupança nesta família de modelos uma vez que está associada com os lucros retidos das firmas. Para maiores detalhes, ver  \textcites[Seção III]{skott_kaldoriansaving_1981}{marglin_foundation_1984}{skott_kaldors_1989}.}. 
%MAIS REFERÊNCIAS


Portanto, no modelo de Cambrigde, existe uma relação simultânea e necessária entre crescimento e distribuição de modo que ser resumido nos seguintes termos:

\begin{citacao}
The main message of the Cambridge
equation is that the warranted growth rate is determined by the rate of capital
accumulation gk that results from the investment decisions of entrepreneurs; this
determines the long-period (or normal) income distribution, which thereby
becomes endogenous and subordinated to the rate of accumulation \cite[p.~158]{cesaratto_neo-kaleckian_2015}
\end{citacao}
Desse modo, obtém-se uma relação positiva entre poupança e crescimento no longo prazo ou ainda uma relação negativa entre salários reais e taxa lucros (como explicitado na Eq. \ref{LucroCambridge}). Consequentemente, para a garantir o equilíbrio entre demanda e capacidade produtiva associado a uma maior taxa de crescimento é necessário que uma parcela menor da renda seja destinada ao consumo. A importância de explicitar esta causalidade em termos do consumo é que destaca a importância do mecanismo de preços no modelo e a respectiva resolução da instabilidade de Harrod. Como mencionado anteriormente, os preços são mais flexíveis do que os salários por hipótese. Assim, se a taxa crescimento da economia estiver acima da taxa garantida, instaura-se um aumento dos preços acima dos salários e opera-se um mecanismo de poupança forçada. O resultado é uma redução dos salários reais e, por definição, aumento da participação dos lucros na renda.  Neste modelo, portanto, é justamente a mudança na distribuição funcional da renda, e consequentemente a propensão marginal a poupar da economia, que promove o ajuste da taxa garantida para a taxa efetiva de crescimento assegurando a estabilidade do modelo.

Desse modo, por mais que tal modelo consiga reproduzir o fato estilizado de que capacidade produtiva e demanda se equilibram no longo prazo, a igualdade entre poupança e investimento não se dá por variações no produto:

\begin{citacao}
	\textit{
		In sum then, both Robinson and Kaldor-Pasinetti rely on	income redistribution via profit margins relative to money wages at a given level of employment to adjust when short-period entrepreneurial sales forecasts are proved incorrect,
		while in The General Theory, Keynes suggested that changes in the level of employment were the primary short-period adjustment mechanism, with income distribution playing a less important role.	
	} \cite[p.~127]{davidson_accumulation_1978}
\end{citacao}
Além disso, não são verificados os resultados decorrentes da teoria gerencialista da firma associados a essa teoria.
Tal microfundamentação implica na flexibilização dos preços e das margens de lucro dadas mudanças na demanda agregada que não é razoável seja no nível micro ou macroeconômico  e, portanto, deve ser rejeitada
\footnote{Para maiores detalhes, ver discussão em \textcites{ciccone_2017}[p.~104--5, n. 17]{serrano_teoria_1988}.}\footnote{
	Outro tipo de crítica pode ser visto em \textcite[p.~127]{davidson_accumulation_1978} em que o autor realça a incompatibilidade com as implicações de uma economia monetária de produção:
	\begin{citacao}
		
		[I]t is an essential characteristic of a monetary economy that offer
		prices and money wages should have short-period stickiness
		and hence employment levels \textit{must} be more adjustable to disequilibrium conditions. Hence, these neo-keynesian models
		will be seen to be deficient in terms of their discussions of
		monetary aspects.
	\end{citacao}
que, no entanto, não desqualifica a contribuição desses autores como pontua Davidson logo em seguida:

	\begin{citacao}
		Nevertheless, the authors of these models
		recognised the limitations of their efforts and have continually
		stressed that their analyses are basically concerned with long-run steady rates of equilibrium growth.
	\end{citacao}
}. \textcite[p.~158]{cesaratto_neo-kaleckian_2015}, por sua vez, destaca a falta de robustez na relação entre taxas de crescimento mais elevadas e mudanças na distribuição de renda a favor dos lucros.
Tais limitações do modelo de Cambridge não devem ser entendidas como uma impossibilidade do crescimento ser \textit{demand-led} no longo prazo. Argumenta-se aqui que a adequação da capacidade produtiva à demanda não precisa lançar mão de tais hipóteses. 

Na tentativa de responder à instabilidade de Harrod, parte da literatura abandona a hipótese de endogeneidade da distribuição de renda por meio da existência de uma estrutura de mercado oligopolista. A título de exemplo, \citeauthor*{steindl_maturity_1952} afirma que em seu livro de 1952 \cite{steindl_maturity_1952} possuia um raciocínio semelhante ao de Kaldor para o caso de estrutura de mercado competitiva em que tanto as taxas de lucro quanto o grau de utilização estariam em seu nível normal no longo prazo. No entanto, quando revisita essa ideia \cite{steindl_stagnation_1979}, afirma que tal análise da distribuição não é adequada para uma economia oligopolizada. Esta proposta será analisada na seção seguinte.

\begin{comment}
DESCARTADOS

\begin{equation}
\frac{I}{K} = g = \gamma + \gamma_r r
\end{equation}
Esse raciocínio pode ser traduzido em termos da equação \ref{Sintetica}\footnote{A versão proposta por \textcite{pasinetti_rate_1962} explicita as condições de \textit{stedy state} em que a taxa de juros e lucros precisam ser iguais no longo prazo. \textcite[p.~101]{kurz_post-keynesian_2010} destacam que a função poupança de Kaldor só é possível no longo prazo se a taxa de juros não exceder a taxa de lucros. Além disso, a exclusão da propensão marginal à poupar dos trabalhadores é decorrência do ``Teorema de Pasinetti'' em que a taxa de lucro independe da poupança dos trabalhadores.
}:

\begin{equation}
\label{Cambridge_Parcial}
\gamma + \gamma_r r = \mybox{$g = g_K$} =  f\frac{s_k\cdot u_N}{\overline v}
\end{equation}
Adiante, decompõe-se a taxa de lucro ($r$) nos termos de \textcite{weisskopf_marxian_1979}:
$$
r = \frac{P}{K} = \frac{P}{Y}\frac{Y}{Y_{FC}}\frac{Y_{FC}}{K}
$$
em que $P$ é a massa de lucros e $\omega$ o \textit{wage-share}. Como a relação capital-produto é considerada constante, a taxa de lucro depende simultaneamente do grau de utilização e distribuição de renda:
\begin{equation}
\label{Decomposicao_Lucro}
r = \frac{(1-\omega)\cdot u}{\overline v}  
\end{equation}
Substituindo a equação \ref{Decomposicao_Lucro} na \ref{Cambridge_Parcial}, obtém-se

$$
\gamma + \gamma_r \frac{(1-\omega)\cdot u}{\overline v} = \mybox{$g = g_K$} =  f\frac{s\cdot u}{\overline v}
$$


No modelo de Cambridge, a propensão marginal a poupar é definida exogenamente e é idêntica a propensão média, logo, $f=1$. Além disso, o grau de utilização converge ao normal e, portanto, não assume quaisquer valores necessários para garantir a igualdade entre demanda e capacidade produtiva e uma vez constante não pode ser a variável de fechamento. Por fim, supondo que o componente autônomo do investimento seja constante, isto é, $\gamma = \overline \gamma$ e isolando as variáveis endógenas restantes,

$$
\gamma_r \frac{(1-\omega)\cdot u}{\overline v} =   \frac{\overline s\cdot \overline u_N}{\overline v} - \gamma
$$

$$
\frac{(1-\omega)\cdot u}{\overline v} =   \left(\frac{\overline s\cdot \overline u_N}{\overline v} - \gamma\right)\frac{1}{\gamma_r}
$$

$$
(1-\omega) =   \left(\frac{\overline s\cdot \overline u_N}{\overline v} - \gamma\right)\frac{\overline v}{\overline \gamma_r \cdot \overline u_N}
$$

\begin{equation}
\label{Cambridge}
\therefore (1-\omega) = \frac{1}{\overline\gamma_r}\left(\overline s - \frac{\overline\gamma\cdot\overline v}{\overline u_N}\right)
\end{equation}

%Em linha com a formalização de \textcite[p.~347-59]{lavoie_post-keynesian_2015}, o raciocínio acima é estendido para a determinação da taxa de acumulação ($g_K$) que depende positivamente ($\gamma_r$) da taxa de lucro ($r$) e dos \textit{animal spirits} ($\gamma$)\footnote{Dentre os critérios para adequar um modelo, \textcite{robinson_model_1962} escolhe aquele que é compatível com os determinantes do comportamento humano em uma economia capitalista (\textit{animal spirit}). Além disso, a autora realça algumas características que considera fundamental em uma economia capitalista, tais como: produção é organizada por firmas (economia monetária de produção) e o consumo é destinado às famílias que, por sua vez, podem ser rentistas ou trabalhadoras. Alguns dos elementos citados anteriormente comporiam o centro da teoria pós-Keynesiana e que mereceriam uma análise mais detalhada. No entanto, dados os objetivos desta investigação, a ênfase recairá sobre a importância da autonomia do investimento.}:

\end{comment}

