\subsection{Autonomia do investimento residencial}
\label{RevResidencial}

O objetivo desta seção é apresentar a teorias em torno do investimento residencial dando ênfase à sua autonomia. Antes de avançar, destaca-se a tentativa de tratar este componente da demanda agregada em termos teóricos uma vez que a discussão empírica será endereçada no capítulo seguinte. 
Sendo assim, é feito um breve mapeamento tanto da ortodoxia quanto da heterodoxia a respeito do tema com especial atenção para sua autonomia, a começar pelos trabalhos ortodoxos.
Neste ponto, convém pontuar que uma grande parte dos trabalhos envolvendo investimento residencial tem deixado de tratá-lo macroeconomicamente de modo que uma parcela considerável tem se restringido a questões microeconômicas e regionais \cite{arestis_u.s._2008}.
É importante ressaltar também que, por se tratar de um gasto não criador de capacidade produtiva, o debate\footnote{Para mais detalhes, ver \textcite{arku_housing_2006}.} sobre a importância do investimento residencial para o desenvolvimento econômico se restringiu em categorizá-lo enquanto absorvedor de recursos produtivos \cite{solow_importance_1995} enquanto parte da literatura de crescimento indicava  a possibilidade de um sobreinvestimento residencial \cite{mills_has_1987}.
Além disso, parte da (diminuta) literatura sobre o tema tem apontado em direção para a perda de importância relativa do investimento residencial para a economia \cite{grebler_capital_1956}\footnote{
	Em relação a perda da relevância do investimento residencial, \textcite[p.~134-142]{grebler_capital_1956} pontuam que as modificações nas preferências dos consumidores que tenderiam a destinar uma parcela maior da renda ao consumo de bens duráveis e menos aos imóveis. No entanto, quanto atualizam a análise com os novos dados disponibilizados após a publicação deste livro, recolocam tal argumento indicando que pode ser o caso de uma reversão da tendência de queda da taxa de investimento residencial.
}\footnote{
	\textcite{rognlie_deciphering_2016}, por sua vez, encontra evidências para a maior importância do investimento residencial na economia.
	Resumidamente, parte de um modelo multisetorial para investigar a queda e retomada da participação dos lucros líquidos na renda.
	Ao avaliar as diferentes elasticidades de substituição entre capital e trabalho, não encontra evidência para a tese da acumulação defendida por \textcite{piketty_o_2014} e defende a hipótese da escassez associada tanto aos maiores custos de construção dos imóveis quanto menor disponibilidade de terra.
	Dito isso, conclui que tais elementos contribuíram para a elevação do preço dos imóveis e subsequente participação na renda.
	Por fim, resta destacar que por centrar a análise na substituição entre os fatores de produção a comparatibilidade deste estudo com aqueles desenvolvidos pela heterodoxia é comprometida e, portanto, optou-se por não analisá-lo pormenorizadamente.
}.

Apesar de não adotarem esta terminologia, alguns destes trabalhos indicam a autônomia do investimento residencial tanto por conta da durabilidade dos imóveis \cite{derksen_long_1940} quanto por fatores populacionais e demográficos \cites{hansen_economic_1939}{grebler_capital_1956}. Como consequência, ambas as constatações apontam em direção para a existência de ciclos imobiliários cuja temporalidade é distinta e mais ampla que a da renda. 
Tais trabalhos, no entanto, têm dado pouca ênfase aos determinantes da demanda --- centram a análise na oferta\footnote{Os fatores populacionais e demográficos mencionados no corpo do texto dizem respeito a oferta de imóveis e não a demanda. Como será discutindo adiante, parte dos modelos heterodoxos consideram este elemento como determinante da demanda por imóveis.} --- e, portanto, não são compatível com a agenda da macroeconomia da demanda efetiva.

Desta breve revisão da literatura, pontua-se o destaque dado tanto ao crescimento populacional quanto às mudanças demográficas na análise dos determinantes da oferta de novos imóveis.
Dentre estes (poucos) autores, apenas \textcite{grebler_capital_1956} pontuam a importância das condições de financiamento\footnote{
	Adicionalmente, é importante ressaltar que tal estudo foi realizado em meio a mudanças institucionais significativas em curso na economia norte-americana que, por sua vez, são reconhecidas pelos autores:
	\begin{citacao}
		\textit{All these observations have ignored what is potentially perhaps the most important variable: the future role of government. The concern
			of government with housing and its financing, though it originated in
			depression emergencies and was intensffied by war and postwar dislocations, is likely to be a lasting and probably increasing influence on
			residential construction}.
		\cite[p.~27]{grebler_capital_1956}
	\end{citacao}
	Tal discussão será retomada no capítulo seguinte em que serão realçados alguns fatos estilizados.
} para a demanda por imóveis\footnote{Além dos condicionantes financeiros, os autores também chamam a atenção tanto para as mudanças nos balanços patrimoniais dos investidores institucionais quanto para o aumento da dívida hipotecária associada a maior taxa de propriedade imobiliária.
}:
	\begin{citacao}
		\textit{Because the demand for housing is particularly sensitive to credit
		terms and depends heavily on debt financing, a general long-term increase
		in the cost of borrowing would tend to affect the volume of
		residential construction more than the volume of business investment,
		a large portion of which is financed from internal funds}
		\cite[p.~322--3]{grebler_capital_1956}
	\end{citacao}

%MIAN E SUFI

Examinada parcela dos trabalhos ortodoxos, cabe inspecionar a forma com que a heterodoxia tratou do tema. 
% Financeirização e hipotecarização
Parte significativa desta literatura  --- emergente no pós-crise imobiliária --- centra esforços na conexão deste tipo de gasto com processos mais gerais como a financeirização \cites{aalbers_financialization_2008}{bibow_financialization_2010} enquanto uma pequena fração aborda investimento residencial e crescimento.
Uma exceção é o trabalho de \textcite{zezza_u.s._2008} em que são investigados os efeitos da diminuição --- apesar da distribuição da renda a favor dos lucros --- da propensão marginal a poupar da economia norte-americana a partir da metodologia \textit{Stock-Flow Consistent} com mercado de ações e imobiliário\footnote{
	Alguns trabalhos seguiram a contribuição de \textcite{zezza_u.s._2008}.
	Um deles é o de \textcite{nikolaidi_securitisation_2015} com dois tipos de agentes demandando imóveis: parcela dos trabalhadores e investidores institucionais.
	Para os primeiros, a demanda por casas é determinada positivamente pela poupança deste setor acrescido de empréstimos hipotecários e negativamente pelo preço dos imóvies de modo que não pode ser considerado estritamente autônomo.
	Já os demais agentes, demandam imóveis tal como outros ativos financeiros, ou seja, depende positivamente de sua taxa de retorno.
	Em conjunto, tais equações comportamentais determinam que a taxa de crescimento do investimento residencial depende tanto da razão entre a demanda por imóveis em relação ao total quanto de sua inflação que, por sua vez, é determinada pelo estoque de imóveis não vendidos.
	Sendo assim, o investimento residencial no trabalho de \textcite{nikolaidi_securitisation_2015} pode ser considerado ``semi-autônomo''.
	No entanto, ao partir do procedimento de \textcite{godley_money_1999} para determinação do portfólio de ativos dos agentes, trata os imóveis como um ativo financeiro qualquer sem considerar suas particularidade, qual seja, durabilidade e baixo risco.
}\footnote{
	Tal resultado, argumenta, decorre dos ganhos de capital nos mercados imobiliário e acionário entre o topo da distribuição, contribuindo para a diminuição da taxa de poupança.
}. 
Por mais que este trabalho seja uma via para a inclusão do investimento residencial nos modelos macroeconômicos, tal gasto não é o principal determinante da dinâmica uma vez que parte de uma especificação kaleckiana do investimento das firmas.
Sendo assim, a influência deste componente da demanda para a dinâmica é bastante limitada.

Outra vertente heterodoxa, tem lançado mão de modelos baseados em agentes (ABM) para avaliar as relações entre instabilidade financeira, endividamento das famílias e distribuição de renda.
Em linha com \textcite{cynamon_inequality_2013} e \textcite{erlingsson_integrating_2013}, \textcite{cardaci_inequality_2018} parte da hipótese de consumo cascata desenvolvida por \textcite{frank_expenditure_2014}\footnote{
	De acordo com esta hipótese do efeito cascata, as famílias aprendem os padrões de consumo através dos grupos de referência a quem comparam seus padrões de vida.
} para conectar a piora da distribuição pessoal da renda ao aumento do preço dos imóveis\footnote{Dentre as contribuições, vale pontuar a endogeinização de um mercado (e rede) de crédito com consumo financiado por empréstimos colateralizados por hipotecas --- como em \textcite{mian_house_2011} --- como uma fonte alternativa de financiamento que compensa a estagnação salarial.}.
Apesar de relevante, tal contribuição não avança em direção a uma qualificação da autonomia do investimento residencial e, portanto, deve-se prosseguir na busca de alternativas na heterodoxia.

Partindo dos mercados externos de Rosa Luxemburgo, \textcite{fiebiger_semi-autonomous_2018} argumenta que o investimento residencial --- além de não criar capacidade produtiva --- é semi-autônomo uma vez que é financiado principalmente --- mas não estritamente --- por crédito

\begin{citacao}
	\textit{The key aspect of an external market is that the expenditures
		are financed by a source external to the firm sector’s current outlays. Worker invest-
		ment in new dwellings and debt-financed consumption possess the characteristics of
		an external market. The reason is because the finance for those expenditures is largely
		external to current wage income and, when financed by banks, also independent of any
		prior income. \textbf{A home purchase will usually exceed the annual wage of non-supervisory
			workers}.
	}
	\cite[p.~3, grifos adicionados]{fiebiger_semi-autonomous_2018}
\end{citacao}
Como consequência, tal tipo de gasto gera uma fonte externa de lucro para as firmas.
Em outras palavras, a renda gerada pelas firmas não é sua a única fonte de receitas.
Em seguida, argumenta que na medida que estes mercados externos lideram a acumulação, induzem o investimento criador de capacidade a se ajustar a demanda.
Desse modo, \textcite{fiebiger_semi-autonomous_2018} evidencia a conexão entre princípio de ajuste do estoque de capital e os mercados externos de Rosa Luxemburgo, ampliando a abrangência das teorias compatíveis com o modelo do supermultiplicador sraffiano.
Desta discussão, conclui-se que o investimento das famílias pode ser considerado (ao menos) como semi-autônomo. 

Em paralelo, --- e seguindo explicitamente o modelo do supermultiplicador sraffiano --- \textcite{gowans_introducing_2014} argumenta que o investimento residencial é autônomo e esta autonomia decorre --- como em \textcite{hansen_economic_1939} --- do crescimento populacional que, por sua vez, cresce a uma taxa exógena\footnote{
	Para uma discussão do primeiro problema de \textcite{harrod_essay_1939} em um modelo do tipo supermultiplicador, ver \textcite{allain_demographic_2018}.
}.
Em seguida estabelece --- também seguindo \textcite{hansen_economic_1939} --- os principais canais pelo qual o crescimento populacional\footnote{
É digno de nota pontua que se o crescimento populacional é o principal determinante do investimento residencial, é esperado que sua autonomia decaia com a queda da taxa de crescimento vegetativo.
No entanto,  \textcite[p.~11]{grebler_capital_1956} argumentam que este não é o caso quando são consideradas mudanças demográficas nas unidades familiares:

\begin{citacao}
	\textit{The manner in which the population arranges itself into households
		occupying separate dwelling units has been subject to marked changes,
		which are associated with trends in longevity and other demographic
		factors, changes in taste and preferences, and the rise in per capita real
		income. Under the influence of these factors the social units occupying
		or seeking separate dwelling units have become more and more fragmentized.}
\end{citacao}
} é repassado para a demanda efetiva, são eles: (i) Formação de capital residencial; (ii) Gastos do governo e; (iii) Produção de bens essenciais.
Dessa discussão, o autor conclui que --- diferentemente de \textcite{robinson_model_1962} --- crescimento populacional e econômico não são fenômenos separados e devem ser tratados conjuntamente.
Apesar destes elementos esclarecerem alguns aspectos sobre a autonomia do investimento residencial, argumenta-se que não são os únicos.

Neste ponto, o ineditismo de \textcite{duesenberry_investment_1958} se destaca por realizar uma descrição detalhada da estrutura do mercado imobiliário.
Em seguida, argumenta que o investimento residencial depende tanto da renda quanto das movimentações demográficas e das condições de financiamento das famílias.
Com esta categorização em mãos, \textcite{duesenberry_investment_1958} apresenta um modelo\footnote{As variáveis são indicadas por letras diferentes do original para manter a consistência ao longo da dissertação.} para a construção de novos imóveis --- mas não incorpora a especulação --- cuja forma simplificada\footnote{
	Na versão mais completa --- mas não formalizada --- a construção de novos imóviesdepende de: (i) alugueis e preços dos imóveis em relação aos demais preços; (ii) taxa de vacância e relação entre demanda e oferta para as casas individuais; (iii) número de construtores especulativos; (iv) ativos e aversão ao risco da comunidade; (v) nível de construção e custos operacionais.} é:

$$
I_{h} = I_h(p_{h_t}, g, g_n)
$$
$$
\Delta I_{sh} =  I_{sh}(u_h, p_{h_t}, s_t)
$$
$$
s_t = s_t\left(\frac{\Delta I_{h}}{I_{hs}}\right)
$$
$$
\Delta p_{h_t} = p_{h_t}\left(\frac{I_{h}}{I_{hs}}\right)
$$
em que $I_h$ indica a demanda por imóveis enquanto $I_{hs}$ corresponde à oferta, $g$ é a taxa de crescimento do produto, $g_n$ é a taxa de crescimento populacional,
$u_h$ é a capacidade de produção do setor imobiliário (semelhante ao grau de utilização das firmas).
Resumidamente, este sistema dinâmico é liderado principalmente pela taxa de crescimento populacional e do produto. No entanto, argumenta que quando tais variáveis apresentam dinâmicas erráticas (representado por $\varepsilon$) ou quando os parâmetros do modelo mudam, a taxa de construção pode ter uma dinâmica bastante distinta da renda. Nestes casos,

\begin{equation}
\label{Duesen_sInfla}
I_h = I_h(p_{h_t}, g_n, \varepsilon)
\end{equation}
A partir deste modelo, o autor chega a conclusão que existem dois tipos de ciclos imobiliários:
\begin{citacao}
\textit{We must think of two typos of housing cycle: (1) Cycles in which the normal relation between housing investment and income movements operate and  in which housing investment tends to follow the path of movement of aggregate income, though with different timming.
In this case, there is not only very little tendency for residential construction to move cyclically by itself, but the charecteristics of the housing industry are such that residential constrution tends to  stabilize the system against other sources of fluctuation.
(2) Cycles in which changes in the structure of the housing industry itself, such as wartime construction backlogs and speculation, tend to produce a fluctuation in the rate of house bulding even when aggregate income is growing steadily. 
In that case, there is a \textbf{genuine independent housing fluctuation} which communicates itself to the movement of agregate income.}
\cite[p.~164. grifos adicionados]{duesenberry_investment_1958}
\end{citacao}
Dito isso e tomando a equação \ref{Duesen_sInfla} em termos de taxa de crescimento, obtém-se um possível caminho para a endogeinização da taxa de crescimento do investimento residencial:
\begin{equation}
g_{I_h} = g_{I_h}(\dot p_{h_t}, \dot g_n)
\end{equation}
em que $\dot p_{h_t}$ indica inflação de imóveis e $\dot g_n$ pode ser compreendido como alterações demográficas.
Apesar desta ser uma primeira aproximação da conexão entre investimento residencial e bolha de ativos, não possui uma forma funcional de modo que apenas principais as variáveis são explicitadas.
De todo modo, evidencia-se a autonomia do investimento residencial para o caso com inflação de imóveis.

Portanto, deste breve levantamento dos (poucos) trabalhos teóricos que incluem investimento residencial, conclui-se que este gasto pode ser considerado ao menos como semi-autônomo uma vez que está associado tanto a fatores demográfico quanto a condições de financiamento.
Além disso, outros trabalhos indicam a temporalidade distinta do ciclo de imóveis em relação ao ciclo da renda decorrente da durabilidade deste ativo.
Por fim, argumenta-se que na presença de especulação, amplia-se a autonomia do investimento residencial em relação a renda. 
Desse modo, na presença de bolha imobiliária e na ausência de restrição de crédito o ciclo de imóveis tende a apresentar --- de um ponto de vista estritamente teórico --- uma dinâmica distinta da renda, ou seja, é autônomo.
Dito isso, cabe a seção seguinte discutir como a literatura tem incluido os gastos autônomos não criadores de capacidade produtiva para então elencar uma forma de tratar o investimento residencial nesses termos.


PONTE SEÇÃO SEGUINTE.

%RESPONDER SKOTT


%DUESENBERRY: INVESTIMENTO RESIDENCIAL E DEMOGRAFICA


%DESTA REVISÃO DA LITERATURA VERIFICA-SE QUE OS MODELOS OU RESTRINGEM A CONTRIBUIÇÃO DO INVESTIMENTO RESIDENCIAL A PERDA DA PARTICIPAÇÃO RELATIVA DOS SALÁRIOS OU CENTRAM A ANÁLISE EM ELEMENTOS DEMOGRÁFICOS ALÉM DE DEIXAR DE LADO A DEMANDA POR IMÓVEIS POR RAZÕES ESPECULATIVAS. COMO SERÁ MELHOR DISCUTIDO NO CAPÍTULO SEGUINTE, A RELEVÂNCIA DO INVESTIMENTO RESIDENCIAL NO EUA SE DÁ PELA CONEXÃO DESTE GASTO COM A AMPLIAÇÃO DO COLATERAL E SUBSEQUENTE ELEVAÇÃO DO ENDIVIDAMENTO/CONSUMO. UMA VEZ QUE ESTE CAPÍTULO É DE TEOR TEÓRICO, PRETENDE-SE DISCUTIR COMO INCLUIR TAL COMPONENTE DA DEMANDA AGREGADA SEM CONFINÁ-LO A UM CASO ESPECÍFICO.

%- A AUTONOMIA DESTE GASTO NÃO SE RESTRINGE AO CRESCIMENTO POPULACIONAL.

%TAXA PRÓPRIA DE JUROS


\begin{comment}
Feita esta descrição, Duesenberry pontua: (i) a flutuação dos imóveis decorrente de um desequilíbrio inicial é amortecida; (ii) a flutuação da renda causará flutuações no investimento residencial com defasagem; (iii) este setor possui demanda de reserva e, portanto, os efeitos de uma recessão nesse setor são mais dispersos ao longo do tempo que, por sua vez, contribui para que a perda de dinamismo da renda seja atenuado.

\end{comment}