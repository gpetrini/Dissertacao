\subsection{Investimento residencial nos modelos macroeconômicos}
\label{RevResidencial}

O objetivo desta seção é apresentar algumas teorias em torno do investimento residencial. 
Antes de avançar, destaca-se a tentativa de tratar este componente da demanda agregada em termos estritamente teóricos uma vez que a discussão empírica será endereçada no capítulo seguinte. 
Neste ponto, convém pontuar que uma grande parte dos trabalhos envolvendo investimento residencial tem deixado de tratá-lo macroeconomicamente de modo que uma parcela considerável tem se restringido a questões microeconômicas e regionais como destacam \textcite{arestis_u.s._2008}.
É importante ressaltar também que, por se tratar de um gasto não criador de capacidade produtiva, o debate sobre a importância do investimento residencial enquanto estratégia para atingir o desenvolvimento econômico se restringiu em categorizá-lo enquanto absorvedor de recursos produtivos \cite{solow_importance_1995} enquanto parte da literatura de crescimento indicava  a possibilidade de um sobreinvestimento residencial \cite{mills_has_1987}\footnote{Para mais detalhes, ver \textcite{arku_housing_2006}.}.
%TODO Mais detalhes Arku
%\footnote{Além disso, parte da (diminuta) literatura sobre o tema tem apontado em direção para a perda de importância relativa do investimento residencial para a economia. \textcite[p.~134-142]{grebler_capital_1956} pontuam que as modificações nas preferências dos consumidores que tenderiam a destinar uma parcela maior da renda ao consumo de bens duráveis e menos aos imóveis. No entanto, quanto atualizam a análise com os novos dados disponibilizados após a publicação deste livro, recolocam tal argumento indicando que pode ser o caso de uma reversão da tendência de queda da taxa de investimento residencial.
%}
%\footnote{
%	\textcite{rognlie_deciphering_2016}, por sua vez, encontra evidências para a maior importância do investimento residencial na economia.
%	Resumidamente, parte de um modelo multisetorial para investigar a queda e retomada da participação dos lucros líquidos na renda.
%	Ao avaliar as diferentes elasticidades de substituição entre capital e trabalho, não encontra evidência para a tese da acumulação defendida por \textcite{piketty_o_2014} e defende a hipótese da escassez associada tanto aos maiores custos de construção dos imóveis quanto menor disponibilidade de terra.
%	Dito isso, conclui que tais elementos contribuíram para a elevação do preço dos imóveis e subsequente participação na renda.
%	Por fim, resta destacar que por centrar a análise na substituição entre os fatores de produção a comparatibilidade deste estudo com aqueles desenvolvidos pela heterodoxia é comprometida e, portanto, optou-se por não analisá-lo pormenorizadamente.
%}

%Apesar de não adotarem esta terminologia, alguns dos trabalhos indicam que o investimento residencial é autônomo tanto por conta da durabilidade dos imóveis \cite{derksen_long_1940} quanto por fatores populacionais e demográficos \cites{hansen_economic_1939}{grebler_capital_1956}. Como consequência, ambas as constatações apontam em direção para a existência de ciclos imobiliários cuja temporalidade é distinta e mais ampla que a da renda. 
%Dentre estes (poucos) autores, apenas \textcite{grebler_capital_1956} pontuam a importância das condições de financiamento\footnote{
%	Adicionalmente, é importante ressaltar que tal estudo foi realizado em meio a mudanças institucionais significativas em curso na economia norte-americana que, por sua vez, são reconhecidas pelos autores:
%	
%	\begin{citacao}
%		\textit{All these observations have ignored what is potentially perhaps the most important variable: the future role of government. The concern
%			of government with housing and its financing, though it originated in
%			depression emergencies and was intensffied by war and postwar dislocations, is likely to be a lasting and probably increasing influence on residential construction}.
%		\cite[p.~27]{grebler_capital_1956}
%	\end{citacao}
%	Tal discussão será retomada no capítulo seguinte em que serão realçados alguns fatos estilizados.
%} para a demanda por imóveis\footnote{Além dos condicionantes financeiros, os autores também chamam a atenção tanto para as mudanças nos balanços patrimoniais dos investidores institucionais quanto para o aumento da dívida hipotecária associada a maior taxa de propriedade imobiliária.
%}:
%
%\begin{citacao}
%	\textit{Because the demand for housing is particularly sensitive to credit
%		terms and depends heavily on debt financing, a general long-term increase
%		in the cost of borrowing would tend to affect the volume of
%		residential construction more than the volume of business investment,
%		a large portion of which is financed from internal funds}
%	\cite[p.~322--3]{grebler_capital_1956}
%\end{citacao}
%Tais trabalhos, no entanto, têm dado pouca ênfase aos determinantes da demanda\footnote{Os fatores populacionais e demográficos mencionados no corpo do texto dizem respeito a oferta de imóveis e não a demanda. Como será discutindo adiante, parte dos modelos heterodoxos consideram este elemento como determinante da demanda por imóveis.} e, portanto, não são compatíveis com a agenda da macroeconomia da demanda efetiva.

Feito este breve panorama, cabe inspecionar a forma com que a literatura recente tratou do tema. 
% Financeirização e hipotecarização e variabilidades de capitalismo
Parte significativa desta literatura  --- emergente no pós-crise imobiliária --- centra esforços na conexão deste tipo de gasto com processos mais gerais como a financeirização \cites{aalbers_financialization_2008}{bibow_financialization_2010}
enquanto uma fração minoritária o relaciona com as variabilidades
de capitalismo e as relações com o \textit{welfare state} \cite{schwartz_politics_2009}. No
entanto, a partir da revisão bibliográfica, verificou-se que uma fração pequena da literatura heterodoxa aborda as relações entre crescimento e investimento residencial.
Uma exceção é o trabalho de \textcite{zezza_u.s._2008} em que são investigados os efeitos da diminuição --- apesar da distribuição da renda a favor dos lucros --- da propensão média a poupar da economia norte-americana por meio da introdução do mercado imobiliário na metodologia \textit{Stock-Flow Consistent}\footnote{
	Alguns trabalhos seguiram a contribuição de \textcite{zezza_u.s._2008}.
	Um deles é o de \textcite{nikolaidi_securitisation_2015} com dois tipos de agentes demandando imóveis: parcela dos trabalhadores e investidores institucionais.
	Para os primeiros, a demanda por casas é determinada positivamente pela poupança deste setor acrescido de empréstimos hipotecários e negativamente pelo preço dos imóvies de modo que não pode ser considerado estritamente autônomo.
	Já os demais agentes, demandam imóveis tal como outros ativos financeiros, ou seja, depende positivamente de sua taxa de retorno.
	Em conjunto, tais equações comportamentais determinam que a taxa de crescimento do investimento residencial depende tanto da razão entre a demanda por imóveis em relação ao total quanto de sua inflação que, por sua vez, é determinada pelo estoque de imóveis não vendidos.
	Sendo assim, o investimento residencial no trabalho de \textcite{nikolaidi_securitisation_2015} pode ser considerado ``semi-autônomo''.
	No entanto, ao partir do procedimento de \textcite{godley_money_1999} para determinação do portfólio de ativos dos agentes, trata os imóveis como um ativo financeiro qualquer sem considerar suas particularidade, qual seja, durabilidade e baixo risco.
}\footnote{
	Tal resultado, argumenta, decorre dos ganhos de capital nos mercados imobiliário e acionário entre o topo da distribuição, contribuindo para a diminuição da taxa de poupança.
}. 
Por mais que este trabalho seja uma via para a inclusão do investimento residencial nos modelos macroeconômicos, tal gasto não é o principal determinante da dinâmica uma vez que parte de uma especificação kaleckiana do investimento das firmas.
Sendo assim, a influência do investimento das famílias para a dinâmica é bastante limitada.

Outra vertente heterodoxa tem lançado mão de modelos baseados em agentes (ABM) para avaliar as relações entre instabilidade financeira, endividamento das famílias e distribuição de renda.
Em linha com \textcite{cynamon_inequality_2013} e \textcite{erlingsson_integrating_2013}, \textcite{cardaci_inequality_2018} parte da hipótese de consumo cascata retomada por \textcite{frank_expenditure_2014}\footnote{
	De acordo com esta hipótese do efeito cascata, as famílias aprendem os padrões de consumo através dos grupos de referência a quem comparam seus padrões de vida.
} para conectar a concentração da renda ao aumento do preço dos imóveis.
%\footnote{Dentre as contribuições, vale pontuar a endogeinização de um mercado (e rede) de crédito com consumo financiado por empréstimos colateralizados por hipotecas --- como em \textcite{mian_house_2011} --- como uma fonte alternativa de financiamento que compensa a estagnação salarial.}.
Apesar de relevante, tal contribuição não avança em direção a uma especificação da taxa de crescimento do investimento residencial e, portanto, deve-se prosseguir na busca de alternativas na heterodoxia.

Partindo dos mercados externos de Rosa Luxemburgo, \textcite{fiebiger_semi-autonomous_2018} argumenta que o investimento residencial --- além de não criar capacidade produtiva --- é semi-autônomo uma vez que é financiado principalmente --- mas não somente --- por crédito

\begin{citacao}
	\textit{The key aspect of an external market is that the expenditures
		are financed by a source external to the firm sector’s current outlays. Worker invest-
		ment in new dwellings and debt-financed consumption possess the characteristics of
		an external market. The reason is because the finance for those expenditures is largely
		external to current wage income and, when financed by banks, also independent of any
		prior income. \textbf{A home purchase will usually exceed the annual wage of non-supervisory
			workers}.
	}
	\cite[p.~3, grifos adicionados]{fiebiger_semi-autonomous_2018}
\end{citacao}
Por ser (semi)autônomo, tal gasto gera uma fonte externa de lucro para as firmas e na medida que estes mercados externos lideram a acumulação, induzem o investimento criador de capacidade a se ajustar a demanda.
Desse modo, \textcite{fiebiger_semi-autonomous_2018} evidencia a conexão entre o princípio de ajuste do estoque de capital e os mercados externos de Rosa Luxemburgo, ampliando a abrangência das teorias compatíveis com o modelo do supermultiplicador sraffiano.
Desta discussão, conclui-se que o investimento das famílias pode ser considerado (ao menos) como semi-autônomo em relação a renda. 

Em paralelo, --- e seguindo explicitamente o supermultiplicador sraffiano --- \textcite{gowans_introducing_2014} argumenta que o investimento residencial é autônomo e esta autonomia decorre --- como em \textcite{hansen_economic_1939} --- do crescimento populacional que, por sua vez, cresce a uma taxa exógena.
Em seguida estabelece --- também seguindo \textcite{hansen_economic_1939} --- os principais canais pelo qual o crescimento populacional\footnote{
É digno de nota pontua que se o crescimento populacional é o principal determinante do investimento residencial, é esperado que sua autonomia decaia com a queda da taxa de crescimento vegetativo.
No entanto,  \textcite[p.~11]{grebler_capital_1956} argumentam que este não é o caso quando são consideradas mudanças demográficas nas unidades familiares:

\begin{citacao}
	\textit{The manner in which the population arranges itself into households
		occupying separate dwelling units has been subject to marked changes,
		which are associated with trends in longevity and other demographic
		factors, changes in taste and preferences, and the rise in per capita real
		income. Under the influence of these factors the social units occupying
		or seeking separate dwelling units have become more and more fragmentized.}
\end{citacao}
} é repassado para a demanda efetiva, são eles: (i) Formação de capital residencial; (ii) Gastos do governo e; (iii) Produção de bens essenciais.
Dessa discussão, o autor conclui que --- diferentemente de \textcite{robinson_model_1962} --- crescimento populacional e econômico não são fenômenos separados e devem ser tratados conjuntamente.
%Apesar destes elementos esclarecerem alguns aspectos sobre a autonomia do investimento residencial, argumenta-se que não são os únicos.

Da discussão acima, 
conclui-se mesmos estes modelos que incluem o investimento residencial pouco avançaram em seu tratamento teórico.
Neste ponto, a descrição detalhada de \textcite{duesenberry_investment_1958} da estrutura do mercado imobiliário se destaca.
Com esta caracterização em mãos, argumenta que o investimento residencial --- sobretudo dos imóveis unifamiliares --- depende tanto da renda quanto das movimentações demográficas e das condições de financiamento das famílias.
Dadas as características demográficas da população, argumenta que a demanda por novos imóveis é dada por:
	(i) renda (corrente e acumulada);
	(ii) crescimento do número de famílias ($g_n$);
	(iii) ativos mobiliários ($K_h$) e não-imobiliários ($V_h$)\footnote{No caso da demanda por alugueis, \textcite{duesenberry_investment_1958} afirma que a importância dos ativos imobiliários é menor.};
	(iv) preços dos imóveis ($p_h$) e de outros bens e;
	(v) condições de crédito ($L_h$).
De forma semelhante ao modelo anterior, a demanda por imóveis ($I_h$) depende positivamente do crescimento populacional, mas avança em outras direções.
Neste ponto, \textcite{duesenberry_investment_1958} destaca que o impacto do aumento dos preços dos imóveis é ambíguo uma vez que permite que as famílias que possuem casas conseguam se mudar para casa melhores (efeito-riqueza) enquanto aquelas que não as possuem  precisam de um maior poder de compra para obtê-las.
Sendo assim, argumenta que a importância dos ativos não-imobiliários é maior para as famílias que não possuem um imóvel.
Em seguida, afirma que a renda corrente afeta a demanda por imóveis positivamente, mas a relevância é menor na medida que o valor dos adiatamentos (\textit{down payments}) necessários para comprar um imóvel aumentam.
Dito isso, conclui que o acesso ao crédito está entre os principais determinantes da demanda efetiva por imóveis uma vez que afeta o poder de compra das famílias\footnote{
Apesar da importância do crédito para a demanda por imóveis, \textcite{duesenberry_investment_1958} não explicita isso em seu modelo formal.
}.
A equação \ref{Demanda_Ih} resume os pontos aqui destacados:

\begin{equation}
\label{Demanda_Ih}
I_{h} = I_h(\underset{?}{p_{h_t}}, \underset{+}{Y}, \underset{+}{g_n}\, [, \underset{+}{L_h}, \underset{+}{V_h}, \underset{+}{K_h}])
\end{equation}

A oferta de novos imóveis ($I_{hs}$), por sua vez, 
depende positivamente da capacidade de produção do setor imobiliário ($u_h$ semelhante ao grau de utilização das firmas) e dos preços dos imóveis em relação aos custos de produção, bem como da velocidade de venda dos imóveis ($\upsilon_t$) definida como a variação da demanda dos imóveis em relação à oferta existente:
\begin{equation}
\label{Oferta_Ih}
\Delta I_{hs} =  I_{hs}(\underset{+}{u_h}, \underset{+}{p_{h_t}}, \underset{+}{\upsilon_t}) \hspace{3cm} \upsilon_t = \upsilon_t\underset{+}{\left(\frac{\Delta I_h}{I_{hs}}\right)}
\end{equation}
Por fim, os preços --- na ausência de especulação --- são dados pela razão entre a demanda e oferta de imóveis.
Desse modo, os preços aumentam na medida que a demanda cresce mais que a oferta e caem na situação oposta:
\begin{equation}
\label{Precos_Ih}
\Delta p_{h_t} = p_{h_t}\underset{+}{\left(\frac{I_{h}}{I_{hs}}\right)}
\end{equation}

Com estes elementos em mãos, \textcite{duesenberry_investment_1958} constroi um sistema de equações\footnote{As variáveis são indicadas por letras diferentes do original para manter a consistência ao longo da dissertação.} sem especulação\footnote{Para \textcite{duesenberry_investment_1958}, a inflação de imóveis corresponderia principalmente a especulação com as residências multifamiliares.} composto pelas equações de demanda (\ref{Demanda_Ih}), oferta (\ref{Oferta_Ih}) e preços (\ref{Precos_Ih}) dos imóveis.
Resumidamente, a dinâmica deste sistema é determinada principalmente pela taxa de crescimento populacional e do produto. No entanto, \textcite{duesenberry_investment_1958} argumenta que quando tais variáveis apresentam dinâmicas erráticas\footnote{\textcite[p.~158]{duesenberry_investment_1958} não faz nenhum detalhamento do que entende por dinâmica errática destas variáveis. No original:

\begin{citacao}

[W]\textit{hen income and population move erradically, or when the parameters of the system are changed (as a result of changes in tates or institutional arrangements), the movements of the rate of building may follow a course which is very different from the path of movements of income}.
\end{citacao}

} (aqui representadas por $\varepsilon$) ou quando os parâmetros do modelo mudam, a taxa de construção pode ter uma dinâmica desassociada da renda. Nestes casos,

\begin{equation}
\label{Duesen_sInfla}
I_h = I_h(\underset{0}{g}, \underset{?}{p_{h_t}}, \underset{+}{g_n}, \underset{?}{\varepsilon})
\end{equation}
A partir deste modelo, o autor chega a conclusão que existem dois tipos de ciclos imobiliários:
\begin{citacao}
\textit{We must think of two types of housing cycle: (1) Cycles in which the normal relation between housing investment and income movements operate and  in which housing investment tends to follow the path of movement of aggregate income, though with different timming.
In this case, there is not only very little tendency for residential construction to move cyclically by itself, but the charecteristics of the housing industry are such that residential constrution tends to  stabilize the system against other sources of fluctuation.
(2) Cycles in which changes in the structure of the housing industry itself, such as wartime construction backlogs and speculation, tend to produce a fluctuation in the rate of house bulding even when aggregate income is growing steadily. 
In that case, there is a \textbf{genuine independent housing fluctuation} which communicates itself to the movement of agregate income.}
\cite[p.~164. grifos adicionados]{duesenberry_investment_1958}
\end{citacao}
Dito isso e tomando a equação \ref{Duesen_sInfla} em termos de taxa de crescimento, obtém-se um possível caminho para a especificação da taxa de crescimento do investimento residencial na presença de especulação imobiliária:
\begin{equation}
\label{gZ_Duesenberry}
g_{I_h} = g_{I_h}(\underset{0}{g}, \underset{+}{\dot p_{h_t}}, \underset{+}{\dot g_n})
\end{equation}
em que $\dot p_{h_t}$ indica inflação de imóveis e $\dot g_n$ pode ser compreendido como alterações demográficas.
Com isso, evidencia-se a autonomia do investimento residencial em relação a renda para o caso com inflação de imóveis.

%%%%%%%%%%%%%%%%%%%%%%%%%%%%%%% TAXA PRÓPRIA

Apesar da equação \ref{gZ_Duesenberry} ser uma primeira aproximação da conexão entre investimento residencial e bolha de ativos, não possui uma forma funcional de modo que se faz necessário investigar na literatura formas de especificá-la.
Uma alternativa é a extensão do supermultiplicador sraffiano por meio da taxa própria de juros do imóveis (Taxa Própria, $own$) elaborada por \textcite{teixeira_crescimento_2015}. 
Esta taxa de juros real específica é definida como a taxa de juros hipotecária ($r_{mo}$) deflacionada pela inflação dos imóveis ({$\dot p_h$}) de modo que a taxa de crescimento do investimento residencial ($g_Z$) é dada por:
$$
g_{I_h} = \phi_0 - \phi_1 \overbrace{\left(\frac{1+r_{mo}}{1+\dot p_h} - 1\right)}^{\text{Taxa Própria}}
$$
\begin{equation}
\label{tx_Propria}
g_{I_h} = \phi_0 - \phi_1\cdot own
\end{equation}
em que os $\phi_i$s são parâmetros e cujo termo em parênteses é a Taxa Própria. 
O primeiro parâmetro se refere aos determinantes de longo prazo (\textit{e.g.} arranjos institucionais do mercado imobiliários e de crédito) enquanto o segundo capta a demanda por imóveis decorrente das expectativas de ganhos de capital resultantes da especulação com o estoque de imóveis existente e diz respeito ao ciclo econômico.

Tal taxa real de juros, argumenta, é a taxa de juros relevante para os demandantes de casas uma vez que os detentores de um ativo levam seu preço em consideração no processo decisório já que sua variação pode gerar perdas/ganhos de capital \cite[p.~144]{teixeira_crescimento_2015}.
Em outras palavras, a taxa de juros das hipotecas capta o serviço da dívida para os ``investidores'' (neste caso, famílias) enquanto a variação do preço dos imóveis permite incorporar mudança no patrimonio líquido. Portanto, aufere o custo real em imóveis de se comprar imóveis \cite[p.~53]{teixeira_crescimento_2015}. 
Esta proposta, portanto, lança luz sobre a influência da inflação imobiliária na construção de novos imóveis e, de acordo com o supermultiplicador sraffiano, sobre o produto como um todo. 
Desse modo, a partir da taxa própria de juros do imóveis é possível assinalar a importância do investimento residencial para além do ciclo e estendê-la para o longo prazo.  
Sendo assim, conclui-se que esta é uma forma apropriada para especificar a taxa de crescimento do investimento residencial.

Em vista deste breve levantamento dos (poucos) trabalhos teóricos que incluem investimento residencial, conclui-se que este gasto pode ser considerado ao menos como semi-autônomo em relação à renda.
Além disso, argumenta-se que na presença de especulação tal autonomia é ampliada. 
Desse modo, na presença de bolha imobiliária e na ausência de restrição de crédito, o ciclo de imóveis tende a apresentar --- de um ponto de vista estritamente teórico --- uma dinâmica distinta da renda, ou seja, é autônomo.
Em seguida, elegeu-se a taxa própria de juros dos imóveis para especificar a taxa de crescimento do investimento residencial.
Dito isso, cabe a seção seguinte apresentar as considerações finais.


%Por fim, resta discutir a autonomia\footnote{
%	Como será melhor discutido na seção \ref{RevResidencial}, o tratamento do investimento residencial não é consensual pela literatura. Dito isso, vale destacar uma questão menos controversa, qual seja, tal gasto não cria capacidade produtiva ao setor privado. Esta conclusão, no entanto, não deve ser estendida sem as devidas mediações a construção de escritórios como bem pontua \textcite{duesenberry_investment_1958}. No entanto, tais questões fogem dos objetivos da presente investigação, cabendo apenas reforçar a não-criação de capacidade produtiva decorrente da construção de novas residências.
%} deste componente da demanda agregada e isso fica a cargo da seção seguinte.



\begin{comment}

%DESTA REVISÃO DA LITERATURA VERIFICA-SE QUE OS MODELOS OU RESTRINGEM A CONTRIBUIÇÃO DO INVESTIMENTO RESIDENCIAL A PERDA DA PARTICIPAÇÃO RELATIVA DOS SALÁRIOS OU CENTRAM A ANÁLISE EM ELEMENTOS DEMOGRÁFICOS ALÉM DE DEIXAR DE LADO A DEMANDA POR IMÓVEIS POR RAZÕES ESPECULATIVAS. COMO SERÁ MELHOR DISCUTIDO NO CAPÍTULO SEGUINTE, A RELEVÂNCIA DO INVESTIMENTO RESIDENCIAL NO EUA SE DÁ PELA CONEXÃO DESTE GASTO COM A AMPLIAÇÃO DO COLATERAL E SUBSEQUENTE ELEVAÇÃO DO ENDIVIDAMENTO/CONSUMO. UMA VEZ QUE ESTE CAPÍTULO É DE TEOR TEÓRICO, PRETENDE-SE DISCUTIR COMO INCLUIR TAL COMPONENTE DA DEMANDA AGREGADA SEM CONFINÁ-LO A UM CASO ESPECÍFICO.

%- A AUTONOMIA DESTE GASTO NÃO SE RESTRINGE AO CRESCIMENTO POPULACIONAL.

%TAXA PRÓPRIA DE JUROS



Feita esta descrição, Duesenberry pontua: (i) a flutuação dos imóveis decorrente de um desequilíbrio inicial é amortecida; (ii) a flutuação da renda causará flutuações no investimento residencial com defasagem; (iii) este setor possui demanda de reserva e, portanto, os efeitos de uma recessão nesse setor são mais dispersos ao longo do tempo que, por sua vez, contribui para que a perda de dinamismo da renda seja atenuado.

\end{comment}