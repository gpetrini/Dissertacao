\subsection{Autonomia do investimento residencial}
\label{RevResidencial}

INTRODUÇÃO

INDICAR QUE SERÁ FEITO UM ESFORÇO PARA TRATAR DO INVESTIMENTO RESIDENCIAL DE FORMA TEÓRICA APESAR DE COMUMENTE ASSOCIADO A EMPIRIA.


INVESTIMENTO RESIDENCIAL E DESENVOLVIMENTO

Por se tratar de um gasto não criador de capacidade produtiva, o debate\footnote{Para mais detalhes, ver \textcite{arku_housing_2006}.} se restringiu em categorizá-lo enquanto absorvedor de recursos produtivos \cite{solow_importance_1995} enquanto parte da literatura de crescimento indicava  a possibilidade de um sobreinvestimento residencial \cite{mills_has_1987}.

TEXTO NBER: PERDA DE IMPORTÂNCIA DESTE SETOR
ROGNLIE: REDUÇÃO DA TAXA DE CAPITAL

% Zezza
Examinada parcela dos trabalhos ortodoxos, cabe inspecionar a forma com que a heterodoxia tratou do tema. 
% Financeirização e hipotecarização
Parte significativa desta literatura  --- emergente no pós-crise imobiliária --- centra esforços na conexão deste tipo de gasto com processos mais gerais como a financeirização \cites{aalbers_financialization_2008}{bibow_financialization_2010} enquanto uma pequena fração aborda investimento residencial e crescimento.
A título de menção, vale destacar também o trabalho de \textcite{zezza_u.s._2008} em que são investigados os efeitos da diminuição da propensão marginal a poupar da economia norte-americana a partir da metodologia \textit{Stock-Flow Consistent} e conclui que o consumo financiado por crédito é o principal determinante do crescimento de modo que o investimento residencial um efeito riqueza via valorização dos imóveis apenas em que os ganhos de capital ajudam a explicar a redução na taxa de poupança apesar da distribuição da renda a favor dos lucros. 


ZEZZA, CARDACI, ETC: DISTRIBUIÇÃO E INVESTIMENTO RESIDENCIAL

HANSEN E DUESENBERRY: INVESTIMENTO RESIDENCIAL E DEMOGRAFICA

GOWANS E ALLAIN: INVESTIMENTO RESIDENCIAL E DEMOGRAFIA EM MODELOS DO TIPO SUPERMULTIPLICADOR

DESTA REVISÃO DA LITERATURA VERIFICA-SE QUE OS MODELOS OU RESTRINGEM A CONTRIBUIÇÃO DO INVESTIMENTO RESIDENCIAL A PERDA DA PARTICIPAÇÃO RELATIVA DOS SALÁRIOS OU CENTRAM A ANÁLISE EM ELEMENTOS DEMOGRÁFICOS ALÉM DE DEIXAR DE LADO A DEMANDA POR IMÓVEIS POR RAZÕES ESPECULATIVAS. COMO SERÁ MELHOR DISCUTIDO NO CAPÍTULO SEGUINTE, A RELEVÂNCIA DO INVESTIMENTO RESIDENCIAL NO EUA SE DÁ PELA CONEXÃO DESTE GASTO COM A AMPLIAÇÃO DO COLATERAL E SUBSEQUENTE ELEVAÇÃO DO ENDIVIDAMENTO/CONSUMO. UMA VEZ QUE ESTE CAPÍTULO É DE TEOR TEÓRICO, PRETENDE-SE DISCUTIR COMO INCLUIR TAL COMPONENTE DA DEMANDA AGREGADA SEM CONFINÁ-LO A UM CASO ESPECÍFICO.

- A AUTONOMIA DESTE GASTO NÃO SE RESTRINGE AO CRESCIMENTO POPULACIONAL.
- INCLUSÃO DE ELEMENTOS DA OFERTA DE TRABALHO NÃO PODEM SER DESASSOCIADOS DE UMA DISCUSSÃO SOBRE TÉCNICAS PRODUTIVAS DE MODO QUE OUTRAS QUESTÕES SURGEM

NIKALDI

RESPONDER SKOTT

TAXA PRÓPRIA DE JUROS
