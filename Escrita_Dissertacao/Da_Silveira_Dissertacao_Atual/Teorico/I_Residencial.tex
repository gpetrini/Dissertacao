\section{Investimento residencial e crescimento: rumo a endogeinização}


INTRODUÇÃO

INVESTIMENTO RESIDENCIAL E DESENVOLVIMENTO

Por se tratar de um gasto não criador de capacidade produtiva, o debate\footnote{Para mais detalhes, ver \textcite{arku_housing_2006}.} se restringiu em categorizá-lo enquanto absorvedor de recursos produtivos \cite{solow_importance_1995} enquanto parte da literatura de crescimento indicava  a possibilidade de um sobreinvestimento residencial \cite{mills_has_1987}.

ROGNLIE: REDUÇÃO DA TAXA DE CAPITAL

% Zezza
Examinada parcela dos trabalhos ortodoxos, cabe inspecionar a forma com que a heterodoxia tratou do tema. 
% Financeirização e hipotecarização
Parte significativa desta literatura  --- emergente no pós-crise imobiliária --- centra esforços na conexão deste tipo de gasto com processos mais gerais como a financeirização \cites{aalbers_financialization_2008}{bibow_financialization_2010} enquanto uma pequena fração aborda investimento residencial e crescimento.
A título de menção, vale destacar também o trabalho de \textcite{zezza_u.s._2008} em que são investigados os efeitos da diminuição da propensão marginal a poupar da economia norte-americana a partir da metodologia \textit{Stock-Flow Consistent} e conclui que o consumo financiado por crédito é o principal determinante do crescimento de modo que o investimento residencial um efeito riqueza via valorização dos imóveis apenas em que os ganhos de capital ajudam a explicar a redução na taxa de poupança apesar da distribuição da renda a favor dos lucros. 


ZEZZA, CARDACI, ETC: DISTRIBUIÇÃO E INVESTIMENTO RESIDENCIAL

HANSEN E DUESENBERRY: INVESTIMENTO RESIDENCIAL E DEMOGRAFICA

GOWANS E ALLAIN: INVESTIMENTO RESIDENCIAL E DEMOGRAFIA EM MODELOS DO TIPO SUPERMULTIPLICADOR

DESTA REVISÃO DA LITERATURA VERIFICA-SE QUE OS MODELOS OU RESTRINGEM A CONTRIBUIÇÃO DO INVESTIMENTO RESIDENCIAL A PERDA DA PARTICIPAÇÃO RELATIVA DOS SALÁRIOS OU CENTRAM A ANÁLISE EM ELEMENTOS DEMOGRÁFICOS ALÉM DE DEIXAR DE LADO A DEMANDA POR IMÓVEIS POR RAZÕES ESPECULATIVAS. COMO SERÁ MELHOR DISCUTIDO NO CAPÍTULO SEGUINTE, A RELEVÂNCIA DO INVESTIMENTO RESIDENCIAL NO EUA SE DÁ PELA CONEXÃO DESTE GASTO COM A AMPLIAÇÃO DO COLATERAL E SUBSEQUENTE ELEVAÇÃO DO ENDIVIDAMENTO/CONSUMO. UMA VEZ QUE ESTE CAPÍTULO É DE TEOR TEÓRICO, PRETENDE-SE DISCUTIR COMO INCLUIR TAL COMPONENTE DA DEMANDA AGREGADA SEM CONFINÁ-LO A UM CASO ESPECÍFICO.

- A AUTONOMIA DESTE GASTO NÃO SE RESTRINGE AO CRESCIMENTO POPULACIONAL.
- INCLUSÃO DE ELEMENTOS DA OFERTA DE TRABALHO NÃO PODEM SER DESASSOCIADOS DE UMA DISCUSSÃO SOBRE TÉCNICAS PRODUTIVAS DE MODO QUE OUTRAS QUESTÕES SURGEM

TAXA PRÓPRIA DE JUROS

Como destaca \textcite[p.~53]{teixeira_crescimento_2015}, os detentores de um ativo levam seu preço em consideração no processo decisório uma vez que sua variação pode gerar perdas/ganhos de capital. Como alternativa, elabora a taxa própria de juros do imóveis (Taxa Própria, $own$) definida como a taxa de juros hipotecária ($r_{mo}$) deflacionada pela inflação dos imóveis ({$\dot p_h$}) de modo que o investimento residencial ($g_Z$) é dado por:
$$
g_Z = \phi_0 - \phi_1 \overbrace{\left(\frac{1+r_{mo}}{1+\dot p_h} - 1\right)}^{\text{Taxa Própria}}
$$

\begin{equation}
g_Z = \phi_0 - \phi_1\cdot own
\end{equation}
em que os $\phi_i$s são parâmetros e cujo termo em parênteses é a Taxa Própria. 
O primeiro parâmetro se refere aos determinantes de longo prazo (\textit{e.g.} arranjos institucionais do mercado imobiliários e de crédito) enquanto o segundo capta a demanda por imóveis decorrente das expectativas de ganhos de capital resultantes da especulação com o estoque de imóveis existente e diz respeito ao ciclo econômico.

Em outras palavras, a taxa de juros das hipotecas capta o serviço da dívida para os ``investidores'' (neste caso, famílias) enquanto a variação do preço dos imóveis permite incorporar mudança no patrimonio líquido. Portanto, aufere de modo satisfatório o custo real em imóveis de se comprar imóveis \cite[p.~53]{teixeira_crescimento_2015}. Desse modo, a partir da taxa própria de juros do imóveis é possível revelar importância do investimento residencial para além do ciclo e estendê-la para o longo prazo.  Tal proposta, portanto, lança luz sobre a influência da inflação imobiliária na construção de novos imóveis e, de acordo com o supermultiplicador sraffiano, sobre o produto como um todo. 


TAXA PRÓPRIA, SUPERMULTIPLICADOR E MÉDIO PRAZO!

Da discussão anterior, verifica-se que a literatura sobre investimento residencial é bastante escassa nos modelos com gastos autônomos. Como será apresentado no capítulo seguinte, parte da literatura empírica (diminuta, mas crescente) destaca a importância deste componente da demanda para a dinâmica da economia norte americana. Diferentemente de grande parte dos trabalhos teóricos e empíricos, argumenta-se que é o investimento residencial que antecipa o ciclo econômico. Tal discussão é endereçada no capítulo seguinte, mas antes resta especificar qual modelo o mais adequado para o capítulo \ref{CapModelo} e isso é feito a seguir.
