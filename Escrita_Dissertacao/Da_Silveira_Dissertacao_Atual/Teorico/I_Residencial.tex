\subsection{Autonomia do investimento residencial}
\label{RevResidencial}

O objetivo desta seção é apresentar a teorias em torno do investimento residencial dando ênfase à sua autonomia (ou não) em relação ao ciclo econômico. Antes de avançar, destaca-se a tentativa de tratar este componente da demanda agregada em termos teóricos uma vez que a discussão empírica será endereçada no capítulo seguinte. 
Sendo assim, é feito um breve mapeamento tanto da ortodoxia quanto da heterodoxia a respeito do tema com especial atenção para sua autonomia, a começar pelos trabalhos ortodoxos.

Neste ponto, convém pontuar que uma grande parte dos trabalhos envolvendo investimento residencial tem deixado de tratá-lo macroeconomicamente de modo que uma parcela considerável tem se restringido a questões microeconômicas e regionais \cite{arestis_u.s._2008}.
É importante ressaltar também que, por se tratar de um gasto não criador de capacidade produtiva, o debate\footnote{Para mais detalhes, ver \textcite{arku_housing_2006}.} sobre a importância do investimento residencial para o desenvolvimento econômico se restringiu em categorizá-lo enquanto absorvedor de recursos produtivos \cite{solow_importance_1995} enquanto parte da literatura de crescimento indicava  a possibilidade de um sobreinvestimento residencial \cite{mills_has_1987}.
Além disso, parte da (diminuta) literatura sobre o tema tem apontado em direção para a perda de importância relativa do investimento residencial para a economia \cite{grebler_capital_1956}\footnote{
	Em relação a perda da relevância do investimento residencial, \textcite[p.~134-142]{grebler_capital_1956} pontuam que as modificações nas preferências dos consumidores que tenderiam a destinar uma parcela maior da renda ao consumo de bens duráveis e menos aos imóveis. No entanto, quanto atualizam a análise com os novos dados disponibilizados após a publicação deste livro, recolocam tal argumento indicando que pode ser o caso de uma reversão da tendência de queda da taxa de investimento residencial.
}\footnote{
	\textcite{rognlie_deciphering_2016}, por sua vez, encontra evidências para a maior importância do investimento residencial na economia.
	Resumidamente, parte de um modelo multisetorial para investigar a queda e retomada da participação dos lucros líquidos na renda.
	Ao avaliar as diferentes elasticidades de substituição entre capital e trabalho, não encontra evidência para a tese da acumulação defendida por \textcite{piketty_o_2014} e defende a hipótese da escassez associada tanto aos maiores custos de construção dos imóveis quanto menor disponibilidade de terra.
	Dito isso, conclui que tais elementos contribuíram para a elevação do preço dos imóveis e subsequente participação na renda.
	Por fim, resta destacar que por centrar a análise na substituição entre os fatores de produção a comparatibilidade deste estudo com aqueles desenvolvidos pela heterodoxia é comprometida e, portanto, optou-se por não analisá-lo pormenorizadamente.
}.

Apesar de não adotarem esta terminologia, alguns destes trabalhos indicam a autônomia do investimento residencial tanto por conta da durabilidade dos imóveis \cite{derksen_long_1940} quanto por fatores populacionais e demográficos \cites{hansen_economic_1939}{grebler_capital_1956}. Como consequência, ambas as constatações apontam em direção para a existência de ciclos imobiliários cuja temporalidade distinta e mais ampla que a da renda. 
Tais trabalhos, no entanto, têm dado pouca ênfase aos determinantes da demanda --- centram a análise na oferta --- por imóveis e, assim,  tratam este gasto como outro qualquer (depende dos preços relativos e da renda das famílias)\footnote{Os fatores populacionais e demográficos mencionados no corpo do texto dizem respeito a oferta de imóveis e não a demanda. Como será discutindo adiante, parte dos modelos heterodoxos consideram este elemento como determinante da demanda por imóveis.}.

Desta breve revisão da literatura ortodoxa sobre o investimento residencial, pontua-se o destaque dado tanto ao crescimento populacional quanto às mudanças demográficas na análise dos determinantes da oferta de novos imóveis.
Dentre estes (poucos) autores, apenas \textcite{grebler_capital_1956} pontuam a importância das condições de financiamento\footnote{
	Adicionalmente, é importante ressaltar que tal estudo foi realizado em meio a mudanças institucionais significativas em curso na economia norte-americana que, por sua vez, são reconhecidas pelos autores:
	\begin{citacao}
		\textit{All these observations have ignored what is potentially perhaps the most important variable: the future role of government. The concern
			of government with housing and its financing, though it originated in
			depression emergencies and was intensffied by war and postwar dislocations, is likely to be a lasting and probably increasing influence on
			residential construction}.
		\cite[p.~27]{grebler_capital_1956}
	\end{citacao}
	Tal discussão será retomada no capítulo seguinte em que serão realçados alguns fatos estilizados.
} para a demanda por imóveis\footnote{Além dos condicionantes financeiros, os autores também chamam a atenção tanto para as mudanças nos balanços patrimoniais dos investidores institucionais quanto para o aumento da dívida hipotecária associada a maior taxa de propriedade imobiliária.
}:
	\begin{citacao}
		\textit{Because the demand for housing is particularly sensitive to credit
		terms and depends heavily on debt financing, a general long-term increase
		in the cost of borrowing would tend to affect the volume of
		residential construction more than the volume of business investment,
		a large portion of which is financed from internal funds}
		\cite[p.~322--3]{grebler_capital_1956}
	\end{citacao}

MIAN E SUFI

Examinada parcela dos trabalhos ortodoxos, cabe inspecionar a forma com que a heterodoxia tratou do tema. 
% Financeirização e hipotecarização
Parte significativa desta literatura  --- emergente no pós-crise imobiliária --- centra esforços na conexão deste tipo de gasto com processos mais gerais como a financeirização \cites{aalbers_financialization_2008}{bibow_financialization_2010} enquanto uma pequena fração aborda investimento residencial e crescimento.
Uma exceção é o trabalho de \textcite{zezza_u.s._2008} em que são investigados os efeitos da diminuição da propensão marginal a poupar da economia norte-americana a partir da metodologia \textit{Stock-Flow Consistent}\footnote{
	Outro modelo na linhagem SFC que incorpora investimento residencial é o de NIKOLAIDI.
	No entanto, ao partir do procedimento de \textcite{godley_money_1999} para determinação do portfólio de ativos dos agentes, trata os imóveis como um ativo financeiro qualquer sem considerar, por sua vez, a durabilidade deste ativo.
}.

ZEZZA

 e conclui que o consumo financiado por crédito é o principal determinante do crescimento de modo que o investimento residencial um efeito riqueza via valorização dos imóveis apenas em que os ganhos de capital ajudam a explicar a redução na taxa de poupança apesar da distribuição da renda a favor dos lucros. 

Outra vertente heterodoxa, tem lançado mão de modelos baseados em agente para avaliar as relações entre instabilidade financeira, endividamento das famílias e distribuição de renda.
Em linha com 
CYNAMON E FAZARRI e \textcite{erlingsson_integrating_2013}, \textcite{cardaci_inequality_2018} parte da hipótese de consumo cascata desenvolvida por \textcite{frank_expenditure_2014}\footnote{
	De acordo com esta hipótese do efeito cascata, as famílias aprendem os padrões de consumo através dos grupos de referência a quem comparam seus padrões de vida.
} para conectar a piora da distribuição pessoal da renda ao aumento do preço dos imóveis\footnote{Dentre as contribuições, vale pontuar a endogeinização de um mercado (e rede) de crédito com consumo financiado por empréstimos colateralizados por hipotecas --- como em \textcite{mian_house_2011} --- como uma fonte alternativa de financiamento que compensa a estagnação salarial.}.
Apesar de relevante, tal contribuição não avança em direção a uma qualificação da autonomia do investimento residencial e, portanto, deve-se prosseguir na busca de alternativas na heterodoxia.

NIKALDI

Seguindo o modelo do supermultiplicador sraffiano, \textcite{gowans_introducing_2014} argumenta que o investimento residencial é autônomo e esta autonomia decorre --- como em \textcite{hansen_economic_1939} --- do crescimento populacional que, por sua vez, cresce a uma taxa exógena\footnote{
	Para uma discussão do primeiro problema de \textcite{harrod_essay_1939} em um modelo do tipo supermultiplicador, ver \textcite{allain_demographic_2018}.
}.
Em seguida estabelece --- também seguindo \textcite{hansen_economic_1939} --- os principais canais pelo qual o crescimento populacional é repassado para a demanda efetiva, são eles: (i) Formação de capital residencial; Gastos do governo e; Produção de bens essenciais.
Dessa discussão, o autor conclui que --- diferentemente de \textcite{robinson_model_1962} --- crescimento populacional e econômico não são fenômenos separados e devem ser tratados conjuntamente.
No entanto, se o crescimento populacional é o principal determinante do investimento residencial, é esperado que sua relevância decaia com a queda da taxa de crescimento vegetativo que não é o caso quando são consideradas mudanças demográficas nas unidades familiares:

\begin{citacao}
	\textit{The manner in which the population arranges itself into households
	occupying separate dwelling units has been subject to marked changes,
	which are associated with trends in longevity and other demographic
	factors, changes in taste and preferences, and the rise in per capita real
	income. Under the influence of these factors the social units occupying
	or seeking separate dwelling units have become more and more fragmentized.}
	\cite[p.~11]{grebler_capital_1956}
\end{citacao}

A autonomia do investimento residencial, por sua vez, não se limita a tais determinantes como pontua \textcite{fiebiger_semi-autonomous_2018} uma vez que é financiado principalmente por crédito.

FIEBIGER

- DESTACAR REDUÇÃO DA TAXA DE CRESCIMENTO POPULACIONAL E AUMENTO DO INSUMO NA PRODUÇÃO DE NOVOS IMÓVEIS


RESPONDER SKOTT


DUESENBERRY: INVESTIMENTO RESIDENCIAL E DEMOGRAFICA


DESTA REVISÃO DA LITERATURA VERIFICA-SE QUE OS MODELOS OU RESTRINGEM A CONTRIBUIÇÃO DO INVESTIMENTO RESIDENCIAL A PERDA DA PARTICIPAÇÃO RELATIVA DOS SALÁRIOS OU CENTRAM A ANÁLISE EM ELEMENTOS DEMOGRÁFICOS ALÉM DE DEIXAR DE LADO A DEMANDA POR IMÓVEIS POR RAZÕES ESPECULATIVAS. COMO SERÁ MELHOR DISCUTIDO NO CAPÍTULO SEGUINTE, A RELEVÂNCIA DO INVESTIMENTO RESIDENCIAL NO EUA SE DÁ PELA CONEXÃO DESTE GASTO COM A AMPLIAÇÃO DO COLATERAL E SUBSEQUENTE ELEVAÇÃO DO ENDIVIDAMENTO/CONSUMO. UMA VEZ QUE ESTE CAPÍTULO É DE TEOR TEÓRICO, PRETENDE-SE DISCUTIR COMO INCLUIR TAL COMPONENTE DA DEMANDA AGREGADA SEM CONFINÁ-LO A UM CASO ESPECÍFICO.

- A AUTONOMIA DESTE GASTO NÃO SE RESTRINGE AO CRESCIMENTO POPULACIONAL.
- INCLUSÃO DE ELEMENTOS DA OFERTA DE TRABALHO NÃO PODEM SER DESASSOCIADOS DE UMA DISCUSSÃO SOBRE TÉCNICAS PRODUTIVAS DE MODO QUE OUTRAS QUESTÕES SURGEM

TAXA PRÓPRIA DE JUROS
