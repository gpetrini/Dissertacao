
\section{Convergência ao grau de utilização normal: dois paradigmas e os dois paradoxos}\label{debate}

Esta seção pretende destacar a discussão entre Kaleckianos e Sraffianos sobre a convergência do grau de utilização da capacidade ao nível normal\footnote{Por grau de utilização normal, adota-se a definição de \textcites[p.~423--4, Original de 1986]{ciccone_2017}: ``\textit{The `normal' utilization of capacity can therefore imply not only the expectation of a certain breadth and frequency of the fluctuations in demand, but also the expectation of the idleness of the excess capacity deliberately chosen by the entrepreneurs;}'' } assim como as implicações para os paradoxos dos custos e da parcimônia. 
Argumenta-se que o grau de utilização não é somente uma variável de fechamento mas também fundamental para a validação desses paradoxos em termos de crescimento. 

A longevidade da discussão sobre a convergência do grau de utilização pode ser vista no modelo de \textcite{vianello_pace_1985} cujo argumento é que sobre/sub-utilização não são persistentes e, portanto, o grau de utilização converge ao normal no \textit{steady-state}. \textcite{amadeo_role_1986} contra-argumenta que, mesmo no longo-prazo, o grau de utilização pode ser diferente do normal uma vez que: (i) as firmas tendem a evitar guerras de preços em uma economia oligopolista e (ii) o grau de utilização é o principal determinante do investimento agregado. Além disso, afirma que não há um mecanismo endógeno que garanta que o grau de utilização irá retornar ao normal\footnote{No entanto, faz uma concessão: se os empresários revisarem suas expectativas, o grau de utilização pode ser igual ao normal. De todo modo, o argumento é que mesmo convergindo ao nível planejado, o grau de utilização continua sendo a variável endógena do modelo.}. Ao analisar o modelo de \textcites[Original de 1986]{ciccone_2017}, afirma:

\begin{citacao}
From our perspective, there is no argument in the analysis to support the idea that the system will acommodate to changes in accumulation independently from changes in distribuition. The system may indeed accomodate leaving distribution unaffected, but the uncertainty of the results leads one to believe that it will not. \cite[p. 160]{amadeo_role_1986}
\end{citacao}
Dessa forma, fica evidente a importância da endogenização do grau de utilização para os modelos Kaleckianos\footnote{\textcite[p.~30]{rowthorn_demand_1981}, por exemplo, destaca a importância dessa endogenização para contrastar com os modelos de Cambridge.}. No limite, é o que garante que a distribuição de renda possa ser exógena sem incorrer na instabilidade de Harrod, ou melhor, para que não precise ser endogeinizada. Nesses termos, a caracterização dos regimes de crescimento (\textit{wage/profit-led}) perpassa pela endogeinização do grau de utilização que, ao não convergir ao normal, acomoda  \textbf{persistentemente} mudanças na distribuição funcional da renda.  O centro do argumento de \textcite[p.~155--160]{amadeo_role_1986} é que se o grau de utilização for exógeno, mudanças na taxa de acumulação recairão necessariamente sobre a distribuição de renda. Além disso, a proposta defendida pelo autor é de que o grau de utilização normal converge, endogenamente, ao efetivo:

\begin{citacao}
Indeed, one may argue that if the equilibrium degree is systematically different from
the planned degree of utilization, entrepreneurs will eventually revise their plans,
thus \textbf{altering the planned degree}. If, for instance, the equilibrium degree of utilization is smaller than the planned degree ($u - u_n$), it is possible that entrepreneurs will
reduce $u_n$. \cite[p. 155, grifos adicionados e variáveis adaptadas]{amadeo_role_1986}
\end{citacao}

Como enfatizam \textcite{hein_harrodian_2012}, a endogeinização do grau de utilização normal é uma das formas encontradas pela literatura Kaleckiana para lidar com a instabilidade de Harrod\footnote{Tais esforços foram endereçados a convergência do grau de utilização e não ao caráter autônomo do investimento. O argumento aqui defendido, diferentemente de \textcite{setterfield_long-run_2017}, é que a equidade $u = u_n$ não é condição necessária nem suficiente para que o problema da instabilidade de Harrod seja resolvido sem impor hipóteses psicológicas adicionais.}. Ao longo da exposição, verifica-se que boa parte desses modelos Kaleckianos não-Tradicionais\footnote{Por modelos Kaleckianos não-Tradicionais, entende-se como aqueles que incorporam a convergência do grau de utilização ao normal ou a existência de gastos autônomos não criadores de capacidade.} resgatam conceitos como: (i) incerteza; (ii) racionalidade limitada; (iii) impossibilidade de maximização restrita; (iv) convenções; (v) conflito de interesses e; (vi) ameaça de novos entrantes. Não está sendo questionada a plausibilidade desses elementos, mas sim, o movimento retórico\footnote{Movimento retórico aqui utilizado no sentido de \textcite{swales_aspects_2011} e, de forma mais específica, fazendo referência ao estabelecimento de um nicho de pesquisa para preservar determinada tradição teórica que neste caso diz respeito a manutenção da endogeneidade do grau de utilização nos modelos Kaleckianos.} em que tais conceitos deixaram de ser somente fundamentos e passaram a ser argumentos. 

A convergência do grau de utilização ao normal, no entanto, não contradiz ou abdica tais conceitos. Como pontua \textcite{steindl_maturity_1952}, as firmas mantém deliberadamente capacidade produtiva ociosa para acomodar movimentações inesperadas na demanda efetiva. Outro exemplo é a existência de conflitos de interesses entre as parcelas da sociedade que, seguindo algumas teorias sraffianas da distribuição como a de \textcite{pivetti_essay_1992}, não se resumem à estrutura de mercado ou à barganha salarial/distribuição de lucros e dividendos\footnote{Para uma crítica aos microfundamentos que dizem respeito a formação de preço nos modelos Kaleckianos, ver \textcite{steedman_questions_1992}.}. Desse modo, usar tais conceitos como argumentos não é suficiente para negar a convergência do grau de utilização.


Outra forma de lidar com a instabilidade de Harrod é abstraí-la através de um corredor de estabilidade como em \textcites{dutt_growth_1990}{setterfield_long-run_2017}. Uma das implicações desta abordagem, argumentam, é que a instabilidade (harrodiana) deixa de ser a regra e passa a ser a exceção. No entanto, uma das características estruturais dos modelos Kaleckianos\footnote{Cabe a menção à crítica de \textcite{nikiforos_distribution_2012} em que os autores apontam a existência de não-linearidades no modelo Kaleckiano que implicam na necessidade de repensar os regimes de crescimento \textit{wage-} e \textit{profit-led}.} destacadas por \textcite{skott_theoretical_2012} é a maior  sensibilidade da poupança em relação ao investimento, implicando que tal corredor de estabilidade precisa ser exageradamente grande \cite[p.~6]{girardi_normal_2018}.

A literatura Kaleckiana também questionou a razoabilidade de considerar o grau de utilização normal como singular e constante\footnote{Ao longo deste capítulo, o grau de utilização normal é considerado constante por simplificação. Como será discutido em maiores detalhes no capítulo 
	%TODO: Concertar referência
	%\ref{CapFatos}, 
	seguinte
	o grau de utilização normal não é necessariamente constante nem igual à média do grau de utilização observado.}. Como resposta, utilizam um argumento convencionalista em que o grau de utilização efetivo é encarado como normal dada a existência da incerteza fundamental \cite{lavoie_kaleckian_1995}. Desse modo, tal como em \textcite{amadeo_role_1986}, o grau de utilização normal se ajusta endogenamente ao efetivo. Além disso, afirmam que as firmas possuem um comportamento adaptativo. Em resposta, \textcite{skott_theoretical_2012} argumenta que um comportamento adaptativo só é razoável em relação à variáveis que os referidos agentes não possuem controle, o que não é o caso para as firmas e o grau de utilização. Dito isso, o autor questiona o porquê do grau de utilização desejado se ajustar e não a taxa de acumulação:

\begin{citacao}
But why adjust the target? Revised plans can take the form of changing the rate of accumulation—the Harrodian argument—rather than the target. Adjustments in the target would only be justified if the experience of low actual utilization makes firms decide that low utilization has now become optimal, and neither Amadeo nor Lavoie presents an argument for this causal link. \cite[p.120]{skott_theoretical_2012}
\end{citacao}
Além disso, destaca que a endogeinização do grau de utilização normal não implica em equivalência com o efetivo no longo prazo. De forma complementar, \textcite{nikiforos_utilization_2016} critica esta ideia convencionalista ao frisar que a necessidade de responder efeitos inesperados na demanda agregada é, acima de tudo, um \textbf{objetivo} das empresas e não um comportamento convencional. 

Retomando os movimentos retóricos apresentados anteriormente, é possível destacar um tratamento assimétrico no que diz respeito ao comportamento maximizador das firmas \cite[p.~123]{skott_theoretical_2012}. Grosso modo, dada a racionalidade limitada dos agentes, não verifica-se um comportamento maximizador estrito, mas sim ``satisfatório''  \cite{dutt_equilibrium_2010}. No entanto, tal raciocínio não é estendido para a capacidade de minimização de custos pelas empresas na determinação da demanda por trabalho. Desse modo, se a racionalidade limitada é utilizada como argumento para impedir que o grau de utilização convirja ao normal, isso deveria implicar na impossibilidade da relação trabalho-produto estar no nível desejado. Caso contrário, tal postura além de ser assimétrica é também inconsistente.

Além da endogeinização do grau de utilização, \textcite{hein_harrodian_2012} apresentam as propostas de \textcite{dallery_conflicting_2011} em que a existência de objetivos conflitantes e (em potencial) mutualmente excludentes impedem que o grau de utilização atinja o nível desejado. Em síntese, acionistas e gerentes possuem interesses distintos, impactando as decisões sobre a taxa de lucro normal. Já a negociação salarial entre empregado e empregador determina os valores da margem de lucro. Sendo assim, os autores argumentam que um objetivo sendo alcançado, o outro necessariamente é deixado em segundo plano. A título de exemplo, ao apresentar o modelo de inflação por conflito distributivo, \textcite[p.~567]{lavoie_post-keynesian_2015} afirma:
\begin{citacao}
 The presence of bargaining power on the part of the workers allows
us to respond to a criticism of the Kaleckian model of growth and distribution: there
cannot be a true long-run steady state unless the normal profit rate and the actual profit
rate are equated. In the above model, the two rates are equated; that is, the actual profit
rate and the target rate of return assessed by firms become equal. Despite this, the rate
of utilization of capacity is still free to vary from its standard or normal value. The key
characteristic of the Kaleckian model, the endogeneity of its rate of capacity utilization,
is thus preserved.
\end{citacao}
Com isso, é possível readequar algumas características dos modelos Kaleckianos com a endogeinização do grau de utilização.

No entanto, \textcite[p.~125]{skott_theoretical_2012} chama atenção que  tal argumento permite que o grau de utilização se torne uma variável livre, mas isso não implica que será uma variável livre. 
Além disso, \textcite{dallery_conflicting_2011} centram o argumento na possibilidade desses objetivos serem conflituosos, mas não apresentam razões para que sejam \textbf{necessariamente} inconciliáveis.  

Antes de avançar no que \textcite{girardi_normal_2018} chamam de ``nova abordagem'' é necessário destacar qual a importância de preservar a endogeinização do grau de utilização no longo prazo. Dadas mudanças na distribuição funcional da renda, o grau de utilização varia e persiste em níveis distintos do normal de modo que a economia apresente características \textit{wage-} ou \textit{profit-led} no longo prazo. Tal implicação decorre de dois conceitos caros para a literatura kaleckiana. O primeiro deles é o paradoxo dos custos proposto por \textcite{rowthorn_demand_1981} em que um aumento na participação dos salários (lucros) na renda geram maiores (menores) taxas de lucro dados os efeitos positivos (negativos) sobre o grau de utilização e, portanto, sobre a taxa de crescimento:

\begin{citacao}
Anything which increases the real cost of production (including taxes and depreciation) will reduce the amount of net profit earned at \textbf{current level of capacity utilization}. However, such an increase in costs will be followed by a rise in output, and so the level of capacity will increase. Since $u < u_N$, the \textbf{economies of scale} resulting from higher capacity utilization will more than offset the effect of higher costs. As result the rate of profits will increase. Thus, under the assumed conditions, \textbf{higher costs lead to higher profits}.
\cite[p.~18, grifos adicionados e variáveis adaptadas]{rowthorn_demand_1981}
\end{citacao}
Já paradoxo da parcimônia exposto por \textcite{keynes_general_1936} trata dos efeitos negativos sobre o grau de utilização/taxa de acumulação de um aumento na propensão marginal a poupar\footnote{Vale destacar que \textcite{keynes_general_1936} não trata dessas questões em termos de crescimento e sim a literatura pós-Keynesiana. Para mais detalhes, ver \textcite[Cap. 6]{lavoie_post-keynesian_2015}.}. Mantida a endogenização do grau de utilização, estes paradoxos podem ser preservados no longo prazo uma vez que o grau de utilização acomoda mudanças na distribuição de renda.

%SUPERMULTIPLICADOR, DISTRIBUIÇÃO DE RENDA E OS PARADOXOS

No modelo do supermultiplicador sraffiano, por outro lado, mudanças na distribuição funcional da renda geram efeitos temporários sobre a taxa de crescimento que, esgotados os efeitos sobre a propensão marginal a investir, retorna a taxa dos gastos autônomos \cite[p.~ 79 VER PÁGINA]{serrano_sraffian_2017}. Isso pode ser visualizado em termos da equação \ref{crescimentosuper} que indica que a taxa de crescimento tendencial, quando $u = u_N$, independe da distribuição:
\begin{citacao}
That lower marginal propensity to save will increase the level of induced consumption and
aggregate demand, and, consequently, also the long-period level of productive capacity.
However, this will be a \textbf{once-and-for-all effect}. Once capacity has adjusted to the new (higher)
\textbf{level} of effective demand implied by the higher (super) multiplier, the economy will settle back
to steady growth grow at the unchanged rate given by the growth of autonomous expenditures.
Therefore, on the demand side, a decrease in the marginal propensity to save brought about by
the rise of the real wage will have a positive long-period \textbf{level} effect (on capacity output), but
will have no effect on the sustainable secular rate of growth of capacity.
\cite[p.~138, grifos adicionados]{serrano_sraffian_1995}
\end{citacao}
Desse modo, uma vez que o grau de utilização converge ao normal, o paradoxo dos custos deixa de ser validado no longo prazo. Já o paradoxo da poupança não afeta diretamente a taxa de crescimento, mas sim o limite superior ao qual a taxa de poupança pode se ajustar para permitir um crescimento liderado pela demanda \cite{serrano_trouble_2017}. DÚVIDA

Por mais que no modelo do supermultiplicador sraffiano o grau de utilização converge ao normal, esta postura não é consensual entre teóricos sraffianos. \textcite[p.~161]{cesaratto_neo-kaleckian_2015} destaca que \textcites{garegnani_notes_1992}{palumbo_growth_2003} argumentam que tal convergência não é representativa de uma economia capitalista. O próprio modelo de \textcite[Orignial de 1986]{ciccone_2017} apresentado anteriormente desassocia preços e taxa geral de lucro normais do grau de utilização normal. Tendo em vista este modelo, \textcite[p.~476]{moreira_demanda_2018} afirmam:

\begin{citacao}
 Enfim, o que importa é que as
quantidades produzidas estejam em consonância com a demanda efetiva, o que não
implica que a capacidade produtiva também esteja e o grau de utilização seja normal.
Entretanto, isso não nega que haja uma tendência de ajuste da capacidade à demanda.
\end{citacao}
adiante
\begin{citacao}
É importante deixar claro que, apesar do grau de utilização normal não ser
uma condição necessária para o ajuste à posição de longo prazo, ainda assim, é a
referência utilizada para o cálculo dos preços normais que deverão prevalecer no
longo prazo. Isso significa que, independente da capacidade instalada efetiva num
dado momento, para cada novo processo de instalação de capacidade, a intenção de
operá-la ao nível normal será sempre uma baliza para as decisões de investimento [...]
\cite[p.~477]{moreira_demanda_2018}
\end{citacao}

Além disso, \textcite[p.~408]{lavoie_post-keynesian_2015} destaca a rejeição de alguns autores sraffianos, dentre eles \textcite{palumbo_growth_2003}, à ênfase ao conceito de \textit{steady state} em detrimento da média ao longo dos períodos de transição. Como consequência indireta da crítica do supermultiplicador sraffiano, não raro encontra-se modelos Kaleckianos com convergência ao grau de utilização normal que destacam a manutenção dos resultados canônicos na média:

\begin{citacao}
Thus, \textbf{on average}, the rate of utilization and the growth rate of the
economy are higher than at the starting and terminal points of the traverse. Thus, what
these Sraffians are telling us is that more attention should be paid to the average values
achieved during the traverse than to the terminal points.
\cite[p.~408, grifos adicionados]{lavoie_post-keynesian_2015}
\end{citacao}
Dito isso, a tabela \ref{computilizacao} resume a discussão sobre a convergência do grau de utilização destacando quais características dos modelos Kaleckianos tradicionais são mantidas. Nesta tabela, fica explicitada a relação entre endogeinização do grau de utilização (efetivo ou normal) e a preservação dos paradoxos kaleckianos. Além disso, verifica-se que a não endogeinização está relacionada com características da teoria clássica da distribuição no longo prazo.

% Please add the following required packages to your document preamble:
% \usepackage{graphicx}
\begin{table}[htb]
\centering
\caption{Modelos Kaleckianos e a  convergência ao grau de utilização: críticas e propostas}
\label{computilizacao}
\begin{tabular}{r|ll}
\hline
\textbf{Autor} & \begin{tabular}[l]{@{}l@{}}\textbf{Convergência ao grau}\\ \textbf{de utilização normal}\end{tabular} & \textbf{Paradoxos preservados} \\\hline\hline
 \textcites{amadeo_role_1986}           &$u_N \to u$
                                        &\begin{tabular}[l]{@{}l@{}}Custos e da
                                        \\parcimônia\end{tabular}\\
 \textcites{committeri_capacity_1987}   &Não converge
                                        &\begin{tabular}[l]{@{}l@{}}Ausentes\\
                                        (Clássico no LP)\end{tabular}\\
 \textcites{lavoie_kaleckian_1995}      &\begin{tabular}[l]{@{}l@{}}$u_N \to u$\\(Convencionalista)\end{tabular}                      
                                        &\begin{tabular}[l]{@{}l@{}}Custos e da
                                        \\parcimônia\end{tabular}\\
 \textcites{serrano_sraffian_1995}      &$u \to u_N$
                                        & Ausentes \\
 \textcite{dutt_dependence_1997}        &\begin{tabular}[l]{@{}l@{}}$u_N \to u$\\(Novos entrantes)\end{tabular}
                                        &\begin{tabular}[l]{@{}l@{}}Custos e da
                                        \\parcimônia\end{tabular}\\
 \textcite{dumenil_being_1999}          &Política Monetária
                                        &\begin{tabular}[l]{@{}l@{}}Ausentes\\
                                        (Clássico no LP)\end{tabular}\\
 \textcite{shaikh_economic_2009}        &\textit{Retation ratio}
                                        &\begin{tabular}[l]{@{}l@{}}Ausentes\\
                                        (Clássico no LP)\end{tabular}\\
 \textcite{dallery_conflicting_2011}    &\begin{tabular}[l]{@{}l@{}}Não converge\\
                                        (Conflitos)\end{tabular}
                                        &\begin{tabular}[l]{@{}l@{}}Custos e da
                                        \\parcimônia\end{tabular}\\
                                        
 \textcite{schoder_endogenous_2012}    &\begin{tabular}[l]{@{}l@{}}Não converge\\
                                        (Relação capital-produto)\end{tabular}
                                        &\begin{tabular}[l]{@{}l@{}}Parcimônia\end{tabular}\\
 \textcite{nikiforos_utilization_2016}  &\begin{tabular}[l]{@{}l@{}}$u_N \to u$\\
                                        (Microfundamentos)\end{tabular}   
                                        &\begin{tabular}[l]{@{}l@{}}Custos e da
                                        \\parcimônia\end{tabular}\\
 \textcite{setterfield_long-run_2017}   &\begin{tabular}[l]{@{}l@{}}Não converge\\
                                        (Corredor de estabilidade)\end{tabular}
                                        &\begin{tabular}[l]{@{}l@{}}Custos e da
                                        \\parcimônia\end{tabular}\\\hline\hline
\end{tabular}
\caption*{\textbf{Fonte:} Elaboração própria}
\end{table}

Retornando às abordagens Kaleckianas, \textcite{girardi_normal_2018} afirmam que mesmo não existindo um único nível de grau de utilização normal, a presença de gastos autônomos que não criam capacidade produtiva implicam na instabilidade harrodiana\footnote{Este tema será abordado em maiores detalhes na seção seguinte.}. Diante destas lacunas, a proposta de \textcite{nikiforos_utilization_2016}\footnote{\textcite{girardi_normal_2018} denominam como ``Nova Abordagem''} se destaca. A principal característica dessa nova abordagem é a reformulação de fundamentos microeconômicos em que impõe-se rendimentos crescentes (que crescem à taxas decrescentes). 

Grosso modo, essa alteração do modelo Kaleckiano tradicional faz com que o grau de utilização normal se torne endógeno e positivamente relacionado com a demanda das empresas ($Q$). O argumento decorre da determinação do grau de utilização desejado pelo princípio de minimização dos custos tal como em \textcite{kurz_normal_1986}. Na presença de economias de escala, o aumento na demanda é atendida por um maior grau de utilização e não por uma expansão das firmas a um mesmo grau de utilização \cite[p.~442]{nikiforos_utilization_2016}:

\begin{equation}
\label{utilNiki}
    \dot u_n = \mu (u - u_N)
\end{equation}

\begin{equation}
\label{NIKI}
\dot Q = \xi (g^* - g_0)
\end{equation}
em que $\mu$ e $\xi$ são parâmetros positivos, $g^*$ é a taxa de acumulação corrente e $g_0$ é a esperada. Nesses termos, uma taxa de acumulação maior do que a esperada faz com que a demanda (individual) da firma aumente implicando em um maior grau de utilização desejado que, na presença de rendimentos crescentes, não está associada a uma escolha técnica mais dispendiosa:

\begin{citacao}
We concluded that via this mechanism the firm will tend to increase the utilization of its resources along with its product as the economy grows and the firm faces increasing demand. At
a macro level this can be ``translated'' into an endogenous adjustment of the desired level of utilization towards its actual rate. \textcite[p.~459]{nikiforos_utilization_2016}
\end{citacao}

Por mais que no modelo de \textcite{nikiforos_utilization_2016} o grau de utilização normal se ajusta endogenamente ao efetivo, o argumento se difere ao de \textcites{amadeo_role_1986}:
\begin{citacao}
However, the logic towards equation (\ref{utilNiki}) is different compared to the rationale that has been provided so far in the literature. We begin at the micro level, from a firm that explicitly sets its desired utilization rate based on a \textbf{cost-minimizing decision} process and then we provide a link of this micro behavior to the adjustment of utilization at the macro level. \cite[p.~456, grifos adicionados e numeração adaptada]{nikiforos_utilization_2016}
\end{citacao}
Adiante, o autor prossegue com uma justificativa empírica do porquê da endogeinização do grau de utilização normal que será analisada mais detidamente no capítulo \ref{CapFatos}. Por ora, resta apresenta as críticas teóricas ao modelo de \textcites{nikiforos_utilization_2016} proposta por \textcites{girardi_normal_2018}. Acertadamente, os autores reconhecem que na presença de rendimentos crescentes, a dinâmica econômica faria com que emergisse uma estrutura de mercado oligopolizada. Dito isso, as críticas são feitas em três frentes\footnote{Adicionalmente, \textcite{girardi_normal_2018} pontuam que a concepção de capital apresentada por \textcites{nikiforos_utilization_2016} não escapa das críticas da controvérsia de Cambridge.}: (i) Dada a estrutura de mercado emergente, não há razões para se supor que as empresas serão tomadoras de preço; (ii) Não há razões para se supor que \textbf{todas} as firmas apresentem rendimentos crescentes e; (iii) A passagem das hipóteses microeconômicas ao nível macro é um salto lógico. 

O salto lógico mencionado decorre das implicações da agregação do nível microeconômico. \textcites{girardi_normal_2018} argumentam que tal mecanismo não implicaria na endogeinização do grau de utilização normal, mas sim que taxas de crescimento positivas fariam com que o grau de utilização ampliasse indefinitivamente até atingir um teto. Além disso, afirmam que não existem justificativas econômicas para se supor que a relação \ref{utilNiki} seja razoável: ``\textit{This assumption does not seem to have any compelling economic justification, other than the desire to be able to derive a macroeconomic adjustment process that can `save' the standard Neo-Kaleckian model from instability problems}'' \cite[p.~15]{girardi_normal_2018}.


Outra implicação importante da equação \ref{NIKI} é que a economia estando na trajetória estável, ou seja, ($g^* = g_0$), uma taxa de crescimento elevada fará com que o crescimento do nível da firma seja nulo. Sendo assim, mesmo que exista uma fonte autônoma de crescimento da economia ($Z > 0$), as firmas não irão reagir a esse estímulo.  Dessa forma, haverá equilíbrio entre demanda e capacidade produtiva se, e apenas se, a taxa de crescimento dos gastos autônomos coincidir com $g_0$. Portanto, mesmo que a equação \ref{NIKI} seja razoável, tal modelo é incapaz de incorporar de forma convincente os gastos autônomos que não criam capacidade. A forma com que os modelos Kaleckianos incorporam tais gastos é endereçada na seção seguinte\footnote{Como discutido na seção \ref{Literatura}, uma das características do modelo do supermultiplicador sraffiano é a inclusão dos mencionados gastos autônomos ($Z>0$). No entanto, \textcite{nikiforos_comments_2018} argumenta que podem ser incluídos nos modelos Kaleckianos convencionais.}.