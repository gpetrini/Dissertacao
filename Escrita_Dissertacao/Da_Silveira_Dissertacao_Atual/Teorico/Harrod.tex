\section{Fechamentos dos modelos heterodoxos de crescimento}
\subsection{Da instabilidade de Harrod à estabilidade fundamental}
\label{SecHarrod}
%TODO Alterar título após modificações
%=============== Inicio: Ligação Harrod ============

As origens da moderna teoria do crescimento econômico devem, em grande parte, às contribuições de \textcite{harrod_essay_1939} em que extrapola o princípio da demanda efetiva formulado por \textcite{keynes_general_1936} para uma economia em crescimento. Tal modelo impôs importantes questões: existe estabilidade do crescimento no longo pra\-zo? É possível equacionar o crescimento da demanda com o crescimento da capacidade produtiva? Se sim, qual variável acomoda essa adequação? A capacidade produtiva se ajusta à demanda ou o inverso? Os modelos de Cambridge, Oxford e supermultiplicador sraffiano responderam essas provocações de formas distintas e serão analisados ao longo desta seção.

Para evitar redundâncias, são apresentadas as hipóteses que permeiam as famílias de modelos aqui avaliadas. 
A presente exposição prioriza a parcimônia e, portanto, trata-se de uma economia sem relações externas, governo, depreciação e progresso tecnológico cuja produção possui retornos constantes de escala. Além do PDE, o que torna os modelos em questão consistentes é o abandono da substitutibilidade entre capital e trabalho e, portanto, adota-se uma função de produção Leontief em que existem dois possíveis produtos potenciais: plena capacidade ($Y_K$) e pleno emprego ($Y_L$) de modo que o produto potencial ($Y_{FC}$) é determinado por:

\begin{equation}
    Y_{FC} = \min (Y_K, Y_L)
\end{equation}
Seguindo a literatura, em que o estoque capital ($K$) é o fator escasso, o produto potencial é dado pela plena utilização do capital
\begin{equation}
\label{Oferta}
    Y_{FC} = Y_K = \frac{1}{v}K_{t-1}
\end{equation}
em que $v$ é a relação técnica capital-produto potencial. 


Considerando as hipóteses anteriores, a determinação do produto pelos componentes da demanda é obtida pela soma do consumo e investimento. Como será visto adiante, a distinção entre os modelos recairá sobre a autonomia (completa ou nula) do investimento das firmas e a existência de gastos autônomos não criadores de capacidade produtiva (denotados por $Z$). De modo a expor o problema deixado por \textcite{harrod_essay_1939}, supõe-se que o consumo é completamente induzido e que não existem gastos ``improdutivos'' ($Z=0$). Assim, o produto determinado pela demanda é dado pelo multiplicador:

\begin{equation}
\label{Demanda}
Y = \frac{I}{1-\omega}
\end{equation}
em que $\omega$ é o \textit{wage-share} e é idêntico à propensão marginal a consumir da economia\footnote{
	Para que a propensão marginal a consumir da economia seja idêntica ao \textit{wage-share} é preciso que a propensão marginal a consumir a partir dos salários seja igual ($c_w = 1$) a unidade enquanto a propensão marginal a consumir a partir dos lucros ($P$) seja nula ($c_k = 0$):
	
	$$
	c = c_w\cdot \frac{W}{Y} + c_k\cdot \frac{P}{Y}
	$$
	$$
	c = 1\cdot \omega + 0\cdot (1-\omega)
	$$
	$$
	\therefore c = \omega
	$$
	
}.
Outra hipótese compartilhada entre todos os modelos aqui avaliados é que os trabalhadores não poupam de modo que a propensão marginal a poupar da economia ($s$) é idêntica à propensão marginal a poupar dos capitalistas ($s_k$). 
Sendo assim, a poupança ($S$) é feita pelos capitalistas a partir dos lucros totais ($FT$):
$$
S = s\cdot FT
$$
Uma vez que todo o consumo é induzido, a taxa de crescimento da economia ($g$) é determinada pela taxa de investimento ($I$):
% TODO: Grifar

\begin{equation}
\label{crescimento_efetivo}
	g = \frac{\Delta I}{I_{t-1}}
\end{equation}
O princípio do acelerador --- neste caso, acelerador rígido e com ajuste completo ---, por sua vez, estabelece que a determinação do investimento decorre das alterações na demanda (efetiva) por meio do princípio de ajuste do estoque de capital:
$$
K = v\cdot Y
$$
\begin{equation}
\Delta K = I = v\Delta Y
\end{equation}


%Tomando o modelo mais genérico, em que consumo é parcialmente induzido ($C$), o investimento criador de capacidade produtiva ao setor privado ($I$) possui uma parcela autônoma ($\overline I$) e outra induzida e os gasto autônomos são não nulos ($Z > 0$), obtém-se a determinação do produto pelos componentes da demanda:
%\begin{equation}
%\label{Demanda}
%    Y = C + I + Z
%\end{equation}
A questão que permeia os modelos analisados são as condições para que exista um crescimento equilibrado da demanda (Eq. \ref{Demanda}) e da capacidade produtiva (Eq. \ref{Oferta}). 
\textcite{harrod_essay_1939} argumenta que a junção do acelerador com o multiplicador permite tratar o Princípio da Demanda Efetiva de forma dinâmica e que esta é a essência do modelo de Harrod cuja equação fundamental pode ser deduzida da identidade entre poupança ($S$) e investimento:

$$
s\cdot Y = S \equiv I
$$
Neste ponto, fica evidente que neste modelo a propensão marginal a poupar ($s$) é igual a propensão média à poupar ($S/Y$) na ausência dos gastos autônomos não criadores de capacidade\footnote{As implicações desta igualdade serão analisadas mais detidamente ao tratar do supermultiplicador sraffiano.}. Em seguida, basta normalizar esta identidade pelo estoque de capital,
$$
\frac{I}{K} = s\frac{Y}{K}
$$
$$
\frac{I}{K} = s\frac{Y}{v\cdot Y_K}
$$
\begin{equation}
\label{tx_gk}
    g_K = \frac{s}{v}u
\end{equation}
em que $g_K$ é a taxa de acumulação e $u$ é o grau de utilização da capacidade definido por:
$$
u = \frac{Y}{Y_{FC}}
$$
e sua taxa de variação pode ser dada por\footnote{Isso pode ser indicado a partir da equação que define o grau de utilização:
	$$
	u = \frac{Y}{Y_{FC}}
	$$
	calculando o diferencial total, obtém-se:
	$$
	\Delta u = \frac{\Delta Y}{ Y_{FC}} - \frac{Y\cdot \Delta Y_{FC}}{Y_{FC}^2}
	$$
	dividindo por $u_{t-1}$ de modo a obter a taxa de crescimento do grau de utilização:
	$$
	\frac{\Delta u}{u_{t-1}} = \frac{\Delta Y}{Y_{t-1}} - \frac{\Delta Y_{FC}}{Y_{FC_{t-1}}}
	$$
	Como indicado no texto, quando a demanda e capacidade produtiva crescerem a taxas distintas ($g \neq g_{Y_{FC}}$), o grau de utilização irá necessariamente variar ($g_u \neq 0$).
}
%TODO Rever alteração para tempo discreto
$$
\frac{\Delta u}{u_{t-1}} = \frac{\Delta Y}{Y_{t-1}} - \frac{\Delta Y_{FC}}{Y_{FC_{t-1}}}
$$
e evidencia que para que o grau de utilização se estabilize, é preciso que, no \textit{steady state}, produto e capacidade produtiva cresçam a uma mesma taxa. 
No entanto, não basta que o grau de utilização se estabilize para obter a equação fundamental de Harrod, mas também que o grau de utilização mantenha-se no nível normal ($u_N$)\footnote{
	Ao longo de sua exposição, \textcite{harrod_essay_1939} considera implicitamente que o grau de utilização normal é igual a unidade de modo que sua equação fundamental é expressa nos seguintes termos:
	$$
	u_N = 1
	$$
	
	$$
	g_w = \frac{s}{v}
	$$
}:

\begin{equation}
    \label{Fundamental}
    g_w = \frac{s}{v}u_N
\end{equation}
em que $g_w$ é a taxa de crescimento que garante que a demanda e capacidade produtiva cresçam dinamicamente equilibradas. Além disso, pelo grau de utilização estar em seu nível desejado ($u_N$), esta taxa corresponde àquela que os empresários estariam satisfeitos e não haveria razões para alterar seu comportamento e/ou planos de investimento. Em outras palavras, a taxa garantida expressa a taxa de equilíbrio entre oferta e demanda ($g_K$ e $g$, respectivamente). 


Neste modelo, quando a taxa efetiva é maior (menor) que a taxa garantida, o grau de utilização da capacidade é maior (menor) que o normal e isso pode ser apresentado partindo da equação \ref{tx_gk}:
$$
g_w = \frac{s}{v}\cdot u_N
$$
$$
\frac{s}{v} = \frac{g_w}{u_N}
$$
$$
g = \frac{g_w}{u_N}\cdot u
$$
$$
\frac{g}{g_w} = \frac{u}{u_N}
$$
de modo que
$$
g > g_w \Rightarrow u > u_N
$$
Sendo este o caso, as firmas buscam ampliar a capacidade produtiva com o objetivo que o grau efetivo de utilização da capacidade convirja ao normal. O aumento da taxa de crescimento do investimento tem um impacto imediato na taxa de crescimento da
economia e, apenas de depois de alguma defasagem, na taxa de crescimento do estoque de capital. O resultado, portanto, é um grau de utilização da capacidade ainda mais distante do planejado. O problema é justamente que o mecanismo de ajuste do modelo leva a economia cada vez mais distante da sua posição de \textit{steady-state}. A esse processo \textcite{harrod_essay_1939} denomina de instabilidade fundamental. Em outras palavras, quando a taxa de crescimento efetiva difere  da garantida, o grau de utilização efetivo é diferente do normal ($u\neq u_N$) e esta diferença se acentua ao longo tempo.



%a taxa de crescimento efetiva se afasta da taxa desejada em função da reação do investimento à variações no nível de atividade. Seguindo o princípio acelerador nos moldes de \textcite{harrod_essay_1939}, a resposta a uma sobreutilização da capacidade ($u>1$) é o aumento da taxa de acumulação que, pelo efeito multiplicador gera  demanda, reforçando o mecanismo de descolamento, para então ampliar a capacidade produtiva \cite[p.~12]{serrano_trouble_2017}.  No entanto, uma vez que essas são diferentes, não há um mecanismo de convergência entre elas.


% =============== Instabilidade fundamental Serrano et all ==========

Tendo em vista que neste modelo o comportamento do investimento é o principal determinante da trajetória, \textcite[p.~26--28]{harrod_essay_1939} procura reduzir tais efeitos incluindo frações do investimento que não estão diretamente relacionadas com a renda corrente. Tal constatação introduz a possibilidade de que exista um componente autônomo do investimento que não é afetado pelo mecanismo de ajuste do estoque de capital no longo prazo e, portanto, permite que a instabilidade harrodiana seja amenizada:

\begin{citacao}
Now, it is probably the case that in any period not the whole of the new capital is destined to look after the increment of output of consumers' goods. There may be  long-range plans of capital development or a transformation  of the method of  producing  the pre-existent level of output. \cite[p.~17]{harrod_essay_1939}
\end{citacao}
adiante
\begin{citacao}
The force  of this  argument [Princípio da instabilidade], however, is somewhat \textbf{weakened} when long-range  capital outlay is taken into account.
\cite[p.~26, grifos adicionados]{harrod_essay_1939}
\end{citacao}
Tal possibilidade, como será discutido adiante, sugere que a instabilidade harrodiana pode ser eliminada por meio da autonomia --- com implicações para a propensão marginal e média a poupar ---  de alguns gastos.
No entanto, argumenta-se (na seção \ref{SecSuper}) que tal gasto autônomo não precisar ser necessariamente o investimento criador de capacidade produtiva para que a instabilidade harrodiana seja eliminada.
Como consequência, tal instabilidade não decorre do princípio de ajuste do estoque de capital \textit{per se}, mas sim, da especificação da propensão marginal (e média) a poupar e da rigidez do acelerador.
% TODO: Grifar
%Isso implica que um modelo em que o investimento é induzido pelo princípio do ajuste de estoque de capital não é necessariamente instável.

%Uma observação importante é que apesar de \textcite[p.~23]{harrod_essay_1939} afirmar que existe uma única taxa de crescimento garantida, \textcite[p.~83]{robinson_model_1962} alerta que isso não implica que o investimento deve se adequar a propensão marginal a poupar determinada \textit{a priori}. %Argumenta que os modelos liderados pela demanda devem ser avaliados pelas respectivas formas de determinar o investimento\footnote{
%	No original: ``\textit{The Keynesian models can be classified classified according to the assumption made about about the inducement to invest}.''
%} uma vez que o Princípio da Demanda Efetiva é o denominador comum entre eles.
Dito isso, os modelos apresentados ao longo deste capítulo serão analisados a partir dos respectivos  fechamentos\footnote{Entende-se por fechamento como variável que assume valores economicamente relevantes de tal forma a tornar determinada relação (e.g. taxa de lucro) válida. Em outras palavras, trata-se da última variável que é resolvida endogenamente. Desse modo, dizer que o fechamento de um modelo é estabelecido por uma variável (digamos, $j$) implica dizer que $j$ é endógena.} uma vez que permitem uma análise comparativa mantidas as hipóteses compartilhadas entre eles. 
Adicionalmente, inclui-se a possibilidade de existência de gastos autônomos não criadores de capacidade produtiva para garantir a comparação entre os modelos analisados. Com essa hipótese adicional, a propensão média à poupar torna-se uma função da razão entre os gastos autônomos ($Z$) e o produto:

\begin{equation}
\label{Poupanca_Super}
    \frac{S}{Y} = s - \frac{Z}{Y}
\end{equation}
Seguindo a exposição de \textcite{serrano_sraffian_1995}, seja $f$ a fração entre propensão média e marginal a poupar
$$
f = \frac{\frac{S}{Y}}{s}
$$
de modo que será igual a unidade quando forem idênticas. 
Com isso, a equação fundamental de Harrod pode ser rearranjada nos seguintes termos:

\begin{equation}
 \label{EqGeral}   
g_K = g_w = f\frac{s\cdot u_N}{v}
\end{equation}
A equação acima permite comparar os modelos analisados de modo a destacar a variável que garanta, 
$$
g_K = g
$$
a começar pelo de Cambridge.
