\section{Autonomia dos gastos autônomos}\label{Literatura}

Da seção anterior, destacou-se o o supermultuplicador sraffiano (SMS) enquanto um modelo adequado para tratar dos gastos autônomos não criadores de capacidade produtiva.
Dito isso, esta seção pretende esclarecer algumas questões envolvendo a autonomia dos gastos ditos ``improdutivos'' para então analisar o investimento residencial.

\section{Modelos de crescimento e os gastos autônomos: uma revisão empírica}
\label{RevF}

O objetivo desta seção é analisar os trabalhos empíricos que analisam a relação entre os gastos autônomos não criadores de capacidade produtiva ao setor privado e crescimento econômico e, assim, complementar a discussão teórica realizada na seção \ref{SecAutonomos}. Mais uma vez, seguindo a categorização de \textcite{cesaratto_technical_2003}, os referidos gastos são:
(i) consumo financiado por crédito ou riqueza acumulada;
(ii) gastos do governo;
(iii) investimento residencial;
(iv) exportações e;
(v) gastos com P\&D\footnote{
	%TODO Nota de rodapé sobre gastos com P\&D
	Nota de rodapé sobre gastos com P\&D?
}. 
Da revisão da literatura empírica, verificou-se três principais preocupações: 
\begin{description}
	\item[(a)] testar a importância dos gastos autônomos sobre a taxa de crescimento de longo prazo; 
	\item[(b)] avaliar a relação entre taxa de investimento e produto;
	\item[(c)] investigar a dinâmica de cada um dos gastos autônomos referidos anteriormente.
\end{description}
Cada um desses pontos será analisado adiante. Por fim, vale pontuar que, dados os objetivos desta investigação, serão privilegiados os trabalhos que tenham o supermultiplicador sraffiano desenvolvido por \textcite{serrano_long_1995} e \textcite{bortis_institutions_1996} como forma de análise.

No que diz respeito ao tema (a), o trabalho de \textcite{girardi_long-run_2016} se destaca por analisar os efeitos de longo prazo dos gastos autônomos sobre o produto bem como por apresentar uma forma de se calcular o supermultiplicador para a economia norte-americana. Para tanto, estimam um VECM e obtém os resultados esperados de acordo com a teoria\footnote{Mais precisamente, tais resultados se sustentam uma vez desconsiderado o consumo financiado por crédito. Como justificativa para tal medida, \textcite[p.~13]{girardi_long-run_2016} argumentam que tal gasto está associado a algumas fases do ciclo econômico e, portanto, apresenta uma parcela consideravelmente induzida.}: (i) gastos autônomos e o produto apresentam uma tendência de longo prazo (são cointegradas); (ii) relação de causalidade parte dos gastos autônomos para o produto e (iii) relação positiva entre taxa de crescimento dos gastos autônomos e taxa de investimento. Já no artigo de \textcite{girardi_autonomous_2018}, o mesmo é feito para alguns países da zona do euro com a diferença que foram utilizadas variáveis instrumentais como \textit{proxy} de alguns gastos autônomos e foram obtidos resultados semelhantes ao do estudo anterior. Por fim, o trabalho de \textcite{goes_supermultiplier_2018} possui semelhanças com o de \textcite{girardi_long-run_2016}, mas se distingue por extendê-lo para mais países e por adotar critérios para agrupá-los bem como por reportar a convergência do grau de utilização ao nível normal.

Os trabalhos empíricos envolvendo o supermultiplicador, no entanto, não estão restringidos aos EUA ou países da OCDE. \textcite{freitas_pattern_2013}, por exemplo, analisam o caso brasileiro para os anos de 1970 a 2005 e concluem que diferentes gastos autônomos (em ordem, gastos do governo e consumo financiado por crédito) lideraram o crescimento em momentos distintos. Paralelamente, \textcite{braga_investment_2018} investiga o efeito acelerador para o caso brasileiro de 1996 a 2017 e conclui que o investimento criador de capacidade produtiva é causado (no sentido de Granger) pelo produto, ou seja, é induzido.  

Enquanto \textcite{freitas_pattern_2013} e \textcite{girardi_autonomous_2015} abordam a importância dos gastos autônomos para o crescimento, \textcite{braga_investment_2018} avalia a relação entre taxa de investimento e crescimento. Assim, estão abarcadas as preocupações (a) e (b) elencadas anteriormente. Resta, portanto, evidenciar os trabalhos que destacam a importância de alguns gastos autônomos em específico. Um deles é o de \textcite{medici_cointegration_2011} em que avalia o caso argentino para os anos de 1980 a 2007 e encontra evidências de cointegração entre renda, consumo do governo e o consumo privado autônomo (\textit{i.e.} não assalariado) em que os últimos granger-causam o primeiro. O modelo apresentado por \textcite{deleidi_mission-oriented_2019}, por sua vez, também analisa a importância dos gastos do governo investigando se o tipo de política fiscal adotada tem impactos sobre o crescimento. Em linhas gerais, os autores concluem que gastos orientados em setores mais intensivos em P\&D e em mudanças estrutuais (correspondente ao gasto v) possuem efeitos maiores do que uma política centrada apenas em incentivos fiscais. Já no que diz respeito às exportações (gasto iv), destaca-se a literatura de restrição por balanço de pagamentos seguindo a lei de \textcite{mccombie_balance--payments_1994} 
%TODO Orig year Thirwall
em que as exportações são os determinantes do crescimento de longo prazo \cites{atesoglu_balance--payments-constrained_1993}{mccombie_empirics_1997}{moreno-brid_mexicos_1999}{bertola_balance--payments-constrained_2002}\footnote{Por mais que tal abordagem não lance mão explicitamente do modelo do supermultiplicador sraffiano, as conclusões são compatíveis uma vez que estão presentes gastos autônomos não criadores de capacidade e a especificação da função investimento pode seguir o princípio do ajuste do estoque de capital.}.

%======================== Investimento residencial: Arestis e gasto induzido

Por fim, no que tange o investimento residencial, verifica-se uma lacuna na literatura empírica heterodoxa de crescimento liderado pela demanda. Vale retomar a compatibilidade deste componente da demanda com o modelo do supermultiplicador uma vez que (i) não cria   capacidade produtiva ao setor privado e (ii) pelas hipotecas serem a principal forma de financiamento (e não salários) de acordo com o \textit{Survey of Construction} \cite{us_census_bureau_characteristics_2017}. Dito isso, caberá a seção seguinte examinar as formas que a literatura econométrica encontrou para incorporar o investimento residencial para então eleger uma alternativa compatível com o supermultiplicador sraffiano.
\subsection{Autonomia do investimento residencial}
\label{RevResidencial}

INTRODUÇÃO

INDICAR QUE SERÁ FEITO UM ESFORÇO PARA TRATAR DO INVESTIMENTO RESIDENCIAL DE FORMA TEÓRICA APESAR DE COMUMENTE ASSOCIADO A EMPIRIA.


INVESTIMENTO RESIDENCIAL E DESENVOLVIMENTO

Por se tratar de um gasto não criador de capacidade produtiva, o debate\footnote{Para mais detalhes, ver \textcite{arku_housing_2006}.} se restringiu em categorizá-lo enquanto absorvedor de recursos produtivos \cite{solow_importance_1995} enquanto parte da literatura de crescimento indicava  a possibilidade de um sobreinvestimento residencial \cite{mills_has_1987}.

TEXTO NBER: PERDA DE IMPORTÂNCIA DESTE SETOR
ROGNLIE: REDUÇÃO DA TAXA DE CAPITAL

% Zezza
Examinada parcela dos trabalhos ortodoxos, cabe inspecionar a forma com que a heterodoxia tratou do tema. 
% Financeirização e hipotecarização
Parte significativa desta literatura  --- emergente no pós-crise imobiliária --- centra esforços na conexão deste tipo de gasto com processos mais gerais como a financeirização \cites{aalbers_financialization_2008}{bibow_financialization_2010} enquanto uma pequena fração aborda investimento residencial e crescimento.
A título de menção, vale destacar também o trabalho de \textcite{zezza_u.s._2008} em que são investigados os efeitos da diminuição da propensão marginal a poupar da economia norte-americana a partir da metodologia \textit{Stock-Flow Consistent} e conclui que o consumo financiado por crédito é o principal determinante do crescimento de modo que o investimento residencial um efeito riqueza via valorização dos imóveis apenas em que os ganhos de capital ajudam a explicar a redução na taxa de poupança apesar da distribuição da renda a favor dos lucros. 


ZEZZA, CARDACI, ETC: DISTRIBUIÇÃO E INVESTIMENTO RESIDENCIAL

HANSEN E DUESENBERRY: INVESTIMENTO RESIDENCIAL E DEMOGRAFICA

GOWANS E ALLAIN: INVESTIMENTO RESIDENCIAL E DEMOGRAFIA EM MODELOS DO TIPO SUPERMULTIPLICADOR

DESTA REVISÃO DA LITERATURA VERIFICA-SE QUE OS MODELOS OU RESTRINGEM A CONTRIBUIÇÃO DO INVESTIMENTO RESIDENCIAL A PERDA DA PARTICIPAÇÃO RELATIVA DOS SALÁRIOS OU CENTRAM A ANÁLISE EM ELEMENTOS DEMOGRÁFICOS ALÉM DE DEIXAR DE LADO A DEMANDA POR IMÓVEIS POR RAZÕES ESPECULATIVAS. COMO SERÁ MELHOR DISCUTIDO NO CAPÍTULO SEGUINTE, A RELEVÂNCIA DO INVESTIMENTO RESIDENCIAL NO EUA SE DÁ PELA CONEXÃO DESTE GASTO COM A AMPLIAÇÃO DO COLATERAL E SUBSEQUENTE ELEVAÇÃO DO ENDIVIDAMENTO/CONSUMO. UMA VEZ QUE ESTE CAPÍTULO É DE TEOR TEÓRICO, PRETENDE-SE DISCUTIR COMO INCLUIR TAL COMPONENTE DA DEMANDA AGREGADA SEM CONFINÁ-LO A UM CASO ESPECÍFICO.

- A AUTONOMIA DESTE GASTO NÃO SE RESTRINGE AO CRESCIMENTO POPULACIONAL.
- INCLUSÃO DE ELEMENTOS DA OFERTA DE TRABALHO NÃO PODEM SER DESASSOCIADOS DE UMA DISCUSSÃO SOBRE TÉCNICAS PRODUTIVAS DE MODO QUE OUTRAS QUESTÕES SURGEM

NIKALDI

RESPONDER SKOTT

TAXA PRÓPRIA DE JUROS



\begin{comment}
DESCARTADOS

Além disso, a inclusão deste componente de gasto revela a resolução parcial da instabilidade harrodiana\footnote{Diferentemente de \textcite{hein_harrodian_2012}, a instabilidade harrodiana é entendida como a incapacidade das expectativas sobre o grau de utilização se ajustarem na direção correta (instabilidade fundamental nos termos de \textcite{serrano_trouble_2017}) e não como o princípio do ajuste do estoque de capital.} nos modelos kaleckianos se não forem feitas modificações adicionais\footnote{Como visto, nos modelos mais convencionais a endogeneidade do grau de utilização é suficiente para contornar esse problema. As complicações mencionadas, decorrem das sofisticações dos modelos kaleckianos.}.
%INICIO EXPOSIÇÃO
Considerando, como em \textcite{amadeo_expectations_1987}, que o investimento reaja às expectativas sobre o grau de utilização ($u^e$), a função de acumulação ($g_I$) pode ser reescrita como:

\begin{equation}
\label{Kalecki_Autonomous}
g_I = \gamma + \gamma_u (u^e - u_n)
\end{equation}
em que $\gamma$ corresponde ao componente autônomo do investimento e pode ser traduzido tanto como \textit{animal spirits} quanto expectativa média da taxa de crescimento de longo prazo \cite[p.~4]{allain_macroeconomic_2014}. A justificativa da mudança da função de acumulação é por permitir tornar explícito o princípio do ajuste do estoque de capital no longo prazo. Como visto, no curto prazo o grau de utilização não é necessariamente igual ao desejado. No entanto, se as firmas ajustam o estoque de capital, 
$$
\Delta g_i = \varphi (u^e - u_n) \hspace{2cm} \varphi > 0
$$
tais expectativas devem ser revistas:

\begin{equation}
\Delta u^e = \xi (u - u^e), \hspace{3cm} \xi > 0
\end{equation}
Da mesma forma, as expectativas em relação à taxa de crescimento secular ($\gamma$) são corrigidas pelas taxas de crescimento efetivas ($g^*$), ou seja, 
\begin{equation}
\label{Autonomo_gamma}
\Delta \gamma = \phi (g^* - \gamma)
\end{equation}
em que $\phi$ indica um fator de correção positivo. Substituindo recursivamente e seguindo os procedimentos de \textcite[p.~5]{allain_macroeconomic_2014}, obtém-se:
\begin{equation}
\label{Autonomo_u}
\Delta \gamma = \phi \gamma_u (u - u^e) \Leftrightarrow \Delta g_I = \varphi (u^e - u_n), \hspace{2cm} \varphi > 0
\end{equation}

Tal equação implica na instabilidade de Harrod uma vez que há uma sobre/sub-estimação do grau de utilização que, por sua vez, se afasta cada vez mais do grau de equilíbrio. Em outras palavras, supondo que os empresários revisem a taxa de crescimento tendencial de acordo com a efetiva e se ambas se distinguirem, não existe um mecanismo que as igualem:
\begin{citation}
When the actual rate of utilization is consistently higher than the normal rate ($u^* > u_n$), this implies that the growth rate of the economy is consistently above the assessed secular growth rate of sales ($g > \gamma$). Thus, as long as entrepreneurs react to this in an adaptive way, they should eventually make a new, \textbf{higher}, assessment of the trend growth rate of
sales, thus making use of a \textbf{larger} $\gamma$ parameter in the investment function.
\cite[p.~144, grifos adicionados e variáveis adaptadas]{hein_harrodian_2012}
\end{citation}
Essa instabilidade\footnote{Vale destacar que não é necessário recorrer à mudanças nos modelos kaleckianos para incorrer em instabilidade, como pontua \textcite{dallery_kaleckian_2007}, o que não implica que todas elas são do tipo Harrodiana.}, argumentam \textcite[p.~144]{hein_harrodian_2012}, decorre do coeficiente $\gamma$ da função de investimento que deixa de ser constante na medida que o grau de utilização se afasta do normal. Nesses termos, não é paradoxal um modelo apresentar estabilidade Keynesiana e não resolver a instabilidade de Harrod. 

A razão do porquê pode ser explicitada seguindo a exposição de \textcite{hein_harrodian_2012} e \textcite{allain_macroeconomic_2014}. 


Vale ressaltar que tal resultado se verifica mesmo com a equação \ref{eqAllain} sendo idêntica à \ref{Autonomo_u}\footnote{A diferença consiste na substituição do grau de utilização efetivo pelo esperado.} como a diferença, nada trivial, da introdução dos gastos autônomos. 


Desse modo, a introdução dos gastos autônomos não criadores de capacidade é capaz de contornar a instabilidade dos modelos kaleckianos uma vez induzido investimento no longo prazo. 

\end{comment}