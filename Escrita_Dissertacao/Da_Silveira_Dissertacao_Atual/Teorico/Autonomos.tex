\subsection{Gastos autônomos no longo prazo?}
\label{CrescimentoAutonomo}

A motivação desta seção é responder algumas críticas feitas ao SMS e que também são aplicáveis aos modelos que serão apresentados na subseção seguinte e, portanto, devem ser esclarecidos antes de se avançar na discussão.
Em linhas gerais, parte das críticas ao SMS recaem sobre a autonomia dos gastos no longo prazo. Ao longo desta seção, argumenta-se que tais críticas decorrem de uma incompreensão tanto sobre o SMS quanto sobre o conceito de autonomia e, assim, são necessárias algumas elucidações para promover um debate mais proveitoso a começar pela definição de autonomia.


Ao partir do fluxo circular da renda, \textcite{serrano_long_1995} afirma que o investimento das é autônomo, ou seja, independe da renda tal como estabelecido pelo princípio da demanda efetiva e o mesmo não pode ser dito do consumo a partir dos salários. 
Neste ponto, cabe destacar outra passagem de \textcite[p.~120, grifos adicionados]{serrano_sraffian_1995}:

\begin{citacao}
	Indeed, the true reason for the lack of balance between capacity and demand in the Oxford theory [Modelos kaleckianos] in the long run is actually much simpler. As we have seen above in this theory, in the long run the level of output adapts itself to the level of aggregate demand. The level of productive capacity, however, cannot adjust to this level of aggregate demand because current capacity has already been determined as the result of previous autonomous investment. Hence it is the idea that investment is \textbf{autonomous} and not \textbf{anything related to oligopoly} or competition that explain the long-run discrepancies between capacity and demand.
\end{citacao}
A rejeição dos modelos kaleckianos tradicionais deve entendidas nesses termos, ou seja, foram descartados pelas implicações de se considerar o investimento enquanto variável autônoma no médio  e longo prazo.
Tais implicações, o entanto, não ocorrem quando o investimento passa a ser indizido pelo princípio do ajuste do estoque de capital:

\begin{citacao}
	Note that from our definition of capacity generating investment expenditures, it follows that when this type of investment is induced, productive capacity is necessarily a consequence of the evolution of effective demand. On the other hand, when capacity generating investment is autonomous it is productive capacity that emerges as a necessary consequence of (autonomous) investment. […] Indeed, the view that capacity of each sector is adjusted to normal level of effectual demand in every long-period position, necessary implies treating the long-period level of capacity generating investment as an endogenous magnitude. \cite[p.~77]{serrano_sraffian_1995}
\end{citacao}
Da citação anterior, fica explicitada a importância do investimento induzido para que demanda agregada e capacidade produtiva cresçam dinamicamente equilibradas.


OUTROS GASTOS AUTÔNOMOS QUE NÃO INVESTIMENTO DAS FIRMAS: Possas e Macedo e Silva + Nota de Rodapé Garagnani

Uma vez que o investimento das firmas é considerado induzido, é preciso considerar outro gasto enquanto autônomo para determinar a dinâmica.
Tal autonomia, por sua vez, pode ser compreendida como desdobramento do princípio da demanda efetiva estabelecido nos seguintes termos: a renda (e o emprego) é determinada pelo dispêndio que, por sua vez, é objeto de decisão autônoma. Em outras palavras, o estabelecimento de uma relação mercantil é assimétrico e decorre da decisão de gasto que implica um ato passivo de recebimento. 
Esta relação causal é resultado da distinção entre renda e poder de compra. Dito de outro modo, a decisão de gasto depende do portfólio dos agentes que, por sua vez, determina o poder de compra disponível e, portanto, as decisões de gasto são autônomas em relação a renda corrente  \cites{possas_demanda_1981}[p.~47--72]{possas_dinamica_1987}[p.~1--9]{macedo_e_silva_macroeconomia_1999}.

Outra forma de pontuar a autonomia dos gastos não criadores de capacidade produtiva é a do próprio \textcite{serrano_sraffian_1995} em que argumenta-se a independência da renda contratual (salários, por exemplo) decorrente das decisões de produção .
Portanto, restaura-se a ideia de que o PDE trata-se de um conceito mais generalizante que não se limita à determinação da dinâmica econômica pelo investimento das firmas\footnote{
	Isto posto, tais pontos demonstram que PDE e a ideia de investimento como ``\textit{causa causans}'' não são equivalentes.}
\cite{brochier_macroeconomics_2017}. 
Logo, é fundamental ter essa distinção em mente uma vez que não torna inconsistentes trabalhos inspirados no PDE que argumentam que  outros gastos estabelecem a dinâmica da economia.
Feitas essas ressalvas e seguindo a tipologia de \textcite{cesaratto_technical_2003} e a categorização de \textcite{serrano_sraffian_1995}, tais gastos autônomos são: (i) gastos do governo; (ii) consumo financiado por crédito; (iii) Investimento residencial\footnote{
	Como será melhor discutido na seção \ref{RevResidencial}, o tratamento do investimento residencial não é consensual pela literatura. Dito isso, vale destacar uma questão menos controversa, qual seja, tal gasto não cria capacidade produtiva ao setor privado. Esta conclusão, no entanto, não deve ser estendida sem as devidas mediações a construção de escritórios como bem pontua \textcite{duesenberry_investment_1958}. No entanto, tais questões fogem dos objetivos da presente investigação, cabendo apenas reforçar a não-criação de capacidade produtiva decorrente da construção de novas residências.
}; (iv) Gastos com P\&D\footnote{
	%TODO Nota sobre especificidade do investimento em P\&D.
} e; (v) Exportações.

Feitos estes esclarecimentos, é possível avançar para as controvérsias envolvendo a autonomia dos gastos no longo prazo. 
Em sua crítica ao SMS, \textcite{nikiforos_comments_2018} destaca os seguinte pontos: 
(i) os gastos autônomos podem ser incorporados nos modelos kaleckianos e, portanto, os resultados para o curto e médio prazo não são inéditos; 
(ii) é um modelo de longo prazo e deve ser avaliado enquanto tal; 
(iii) no longo prazo, é o grau de utilização normal que se endogeiniza; 
(iv) investimento apesar de não depender da poupança, não desempenha um papel relevante; 
(v) no longo prazo, $Z$ deixa de ser autônomo\footnote{Os pontos (i) e (iii) já foram abordados indiretamente ao longo da exposição enquanto os demais devem ser analisados mais detidamente.}.  No que diz respeito ao ponto (i), toda a discussão feita nesta seção indica que os gastos autônomos podem sem ser incluidos nos modelos kaleckianos, enquanto no longo prazo deve ser válido o princípio do ajuste do estoque de capital para não incorrer na instabilidade Harrodianda. 

No que diz respeito à convergência ao grau de utilização ao nível normal, \citeauthor*{nikiforos_comments_2018} afirma:

\begin{citacao}
	
	[T]he acceptance that in the long run the economy converges to a supply-determined rate of utilization means that either the role of demand vanishes and the model becomes ``classical in the long run'' (Duménil and Lévy 1999), or that demand remains independent but distribution becomes endogenous to allow for the convergence. \cite[p.~9]{nikiforos_comments_2018}
\end{citacao}
Tal raciocínio é infundado por duas razões: (i) convergência ao grau de utilização normal não implica que a demanda é irrelevante. De acordo com o supermultiplicador sraffiano, é a capacidade produtiva que se ajusta à demanda e não o inverso; (ii) se o grau de utilização converge ao normal, a distribuição de renda não é endogeneizada e isso é verificado pela endogeinização da propensão média a poupar decorrente de $Z>0$.

No que diz respeito aos gastos autônomos, o item (v) é o mais problemático e, portanto, esclarecê-lo permite uma melhor compreensão dos anteriores. Ao longo de seu artigo, \textcite[p.~4]{nikiforos_comments_2018} mescla a noção de autônomia com a de exogeneidade\footnote{Uma forma de verificar ambiguidade na definição de Nikiforos é que nos modelos kaleckianos analisados nesta seção tornam o investimento induzido no longo prazo por meio da endogeinização do parâmetro $\gamma$, antes autônomo. }: ``\textit{Autonomous means that is not affected by other economic variables within the system}''. Em linhas gerais, autonomia pode ser associada a independência relativa das demais variáveis econômicas enquanto exogeneidade é uma independência absoluta. Como destacado anteriormente, um gasto é considerado autônomo se independer das decisões de produção (renda) enquanto \textcite{serrano_sraffian_1995} acrescenta a não criação de capacidade produtiva a essa categorização. 

A incompreensão reportada anteriormente pode ser verificada no item (iv) em que afirma que uma das implicações da resolução da instabilidade de Harrod \textit{à la} supermultiplicador é a quase-endogeinização do investimento. No entanto, seguindo o princípio de ajuste do estoque de capital, o investimento se torna \textbf{induzido} (inverso de autônomo) e não endógeno (inverso de \textbf{exógeno}). Partindo da noção de autonomia apresentada anteriormente, \citeauthor*{nikiforos_comments_2018} afirma:
\begin{citacao}
	
	[N]one of the arguments of the investment function play any role whatsoever in the long run. This is strangely
	reminiscent of supply-side models or the FKP [Modelo de Cambrigde] where the accumulation rate converges to the
	exogenous natural growth rate in the long run. \cite[p.~11--12, comentario adicionado]{nikiforos_comments_2018}
\end{citacao}
O modelo do supermultiplicador, no entanto, não afirma que o investimento deixa de ser relevante. \textcite{dejuan_hidden_2017}, por exemplo, destaca que as economias que apresentam uma maior taxa de crescimento são aquelas com uma maior taxa de investimento e não aquelas com grau de utilização persistentemente mais elevado. A maior taxa de investimento, por sua vez, é um resultado esperado do supermultiplicador uma vez que a participação dos gastos autônomos na renda diminui diante de uma taxa de crescimento destes gastos ampliada.

Resta, portanto, a questão referente a temporalidade do supermultiplicador. O argumento de \citeauthor*{nikiforos_comments_2018} é que tal modelo é adequado somente para o longo prazo. No entanto, uma vez que $Z$ deixa de ser autônomo (como visto, exógeno), passa a ter a capacidade explicativa comprometida, ou nas palavras do autor, ``\textit{a cart without its horse}''. Dois pontos desta afirmação devem ser reformulados diante da discussão anterior a despeito do supermultiplicador sraffiano: (i) no longo prazo, $Z$ permanece autônomo e, portanto, continua sendo um modelo válido; (ii) por mais que tenha sido desenvolvido para tratar de questões de longo prazo não impede de utilizá-lo para analisar o ciclo econômico e isso será discutido em maior detalhes nas seções seguintes. 
