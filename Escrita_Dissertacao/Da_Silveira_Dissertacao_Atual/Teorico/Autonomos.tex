\subsection{Gastos autônomos no longo prazo?}
\label{CrescimentoAutonomo}

A motivação desta seção é responder algumas críticas feitas ao SMS e que também são aplicáveis aos modelos que serão apresentados na subseção \ref{Hibridos} e, portanto, devem ser esclarecidos antes de se avançar na discussão.
Em linhas gerais, parte das críticas ao SMS recaem sobre a autonomia dos gastos no longo prazo \cites{skott_autonomous_2017}{nikiforos_comments_2018}. 
Ao longo desta seção, argumenta-se que tais críticas decorrem de uma incompreensão tanto do SMS quanto do conceito de autonomia e, assim, são necessárias algumas elucidações para promover um debate mais proveitoso.


Ao partir do fluxo circular da renda, \textcite{serrano_long_1995} afirma que o investimento das \textbf{firmas} é autônomo, ou seja, independe da renda tal como estabelecido pelo princípio da demanda efetiva e o mesmo não pode ser dito do consumo a partir dos salários.
No entanto, se ...
 
Sendo assim, se  o investimento das firmas for induzido é preciso que outro gasto seja autônomo para determinar a dinâmica.

CONEXÃO ENTRE OS PARÁGRAFOS

Tal autonomia, por sua vez, pode ser compreendida como desdobramento do princípio da demanda efetiva estabelecido nos seguintes termos: a renda (e o emprego) é determinada pelo dispêndio. Em outras palavras, o estabelecimento de uma relação mercantil é assimétrico e decorre da decisão de gasto que implica um ato passivo de recebimento. 
Esta relação causal é resultado da distinção entre renda e poder de compra. Dito de outro modo, a decisão de gasto depende do portfólio dos agentes que, por sua vez, determina o poder de compra disponível e, portanto, as decisões de gasto são autônomas em relação a renda corrente.
Resumidamente, gasta-se poder de compra e não renda  \cites{possas_demanda_1981}[p.~47--72]{possas_dinamica_1987}[p.~1--9]{macedo_e_silva_macroeconomia_1999}.
Esta noção de autonomia, por sua vez, é estendida --- dada a indução do investimento das firmas --- aos gastos não criadores de capacidade produtiva por serem independentes da renda contratual (salários, por exemplo) decorrente das decisões de produção \cite{serrano_sraffian_1995}.


%TODO Pontuar diferença do PDE para Garegnani?

Portanto, defende-se a ideia de que o PDE é um conceito generalizante que não se limita à determinação da dinâmica econômica pelo investimento das firmas\footnote{
	Isto posto, tais pontos demonstram que PDE e a ideia de investimento como ``\textit{causa causans}'' não são equivalentes.}
\cite{brochier_macroeconomics_2017}. 
Logo, é fundamental ter essa distinção em mente uma vez que não torna inconsistentes trabalhos inspirados no PDE que argumentam que  outros gastos estabelecem a dinâmica da economia.
Feitas essas ressalvas e seguindo a tipologia de \textcite{cesaratto_technical_2003} e a categorização de \textcite{serrano_sraffian_1995}, tais gastos autônomos são: (i) gastos do governo; (ii) consumo financiado por crédito; (iii) Investimento residencial; (iv) Gastos com P\&D\footnote{
	%TODO Nota sobre especificidade do investimento em P\&D.
	Por mais que os gastos com P\&D são contabilmente considerados investimento das firmas, optou-se por incluí-los para avaliar modelos que partem do supermultiplicador sraffiano e consideram tais gastos como sendo os determinantes da dinâmica.
} e; (v) Exportações.

Feitos estes esclarecimentos, é possível avançar para as controvérsias envolvendo a autonomia dos gastos no longo prazo. 
Em sua crítica ao SMS, \textcite{nikiforos_comments_2018} destaca os seguintes pontos: 
(i) os gastos autônomos podem ser incorporados nos modelos kaleckianos e, portanto, os resultados para o curto e médio prazo não são inéditos; 
(ii) é um modelo de longo prazo e deve ser avaliado enquanto tal; 
(iii) no longo prazo, é o grau de utilização normal que se endogeiniza; 
(iv) investimento apesar de não depender da poupança, não desempenha um papel relevante; 
(v) no longo prazo, $Z$ deixa de ser autônomo.  
Pontuas as críticas, seguem algumas respostas em que a discussão do ponto (i) ficará a cargo da subseção \ref{Hibridos} para ser debatida mais pormenorizadamente.

O primeiro ponto a ser discutido é o da endogeinização do grau de utilização normal.
Antes de avançar, é importante destacar a importância da endogeinização do grau de utilização para os modelos kaleckianos\footnote{
	Por mais que no modelo do supermultiplicador sraffiano o grau de utilização converge ao normal, esta postura não é consensual entre teóricos sraffianos. \textcite[p.~161]{cesaratto_neo-kaleckian_2015} destaca que \textcites{garegnani_notes_1992}{palumbo_growth_2003} argumentam que tal convergência não é representativa de uma economia capitalista. O modelo de \textcite[Orignial de 1986]{ciccone_2017}, por exemplo, desassocia preços e taxa geral de lucro normais do grau de utilização normal. Tendo em vista este modelo, \textcite[p.~476]{moreira_demanda_2018} afirmam:
	
	\begin{citacao}
		Enfim, o que importa é que as
		quantidades produzidas estejam em consonância com a demanda efetiva, o que não
		implica que a capacidade produtiva também esteja e o grau de utilização seja normal.
		Entretanto, isso não nega que haja uma tendência de ajuste da capacidade à demanda.
	\end{citacao}
	adiante
	\begin{citacao}
		É importante deixar claro que, apesar do grau de utilização normal não ser
		uma condição necessária para o ajuste à posição de longo prazo, ainda assim, é a
		referência utilizada para o cálculo dos preços normais que deverão prevalecer no
		longo prazo. Isso significa que, independente da capacidade instalada efetiva num
		dado momento, para cada novo processo de instalação de capacidade, a intenção de
		operá-la ao nível normal será sempre uma baliza para as decisões de investimento [...]
		\cite[p.~477]{moreira_demanda_2018}
	\end{citacao}}.
Resumidamente, a caracterização dos regimes de crescimento (\textit{wage/profit-led}) perpassa pela endogeinização do grau de utilização que, ao não convergir ao normal, acomoda  \textbf{persistentemente} mudanças na distribuição funcional da renda\footnote{
	No modelo do supermultiplicador sraffiano, por outro lado, mudanças na distribuição funcional da renda geram efeitos temporários sobre a taxa de crescimento que, esgotados os efeitos sobre a propensão marginal a investir, retorna a taxa dos gastos autônomos \cite[p.~ 79]{serrano_sraffian_2017}.
}.  
No limite, é o que garante que a distribuição de renda possa ser exógena sem incorrer na instabilidade de Harrod\footnote{
	Outra forma de lidar com a instabilidade de Harrod é abstraí-la através de um corredor de estabilidade como em \textcites{dutt_growth_1990}{setterfield_long-run_2017}.
}, ou melhor, para que não precise ser endogeinizada\footnote{
	O centro do argumento de \textcite[p.~155--160]{amadeo_role_1986} é que se o grau de utilização for exógeno, mudanças na taxa de acumulação recairão necessariamente sobre a distribuição de renda.
}.
Além de rejeitar a convergência ao grau de utilização normal, a literatura kaleckiana também questionou a razoabilidade de considerá-lo singular e constante. 
Como resposta, utilizam um argumento convencionalista em que o grau de utilização efetivo é encarado como normal dada a existência da incerteza fundamental e do comportamento adaptativo das firmas\footnote{
	Em resposta, \textcite{skott_theoretical_2012} argumenta que um comportamento adaptativo só é razoável em relação à variáveis que os referidos agentes não possuem controle, o que não é o caso para as firmas e o grau de utilização. Dito isso, o autor questiona o porquê do grau de utilização desejado se ajustar e não a taxa de acumulação:
	
	\begin{citacao}
		But why adjust the target? Revised plans can take the form of changing the rate of accumulation—the Harrodian argument—rather than the target. Adjustments in the target would only be justified if the experience of low actual utilization makes firms decide that low utilization has now become optimal, and neither Amadeo nor Lavoie presents an argument for this causal link. \cite[p.120]{skott_theoretical_2012}
	\end{citacao}
	De forma complementar, \textcite{nikiforos_utilization_2016} também critica esta ideia convencionalista ao frisar que a necessidade de responder efeitos inesperados na demanda agregada é, acima de tudo, um \textbf{objetivo} das empresas e não um comportamento convencional. 
} \cite{lavoie_kaleckian_1995}. Desse modo, tal como em \textcite{amadeo_role_1986}, o grau de utilização normal se ajusta endogenamente ao efetivo. 

Além da endogeinização do grau de utilização normal, \textcite{hein_harrodian_2012} apresentam as propostas de \textcite{dallery_conflicting_2011} em que a existência de objetivos conflitantes e mutualmente excludentes impedem que o grau de utilização atinja o nível desejado. Em síntese, acionistas e gerentes possuem interesses distintos, impactando as decisões sobre a taxa de lucro normal. Já a negociação salarial entre empregado e empregador determina os valores da margem de lucro. Sendo assim, os autores argumentam que um objetivo sendo alcançado, o outro necessariamente é deixado em segundo plano\footnote{
	A título de exemplo, ao apresentar o modelo de inflação por conflito distributivo, \textcite[p.~567]{lavoie_post-keynesian_2015} afirma:
	\begin{citacao}
		The presence of bargaining power on the part of the workers allows
		us to respond to a criticism of the Kaleckian model of growth and distribution: there
		cannot be a true long-run steady state unless the normal profit rate and the actual profit
		rate are equated. In the above model, the two rates are equated; that is, the actual profit
		rate and the target rate of return assessed by firms become equal. Despite this, the rate
		of utilization of capacity is still free to vary from its standard or normal value. The key
		characteristic of the Kaleckian model, the endogeneity of its rate of capacity utilization,
		is thus preserved.
	\end{citacao}
}.

Compreendida a relevância e a dimensão do debate acerca da endogeinização do grau de utilização (seja o efetivo ou normal)\footnote{
	A longevidade da discussão sobre a convergência do grau de utilização pode ser vista no modelo de \textcite{vianello_pace_1985} cujo argumento é que sobre/sub-utilização não são persistentes e, portanto, o grau de utilização converge ao normal no \textit{steady-state}.
}, é possível avançar para o ponto (iii) em que \citeauthor*{nikiforos_comments_2018} rejeita a endogeinização do grau de utilização efetivo ao normal\footnote{
	Por mais que no modelo de \textcite{nikiforos_utilization_2016} o grau de utilização normal se ajusta endogenamente ao efetivo, o argumento se difere ao de \textcite{amadeo_role_1986}:
	\begin{citacao}
		However, the logic towards equation (8) is different compared to the rationale that has been provided so far in the literature. We begin at the micro level, from a firm that explicitly sets its desired utilization rate based on a \textbf{cost-minimizing decision} process and then we provide a link of this micro behavior to the adjustment of utilization at the macro level. \cite[p.~456, grifos adicionados e numeração adaptada]{nikiforos_utilization_2016}
	\end{citacao}
	Para uma crítica a essa ``Nova abordagem'', ver \textcite{girardi_normal_2018}.
	
} ao afirmar:

\begin{citacao}
	
	[T]he acceptance that in the long run the economy converges to a supply-determined rate of utilization means that either the role of demand vanishes and the model becomes ``classical in the long run'' (Duménil and Lévy 1999), or that demand remains independent but distribution becomes endogenous to allow for the convergence. \cite[p.~9]{nikiforos_comments_2018}
\end{citacao}
Tal conclusão é incorreta por duas razões: (i) convergência ao grau de utilização normal não implica que a demanda é irrelevante. De acordo com o supermultiplicador sraffiano, é a capacidade produtiva que se ajusta à demanda e não o inverso; (ii) se o grau de utilização converge ao normal, a distribuição de renda não é endogeneizada e isso é verificado pela endogeinização da propensão média a poupar decorrente da existência de gastos autônomos não criadores de capacidade ($Z>0$).

Em relação aos tais gastos autônomos, o item (v) é o mais problemático e, portanto, esclarecê-lo permite uma melhor compreensão dos demais. Ao longo de seu artigo, \textcite[p.~4]{nikiforos_comments_2018} mescla a noção de autonomia com a de exogeneidade: ``\textit{Autonomous means that is not affected by other economic variables within the system}''\footnote{
	Definição semelhante pode ser encontrada em \textcite[p.~2, grifos adicionados]{skott_autonomous_2017} em que o autor afirma:
	\begin{citacao}
		A brief comment on the meaning on autonomous demand in the present context may be useful. To
		be autonomous a component of aggregate demand must satisfy two conditions. It must be \textbf{exogenous}
		(i.e., independent of other variables in the model, including aggregate income and employment), and
		second, its movements must not be offset automatically by changes in other components of demand,
		leaving aggregate demand unchanged.
	\end{citacao}	
}. Diferentemente de Nikiforos, compreende-se autonomia como independência relativa das demais variáveis econômicas enquanto exogeneidade é uma independência absoluta. 
%Como destacado anteriormente, um gasto é considerado autônomo se independer das decisões de produção (renda) enquanto \textcite{serrano_sraffian_1995} acrescenta a não criação de capacidade produtiva a essa categorização. 
A incompreensão reportada anteriormente pode ser verificada no item (iv) em que afirma que uma das implicações da resolução da instabilidade de Harrod \textit{à la} supermultiplicador é a quase-endogeinização do investimento. No entanto, seguindo o princípio de ajuste do estoque de capital, o investimento se torna \textbf{induzido} (inverso de autônomo) e não endógeno (inverso de \textbf{exógeno}). Partindo da noção de autonomia apresentada anteriormente, \citeauthor*{nikiforos_comments_2018} afirma:
\begin{citacao}
	
	[N]one of the arguments of the investment function play any role whatsoever in the long run. This is strangely
	reminiscent of supply-side models or the FKP [Modelo de Cambrigde] where the accumulation rate converges to the
	exogenous natural growth rate in the long run. \cite[p.~11--12, comentario adicionado]{nikiforos_comments_2018}
\end{citacao}
No modelo do supermultiplicador, no entanto, o investimento não deixa de ser relevante. \textcite{dejuan_hidden_2017}, por exemplo, destaca que as economias que apresentam uma maior taxa de crescimento são aquelas com uma maior taxa de investimento e não aquelas com grau de utilização persistentemente mais elevado. A maior taxa de investimento, por sua vez, é um resultado esperado do supermultiplicador uma vez que a participação dos gastos autônomos na renda diminui diante de uma taxa de crescimento destes gastos ampliada.

Resta, portanto, a questão referente a temporalidade do supermultiplicador. O argumento de \citeauthor*{nikiforos_comments_2018} é que tal modelo é adequado somente para o longo prazo. No entanto, uma vez que $Z$ deixa de ser autônomo (leia-se, exógeno), passa a ter a capacidade explicativa comprometida, ou nas palavras do autor, ``\textit{a cart without its horse}''. 
Dois pontos desta afirmação devem ser reformulados diante da discussão anterior a despeito do supermultiplicador sraffiano: (i) no longo prazo, $Z$ permanece autônomo mas pode ser endogeneizado e, portanto, continua sendo um modelo válido; (ii) por mais que tenha sido desenvolvido para tratar de questões de longo prazo nada impede de utilizá-lo para analisar o ciclo econômico e isso será discutido em maior detalhes na seção \ref{Medium}, mas antes é necessário avaliar o investimento residencial.

