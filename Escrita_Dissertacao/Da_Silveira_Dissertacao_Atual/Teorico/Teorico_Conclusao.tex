\section{Considerações finanais}
\label{Concl1}

AUTONOMIA DO INVESTIMENTO

POSSIBILIDADE DE USAR O SUPER PARA O MÉDIO PRAZO

TAXA PRÓPRIA DE JUROS DOS IMÓVEIS

MODELO LÍDIA E SFC

Seguindo a sugestão de \textcite[p.~280]{freitas_growth_2015}, esta investigação avança no sentido de contribuir para a compreensação de outros componentes da demanda agregada não criadores de capacidade. Destaca-se ainda que a instabilidade da economia não decorre da junção do acelerador com o multiplicador\footnote{Como mostram \textcite{serrano_trouble_2017}, dadas certas condições, o SSM é dinamicamente \textit{estável}.}, mas sim da própria instabilidade da demanda agregada \cite{dejuan_hidden_2017}.


Da discussão anterior, verifica-se que a literatura sobre investimento residencial é bastante escassa nos modelos com gastos autônomos. Como será apresentado no capítulo seguinte, parte da literatura empírica (diminuta, mas crescente) destaca a importância deste componente da demanda para a dinâmica da economia norte americana. Diferentemente de grande parte dos trabalhos teóricos e empíricos, argumenta-se que é o investimento residencial que antecipa o ciclo econômico. Tal discussão é endereçada no capítulo seguinte.

CONCLUSÃO ANTIGA.

%============================== Início Retomada ==========================

\textcite{harrod_essay_1939} apresenta um aparato teórico que permite analisar modelos em sua forma dinâmica sem precisar recorrer à defasagens entre as variáveis. Apresenta uma equação que engloba tanto o efeito multiplicador quanto o princípio acelerador cuja implicação é que o equilíbrio dinâmico não é estável. Diante desta problemática, surgiram os modelos de Cambridge, Kaleckianos e supermultiplicador sraffiano na tentativa para domar tal instabilidade (ver tabela \ref{crescimento}). Na seção \ref{SecHarrod}, foram apresentadas tais alternativas em que o modelo de Cambridge não se mostrou adequado dadas as incompatibilidades com o comportamento das firmas associada a essa teoria. Desse modo, restaram os modelos kaleckianos e o SSM. 

A seção seguinte abordou a controvérsia em torno do grau de utilização e sua convergência ao normal no longo prazo e as implicações para os paradoxos dos custos e da parcimônia. Além disso, foram realçadas algumas críticas aos modelos Kaleckianos relacionadas a convergência/endogeinização ao/do grau de utilização normal. 
Coube a seção \ref{Literatura} apresentar a resposta kaleckiana a crítica envolvendo o pricípio do ajuste do estoque de capital em que foram incluídos gastos autônomos não criadores de capacidade.  
%=================================================================================
%								Tabela: modelos de crescimento
%=================================================================================
\begin{table}[htb]
	\centering
	\caption{Fechamento das principais teorias de crescimento heterodoxas}
	\label{crescimento}
	\resizebox{\textwidth}{!}{%
		\begin{tabular}{|l|ccccl|}
			\hline 
			\textbf{Modelo} & \begin{tabular}[c]{@{}c@{}} \textbf{Regime de} \\\textbf{crescimento} \end{tabular} &  \begin{tabular}[c]{@{}c@{}} \textbf{Distribuição} \\\textbf{de renda} \end{tabular} & \begin{tabular}[c]{@{}c@{}}\textbf{Grau de utilização} \\ \textbf{da capacidade}\end{tabular} & \begin{tabular}[c]{@{}c@{}} \textbf{Capacidade}  \\ \textbf{produtiva} \end{tabular} & \textbf{Fechamento} \\ \hline
			\textbf{Cambridge} & Ausente  & Endógena & \begin{tabular}[c]{@{}c@{}} Exógena \\ \end{tabular} & Exógena & Distribuição de renda\\
			\textbf{Kaleckiano} & Wage/Profit-led &  \begin{tabular}[c]{@{}c@{}} Exógena \\ (\textit{Mark-up}) \end{tabular} & Endógena   & Exógena & Grau de utilização \\ 
			\begin{tabular}[l]{@{}l@{}}\textbf{Supermultiplicador} \\\textbf{Sraffiano} \end{tabular} & Ausente & \begin{tabular}[c]{@{}c@{}} Exógena \\ (Teoria Monetária\\da distribuição)  \end{tabular} & Tende ao normal & Endógena & \begin{tabular}[c]{@{}c@{}} Propensão média \\ a poupar \end{tabular} \\ \hline
		\end{tabular}%
	}
\caption*{\textbf{Fonte:} Elaboração própria}
\end{table}


%============================== Fim Retomada ==========================

  Dito isso,  o capítulo seguinte irá esboçar alguns esforços para evidenciar a importância do investimento residencial para a dinâmica do ciclo econômico norte americano e, assim, preencher uma das lacunas dos modelos de crescimento com gastos autônomos.



\begin{comment}
DESCARTADOS
Revisitando a instabilidade de Harrod, \textcite{allain_macroeconomic_2014} destaca que foi tratada majoritariamente de duas formas. A primeira delas é eliminar o comportamento  ``\textit{knife-edge}'' do investimento tornando-o autônomo de modo que a taxa garantida se adeque à taxa de crescimento efetiva. No entanto, tal categorização não permite captar as distinções entre esses modelos e, por conta disso, serão discutido através dos fechamentos tal como em \textcite{serrano_long_1995} --- e revisitado por \textcite{serrano_har_2018}. No modelo de Cambridge, por exemplo, é a distribuição de renda que elimina a instabilidade harrodiana. Nos modelos Kaleckianos, por outro lado, tal eliminação  se dá pela endogeinização do grau de utilização
%\footnote{Uma outra maneira descrita pelo autor é por meio das características do ciclo econômico nos moldes de \textcite{hicks_contribution_1972} em que gastos autônomos determinam o limite inferior enquanto o pleno-emprego determina o superior, abstraindo a instabilidade.}. 
.
A segunda via de solução, ainda na categorização de \textcite{allain_macroeconomic_2014}, é por meio de modelos do tipo supermultiplicador que introduzem gastos autônomos que não criam capacidade\footnote{Vale destacar que a inclusão de gastos autônomos que não criam capacidade produtiva não é suficiente para que um modelo seja qualificado enquanto um supermultiplicador, mas sim, o princípio do ajuste do estoque de capital. A importância desses gasto recai sobre a estabilidade do modelo.} em que o investimento é determinado pelo princípio de ajuste do estoque de capital \cites{serrano_long_1995}{serrano_sraffian_1995}{bortis_institutions_1996}
%\footnote{\textcite[p.~7]{allain_macroeconomic_2014} afirma que o modelo de \textcite{serrano_long_1995} elimina a instabilidade de Harrod por hipótese uma vez que as firmas preveem corretamente a trajetória da demanda efetiva. Argumenta-se que esta interpretação não está alinhada com o supermultiplicador proposto por \textcite{serrano_sraffian_1995} e, ao final deste capítulo, mostra-se que tal problema foi solucionado por meio de: (i) existência de gastos autônomos não criadores de capacidade e (ii) investimento induzido (princípio do ajuste de estoque de capital). No supermultiplicador sraffiano, portanto, a instabilidade não é eliminada por hipótese. Mais detalhes na seção \ref{Literatura}}.
.

%TODO Rever aderência do trecho acima

\end{comment}


