\subsubsection{Modelo(s) kaleckiano(s)}

No pós-segunda guerra, parte da literatura tem dado ênfase para 
as implicações de uma tendência à estagnação secular\footnote{Em linhas gerais, \textcite{steindl_stagnation_1979} afirma que na ausência de políticas expansionistas, de superávits na balança comercial e com a redução do consumo,  tais economias tenderiam a uma estagnação secular e, assim,  não haveria nenhuma razão para que opere ao nível do seu produto potencial ($u \neq u_N$).} \cite{setterfield_distribution_2002}.
Seguindo as contribuições de Kalecki, \textcite{steindl_maturity_1952} argumenta que o ajuste entre demanda e capacidade produtiva em economias com  estruturas de mercado oligopolistas --- ditas maduras --- não se dá por meio da distribuição.
Tal conclusão decorre da rigidez dos preços inclusive no longo prazo devido a \textit{mark-ups} também rígidos resultantes do poder de mercado das firmas oligopolistas de modo que a distribuição de renda não poderia funcionar como fechamento\footnote{
	\textcite{serrano_sraffian_1995} argumenta que a negação da flexibilização do \textit{mark-up} no longo prazo independe da estrutura de mercado uma vez que os preços são predominantemente \textit{fix-price}. Desse modo, a distribuição de renda pode ser exógena mesmo em uma economia concorrencial. Tal argumento decorre de uma análise centrada no longo período (ver \textcite{milgate_capital_1982}) em que a livre concorrência entre capitais --- diferente de uma concorrência perfeita e instantânea nos moldes neoclássicos ---  é uma força sistemática, impondo uma tensão permanente em uma economia capitalista. 
	Desse modo, seguindo microfundamentação da teoria do preço-limite de \textcite{labini} os \textit{mark-ups} também não são flexíveis no longo prazo.
	Portanto, o argumento kaleckiano apesar de ser suficiente,  não é necessário para  rejeitar a flexibilixação do \textit{mark-up} de Cambridge, mas tal discussão foge dos objetivos desta pesquisa.
}.
Sendo assim, por mais que Steindl compartilha da visão de Cambridge de que o investimento das firmas é autônomo, afirma que o modelo de Cambridge não pode ser aplicado (ao menos) para as economias maduras.  


Inspirados em grande parte pelas contribuições de \textcite{steindl_stagnation_1979}, surgem os modelos kaleckianos\footnote{Por conveniência, os modelos Neo-kaleckianos e pós-kaleckianos são referenciados como kaleckianos. Para uma discussão das raízes steindlianas dos modelos ditos kaleckianos, ver \textcite{santiago_uma_2008}.
} \cites{rowthorn_demand_1981}{dutt_stagnation_1984}{taylor_stagnationist_1985}{amadeo_role_1986}{bhaduri_unemployment_1990}. Seguindo a caracterização de \textcite[p.~790]{lavoie_kaleckian_1995}, tais modelos apresentam os seguintes elementos em comum: (i) o investimento é autônomo\footnote{
	A rigor, o investimento é parcialmente induzido na maioria dos modelos kaleckianos. No entanto, tal característica não altera  a essência dos resultados apresentados no corpo do texto.
}; (ii) os preços ($p$) são definidos em relação aos custos diretos do trabalho; (iii) custos marginais constantes abaixo da plena utilização da capacidade; (iv) existe capacidade ociosa e; (v) não existem restrições no mercado de trabalho. 

A hipótese adicional (ii) sobre determinação dos preços implica que 
$$
p = (1+\theta)\cdot \frac{w\cdot L}{Y}
$$
em que $w$ é a taxa de salário nominal, $L/Y$ é a unidade de trabalho utilizada na produção --- e o inverso da produtividade do trabalho --- e $\theta$ é o \textit{markup} (rígido) sobre os custos unitários do trabalho ($W/Y$),  de modo que a participação dos lucros na renda ($1-\omega$) é definida por:
$$
1 - \omega = \frac{\theta}{1+\theta}
$$
logo, a distribuição de renda é exogenamente determinada por microfundamentos relacionados à estrutura de mercado. 
Da equação \ref{EqGeral}, é possível ver o fechamento do modelo, conforme apresentado em \textcite{serrano_trouble_2017}

$$
g = \frac{I}{K}
$$

$$
\frac{I}{K} \equiv \frac{S}{K} =  \frac{FT}{Y}\frac{Y}{Y_{fc}}\frac{Y_{fc}}{K}
$$

$$
g = s_k\cdot \frac{(1-\omega)\cdot u}{ v}
$$

\begin{equation}
\label{FechKalecki}
u = \frac{g\cdot v}{s_k\cdot (1-\omega)}
\end{equation}

Tal exposição permite explicitar que quando a taxa de crescimento não for igual à garantida, o grau de utilização da capacidade necessariamente irá variar para adequar o equilíbrio dinâmico entre demanda e capacidade produtiva.
Mudanças na taxa de crescimento da economia teriam como consequência, portanto, variações do grau de utilização da capacidade, que pode ficar de forma permanente em um patamar diferente do normal.
Nesses termos, a equação \ref{FechKalecki} explicita que o grau de utilização é a variável de fechamento do modelo.

A partir das modificações de \textcite{bhaduri_unemployment_1990} ao modelo kaleckiano canônico, a literatura avançou --- teórica e empiricamente --- em direção aos regimes de crescimento e às formas que a distribuição determina o crescimento.
No entanto, não  cabe aqui elencar se a  literatura heterodoxa (majoritariamente kaleckiana)  categoriza as economias como \textit{wage} ou \textit{profit-led}
\footnote{Ver 
	\textcite{setterfield_distribution_2002} 
	e \textcite{onaran_is_2013} para um  \textit{survey} sobre o tema e \textcite{blecker_wage_led_2016} para uma discussão sobre a importância da temporalidade do regime de crescimento enquanto \textcite{lavoie_origins_2017} apresenta as origens deste debate.} 
e sim ressaltar algumas  características essenciais dessa família de modelos. Em linhas gerais, mudanças na distribuição funcional da renda têm impactos \textbf{persistentes} sobre a taxa de crescimento. 
Tal resultado, por sua vez, decorre da não-convergência --- ao menos nas versões mais tradicionais\footnote{
	Adiante, na seção \ref{Hibridos}, serão analisados os modelos kaleckianos que apresentam convergência ao grau de utilização normal.
} --- ao grau de utilização normal. Sendo assim, capacidade produtiva e demanda só se ajustam se o grau de utilização acomodar tais mudanças\footnote{Uma crítica endereçada especificamente aos modelos kaleckianos diz respeito a razoabilidade do grau de utilização estar \textbf{persistentemente} em níveis (arbitrários) diferentes do desejado no logo prazo.} dada a existência de um componente autônomo do investimento das firmas \cite[p.~84--86]{serrano_sraffian_2017}.
No entanto, parte da literatura tem criticado esta não-convergência ao grau de utilização normal no longo prazo \cites{skott_finance_1988}{skott_theoretical_2012}
bem como a instabilidade do modelo kaleckiano canônico caso o componente autônomo do investimento seja endogeneizado \cites{hein_harrodian_2012}{allain_tackling_2015}{nah_long-run_2017}.
%TODO Rever resolução parcial da instabilidade de harrod

Outra crítica aos modelos kaleckianos é a ausência de gastos autônomos não criadores de capacidade produtiva ($Z$) como um elemento necessário.
A não inclusão destes gastos faz com que o investimento não possa crescer a uma taxa diferente da demanda agregada (isto é $g_I \equiv g$) de modo que mudanças no crescimento não são capazes de alterar a taxa de investimento\footnote{Uma vez que o investimento e renda crescem a uma mesma taxa, a taxa de investimento não se altera e permanece igual a taxa de poupança que, como visto, é idêntica a propensão marginal a poupar exogenamente determinada.}. 
Como consequência, a propensão marginal e média a poupar são idênticas e, portanto, a taxa de poupança ($S/Y = s$) determina a taxa de investimento \cite[p.~5--7]{fagundes_role_2017}.

 
Os modelos até então analisados possuem a hipótese compartilhada de que o investimento criador de capacidade preserva sua autonomia no longo prazo.  
Destaca-se ainda a impossibilidade desses modelos --- em sua forma mais genérica --- reproduzirem os seguintes fatos estilizados: 
(i) grau de utilização acompanha o nível normal apesar de sua volatilidade elevada \cites[p.~110--111]{serrano_long_1995-1}{gahn_empirical_2019} \footnote{Sobre este ponto, destaca-se o debate sobre a instabilidade implícita nos modelos kaleckianos com convergência ao grau normal como pontuado em \textcite{hein_instability_2011} e em \textcite{allain_tackling_2015}.}; 
(ii) relação positiva entre crescimento do produto e participação do investimento na renda \cites{braga_investment_2018}{haluska_growth_2019} e;
(iii) existência --- em sua formulação básica --- de gastos autônomos não criadores de capacidade produtiva ($Z$).
Pontuadas estas críticas a subseção seguinte aborda outra proposta à instabilidade de Harrod.


\begin{comment}
COMENTÁRIO: Esta parte estava realmente confusa. Mesmo se estivesse clara, estaria desconexa com o parágrafo seguinte

\textcite{steindl_stagnation_1979}, por sua vez, define maturidade como a inadequação da função de lucros diante da taxa de crescimento da economia em que o menor grau de utilização da capacidade em uma estrutura de mercado oligopolista (como em \textcite{kalecki_theory_1954}) acomoda essa menor taxa de investimento, explicando a estagnação. 




Nesta família de modelos, o investimento\footnote{Vale destacar que a função poupança não difere nesses modelos, mas pode ser modificada para permitir uma primeira aproximação  da distribuição pessoal da renda \cites{carvalho_personal_2016}{palley_wage-_2017}. A essência do modelo, como mencionado, está contida na função investimento \ref{InvestoKalecki}.} é determinado por:

\begin{equation}
\label{InvestoKalecki}
\frac{I}{K} = \gamma + \gamma_u\cdot u + \gamma_{\pi}\pi = g
\end{equation}
em que $\gamma$ é a parcela autônoma do investimento, $\gamma_u$ representa a sensibilidade do investimento à mudanças no grau de utilização e $\gamma_{\pi}$ em relação ao \textit{profit-share}\footnote{Esse último termo é destacado para evidenciar a crítica de \textcite{bhaduri_unemployment_1990} que inaugura os modelos pós-kaleckianos. Argumenta-se a inclusão deste componente não altera o mecanismo de funcionamento do modelo, mas amplia os resultados possíveis.}. Partindo da versão mais simplificada em que o investimento induzido depende apenas do grau de utilização ($\gamma_{\pi} = 0$),  a equação \ref{InvestoKalecki} pode ser tratada em termos da equação \ref{Sintetica}:

$$
\gamma + \gamma_u\cdot u = \mybox{g = g_K} = f\frac{s u}{\overline v}
$$
Tal como no modelo de Cambridge, supõe-se que o componente autônomo do investimento seja exógeno ($\gamma = \overline \gamma$) e que a propensão marginal a poupar é definida exogenamente e idêntica a propensão média ($f=1$), ou seja

$$
\gamma_u\cdot u = \frac{\overline s u}{\overline v} - \overline \gamma
$$
rearranjando:
$$
u = \left(\frac{\overline s u}{\overline v} - \overline \gamma\right)\frac{1}{\gamma_u}
$$

\begin{equation}
\label{FechKalecki}
\therefore u =  \left(\frac{\overline \gamma\cdot v}{\overline s - \gamma_u}\right)
\end{equation}


Antes de prosseguir para a análise do supermultiplicador sraffiano, é oportuno apresentar este modelo kaleckiano em sua forma ampliada.
Em linhas gerais, tal vertente --- iniciada por \textcite{bhaduri_unemployment_1990} ---  argumenta a estagnação não é um resultado necessário mesmo considerando uma estrutura de mercado oligopolista, preços determinados via \textit{mark-up} e excesso de capacidade.
Adaptando as exposições de \textcite{setterfield_distribution_2002} e de \textcite[Cap, 6]{lavoie_post-keynesian_2015}, é possível escrever a função de investimento em que uma maior participação dos lucros na renda estão relacionados a um maior nível de atividade (e maior grau de utilização):

\begin{equation}
\label{PostKalecki}
\frac{I}{K} = \gamma + \gamma_u\cdot u + \gamma_{\pi}\pi = g
\end{equation}
em que $\gamma$ é a parcela autônoma do investimento, $\gamma_u$ representa a sensibilidade do investimento à mudanças no grau de utilização e $\gamma_{\pi}$ em relação ao \textit{profit-share}. Igualando o identidade entre poupança e investimento obtém-se o grau de utilização que fecha o modelo no curto prazo:

\begin{equation}
\label{KaleckiSR}
u = \frac{\gamma + \gamma_{\pi}(1-\omega)}{s_k\cdot (1-\omega) - v\gamma_u}
\end{equation}
em que o denominador indica a condição de estabilidade (keynesiana) do modelo em que o investimento precisa ser menos sensível do que a poupança à mudanças no nível de atividade, ou melhor, a propensão marginal a gastar precisa ser menor que a unidade\footnote{Para uma crítica à ausência de relações entre crescimento e distribuição assim como às limitações do debate \textit{wage/profit-led} em um aparato
Harrodiano, 
ver 
\textcite{skott_weaknesses_2017}.}.
Nesta formulação, o grau de utilização pode reagir de formas distintas à mudanças na distribuição funcional da renda. Deste modelo, emergem regimes de acumulação a depender da relações (unidirecionais) entre distribuição de renda e crescimento. Utilizando a terminologia convencional, se um aumento da participação dos lucros na renda implicar em maiores taxas de crescimento, tal economia apresenta uma dinâmica \textit{profit-led} enquanto um regime \textit{wage-led} é caracterizado pelo inverso. Esquematicamente:

\begin{center}
$$
\begin{cases}
\gamma_u > \gamma_{\pi}:\frac{dg}{d\omega} > 0\hspace{2cm} \text{\textit{Wage-led}}\\
\gamma_u < \gamma_{\pi}:\frac{dg}{d\omega} < 0 \hspace{2cm} \text{          \textit{Profit-led}}
\end{cases}
$$
\end{center}
para que aumentos na participação dos salários na renda gerem efeitos positivos sobre a taxa de crescimento, é preciso que o investimento seja mais sensível a mudanças no grau de utilização do que à participação dos lucros, configurando um regime \textit{wage-led}\footnote{
Partindo de um modelo sensivelmente diferente do apresentado, \textcite{dutt_stagnation_1984} argumenta que dada uma estrutura de mercado oligopolista, há uma relação positiva entre taxa de crescimento e melhora distributiva. Nesses termos, afirma que a estagnação da economia indiana pode ser explicada como resultado de uma piora na distribuição de renda assim como maior concentração industrial. No entanto, por não incluir o parâmetro $\gamma_\pi$ só é possível que o regime seja \textit{wage-led}.}.  Caso prevaleça o inverso, diz-se que é um regime de acumulação \textit{profit-led}\footnote{\textcite{bhaduri_unemployment_1990} incluem ramificações destas duas possibilidades que não serão exploradas em maior detalhe por não alterarem o mecanismo do modelo.}.

A qualificação anterior trata dos efeitos sobre a taxa de acumulação, que podem ser positivos ou negativos a depender da sensibilidade do investimento ao \textit{profit-share} ($\gamma_{\pi}$), resta analisar os efeitos sobre o grau de utilização. Nesses modelos, existe sempre uma relação negativa entre participação dos lucros na renda e nível de atividade/taxa de lucros (ver equação \ref{FechKalecki}). Resumidamente, a taxa de lucro depende positivamente da participação dos lucros na renda enquanto a relação entre taxa de acumulação e participação dos lucros não é definida \textit{à priori}, como sugere \textcite{bhaduri_unemployment_1990}, mas depende de parâmetros estruturais e isso faz com que surja uma vasta literatura kaleckiana empírica\footnote{\textcite{pariboni_autonomous_2015} ressalta que a convergência para uma discussão empírica na literatura kaleckiana sugere que as questão teóricas tornem-se de uma magnitude menor. Este capítulo, em linha com este autor, pretende fazer uma discussão essencialmente teórica e este tema será endereçado em maiores detalhes na seção \ref{Literatura}.}.

% TODO: Fix error below


\end{comment}
