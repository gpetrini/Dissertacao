\subsection{Princípio da demanda efetiva no médio prazo: um paradigma e duas alternativas?}
\label{Medium}


Para encerrar essa discussão, é feita uma comparação entre as duas alternativas restantes, qual sejam, kalekiana não-convencional e sraffiana. Em linha com \textcite{fagundes_role_2017}, argumenta-se que no \textbf{longo prazo} os modelos kaleckianos não-convencionais respondem suficientemente bem à convergência do grau de utilização sem incorrer na instabilidade de Harrod. Diante disso, existem duas questões importantes em aberto: (i) dadas as hipóteses compartilhadas, qual a distinção fundamental entre ambos os modelos? (ii) dados os objetivos desta investigação, qual modelo a ser adotado? Resta a esta seção responder tais questões.


O primeiro ponto pode ser respondido de forma mais direta: a principal diferença é a autonomia do investimento no curto e médio prazo.  Resumidamente, se o investimento produtivo for induzido, a convergência ao grau de utilização é uma derivação do princípio do ajuste do estoque de capital e, dados certos limites, a capacidade produtiva se ajusta à demanda efetiva:
\begin{citacao}
	Indeed, the true reason for the lack of balance between capacity and demand in the Oxford theory [Modelos kaleckianos] in the long run is actually much simpler. As we have seen above in this theory, in the long run the level of output adapts itself to the level of aggregate demand. The level of productive capacity, however, cannot adjust to this level of aggregate demand because current capacity has already been determined as the result of previous autonomous investment. Hence it is the idea that investment is \textbf{autonomous} and not \textbf{anything related to oligopoly} or competition that explain the long-run discrepancies between capacity and demand.
\end{citacao}
Por outro lado, se o investimento possuir um componente autônomo, como nos modelos kaleckianos convencionais, a demanda efetiva se ajusta à capacidade produtiva que está definida aprioristicamente:
\begin{citacao}
	Note that from our definition of capacity generating investment expenditures, it follows that when this type of investment is induced, productive capacity is necessarily a consequence of the evolution of effective demand. On the other hand, when capacity generating investment is autonomous it is productive capacity that emerges as a necessary consequence of (autonomous) investment. […] Indeed, the view that capacity of each sector is adjusted to normal level of effectual demand in every long-period position, necessary implies treating the long-period level of capacity generating investment as an endogenous magnitude. \cite[p.~77]{serrano_sraffian_1995}
\end{citacao}
Como mostrado na seção \ref{Hibridos}, isso deixa de ser o caso nos modelos kaleckianos com investimento induzido no longo prazo.

Para responder a segunda questão, resta esclarecer um possível ponto de estranhamento. O principal objetivo desta pesquisa é investigar os determinantes do ciclo econômico norte americano e desenvolver um modelo que replique alguns dos fatos estilizados. Sendo este o caso, a ênfase na discussão de modelos de longo prazo parece ser desconexa. 
No entanto, como mencionado na introdução, os modelos elegíveis são aqueles reportam alguns fatos estilizados (\textit{e.g.} relação positiva entre taxa de investimento e crescimento)  no curto, médio e longo prazo.
Desse modo, optar por modelos que se mostram adequados para o curto e longo prazo, mas não para o médio-prazo se mostra questionável uma vez que a validade dos resultados está restrita a uma certa temporalidade. 

Como visto, os modelos restantes represental tais fato estilizados no curto e longo prazo. Resta verificar se o mesmo vale para \textbf{médio prazo}. Dito isso, dentre os modelos kaleckianos com gasto autônomo e com principio de ajuste do estoque de capital e supermultiplicador sraffiano, resta selecionar aquele reproduza o fato estilizado da relação positiva entre taxa de investimento e taxa de crescimento uma vez que a convergência ao grau de utilização normal ocorre no longo período \cites[p.~172]{cesaratto_neo-kaleckian_2015}[p.~8--9]{fiebiger_trend_2017}\footnote{Esta parte da exposição é inspirada na contribuição de \textcite{fagundes_role_2017} no que diz respeito ao médio prazo.}. Dito isso, seja $i$ a taxa de investimento, $\gamma_A$ a parcela autônoma e $h$ a parcela induzida do investimento (das firmas) de modo que a representar a função de acumulação kaleckiana pode ser reescrita como\footnote{
	As etapas são:
	$$
	\frac{I}{K}  = \gamma + \gamma_uu - \gamma_uu_N
	$$
	
	$$
	\gamma_A = \gamma - \gamma_uu_N
	$$
	
	$$
	I = (\gamma_A + \gamma_uu)K
	$$
	
	$$
	\gamma_u\cdot u \cdot K \equiv \gamma_u\frac{Y}{Y_{FC}}K \equiv \gamma_u\cdot v\cdot Y
	$$
	
	$$
	I = \gamma_A\cdot K + \gamma_u\cdot v\cdot Y
	$$
}

\begin{equation}
\tag{kaleckiana}
I = \gamma_A\cdot K + h\cdot Y
\end{equation}
enquanto a do supermultiplicador (adiante, SSM) continua sendo

\begin{equation}
\tag{SSM}
I = h\cdot Y
\end{equation}
Como destacado na seção \ref{SecHarrod}, na ausência de gastos autônomos, a propensão marginal e média a poupar são iguais e, portanto, no modelo kaleckiano convencional, a taxa de investimento é determinada pela taxa de poupança definida exogenamente. Incluindo os gastos autônomos neste modelo, obtém-se:

$$
i = \frac{i_{Trad}\gamma_A + hz}{\gamma_A + z}
$$
em que $i_{Trad}$ denota, tal como em \textcite{fagundes_role_2017}, a taxa de investimento no modelo kaleckiano canônico. Nos modelos kaleckianos não-tradicionais, a ausência dos gastos autônomos implica na volta ao modelo kaleckiano convencional enquanto no supermultiplicador retorna-se ao \textcite{harrod_essay_1939}. Mais uma vez, a introdução de tais gastos não é capaz, por si só, de eliminar a instabilidade  mas sim pela modificação da função investimento \textit{à la} acelerador flexível cuja alteração é feita somente no longo prazo nos modelos derivados de \textcite{allain_tackling_2015}. 

Prosseguindo com a exposição e analisando o equilíbrio de \textit{steady growth} com gastos autônomos ($Z > 0$), verifica-se que no médio prazo dos modelos kaleckianos não-convencionais ($g\to g_Z$) a taxa de investimento ($i_{MR}$) é dada por:
\begin{equation}
\label{investoFagundes}
i_{MR} = \frac{h\cdot g_Z}{g_Z - \gamma_A}
\end{equation}
Diante deste resultado, \textcite{fagundes_role_2017} argumentam que a inclusão dos gastos autônomos no modelo não garante a convergência do grau de utilização ao normal. Para que tal tendência ocorra, por sua vez, é necessário que a participação da parcela autônoma do investimento convirja a zero ($\gamma_A \to 0$) e isto ocorre no modelo de \textcite{allain_tackling_2015}. 
No entanto, \textcite{fagundes_role_2017} reportam uma relação negativa entre taxa de crescimento e taxa de investimento. Supondo, por simplificação, que as variações são infinitesimais, isto pode ser explicitado em termos da equação \ref{investoFagundes} por derivadas parciais:
$$
\frac{\partial i_{MR}}{\partial g_Z} = - \frac{\gamma_A h}{[g_Z - \gamma_A]^2} < 0 \Leftrightarrow \gamma_A > 0
$$
Além disso, os autores pontuam um problema de ``dupla indentidade'' nos modelos \textit{à la} \textcite{allain_tackling_2015} decorrente das diferentes condições de equilíbrio, um em $Z = 0$ e no outro $Z>0$, cujos padrões de crescimento são mutualmente excludentes. No primeiro, obtém-se um regime liderado pelo investimento mas incapaz de gerar a tendência do grau de utilização ao normal e de destacar a importância dos gastos autônomos ($Z\to 0$). No outro, ocorre o inverso, um regime liderado pelos gastos autônomos ($\gamma_A \to 0$) mas que evidencia uma relação negativa entre crescimento e taxa de investimento. Ambos os casos, contraria-se alguns fatos estilizados. Portanto, a aceitação a conclusão de \textcite[p.~13]{fagundes_role_2017} é imediata:

\begin{citacao}
	
	[I]f we think of such a model as an intermediate step towards the long-run model, then we
	believe that there is no problem in using it. The problem occurs when we think of the medium-run
	model as a contribution to the understanding of economic reality in itself, independent from the long-run model.
\end{citacao}
Neste ponto, o trecho a seguir é esclarecedor:

\begin{citacao}
	What the supermultiplier adds to the neo-Kaleckian framework is a plausible mechanism for explaining phases
	of the business cycle when the output share of capacity investment is rising amidst robust rates of output growth. \cite[p.~9]{fiebiger_trend_2017}
\end{citacao}

Resta checar se a alternativa pelo SMS incorre nos mesmos problemas. Para isso, basta verificar os resultados para o caso em que o investimento é completamente induzido. Como a alternativa kaleckiana com gastos autônomos pode ser considerada como híbrida entre o modelo kaleckiano convencional e o SSM, substituindo $\gamma_A = 0$ na equação \ref{investoFagundes}, obtém-se:
$$
i_{MR} = \frac{I}{Y} =  h
$$
Seguindo a proposta do supermultiplicador em que o investimento é completamente induzido:
$$
g = \frac{h\cdot u}{v} \Rightarrow h^* = i_{MR} = \frac{g_Z\cdot v}{u}
$$

$$
\frac{\partial i_{MR}}{\partial g_Z} = \frac{v}{u} > 0
$$
Portanto, a relação negativa entre crescimento e taxa de investimento deixa de existir e isso não é feito às custas da não convergência do grau de utilização ou da relevância dos gastos autônomos no longo prazo. 
Diante desta discussão, conclui-se que o modelo do SSM não é incompatível para analisar o médio prazo ou restrito ao longo prazo como afirma \textcite{nikiforos_comments_2018}. Com isso, elege-se o supermultiplicador sraffiano como o mais adequado para atender os objetivos desta pesquisa. 
 
