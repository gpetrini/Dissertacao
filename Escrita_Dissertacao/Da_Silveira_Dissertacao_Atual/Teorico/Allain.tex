\subsection{Um breve mapeamento da fronteira heterodoxa}
\label{Hibridos}


Ao longo desta seção, serão mapeados os modelos de crescimento, 
sejam eles kaleckianos ou sraffianos, liderados pelos gastos autônomos não criadores de capacidade produtiva ao setor privado. Isso não implica que são os únicos modelos com gastos autônomos, mas sim, que são os modelos em que a participação destes gastos não converge a zero\footnote{
	No que diz respeito ao consumo financiado por crédito, por exemplo, destacam-se os trabalhos de \textcite{dutt_maturity_2006}, \textcite{palley_inside_2010} e \textcite{hein_finance-dominated_2012}.
	No entanto, esses autores trabalham no arcabouço kaleckiano básico. Por isso, a estabilidade só é garantida se o consumo financiado por crédito a crescer a mesma taxa que a acumulação --- ou seja, este componente do consumo não pode ser tratado como de fato um gasto autônomo.
	Diante desta limitação, \textcite{pariboni_household_2016} argumenta que os gastos autônomos desempenham um papel passivo e sugere uma alternativa a partir do supermultiplicador sraffiano. 
}.
Dado que estes modelos partem do fechamento do supermultiplicador sraffiano no longo prazo, os resultados esperados são: (i) mudanças na distribuição de renda não afetam a taxa de crescimento do produto; (ii) o mesmo vale para as alterações na propensão marginal propensão a poupar; (ii) convergência do grau de utilização ao normal; (iv) taxa de crescimento da economia converge à taxa dos gastos autônomos.
Sendo assim, as especificidades de cada modelo serão explicitadas somente se os resultados anteriores se alterarem de modo que serão analisadas as implicações da inclusão dos referidos gastos autônomos na medida que contribuam para os objetivos dessa pesquisa.
%Para manter a comparatividade entre os modelos apresentados, serão realçados alguns dos resultados de \textbf{longo prazo}, são eles: (i) mudanças na distribuição de renda; (ii) alterações na propensão marginal propensão a poupar; (ii) convergência do grau de utilização ao normal; (iv) aumento da taxa de crescimento dos gastos autônomos. Por fim, as variáveis serão adaptadas de modo que $\gamma$ é o componente autônomo do investimento,  $z$ é a participação dos gastos autônomos ($Z$) na renda que crescem a taxa $g_Z$.
Feitas essas ressalvas e seguindo a tipologia de \textcite{cesaratto_technical_2003} e a categorização de \textcite{serrano_sraffian_1995}, tais gastos autônomos são: (i) gastos do governo; (ii) consumo financiado por crédito; (iii) Investimento residencial; e (iv) Exportações.


%% Modelo Allain
%TODO Comparar com a versão do artigo dele
No já mencionado modelo de \textcite{allain_tackling_2015}, os gastos não criadores de capacidade produtiva são os gastos do governo e são financiados por impostos que se ajustam endogenamente para manter o saldo primário equilibrado\footnote{
	Dentre os resultados particulares do modelo de \textcite{allain_tackling_2015}, pontuam-se os efeitos contra-cíclicos do gasto público sobre o nível de atividade e seu papel enquanto estabilizador automático do crescimento.
}.  
\textcite{hein_autonomous_2018}, por sua vez, critica este modelo por não incluir uma discussão sobre a dinâmica do \textit{déficit} e da dívida pública no longo prazo. 
Sendo assim, insere o fechamento de \textcite{allain_tackling_2015} no arcabouço contábil da metodologia SFC de modo que os gastos do governo passam a ser financiados por crédito e emissão monetária. Como consequência, \textcite{hein_autonomous_2018} afirma que este modelo passa a incluir o paradoxo da dívida, ou seja, redução da dívida pública como resultado de um aumento dos gastos do governo dado o aumento do consumo a partir da riqueza financeira. 
Dito isso, cabe realçar que neste modelo em particular, um aumento na taxa de crescimento dos gastos do governo afeta positivamente o grau de utilização que, por sua vez, não converge ao normal\footnote{
	Vale mencionar que uma das peculiaridades deste modelo é a endogeinização da distribuição funcional da renda pelo grau de utilização. No entanto, tal resultado pode decorrer da diferenciação feita por \textcite{hein_autonomous_2018} entre renda decorrente da produção e renda financeira.
}.
Tal resultado por ser visualizado por meio da equação do grau de utilização no longo prazo:
\begin{equation}
\label{Eq_Hein}
u^* = \frac{g_Z - \gamma}{\gamma_u}
\end{equation}
em que $g_Z$ é a taxa de crescimento dos gastos do governo, $\gamma$ representa os \textit{animal spirits} e $\gamma_u$ é a parcela induzida do investimento das firmas.
Em linhas gerais, a equação \ref{Eq_Hein} indica que o grau de utilização não converge ao normal.
No entanto, se os gastos autônomos crescerem a uma mesma taxa que o valor do \textit{animal spirits}, o grau de utilização será nulo.
Como a estabilidade deste modelo independe de $\gamma$, não existem restrições para esse parâmetro de modo que possa zerar o grau de utilização. Dito isso, conclui-se que tal equação pode estar incompleta e, assim, não se sabe o resultado particular reportado acima decorre desta especificação do grau de utilização diferente em relação ao modelo de \textcite{allain_tackling_2015} retomado abaixo
$$
u^* = \frac{g_Z - \gamma}{\gamma_u} + u_n
$$

%MODELO BROCHIER (2018): RIQUEZA FINANCEIRA ACUMULADA
Outro modelo SFC é o de \textcite{brochier_supermultiplier_2018} em que o gasto autônomo é o consumo financiado pela riqueza acumulada\footnote{
	Outro modelo com consumo a ser destacado é o de \textcite{nah_role_2019} %NAH E LAVOIE (2019): INFLAÇÃO E DISTRIBUIÇÃO ENDÓGENA
	que inclui inflação por conflito distributivo. Por mais que tal modelo apresente gastos autônomos como os demais nesta seção, a endogeinização da distribuição de renda elimina uma das hipóteses compartilhadas entre os modelos analisados e, portanto, compromente a comparação e por isso optou-se por não apresentá-lo em maiores detalhes.
}. 
Por mais que este modelo parta do fechamento do supermultiplicador sraffiano,
alguns dos resultados apresentados anteriormente se alteram: 
	(i) alterações na distribuição de renda impactam a taxa de acumulação no longo prazo; 
	(ii) aumento na propensão marginal a consumir a partir da renda disponível aumenta a taxa de crescimento de longo prazo; 
	(iii) grau de utilização converge ao normal; 
	(iv) aumento na propensão a consumir a partir da riqueza aumenta a taxa de acumulação. 
	Neste modelo, os paradoxos dos custos e da parcimônia são mantidos --- apesar dos mecanismos causais não estarem claros dado o grau de endogeneidade do sistema --- inclusive com o grau de utilização convergindo ao normal, configurando uma possível exceção ao que foi exposto até então.

%TODO: Dúvida - Mencionar artigo sobre distribuição de renda.

%No entanto, a menção anterior ao governo não é ocasional. MIMEO argumentam que a presença do governo no modelo atua como a geração de um gasto autônomo que persiste enquanto efeito riqueza. Desse modo, a reformulação deste modelo para uma economia sem governo gera os resultados esperados tais como aqueles apresentados anteriormente: (i) mudanças na distribuição de renda não afetam a taxa de acumulação; (ii) o mesmo vale para a propensão marginal a poupar (via propensão marginal a consumir a partir dos salários); (iii) grau de utilização converge ao desejado; (iv) mudanças na propensão marginal a consumir a partir da riqueza acumulada não alteram a taxa de acumulação. Portanto, feitas essas modificações, os resultados apresentados anteriormente são restaurados e as exceções são eliminadas\footnote{Outra modificação verificada por MIMEO é o desenho da política fiscal e, tal como na retirado do governo, alteram os resultados obtidos na simulação.}.

%TODO: Evidenciar que os resultados diferentes da Lídia não estão no modelo do Mandarino


%TESE MANDARINO: CONSUMO FINANCIADO POR CRÉDITO
Outro modelo na linha do anterior é o de \textcite{mandarino_financing_2018} em que o consumo dos trabalhadores é financiado por crédito --- adicionando um tratamento das relações financeiras ao modelo de \textcite{pariboni_household_2016} e de \textcite{fagundes_dinamica_2017} --- e obtém os resultados de longo prazo esperados dado o fechamento do supermultiplicador sraffiano (não há retornos aos paradoxos kaleckianos).
Vale destacar que este modelo é centrado nas condições de estabilidade do endividamento dos trabalhadores no longo prazo e conclui que aumentos da taxa de crescimento dos gastos autônomos, bem como na taxa de juros implicam em diminuição da taxa de endividamento dos trabalhadores e das firmas. 
Em outras palavras, tal como em \textcite{hein_autonomous_2018}, o modelo de \textcite{mandarino_financing_2018} apresenta o paradoxo da dívida.


%MODELO NAH AND LAVOIE: EXPORTAÇÃO
Analisados o consumo autônomo (financiado por crédito e riqueza) e os gastos do governo, restam os demais componentes da demanda agregada.
No modelo de \textcite{nah_long-run_2017}, semelhante ao de \textcite{dejuan_hidden_2017}, as exportações desempenham o papel dos gastos autônomos. Mais especificamente, é uma proposta para estender a contribuição de \textcite{serrano_sraffian_1995} para o caso de uma economia aberta suficientemente pequena. Os resultados de longo prazo são iguais aos apresentados anteriormente e por conta disso não serão repetidos. No entanto, este modelo se destaca por tentar reconciliar os resultados do supermultiplicador sraffiano com os regimes de crescimento da literatura kaleckiana, pontuando que os efeitos sobre o nível do produto estão sujeitos à sensibilidade da taxa de câmbio real a mudanças na distribuição de renda. 

%MODELO DUTT: INOVAÇÃO
% \textcite{dutt_observations_2018} afirma que são incapazes de fazer com que o investimento (criador de capacidade produtiva) seja determinante do crescimento no longo prazo tal como em Kalecki. Para tanto, inclui um componente de crescimento que expressa o progresso tecnológico determinado autonomamente. No entanto, tal formulação não faz com que o grau de utilização convirja ao normal e que a taxa de crescimento seja determinada pelos gastos autônomos uma vez que essa nova variável afeta a capacidade produtiva no longo prazo. Para garantir as propriedades do supermultiplicador, o progresso técnico é endogeneizado pelos gastos com P\&D ($g_R$) de forma que:
%$$
%g_I + g_R = g_S
%$$
%Neste modelo, uma vez cessados os efeitos do progresso tecnológico: 
%	(i) distribuição afeta a taxa de crescimento de médio prazo apenas; 
%	(ii) propensão marginal a poupar também não afeta o crescimento, mas determina a condição de estabilidade; 
%	(iii) grau de utilização converge ao normal; 
%	(iv) taxa de crescimento converge para a taxa de crescimento dos gastos com P\&D e o resultado se preserva com mais de um gasto autônomo. 
%Portanto, partindo desta formulação, o progresso tecnológico pode determinar o ritmo de crescimento no longo prazo sem afetar o investimento.

Por mais que estes modelos consigam dar atenção para diferentes gastos autônomos, destaca-se a escassez daqueles que tratam do investimento residencial em específico. 
Sendo assim, cabe a seção seguinte avaliar como incluí-lo na agenda de pesquisa dos modelos de crescimento liderados pela demanda.
