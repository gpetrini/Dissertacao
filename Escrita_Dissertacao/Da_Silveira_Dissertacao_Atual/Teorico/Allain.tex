\subsection{Um breve mapeamento da fronteira heterodoxa}
\label{Hibridos}


Ao longo desta seção, serão mapeados os modelos de crescimento, 
sejam eles kaleckianos ou sraffianos, liderados pelos gastos autônomos não criadores de capacidade produtiva ao setor privado ($Z$). Isso não implica que são os únicos modelos com gastos autônomos, mas sim, que são os modelos em que a dinâmica destes gastos não é esgotada no longo prazo\footnote{
	No que diz respeito ao consumo financiado por crédito, por exemplo, destacam-se os trabalhos de \textcite{dutt_maturity_2006}, \textcite{palley_inside_2010} e \textcite{hein_finance-dominated_2012}.
	No entanto, esses autores trabalham no arcabouço kaleckiano básico. Por isso, a estabilidade só é garantida se o consumo financiado por crédito a crescer a mesma taxa que a acumulação --- ou seja, este componente do consumo não pode ser tratado como de fato um gasto autônomo.
	Diante desta limitação, \textcite{pariboni_household_2016} argumenta que os gastos autônomos desempenham um papel passivo e sugere uma alternativa a partir do supermultiplicador sraffiano. Desta forma, a causalidade anterior é revertida de modo que a taxa de acumulação convirja gradualmente a taxa de crescimento consumo financiado por crédito.
}.
Para manter a comparatividade entre os modelos apresentados, serão realçados alguns dos resultados de \textbf{longo prazo}, são eles: (i) mudanças na distribuição de renda; (ii) alterações na propensão marginal propensão a poupar; (ii) convergência do grau de utilização ao normal; (iv) aumento da taxa de crescimento dos gastos autônomos. Por fim, as variáveis serão adaptadas de modo que $\gamma$ é o componente autônomo do investimento,  $z$ é a participação dos gastos autônomos ($Z$) na renda que crescem a taxa $g_Z$.
Feitas essas ressalvas e seguindo a tipologia de \textcite{cesaratto_technical_2003} e a categorização de \textcite{serrano_sraffian_1995}, tais gastos autônomos são: (i) gastos do governo; (ii) consumo financiado por crédito; (iii) Investimento residencial; (iv) Gastos com P\&D\footnote{
	%TODO Nota sobre especificidade do investimento em P\&D.
	Por mais que os gastos com P\&D são contabilmente considerados investimento das firmas, optou-se por incluí-los para avaliar modelos que partem do supermultiplicador sraffiano e consideram tais gastos como sendo os determinantes da dinâmica.
} e; (v) Exportações.


%% Modelo Allain
%TODO Comparar com a versão do artigo dele
No já mencionado modelo de \textcite{allain_tackling_2015}, os gastos do governo além de serem autônomos não criam capacidade ($Z$) e são financiados por impostos que se ajustam endogenamente para manter o saldo primário equilibrado\footnote{
	Dentre os resultados particulares do modelo de \textcite{allain_tackling_2015}, pontua-se os efeitos contra-cíclicos do gasto público sobre o nível de atividade e seu papel enquanto estabilizador automático do crescimento.
}.  
Os resultados de longo prazo são:
	(i) Mudanças na distribuição de renda geram alterações no nível, mas não na taxa de crescimento, eliminado o paradoxo dos custos; 
	(ii) o mesmo vale para alterações na propensão marginal a poupar e o paradoxo da parcimônia; 
	(iii) o grau de utilização converge ao normal e 
	(iv) aumento da taxa de crescimento dos gastos autônomos tem impactos positivos sobre a taxa de acumulação\footnote{
		A convergência do grau de utilização ao nível normal implica na eliminação dos paradoxos kaleckianos em termos de taxas. Além disso, vale destacar que tal convergência decorre de uma das soluções do modelo é a equivalência entre o componente autônomo do investimento e a taxa de crescimento dos gastos autônomos. Como consequência indireta da crítica do supermultiplicador sraffiano, não raro encontram-se modelos kaleckianos (não-convencionais) com convergência ao grau de utilização normal que destacam a manutenção dos resultados canônicos na média:
	
		\begin{citacao}
			Thus, \textbf{on average}, the rate of utilization and the growth rate of the economy are higher than at the starting and terminal points of the traverse. Thus, what these Sraffians are telling us is that more attention should be paid to the average values
			achieved during the traverse than to the terminal points. \cite[p.~408, grifos adicionados]{lavoie_post-keynesian_2015}
	\end{citacao}
	
}.


%MODELO HEIN (2018): GASTOS DO GOVERNO E SUSTENTABILIDADE DA DÍVIDA
Por mais que o modelo de \textcite{allain_tackling_2015} inclua os gastos do governo como sendo os gastos autônomos e preserve as características dos modelos kaleckianos (em nível), \textcite{hein_autonomous_2018} argumenta que não inclui uma discussão sobre a dinâmica do \textit{déficit} e da dívida pública no longo prazo. 
Dito isso, insere as modificações de \textcite{allain_tackling_2015} no arcabouço contábil da metodologia \textit{Stock-Flow Consistent} (adiante, SFC) de modo que os gastos do governo passam a ser financiados por crédito e emissão monetária\footnote{
	Como consequência, \textcite{hein_autonomous_2018} afirma que este modelo permite incluir o paradoxo da dívida, ou seja, redução da dívida pública como resultado de um aumento dos gastos do governo dado o aumento do consumo a partir da riqueza financeira. 
}. Uma distinção deste modelo é que o autor julga não ser razoável, dada a incerteza keynesiana fundamental, que o grau de utilização convirja ao normal no longo prazo\footnote{
	Dentre as equações para o equilíbrio de longo prazo, cabe mencionar àquela que diz respeito ao grau de utilização. Adaptando as variáveis,
	$$
	u = \frac{g_z - \gamma}{\gamma_u}
	$$
	que indica que o grau de utilização não converge ao nível normal e pode se manter persistentemente em um patamar elevado a depender dos parâmetros. Além disso, se os gastos autônomos crescerem a uma mesma taxa que o valor do \textit{animal spirits}, o grau de utilização será nulo. Em outras palavras, como a estabilidade independe de ($\gamma$), não existem restrições para esse parâmetro de modo que possa zerar o grau de utilização. Dito isso, conclui-se que tal equação deve estar incompleta e deveria ser
	$$
	u = \frac{g_z - \gamma}{\gamma_u} + u_n
	$$
}. 
Dito isso, cabe realçar os resultados que tocam os objetivos desta seção\footnote{
	Dentre os resultados restritos a esse modelo, destaca-se:  
	(a) Mudanças nos \textit{animal spirits} afetam negativamente o grau de utilização, mas não possuem efeitos na taxa de crescimento; 
	(b) redução do \textit{déficit} e da dívida do governo em decorrência de: 
	(b.i) aumento nos \textit{animal spirits}; 
	(b.ii) diminuição da propensão marginal a poupar e aumento da propensão a consumir a partir da riqueza. 
	Além disso, o autor conclui, tal como \textcite[Capítulo 11]{godley_growth_2011} e \textcite{arestis_effectiveness_2012}, que uma política fiscal ativa pode atuar para aquecer a economia sem implicar em insustentabilidade da dívida pública.
}: 
	(i) Mudanças na distribuição de renda não afetam a taxa de crescimento de longo prazo; 
	(ii) o mesmo vale para mudanças na propensão marginal a poupar e a consumir a partir da riqueza; 
	(iii) Aumento na taxa de crescimento dos gasto do governo ($g_Z$) afetam positivamente o grau de utilização que, por sua vez, não converge ao normal\footnote{
		Vale mencionar que uma das peculiaridades deste modelo é a endogeinização da distribuição funcional da renda pelo grau de utilização. No entanto, tal resultado pode decorrer da diferenciação feita por \textcite{hein_autonomous_2018} entre renda decorrente da produção e renda financeira.
		}; 
	(iv) A taxa de crescimento da economia converge para a taxa de crescimento dos gastos autônomos no longo prazo. 

%MODELO BROCHIER (2018): RIQUEZA FINANCEIRA ACUMULADA
Outro modelo SFC com gasto do governo é o de \textcite{brochier_supermultiplier_2018}.
A principal diferença em relação é a endogeinização dos gastos autônomos por meio do consumo a partir da riqueza financeira acumulada\footnote{
	Outro modelo com consumo a ser destacado é o de \textcite{nah_role_2019} %NAH E LAVOIE (2019): INFLAÇÃO E DISTRIBUIÇÃO ENDÓGENA
	que inclui inflação por conflito distributivo. Por mais que tal modelo apresente gastos autônomos como os demais nesta seção, a endogeinização da distribuição de renda elimina uma das hipóteses compartilhadas entre os modelos analisados e, portanto, compromente a comparação e deve ser discutido a parte.
}. 
Alguns dos resultados apresentados anteriormente se alteram: 
	(i) alterações na distribuição de renda impactam a taxa de acumulação no longo prazo; 
	(ii) aumento na propensão marginal a consumir a partir da renda disponível aumenta a taxa de crescimento de longo prazo; 
	(iii) grau de utilização converge ao normal; 
	(iv) aumento na propensão a consumir a partir da riqueza (componente correspondente ao $Z$) aumenta a taxa de acumulação. 
	Neste modelo, os paradoxos dos custos e da parcimônia são mantidos --- apesar dos mecanismos causais não estarem claros --- inclusive com o grau de utilização convergindo ao desejado, configurando uma possível exceção ao que foi exposto até então.

%TODO: Dúvida - Mencionar artigo sobre distribuição de renda.

%No entanto, a menção anterior ao governo não é ocasional. MIMEO argumentam que a presença do governo no modelo atua como a geração de um gasto autônomo que persiste enquanto efeito riqueza. Desse modo, a reformulação deste modelo para uma economia sem governo gera os resultados esperados tais como aqueles apresentados anteriormente: (i) mudanças na distribuição de renda não afetam a taxa de acumulação; (ii) o mesmo vale para a propensão marginal a poupar (via propensão marginal a consumir a partir dos salários); (iii) grau de utilização converge ao desejado; (iv) mudanças na propensão marginal a consumir a partir da riqueza acumulada não alteram a taxa de acumulação. Portanto, feitas essas modificações, os resultados apresentados anteriormente são restaurados e as exceções são eliminadas\footnote{Outra modificação verificada por MIMEO é o desenho da política fiscal e, tal como na retirado do governo, alteram os resultados obtidos na simulação.}.

%TODO: Evidenciar que os resultados diferentes da Lídia não estão no modelo do Mandarino


%TESE MANDARINO: CONSUMO FINANCIADO POR CRÉDITO
Outro modelo na linha do anterior --- mas sem retorno dos paradoxos kaleckianos --- é o de \textcite{mandarino_financing_2018} em que o consumo dos trabalhadores é financiado por crédito, adicionando um tratamento das relações financeiras ao modelo de \textcite{pariboni_household_2016} e de \textcite{fagundes_dinamica_2017}. No que diz respeito às implicações para o longo prazo, pontua-se\footnote{
	O modelo de \textcite{mandarino_financing_2018} apresenta diferentes cenários mas foram realçadas as conclusões que tangenciam os quatro pontos de comparação, qual sejam, mudanças: (i) na distribuição; (ii) na propensão marginal a poupar; (iii) do grau de utilização; (iv) decorrentes das variações de $g_Z$.
}: 
	(i) não foram simulados os efeitos de mudanças na distribuição de renda; 
	(ii) diminuição na propensão marginal média a poupar (via aumento na propensão marginal a consumir dos capitalistas) afeta negativamente o nível de atividade mas não a taxa de crescimento de longo prazo e; 
	(iii) grau de utilização converge ao desejado em todos os cenários; 
	(iv) aumento em $g_Z$ aumenta a taxa de acumulação de longo prazo.
Adicionalmente, este modelo é centrado nas condições de estabilidade do endividamento dos trabalhadores no longo prazo e conclui que aumentos da taxa de crescimento dos gastos autônomos bem como na taxa de juros implicam em diminuição da taxa de endividamento dos trabalhadores e das firmas. 


% MODELO NAH E LAVOIE (2019): CONSUMO AUTÔNOMO E INFLAÇÃO POR CONFLITO DISTRIBUTIVO E ENDOGEINIZAÇÃO DA DISTRIBUIÇÃO

%MODELO NAH AND LAVOIE: EXPORTAÇÃO
Analisados o consumo autônomo (financiado por crédito e riqueza) e os gastos do governo, restam os demais componentes da demanda agregada.
No modelo de \textcite{nah_long-run_2017}, semelhante ao de \textcite{dejuan_hidden_2017}, as exportações desempenham o papel dos gastos autônomos. Mais especificamente, é uma proposta para estender a contribuição de \textcite{serrano_sraffian_1995} para o caso de uma economia aberta suficientemente pequena. Os resultados de longo prazo são iguais aos apresentados anteriormente e por conta disso não serão repetidos. No entanto, este modelo se destaca pelo regime de acumulação pode ser caracterizado como \textit{wage-} ou \textit{profit-led} a depender da sensibilidade da taxa de câmbio real a mudanças na distribuição de renda. 

%MODELO DUTT: INOVAÇÃO
Por mais que estes modelos consigam dar atenção para diferentes gastos autônomos, \textcite{dutt_observations_2018} afirma que são incapazes de fazer com que o investimento (criador de capacidade produtiva) seja determinante do crescimento no longo prazo tal como em Kalecki. Para tanto, inclui um componente de crescimento que expressa o progresso tecnológico determinado autonomamente. No entanto, tal formulação não faz com que o grau de utilização convirja ao normal e que a taxa de crescimento seja determinada pelos gastos autônomos uma vez que essa nova variável afeta a capacidade produtiva no longo prazo. Para garantir as propriedades do supermultiplicador, o progresso técnico é endogeneizado pelos gastos com P\&D ($g_R$) de forma que:
$$
g_I + g_R = g_S
$$
Neste modelo, uma vez cessados os efeitos do progresso tecnológico: 
	(i) distribuição afeta a taxa de crescimento de médio prazo apenas; 
	(ii) propensão marginal a poupar também não afeta o crescimento, mas determina a condição de estabilidade; 
	(iii) grau de utilização converge ao normal; 
	(iv) taxa de crescimento converge para a taxa de crescimento dos gastos com P\&D e o resultado se preserva com mais de um gasto autônomo. 
Portanto, partindo desta formulação, o progresso tecnológico pode determinar o ritmo de crescimento no longo prazo sem afetar o investimento.

Apesar dessa variabilidade de modelos, destaca-se a escassez daqueles que tratam do investimento residencial em específico. 
Sendo assim, cabe a seção seguinte avaliar como incluí-lo na agenda de pesquisa dos modelos de crescimento liderados pela demanda.
