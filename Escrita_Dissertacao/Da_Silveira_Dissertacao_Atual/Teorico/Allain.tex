\subsection{Um breve mapeamento da fronteira heterodoxa}
\label{Hibridos}
Ao longo desta seção, serão mapeados os modelos de crescimento, 
sejam eles kaleckianos ou sraffianos, liderados pelos gastos autônomos não criadores de capacidade produtiva ao setor privado ($Z$). Isso não implica que são os únicos modelos com gastos autônomos, mas sim, que são os modelos em que a dinâmica destes gastos não é esgotada no longo prazo\footnote{
	No que diz respeito ao consumo financiado por crédito, por exemplo, destacam-se os trabalhos de \textcite{dutt_maturity_2006}, \textcite{palley_inside_2010} e \textcite{hein_finance-dominated_2012}.
	No entanto, esses autores trabalham no arcabouço kaleckiano básico. Por isso, a estabilidade só é garantida se o consumo financiado por crédito crescer a crescer a mesma taxa que a acumulação - ou seja, este componente do consumo não pode ser tratado como de fato um gasto autônomo.
	Diante desta limitação, \textcite{pariboni_household_2016} argumenta que os gastos autônomos desempenham um papel passivo e sugere uma alternativa a partir do supermultiplicador sraffiano. Desta forma, a causalidade anterior é revertida de modo que a taxa de acumulação convirja gradualmente a taxa de crescimento consumo financiado por crédito.
}.
Antes de prosseguir, no entanto, cabe destacar que por serem modelos na fronteira da literatura, podem não ser representativos do que se entende por modelo kaleckiano e, por conta disso, serão denominados ``não-tradicionais'' ao longo desta seção\footnote{Em especial,  serão investigados os modelos kaleckianos que introduziram mecanismos de ajuste do grau de utilização da capacidade e/ou os que seguem o supermultiplicador sraffiano com a inclusão dos referidos gastos autônomos.}.


Em linhas gerais, as modificações nos modelos kaleckianos estão associadas a algumas críticas envolvendo tanto a não convergência ao grau de utilização normal no longo prazo quanto a resolução parcial da instabilidade de Harrod \cites{dallery_kaleckian_2007}{skott_theoretical_2012}{hein_harrodian_2012}.
A partir da contribuição de \textcite{allain_tackling_2015}, as alterações no modelo canônico têm a inclusão de gastos autônomos como denominador comum, mas uma mediação se faz necessária. 
Tal como o autor pontua, os resultados são distintos a depender de quais gastos são considerados autônomos e isso será avaliado adiante. 
Dito isso, o objetivo da presente seção é destacar como os modelos incorporam os diferentes gastos autônomos não criadores de capacidade produtiva ao setor privado ($Z$). 

Para manter a comparatividade entre os modelos apresentados, serão realçados alguns dos resultados que dizem respeito a efeitos em comum no \textbf{longo prazo}, são eles: (i) mudanças na distribuição de renda; (ii) alterações na propensão marginal propensão à poupar; (ii) efeitos sobre o grau de utilização; (iv) impactos do aumento da taxa de crescimento dos gastos autônomos. Já aqueles resultados que são exclusivos do modelo analisado serão postos em evidência quando necessário. Por fim, as variáveis serão adaptadas de modo que $\gamma$ é o componente autônomo do investimento,  $z$ é a participação dos gastos autônomos ($Z$) na renda que crescem a taxa $g_Z$.


%% Modelo Allain
%TODO Comparar com a versão do artigo dele
Dado o ineditismo, inicia-se pela exposição do modelo de \textcite{allain_tackling_2015} em que os gastos do governo além de serem autônomos não criam capacidade ($Z$) e são financiados por impostos que se ajustam endogenamente para manter o saldo primário equilibrado.  Uma vez introduzidos os gastos do governo, a poupança agregada após os impostos torna-se\footnote{Fazendo as devidas mediações, chega-se à equação \ref{Poupanca_Super} apresentada anteriormente mas uma das etapas deve ser esclarecida. Em linha com a literatura kaleckiana, \textcite{allain_tackling_2015} define grau de utilização como sendo a razão entre renda e estoque de capital. Desse modo, 
	$$
	\frac{S}{K} = s\left(\frac{Y}{K} - \frac{Z}{K}\right) 
	$$
	Multiplicando pelo estoque de capital e dividindo pela renda:
	$$
	\frac{S}{Y} = s - \frac{Z}{Y}
	$$}:
$$
\frac{S}{Y} = s - z
$$
 Neste ponto, \textcite[p.~10]{allain_tackling_2015} segue \textcite{serrano_sraffian_1995} em que a presença dos gastos autônomos possibilitam o ajuste da identidade entre investimento e poupança pela participação desses gastos na renda e não por mudanças no grau de utilização. 

Dito isso, o autor prossegue para o médio prazo\footnote{
	Uma das distinções \textcite{allain_tackling_2015} são as caracterizações do curto, médio e longo prazo. O primeiro é definido pela não alteração dos gastos autônomos enquanto o segundo pode ser definido como aquele que tais gastos crescem a taxa exogenamente determinada. Por fim, o longo prazo é caracterizado por uma função de investimento Harrodiana com o grau de utilização convergindo ao desejado. Vale pontuar a distinção com a temporalidade encontrada em \textcite{freitas_growth_2015} em que a convergência ao grau de utilização normal se dá na \textit{fully-adjusted position} enquanto a convergência da taxa de crescimento a $g_Z$ é dada no longo prazo. Para manter a comparatividade entre os modelos kalekicanos não-tradicionais, adota-se a caracterização de \textcite{allain_tackling_2015} ao longo desta seção.
} em que a participação dos gastos do governo na renda ($z$)\footnote{A rigor, o autor define essa participação dos gastos autônomos normalizados pelo estoque de capital e não pela renda, mas tal apresentação não altera a exposição ao longo do texto.} varia de acordo com a diferença entre as taxas de crescimento dos gastos autônomos e a efetiva ($g^*$):
\begin{equation}
    \Delta z = z (g_Z - g^*)
\end{equation}
Assim, quando a taxa de crescimento efetiva da economia se difere da taxa de crescimento dos gastos autônomos ($g^*\neq g_Z$)  haverá uma variação na participação dos gastos públicos ($\Delta z$), impactando a demanda agregada e a poupança. No médio prazo, em que a taxa de crescimento converge a taxa dos gastos autônomos, o mecanismo de ajuste de $z$ é encerrado e o grau de utilização é dado por:

\begin{equation}
u^* = \frac{g_z - \gamma}{\gamma_u} + u_N
\end{equation}
Esta equação evidencia que se, e somente se, a expectativa tendencial de crescimento ($\gamma$) for igual à taxa de gastos autônomos, o grau de utilização convergirá ao normal no médio prazo e, portanto, é meramente acidental. No entanto, a convergência do grau de utilização ao normal é uma característica do longo prazo que decorre do princípio de ajuste do estoque de capital em que a taxa de crescimento esperada se adequa aos distanciamentos entre o grau de utilização efetivo e normal. Em linhas gerais, para evitar o deflagrar da instabilidade de Harrod é necessário que o investimento deixe de ser autônomo: 
\begin{equation}
\label{eqAllain}
    \dot \gamma = \phi\gamma_u(u^* - u_N)
\end{equation}
em que $\phi$ é um fator de ajuste positivo e suficientemente pequeno\footnote{
	Além disso, \textcite[p.~14]{allain_tackling_2015} argumenta que a novidade  é o parâmetro  de ajustamento positivo ($\phi > 0$) que além de não implicar na instabilidade Harrodiana, é também condição de estabilidade do modelo.
} de modo que:
$$
\gamma = g = g_Z \Leftrightarrow u^* \to u_N
$$
A razão pela qual este modelo não incorre na instabilidade harrodiana é apresentada a seguir.

Partindo do equilíbrio de médio prazo ($\dot z = 0$ com a solução assintótica para $Z >0$\footnote{Ver \textcite[Apêndice A]{allain_tackling_2015} para verificar que com $Z = 0$, retorna-se ao modelo kaleckiano convencional em que a instabilidade harrodiana é reestabelecida.}\footnote{Para maiores detalhes, ver \textcite{fagundes_role_2017}.}) e supondo um aumento na taxa esperada de crescimento ($\uparrow\gamma$), obtém-se um cenário em que a taxa efetiva é maior que a taxa dos gastos autônomos ($g^* > g_z$). Como consequência, a participação dos gastos autônomos na renda diminui de modo que  a poupança agregada aumenta. Essa redução ($\Downarrow z$), por sua vez, tem um efeito negativo tanto sobre a taxa de crescimento efetiva quanto sobre o grau de utilização. Esse processo se esgota com a taxa de crescimento efetiva se ajustando a taxa dos gastos autônomos ($g^* = g_z$) mas com um grau de utilização menor  (equilíbrio de médio prazo reestabelecido). 

No longo prazo, instaura-se o princípio do ajuste do estoque de capital de modo que o grau de utilização converge ao normal ($u^* \to u_N$) em que: (i) Mudanças na distribuição de renda geram alterações no nível, mas não na taxa de crescimento, eliminado o paradoxo dos custos; (ii) o mesmo vale para alterações na propensão marginal a poupar e o paradoxo da parcimônia; (iii) o grau de utilização converge ao normal no longo prazo e não é afetado por modificações no comportamento do investimento/poupança dada a introdução de $Z$ que cresce exogenamente e dado o ajuste na propensão marginal a investir e (iv) aumento da taxa de crescimento dos gastos autônomos tem impactos positivos sobre a taxa de acumulação\footnote{
	Tal como destacado na seção anterior, a convergência do grau de utilização ao nível normal implica na eliminação dos paradoxos kaleckianos em termos de taxas. Além disso, vale destacar que tal convergência decorre de uma das soluções do modelo é a equivalência entre o componente autônomo do investimento e a taxa de crescimento dos gastos autônomos. Como consequência indireta da crítica do supermultiplicador sraffiano, não raro encontra-se modelos kaleckianos (não-tradicionais) com convergência ao grau de utilização normal que destacam a manutenção dos resultados canônicos na média:
	
	\begin{citacao}
		Thus, \textbf{on average}, the rate of utilization and the growth rate of the
		economy are higher than at the starting and terminal points of the traverse. Thus, what
		these Sraffians are telling us is that more attention should be paid to the average values
		achieved during the traverse than to the terminal points.
		\cite[p.~408, grifos adicionados]{lavoie_post-keynesian_2015}
	\end{citacao}
	
}.  Dentre os resultados particulares do modelo de \textcite{allain_tackling_2015}, pontua-se os efeitos contra-cíclicos do gasto público sobre o nível de atividade e seu papel enquanto estabilizador automático do crescimento.

%MODELO DUTT: Gastos do governo
%MODELO GODLEY E LAVOIE (2011): Gastos do governo

%MODELO Superhavelmo / Freitas?

%MODELO HEIN (2018): GASTOS DO GOVERNO E SUSTENTABILIDADE DA DÍVIDA

Por mais que o modelo de \textcite{allain_tackling_2015} inclua os gastos do governo como sendo os gastos autônomos e preserve as características dos modelos kaleckianos (em nível), \textcite{hein_autonomous_2018} argumenta que não inclui uma discussão sobre a dinâmica do \textit{déficit} e da dívida pública no longo prazo. Os gastos do governo, agora financiados por crédito e emissão monetária, crescem a uma taxa exógena tal como em \textcite{allain_tackling_2015}. Uma distinção deste modelo é que o autor julga não ser razoável, dada a incerteza keynesiana fundamental, que o grau de utilização convirja ao normal no longo prazo\footnote{
	Dentre as equações para o equilíbrio de longo prazo, cabe mencionar àquela que diz respeito ao grau de utilização. Adaptando as variáveis,
	$$
	u = \frac{g_z - \gamma}{\gamma_u}
	$$
	que indica que o grau de utilização não converge ao nível normal e pode se manter persistentemente em um patamar elevado a depender dos parâmetros. Além disso, se os gastos autônomos crescerem a uma mesma taxa que o valor do \textit{animal spirits}, o grau de utilização será nulo. Em outras palavras, como a estabilidade independe de ($\gamma$), não existem restrições para esse parâmetro de modo que possa zerar o grau de utilização. Dito isso, conclui-se que tal equação deve estar incompleta e deveria ser
	$$
	u = \frac{g_z - \gamma}{\gamma_u} + u_n
	$$
}. 
Dito isso, cabe realçar os resultados que tocam os objetivos desta seção: 
	(i) Mudanças na distribuição de renda não afetam a taxa de crescimento de longo prazo; 
	(ii) o mesmo vale para mudanças na propensão marginal a poupar e a consumir a partir da riqueza; 
	(iii) Aumento na taxa de crescimento dos gasto do governo ($g_z$) afetam positivamente o grau de utilização\footnote{
		Vale mencionar que uma das peculiaridades deste modelo é a endogeinização da distribuição funcional da renda pelo grau de utilização. No entanto, tal resultado pode decorrer da diferenciação feita por \textcite{hein_autonomous_2018} entre renda decorrente da produção e renda financeira.
		}; 
	(iv) o mesmo vale para a taxa de crescimento de longo prazo. 
Dentre os resultados restritos a esse modelo, destaca-se:  
	(a) Mudanças nos \textit{animal spirits} afetam negativamente o grau de utilização mas não possuem efeitos na taxa de crescimento; 
	(b) redução do \textit{déficit} e da dívida do governo em decorrência de: 
		(b.i) aumento nos \textit{animal spirits}; 
		(b.ii) diminuição da propensão marginal a poupar e aumento da propensão a consumir a partir da riqueza. 

Por se tratar de um modelo do tipo \textit{Stock-Flow Consistent} (adiante, SFC), a dívida do governo é tratada como riqueza financeira privada. Nesses termos, \textcite{hein_autonomous_2018} afirma que este modelo permite incluir o paradoxo da dívida, ou seja, redução da dívida pública como resultado de um aumento dos gastos do governo dado o aumento do consumo a partir da riqueza financeira. Por fim, o autor conclui, tal como \textcite[Capítulo 11]{godley_growth_2011} e \textcite{arestis_effectiveness_2012}, que uma política fiscal ativa pode atuar para aquecer a economia sem implicar em insustentabilidade da dívida pública.

%MODELO BROCHIER (2018): RIQUEZA FINANCEIRA ACUMULADA
Um modelo SFC com supermultiplicador que merece ser pontuado é o de \textcite{brochier_supermultiplier_2018}. Os gastos autônomos foram endogeneizados --- e não induzidos, como diria \textcite{nikiforos_comments_2018} --- e são determinados pelo consumo a partir da riqueza financeira acumulada em uma economia fechada e com governo. Dentre os objetivos do modelo, destaca-se a inclusão de um tratamento das relações financeiras ao supermultiplicador sraffiano. Como consequência, alguns dos resultados apresentados anteriormente se alteram: (i) alterações na distribuição de renda impactam a taxa de acumulação no longo prazo; (ii) aumento na propensão marginal a consumir a partir da renda disponível (semelhante a uma redução na propensão marginal a consumir) aumenta a taxa de crescimento de longo prazo; (iii) grau de utilização converge ao normal; (iv) aumento na propensão a consumir a partir da riqueza (componente correspondente ao $Z$) aumenta a taxa de acumulação. Desse modo, este modelo apresenta uma exceção importante em que os paradoxos dos custos e da parcimônia são mantidos inclusive com o grau de utilização convergindo ao desejado em um modelo com governo, configurando uma possível exceção ao que foi exposto até então.

%TODO: Dúvida - Mencionar artigo sobre distribuição de renda.

%No entanto, a menção anterior ao governo não é ocasional. MIMEO argumentam que a presença do governo no modelo atua como a geração de um gasto autônomo que persiste enquanto efeito riqueza. Desse modo, a reformulação deste modelo para uma economia sem governo gera os resultados esperados tais como aqueles apresentados anteriormente: (i) mudanças na distribuição de renda não afetam a taxa de acumulação; (ii) o mesmo vale para a propensão marginal a poupar (via propensão marginal a consumir a partir dos salários); (iii) grau de utilização converge ao desejado; (iv) mudanças na propensão marginal a consumir a partir da riqueza acumulada não alteram a taxa de acumulação. Portanto, feitas essas modificações, os resultados apresentados anteriormente são restaurados e as exceções são eliminadas\footnote{Outra modificação verificada por MIMEO é o desenho da política fiscal e, tal como na retirado do governo, alteram os resultados obtidos na simulação.}.

%TODO: Evidenciar que os resultados diferentes da Lídia não estão no modelo do Mandarino


%TESE MANDARINO: CONSUMO FINANCIADO POR CRÉDITO
Outro modelo na linha do anterior --- mas sem retorno dos paradoxos kaleckianos --- é o de \textcite{mandarino_financing_2018} em que o consumo dos trabalhadores é financiado por crédito ($Z$)\footnote{
	% LAVOIE (2016): CONSUMO AUTÔNOMO
	Vale a menção ao modelo de \textcite{lavoie_convergence_2016} que obtém resultados semelhantes aos de \textcite{allain_tackling_2015} para o caso do consumo dos capitalistas como gasto autônomo. Outro modelo com consumo a ser destacado é o de \textcite{nah_role_2019} %NAH E LAVOIE (2019): INFLAÇÃO E DISTRIBUIÇÃO ENDÓGENA
	que inclui inflação por conflito distributivo. Por mais que tal modelo apresente gastos autônomos como os demais nesta seção, a endogeinização da distribuição de renda elimina uma das hipóteses compartilhadas entre os modelos analisados e, portanto, compromente a comparação e deve ser discutido a parte.
}, adicionando um tratamento das relações financeiras ao modelo de \textcite{pariboni_household_2016} e de \textcite{fagundes_dinamica_2017}. No que diz respeito às implicações para o longo prazo, pontua-se\footnote{
	O modelo de \textcite{mandarino_financing_2018} apresenta diferentes cenários mas foram realçadas as conclusões que tangenciam os quatro pontos de comparação, qual sejam, mudanças: (i) na distribuição; (ii) na propensão marginal a poupar; (iii) do grau de utilização; (iv) decorrentes das variações de $g_Z$.
}: 
	(i) não foram simulados os efeitos de mudanças na distribuição de renda; 
	(ii) diminuição na propensão marginal média a poupar (via aumento na propensão marginal a consumir dos capitalistas) afeta negativamente o nível de atividade mas não a taxa de crescimento de longo prazo e; 
	(iii) grau de utilização converge ao desejado em todos os cenários; 
	(iv) aumento em $g_Z$ aumenta a taxa de acumulação de longo prazo.
Adicionalmente, este modelo é centrado nas condições de estabilidade do endividamento dos trabalhadores no longo prazo e conclui que aumentos da taxa de crescimento dos gastos autônomos ($\uparrow g_Z$) bem como na taxa de juros implicam em diminuição da taxa de endividamento dos trabalhadores e das firmas. 


% MODELO NAH E LAVOIE (2019): CONSUMO AUTÔNOMO E INFLAÇÃO POR CONFLITO DISTRIBUTIVO E ENDOGEINIZAÇÃO DA DISTRIBUIÇÃO

%MODELO NAH AND LAVOIE: EXPORTAÇÃO
Analisados o consumo autônomo (financiado por crédito e riqueza) e os gastos do governo, restam os demais componentes da demanda agregada.
No modelo de \textcite{nah_long-run_2017}, semelhante ao de \textcite{dejuan_hidden_2017}, as exportações desempenham o papel dos gastos autônomos. Mais especificamente, é uma proposta para estender a contribuição de \textcite{serrano_sraffian_1995} para o caso de uma economia aberta suficientemente pequena. Os resultados de longo prazo são iguais aos apresentados anteriormente e por conta disso não serão repetidos. No entanto, este modelo se destaca pelo regime de acumulação pode ser caracterizado como \textit{wage-} ou \textit{profit-led} a depender da sensibilidade da taxa de câmbio real a mudanças na distribuição de renda. 

%MODELO DUTT: INOVAÇÃO
Apesar dessa variabilidade de modelos, \textcite{dutt_observations_2018} afirma que são incapazes de fazer com que o investimento (criador de capacidade produtiva) seja determinante do crescimento no longo prazo tal como em Kalecki. Para tanto, inclui um componente de crescimento que expressa o progresso tecnológico determinado autonomamente ($\gamma$). No entanto, tal formulação não faz com que o grau de utilização convirja ao normal e que a taxa de crescimento seja determinada pelos gastos autônomos uma vez que essa nova variável afeta a capacidade produtiva no longo prazo. Para garantir as propriedades do supermultiplicador, o progresso técnico é endogeneizado pelos gastos com P\&D ($g_R$) de forma que:
$$
g_I + g_R = g_S
$$
Neste modelo, uma vez cessados os efeitos do progresso tecnológico ($\dot \gamma = 0$): 
	(i) distribuição afeta a taxa de médio prazo apenas; 
	(ii) propensão marginal a poupar também não afeta o crescimento, mas determina a condição de estabilidade; 
	(iii) grau de utilização converge ao normal; 
	(iv) taxa de crescimento converge para $g_Z$ e o resultado se preserva com mais de um gasto autônomo. Portanto, partindo desta formulação, o progresso tecnológico pode determinar o ritmo de crescimento no longo prazo sem afetar o investimento.

No que tange o investimento residencial, destaca-se a escassez da literatura dos modelos de crescimento liderados pela demanda que tratam deste gasto que, como visto na seção \ref{RevResidencial}, restringem sua autonomia ao crescimento populacional \cite{gowans_introducing_2014}. Uma exceção é o trabalho de \textcite{teixeira_crescimento_2015} ---  também \textit{à la} supermultiplicador sraffiano --- em que elabora a taxa própria de juros do imóveis (Taxa Própria, $own$) definida como a taxa de juros hipotecária ($r_{mo}$) deflacionada pela inflação dos imóveis ({$\dot p_h$}) de modo que a taxa de crescimento do investimento residencial ($g_Z$) é dada por:
$$
g_Z = \phi_0 - \phi_1 \overbrace{\left(\frac{1+r_{mo}}{1+\dot p_h} - 1\right)}^{\text{Taxa Própria}}
$$

\begin{equation}
g_Z = \phi_0 - \phi_1\cdot own
\end{equation}
em que os $\phi_i$s são parâmetros e cujo termo em parênteses é a Taxa Própria. 
O primeiro parâmetro se refere aos determinantes de longo prazo (\textit{e.g.} arranjos institucionais do mercado imobiliários e de crédito) enquanto o segundo capta a demanda por imóveis decorrente das expectativas de ganhos de capital resultantes da especulação com o estoque de imóveis existente e diz respeito ao ciclo econômico.

Tal taxa real de juros, argumenta, é a taxa de juros relevante para os demandantes de casas uma vez que os detentores de um ativo levam seu preço em consideração no processo decisório já que sua variação pode gerar perdas/ganhos de capital \cite[p.~144]{teixeira_crescimento_2015}.
Em outras palavras, a taxa de juros das hipotecas capta o serviço da dívida para os ``investidores'' (neste caso, famílias) enquanto a variação do preço dos imóveis permite incorporar mudança no patrimonio líquido. Portanto, aufere o custo real em imóveis de se comprar imóveis \cite[p.~53]{teixeira_crescimento_2015}. 

Tal proposta, portanto, lança luz sobre a influência da inflação imobiliária na construção de novos imóveis e, de acordo com o supermultiplicador sraffiano, sobre o produto como um todo. 
Desse modo, a partir da taxa própria de juros do imóveis desenvolvida por \textcite{teixeira_crescimento_2015} é possível evidenciar a importância do investimento residencial para além do ciclo e estendê-la para o longo prazo.  
Como mostrado na seção anterior, o investimento residencial é um componente de gasto autônomo e, portanto,
esta é uma forma apropriada para especificar sua taxa de crescimento.
Dito isso, resta a seção seguinte selecionar o caminho a ser adotado.



%Por fim, resta discutir a autonomia\footnote{
%	Como será melhor discutido na seção \ref{RevResidencial}, o tratamento do investimento residencial não é consensual pela literatura. Dito isso, vale destacar uma questão menos controversa, qual seja, tal gasto não cria capacidade produtiva ao setor privado. Esta conclusão, no entanto, não deve ser estendida sem as devidas mediações a construção de escritórios como bem pontua \textcite{duesenberry_investment_1958}. No entanto, tais questões fogem dos objetivos da presente investigação, cabendo apenas reforçar a não-criação de capacidade produtiva decorrente da construção de novas residências.
%} deste componente da demanda agregada e isso fica a cargo da seção seguinte.
