\titleformat{\chapter}[display]{\normalfont\huge}{\appendixname{} \thechapter}{20pt}{\bfseries\huge}
\chapter{Modelo SFC}
\label{Append_Fraction}


\section{\textit{Fraction} e o investimento residencial}

O objetivo deste apêndice é mostrar que a inclusão de um gasto autônomo não criador de capacidade produtiva ao setor privado não é condição suficiente para que a propensão marginal ($s$) e média a poupar ($S/Y$) sejam distintas. Tal como no corpo do texto, seja $Y$ a renda, $C$ o consumo induzido, $I_t$ o investimento total e $Z$ os gastos autônomos que serão distindos em cada uma das seções subsequentes.

\subsection{Gastos ``despoupadores''}

Considere que, momentaneamente, os gastos autônomos não criam poupança como o consumo financiado por crédito. 
$$
Y = C + I_t + Z
$$
Partindo da identidade contábil entre investimento e poupança
$$
S = Y - C - Z
$$
Para o caso mais simplificado em que a propensão marginal a consumir a partir do salários é igual a unidade de modo que o consumo induzido é dado por
$$
C = \omega\cdot Y
$$
em que $\omega$ é a participação dos salários na renda. Substituindo na equação anterior,
$$
S = (1-\overline \omega)Y - Z \Rightarrow S = s\cdot Y - Z
$$
Por fim, dividindo a equação anterior pela renda, obtém-se a propensão média a poupar em função dos gastos autônomos:
\begin{equation}
\label{_Fraction}
    \frac{S}{Y} = s - \frac{Z}{Y}
\end{equation}
A equação \ref{_Fraction} explicita que na presença dos gastos autônomos não criadores de capacidade produtiva, propensão marginal e média a poupar são distintas. No entanto, tal afirmação exige uma qualificação adicional. Tais gastos precisam também não serem geradores de poupança, caso contrário, as propensões a poupar são idênticas e, portanto, a \textit{fraction} é igual a unidade.

\subsection{Investimento residencial}

Seguindo os mesmos procedimentos, mas com os gastos autônomos sendo agora o investimento residencial enquanto o consumo volta a ser totalmente induzido. Com isso, o investimento da economia é composto por duas parcelas, o investimento das firmas ($I_f$) e das famílias ($I_h = Z$):
$$
I_t = I_f + I_h
$$
De modo que a renda é determinada por:
$$
Y = C + I_f + Z
$$
Mais uma vez, partindo da identidade entre poupança e investimento
$$
S = Y - C
$$
$$
S = Y - \overline \omega\cdot Y
$$
$$
S = sY
$$
Dividindo a equação anterior pela renda, obtém-se que a propensão média e marginal a poupar são idênticas
$$
\frac{S}{Y} = s = \frac{I_t}{Y}
$$
De modo que a \textit{fraction} seja igual a unidade
$$
\frac{\frac{S}{Y}}{s} = 1
$$
A explicação deste resultado decorre pelo investimento residencial ser contabilmente investimento e, portanto, não é um gasto autônomo ``despoupador''. 

Tal conclusão, no entanto, é problemática por duas razões: (i) diferentemente dos modelos com supermultiplicador apresentados, a \textit{fraction} não é a variável de fechamento; (ii) implica na não replicabilidade do fato estilizado da relação positiva entre crescimento e taxa de investimento. Ambos os pontos podem ser mostrados conjuntamente. Partindo da participação dos componentes da demanda na renda,
$$
\omega + h + \frac{I_h}{Y} = 1
$$
Da equação acima, destaca-se que a propensão marginal a consumir é determinada exogenamente e o mesmo vale para a participação do investimento total, ou seja
$$
\overline \omega + \overline s = 1
$$
Desse modo, um aumento na taxa de investimento das famílias implica necessariamente na redução do investimento produtivo:
$$
\frac{I_t}{Y} = \frac{I_f + I_h}{Y} = s
$$
Isso implica que o investimento e a renda crescem a uma mesma taxa e, portanto, não reproduz o fato estilizado da relação positiva entre crescimento e taxa de investimento. Dito isso e retomando a equação \ref{EqGeral}:
$$
\frac{\overline s\cdot \overline u_N}{v} = \overline g_Z = h\frac{\overline u_N}{\overline v}
$$

\begin{equation}
h = \frac{\overline g_Z\cdot \overline u_N}{\overline v}
\end{equation}
\begin{equation}
    \label{Fechamento_Modelo}
\frac{I_h}{Y} = 1 - \overline\omega - \frac{\overline g_Z\cdot \overline u_N}{\overline v}
\end{equation}
Portanto, como o investimento residencial cresce a uma taxa exógena, é a participação dos gastos autônomos que fecha o modelo como indicado pela equação \ref{Fechamento_Modelo}. 

\section{Simulações}


\begin{table}[H]
	\centering
	\caption{Parâmetros das simulações}
	\label{Resumo_Simulacao}
	\begin{tabular}{c}
\toprule
{} &  Base scenario &  $\Delta \phi_0$ &  $\Delta \omega$ &   $\Delta rm$ &  $\Delta p_h$ \\
\midrule
\textbf{alpha    } &   1,000000e+00 &     1,000000e+00 &     1,000000e+00 &  1,000000e+00 &  1,000000e+00 \\
\textbf{gamma_F  } &   8,000000e-02 &     8,000000e-02 &     8,000000e-02 &  8,000000e-02 &  8,000000e-02 \\
\textbf{gamma_u  } &   1,000000e-02 &     1,000000e-02 &     1,000000e-02 &  1,000000e-02 &  1,000000e-02 \\
\textbf{omega    } &   4,000000e-01 &     4,000000e-01 &     3,000000e-01 &  4,000000e-01 &  4,000000e-01 \\
\textbf{rm       } &   1,000000e-02 &     1,000000e-02 &     1,000000e-02 &  1,000000e-02 &  1,000000e-02 \\
\textbf{spread_l } &   0,000000e+00 &     0,000000e+00 &     0,000000e+00 &  0,000000e+00 &  0,000000e+00 \\
\textbf{spread_mo} &   0,000000e+00 &     0,000000e+00 &     0,000000e+00 &  2,000000e-01 &  0,000000e+00 \\
\textbf{un       } &   8,000000e-01 &     8,000000e-01 &     8,000000e-01 &  8,000000e-01 &  8,000000e-01 \\
\textbf{v        } &   2,500000e+00 &     2,500000e+00 &     2,500000e+00 &  2,500000e+00 &  2,500000e+00 \\
\textbf{phi_0    } &   2,000000e-02 &     2,500000e-02 &     2,000000e-02 &  2,000000e-02 &  2,000000e-02 \\
\textbf{phi_1    } &   1,000000e-01 &     1,000000e-01 &     1,000000e-01 &  1,000000e-01 &  1,000000e-01 \\
\textbf{phparam  } &   1,000000e+00 &     1,000000e+00 &     1,000000e+00 &  1,000000e+00 &  1,000000e+00 \\
\textbf{pe       } &   1,000000e+00 &     1,000000e+00 &     1,000000e+00 &  1,000000e+00 &  1,000000e+00 \\
\textbf{R        } &   1,000000e-01 &     1,000000e-01 &     1,000000e-01 &  1,000000e-01 &  1,000000e-01 \\
\textbf{infla    } &   0,000000e+00 &     0,000000e+00 &     0,000000e+00 &  0,000000e+00 &  5,000000e-02 \\
\textbf{_K_f__1  } &   9,590873e+10 &     5,050035e+21 &     1,336341e+21 &  3,013274e+19 &  4,188039e+21 \\
\textbf{_M_h__1  } &   8,925247e+11 &     3,723122e+22 &     1,449773e+22 &  2,265871e+20 &  3,112247e+22 \\
\textbf{_MO__1   } &   7,860142e+11 &     3,181549e+22 &     1,297658e+22 &  2,498456e+20 &  2,662611e+22 \\
\textbf{_Lk__1   } &   8,733491e+10 &     3,535054e+21 &     1,441843e+21 &  2,776062e+19 &  2,958456e+21 \\
\textbf{_Eq__1   } &   0,000000e+00 &     0,000000e+00 &     0,000000e+00 &  0,000000e+00 &  0,000000e+00 \\
\textbf{_M_f__1  } &   0,000000e+00 &     0,000000e+00 &     0,000000e+00 &  0,000000e+00 &  0,000000e+00 \\
\textbf{_Lf__1   } &   1,917555e+10 &     1,880678e+21 &     7,929786e+19 &  5,758750e+18 &  1,537904e+21 \\
\textbf{_h__1    } &   5,900000e-02 &     7,500000e-02 &     5,900000e-02 &  5,900000e-02 &  7,400000e-02 \\
\textbf{_L__1    } &   1,065105e+11 &     5,415732e+21 &     1,521141e+21 &  3,351937e+19 &  4,496360e+21 \\
\textbf{_M__1    } &   8,925247e+11 &     3,723122e+22 &     1,449773e+22 &  2,265871e+20 &  3,112247e+22 \\
\textbf{_K_HD__1 } &   7,860142e+11 &     3,181549e+22 &     1,297658e+22 &  2,498456e+20 &  2,662611e+22 \\
\textbf{_I_h__1  } &   1,465581e+10 &     7,456755e+20 &     2,419579e+20 &  4,610421e+18 &  6,192118e+20 \\
\textbf{_ph__1   } &   1,000000e+00 &     1,000000e+00 &     1,000000e+00 &  1,000000e+00 &  1,472685e+21 \\
\bottomrule
\end{tabular}

	\caption*{\textbf{Fonte:} Elaboração própria}
\end{table}