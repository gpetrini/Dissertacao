\titleformat{\chapter}[display]{\normalfont\huge}{\appendixname{} \thechapter}{20pt}{\bfseries\huge}
\chapter{Modelo SFC}
\label{Append_Fraction}
%TODO Gráfico sensibilidade
%TODO Desenvolvimento solução de k


\section{\textit{Fraction} e o investimento residencial}

O objetivo deste apêndice é mostrar que a inclusão de um gasto autônomo não criador de capacidade produtiva ao setor privado não é condição suficiente para que a propensão marginal ($s$) e média a poupar ($S/Y$) sejam distintas. Tal como no corpo do texto, seja $Y$ a renda, $C$ o consumo induzido, $I_t$ o investimento total e $Z$ os gastos autônomos que serão distindos em cada uma das seções subsequentes.

\subsection{Gastos ``despoupadores''}

Considere, primeiramente, que os gastos autônomos são o consumo financiado por crédito.

$$
Y = C + I_t + Z
$$
Partindo da identidade contábil entre investimento e poupança
$$
S = Y - C - Z
$$
Para o caso mais simplificado em que a propensão marginal a consumir a partir do salários é igual a unidade de modo que o consumo induzido é dado por
$$
C = \omega\cdot Y
$$
em que $\omega$ é a participação dos salários na renda. Substituindo na equação anterior,
$$
S = (1- \omega)Y - Z \Rightarrow S = s\cdot Y - Z
$$
Por fim, dividindo a equação anterior pela renda, obtém-se a propensão média a poupar em função dos gastos autônomos:
\begin{equation}
\label{_Fraction}
    \frac{S}{Y} = s - \frac{Z}{Y}
\end{equation}
A equação \ref{_Fraction} explicita que na presença dos gastos autônomos não criadores de capacidade produtiva, propensão marginal e média a poupar são distintas. No entanto, tal afirmação exige uma qualificação adicional. Tais gastos precisam ser ``despoupadores'' para se ter esse resultado. Mas seção na seguir, será apresentado o caso de gastos autônomos que geram poupança. 

\subsection{Investimento residencial}

Seguindo os mesmos procedimentos, mas com os gastos autônomos sendo agora o investimento residencial enquanto o consumo volta a ser totalmente induzido. Com isso, o investimento da economia é composto por duas parcelas, o investimento das firmas ($I_f$) e das famílias ($I_h = Z$):
$$
I_t = I_f + I_h
$$
De modo que a renda é determinada por:
$$
Y = C + I_f + Z
$$
Mais uma vez, partindo da identidade entre poupança e investimento
$$
S = Y - C
$$
$$
S = Y -  \omega\cdot Y
$$
$$
S = sY
$$
Dividindo a equação anterior pela renda, obtém-se que a propensão média e marginal a poupar são idênticas
$$
\frac{S}{Y} = s = \frac{I_t}{Y}
$$
De modo que a \textit{fraction} seja igual a unidade
$$
f = \frac{\frac{S}{Y}}{s} = 1
$$
A explicação deste resultado decorre pelo investimento residencial ser contabilmente investimento e, portanto, não é um gasto autônomo ``despoupador''. 

Tal conclusão, no entanto, é problemática por duas razões: (i) diferentemente dos modelos com supermultiplicador apresentados, a \textit{fraction} não é a variável de fechamento; (ii) implica uma relação inversa entre taxa de investimento e crescimento econômico, resultado esse que não encontra sustentação empírica. Ambos os pontos podem ser mostrados conjuntamente. Partindo da participação dos componentes da demanda na renda,
$$
\omega + h + \frac{I_h}{Y} = 1
$$
Da equação acima, destaca-se que a propensão marginal a consumir é determinada exogenamente e o mesmo vale para a participação do investimento total, ou seja
$$
 \omega +  s = 1
$$
Desse modo, um aumento na taxa de crescimento do investimento residencial --- mantida a taxa de investimento total --- implica aumento da propensão marginal a investir plenamente ajustada e subsequente redução da taxa de gastos autônomos
$$
\frac{I_t}{Y} = \frac{I_f + I_h}{Y} = s
$$
Dito isso e retomando a equação \ref{EqGeral}:
$$
\frac{ s\cdot  u_N}{v} =  g_Z = h\frac{ u_N}{ v}
$$

\begin{equation}
h = \frac{ g_Z\cdot  u_N}{ v}
\end{equation}
\begin{equation}
    \label{Fechamento_Modelo}
\frac{I_h}{Y} = 1 - \omega - \frac{ g_Z\cdot  u_N}{ v}
\end{equation}
Portanto, como o investimento residencial cresce a uma taxa exógena, é a participação do próprio investimento residencial na renda que fecha o modelo como indicado pela equação \ref{Fechamento_Modelo}. 

Feitas essas observações, cabe destacar que o modelo apresentado no capítulo \ref{CapModelo} não incorre nesses problemas uma vez que foi adicionado o consumo capitalista autônomo financiado por crédito ($C_k$) que mantem uma proporção fixa em relação aos gastos autônomos totais
$$
C_k =  R \cdot Z
$$
de modo que a relação positiva entre taxa de investimento (total) e crescimento é restabelecida
$$
\frac{I_t}{Y} = 1 -  \omega - \frac{C_k}{Y}
$$
\begin{equation}
\label{retorno_fraction}
\frac{I_t}{Y} = 1 -  \omega - \frac{ R\cdot Z}{Y}
\end{equation}
$$
\frac{I_t}{Y} = 1 -  \omega -  R \cdot (1 -  \omega - h^*)
$$
$$
\frac{I_t}{Y} = (1-R)(1 -  \omega) +  R \cdot h^*
$$
$$
\frac{I_t}{Y} = (1-R)(1 -  \omega) +  R \cdot \frac{g_Z v}{u_N}
$$
$$
\frac{\partial I_t/Y}{\partial g_Z} = \frac{R}{v}\cdot u_N > 0
$$
Como pode ser visto na equação \ref{retorno_fraction}, com o gasto autônomo ``despoupador'', retorna-se ao caso em que a fração não é igual à unidade tal como no fechamento do supermultiplicador sraffiano. 

\section{Simulações}


\begin{table}[H]
	\centering
	\caption{Parâmetros das simulações}
	\label{Resumo_Simulacao}
	\begin{tabular}{c}
\toprule
{} &  Base scenario &  $\Delta \phi_0$ &  $\Delta \omega$ &   $\Delta rm$ &  $\Delta $ Infla \\
\midrule
\textbf{alpha    } &   1,000000e+00 &     1,000000e+00 &     1,000000e+00 &  1,000000e+00 &     1,000000e+00 \\
\textbf{alpha_2  } &   3,000000e-01 &     3,000000e-01 &     3,000000e-01 &  3,000000e-01 &     3,000000e-01 \\
\textbf{gamma_F  } &   9,000000e-01 &     9,000000e-01 &     9,000000e-01 &  9,000000e-01 &     9,000000e-01 \\
\textbf{gamma_u  } &   1,000000e-02 &     1,000000e-02 &     1,000000e-02 &  1,000000e-02 &     1,000000e-02 \\
\textbf{omega    } &   3,000000e-01 &     3,000000e-01 &     2,500000e-01 &  3,000000e-01 &     3,000000e-01 \\
\textbf{rm       } &   2,000000e-02 &     2,000000e-02 &     2,000000e-02 &  2,000000e-02 &     2,000000e-02 \\
\textbf{spread_l } &   0,000000e+00 &     0,000000e+00 &     0,000000e+00 &  0,000000e+00 &     0,000000e+00 \\
\textbf{spread_mo} &   0,000000e+00 &     0,000000e+00 &     0,000000e+00 &  5,000000e-03 &     0,000000e+00 \\
\textbf{un       } &   8,000000e-01 &     8,000000e-01 &     8,000000e-01 &  8,000000e-01 &     8,000000e-01 \\
\textbf{v        } &   2,500000e+00 &     2,500000e+00 &     2,500000e+00 &  2,500000e+00 &     2,500000e+00 \\
\textbf{phi_0    } &   4,000000e-02 &     5,000000e-02 &     4,000000e-02 &  4,000000e-02 &     4,000000e-02 \\
\textbf{phi_1    } &   2,000000e-02 &     2,000000e-02 &     2,000000e-02 &  2,000000e-02 &     2,000000e-02 \\
\textbf{phparam  } &   1,000000e+00 &     1,000000e+00 &     1,000000e+00 &  1,000000e+00 &     1,000000e+00 \\
\textbf{infla    } &   0,000000e+00 &     0,000000e+00 &     0,000000e+00 &  0,000000e+00 &     1,000000e-02 \\
\textbf{gZn      } &   3,900000e-02 &     3,922000e-02 &     3,900000e-02 &  3,900000e-02 &     3,900000e-02 \\
\textbf{_K_f__1  } &   4,244058e+19 &     1,082934e+44 &     3,585057e+30 &  1,974191e+37 &     7,896508e+45 \\
\textbf{_M__1    } &  -1,326991e+20 &    -1,611049e+44 &    -1,223085e+31 & -6,225215e+37 &    -2,427467e+46 \\
\textbf{_MO__1   } &   1,855990e+20 &     3,598007e+44 &     1,704715e+31 &  8,659263e+37 &     3,432732e+46 \\
\textbf{_Lf__1   } &  -3,182981e+20 &    -5,209056e+44 &    -2,927800e+31 & -1,488448e+38 &    -5,860199e+46 \\
\textbf{_h__1    } &   1,240000e-01 &     1,550000e-01 &     1,240000e-01 &  1,230000e-01 &     1,240000e-01 \\
\textbf{_L__1    } &  -3,182981e+20 &    -5,209056e+44 &    -2,927800e+31 & -1,488448e+38 &    -5,860199e+46 \\
\textbf{_K_HD__1 } &   1,855990e+20 &     3,598007e+44 &     1,704715e+31 &  8,659263e+37 &     3,432732e+46 \\
\textbf{_I_h__1  } &   7,069758e+18 &     1,700278e+43 &     6,493526e+29 &  3,290437e+36 &     1,313996e+45 \\
\textbf{_ph__1   } &   1,000000e+00 &     1,000000e+00 &     1,000000e+00 &  1,000000e+00 &     3,004272e+06 \\
\bottomrule
\end{tabular}

	\caption*{\textbf{Fonte:} Elaboração própria}
\end{table}