\section{Solução analítica}
\label{SecAnalitica}

Apresentada a estrutura do modelo, resta expor a solução analítica de modo que fiquem explicitadas as condições de estabilidade bem como as relações dinâmicas entre as variáveis. Para obter o nível da renda de longo prazo, basta substituir \ref{_C}, \ref{_It} em \ref{_Y} para então substituir \ref{_W}, \ref{_If} e considerar $I_h = Z$ como em \ref{_Z}:
\begin{equation}
    \label{AnaliticaNivel}
    Y = \left(\frac{1}{1-\omega - h}\right)Z
\end{equation}
Dito isso, resta apresentar o modelo em termos de crescimento. Partindo contribuição dos componentes da demanda para a variação da renda e resolvendo para a taxa de crescimento, obtém-se
$$
g = g\cdot \omega + \Delta h + g\cdot h + g_z\cdot \frac{Z}{Y}
$$
\begin{equation}
\label{gLP}
    g = \frac{\Delta h}{1 - \omega - h} + g_z
\end{equation}
Como indicado anteriormente, a propensão marginal a investir se ajusta de acordo com o princípio do estoque de capital e, dessa forma, quando o grau de utilização convergir ao normal, a taxa de crescimento da economia tende a taxa de crescimento dos gastos autônomos (neste caso, investimento residencial) definida exogenamente. 
\begin{equation}
u \to u_N \Leftrightarrow g \to g_z
\end{equation}
De modo que a propensão marginal a investir necessária é
\begin{equation}
\label{hLP}
h = \overline g_z\frac{\overline v}{\overline u_N}
\end{equation}
que, como mostram \textcite{fagundes_role_2017}, explicita a relação positiva entre taxa de investimento das firmas e crescimento. Com isso, substituindo \ref{gLP} nas equações \ref{hLP} e \ref{Aux} é possível construir o seguinte sistema de equações:

$$
\begin{cases}
\dot u = \left(\frac{\Delta h}{1 - \omega - h} + g_z - \frac{h{\left(t \right)} u{\left(t \right)}}{v}\right) u{\left(t \right)}\\
\dot h = \gamma_{u} \left(- un + u{\left(t \right)}\right) h{\left(t \right)}
\end{cases}
$$

Para fins de simplificação, supões temporariamente um sistema para o tempo contínuo de modo que seja possível construir o seguinte jacobiano em torno do equilíbrio ($u = u_N$ e $h = h^*$):

$$
J = 
\left[\begin{matrix}
\frac{\partial \dot h}{\partial h} & \frac{\partial \dot h}{\partial u}\\
\frac{\partial \dot u}{\partial h} & \frac{\partial \dot u}{\partial u}
\end{matrix}\right]
$$

\begin{equation}
J = 
\label{Jacobiano}
\left[\begin{matrix}0 & \frac{g_Z \gamma_{u} v}{un}\\- \frac{un^{2}}{v} & - g_Z\end{matrix}\right]
\end{equation}
Seguindo os procedimentos de GANDOLFO, para que um sistema de duas equações seja estável, basta que o determinante de \ref{Jacobiano} seja positivo enquanto o traço seja negativo:

$$
Det(J) = g_Z \gamma_{u} u_N > 0
$$

$$
Tr(J) = -g_Z < 0
$$
uma vez que $\gamma_u$ e $u_N$ são necessariamente positivos, basta que a condição do traço seja atendida. Diferente de \textcite{freitas_growth_2015}, tal condição não é uma das hipóteses iniciais do modelo e, portanto, requer que a taxa própria obedeça a seguinte desigualdade:

\begin{equation}
own < \frac{\phi_0}{\phi_1}
\end{equation}

Além disso, se o sistema é estável em torno do equilíbrio de longo prazo, o grau de utilização deve convergir. Partindo da Eq. \ref{Aux}, é preciso que a seguinte condição seja atentida:
$$
\frac{\partial g}{\partial u} < \frac{\partial g_K}{\partial u}
$$

$$
- \frac{g_Z \gamma_{u} v}{\alpha \omega un + g_Z v - un} < \frac{g_Z}{un}
$$
reescrevendo, obtém-se
\begin{equation}
\alpha \omega + \frac{g_Z v}{u_N} + \gamma_u\cdot v
\end{equation}
Portanto, além do termo em parêntese da Eq. \ref{AnaliticaNivel} ser o supermultplicador, fornece as condições para que o modelo seja estável: 
A condição anterior, como em \textcite{freitas_growth_2015}, significa que a propensão marginal a gasta (consumir e investir) seja menor que a unidade, caso contrário, vigora-se a lei de Say.


%%%%%%%%%%%%%%%%%% Ponte para apêndice

Tal exposição, no entanto, não se distingue da apresentada por \textcite{freitas_growth_2015} uma vez que formalização é semelhante em que o investimento residencial assume o papel dos gastos autônomos. A principal diferença é que este gasto autônomo também forma o estoque de capital da economia que, diferentemente do capital das firmas, não cria capacidade produtiva. Resta, portanto, explicitar a dinâmica entre este dois estoques de capital distintos, captados por $K_k$. A equação que define o grau de utilização da capacidade pode ser reescrita como

$$
u = \frac{Y\cdot v}{K \cdot (1-K_k)}
$$
Dessa forma, dividir o produto pelo estoque de imóveis é o mesmo que
$$
\frac{Y}{K_k\cdot K}
$$
multiplicando pela relação técnica ($v$)
$$
\frac{Y}{K_k\cdot K}\cdot v = \frac{Y\cdot v}{K}\cdot \left(\frac{1}{K_k}\right)
$$
e multiplicando e dividindo por $1-K_k$ obtém-se a seguinte relação com o grau de utilização:
$$
\frac{Y\cdot v}{K\cdot (1-K_k)}\cdot \left(\frac{1-K_k}{K_k}\right) = u \cdot \left(\frac{1-K_k}{K_k}\right)
$$
Portanto,
$$
Y\frac{v}{K_h} =  u \cdot \left(\frac{1-K_k}{K_k}\right)
$$
$$
u = Y\frac{v}{K_h} \cdot \left(\frac{K_k}{1-K_k}\right)
$$
Substituindo as variáveis endógenas de modo a explicitar apenas em termos de parâmetros e variáveis exógenas,
$$
u{\left(t \right)} = - \frac{K_{k} v \left(\phi_{0} - \phi_{1} \left(-1 + \frac{rm + spread_{mo} + 1}{infla + 1}\right)\right)}{\left(1 - K_{k}\right) \left(\alpha \omega - 1 + \frac{v \left(\phi_{0} - \phi_{1} \left(-1 + \frac{rm + spread_{mo} + 1}{infla + 1}\right)\right)}{un}\right)}
$$
e resolvendo para o longo prazo
$$
\frac{K_{k}}{1 - K_{k}} = \frac{un \left(- \alpha \omega \left(infla + 1\right) + infla + 1\right) - v \left(\phi_{0} \left(infla + 1\right) - \phi_{1} \left(- infla + rm + spread_{mo}\right)\right)}{v \left(\phi_{0} \left(infla + 1\right) - \phi_{1} \left(- infla + rm + spread_{mo}\right)\right)}
$$
Por fim, resolvendo para $K_k$:
$$
K_{k} = \frac{un \left(- \alpha \omega \left(infla + 1\right) + infla + 1\right) - v \left(\phi_{0} \left(infla + 1\right) - \phi_{1} \left(- infla + rm + spread_{mo}\right)\right)}{un \left(- \alpha \omega \left(infla + 1\right) + infla + 1\right)}
$$
\begin{equation}
\label{kAnali}
K_{k} = 1 - \frac{v \left(\phi_{0} - \phi_{1} \left(-1 + \frac{rm + spread_{mo} + 1}{infla + 1}\right)\right)}{un \left(- \alpha \omega + 1\right)}
\end{equation}
Cuja forma simplificada é\footnote{As etapas realizadas podem estão no \textit{script} utilizado.}:
$$
K_k = 1 - \frac{h^*}{(1 - \alpha\cdot\omega)}
$$

A equação \ref{kAnali} mostra que a participação dos imóveis no estoque de capital total depende 
positivamente do \textit{spread} bancário da taxa de juros das hipotecas e 
negativamente do componente autônomo de $g_z$ ($\phi_0$), da inflação de ativos ($infla$) e da distribuição dos salários na renda ($\omega$). Formalmente:
\begin{equation}
\frac{\partial K_k}{\partial \phi_0} = \frac{v}{un \left(\alpha \omega - 1\right)} < 0
\end{equation}
\begin{equation}
\frac{\partial K_k}{\partial infla} = \frac{\phi_{1} v \left(rm + spread_{mo} + 1\right)}{un \left(infla + 1\right)^{2} \left(\alpha \omega - 1\right)} < 0
\end{equation}
\begin{equation}
\frac{\partial K_k}{\partial \omega} = - \frac{\alpha v \left(\phi_{0} \left(infla + 1\right) - \phi_{1} \left(- infla + rm + spread_{mo}\right)\right)}{un \left(infla + 1\right) \left(\alpha \omega - 1\right)^{2}} < 0
\end{equation}
\begin{equation}
\frac{\partial K_k}{\partial spread_{mo}} = - \frac{\phi_{1} v}{un \left(infla + 1\right) \left(\alpha \omega - 1\right)} > 0
\end{equation}
Tais resultados, apesar de contra intuitivos, estão em linha com o supermultiplicador sraffiano uma vez que quanto maior $g_z$, menor será a participação dos gastos autônomos na renda e que mudanças na distribuição (isto é, alterações no multiplicador) terão efeito nível, alterando apenas a composição dos diferentes tipos de estoque de capital. 



