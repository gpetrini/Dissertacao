\section{Solução analítica}
\label{SecAnalitica}

Apresentada a estrutura do modelo, resta expor a solução analítica de modo que fiquem explicitadas as condições de estabilidade bem como as relações dinâmicas entre as variáveis. 
Para isso, as equações serão transformadas --- somente nessa seção --- em seu equivalente em tempo contínuo de modo que algumas relações possam ser expressas em termos de derivadas parciais.
Feita esta ressalva, avança-se em direção das condições de estabilidade para então obter algumas razões de longo prazo (indicadas por $^*$)\footnote{O \textit{script} com as etapas realizadas esta disponível sob solicitação.}.
Para obter o nível da renda, basta substituir \ref{ConsumoTotal}, \ref{_It} em \ref{_Y} para então substituir \ref{_W}, \ref{_If} e considerar $Z = I_h + C_k$ como em \ref{_Z}:
\begin{equation}
    \label{AnaliticaNivel}
    Y = \left(\frac{1}{1-\omega - h}\right)Z
\end{equation}
Dito isso, resta apresentar o modelo em termos de crescimento tal como \textcite{freitas_growth_2015}. Partindo contribuição dos componentes da demanda para a variação da renda e resolvendo para a taxa de crescimento, obtém-se
$$
g = g\cdot \omega + \dot h + g\cdot h + g_z\cdot \frac{Z}{Y}
$$
\begin{equation}
\label{gLP}
    g = \frac{\dot h}{1 - \omega - h} + g_z
\end{equation}
Como indicado anteriormente, a propensão marginal a investir se ajusta de acordo com o princípio do ajuste do estoque de capital e, dessa forma, quando o grau de utilização convergir ao normal, a taxa de crescimento da economia tende a taxa de crescimento dos gastos autônomos (neste caso, investimento residencial) definida exogenamente. 
\begin{equation}
u \to u_N \Leftrightarrow g \to g_z
\end{equation}
De modo que a propensão marginal a investir necessária é
\begin{equation}
\label{hLP}
h^* = \overline g_z\frac{\overline v}{\overline u_N}
\end{equation}
que, como mostram \textcite{fagundes_role_2017}, explicita a relação positiva entre taxa de investimento das firmas e crescimento. Com isso, substituindo \ref{gLP} nas equações \ref{hLP} e \ref{Aux} é possível construir o seguinte sistema de equações para verificar as condições de estabilidade:

$$
\begin{cases}
\dot u = \left(\frac{\dot h}{1 - \omega - h} + g_z - \frac{h_t u_t}{v}\right) u_t\\
\dot h = \gamma_{u} \left(- u_N + u_t\right) h_t
\end{cases}
$$
para então construir o seguinte jacobiano em torno do equilíbrio ($u = u_N$ e $h = h^*$):

$$
J = 
\left[\begin{matrix}
\frac{\partial \dot h}{\partial h} & \frac{\partial \dot h}{\partial u}\\
\frac{\partial \dot u}{\partial h} & \frac{\partial \dot u}{\partial u}
\end{matrix}\right]
$$

\begin{equation}
J = 
\label{Jacobiano}
\left[\begin{matrix}0 & \frac{g_Z \gamma_{u} v}{u_N}\\- \frac{u_N^{2}}{v} & - g_Z\end{matrix}\right]
\end{equation}
Seguindo os procedimentos de \textcite{gandolfo_economic_2010}, para que um sistema de duas equações seja estável, basta que o determinante de \ref{Jacobiano} seja positivo enquanto o traço seja negativo:

$$
Det(J) = g_Z \gamma_{u} u_N > 0
$$

$$
Tr(J) = -g_Z < 0
$$
uma vez que $\gamma_u$ e $u_N$ são necessariamente positivos, basta que a condição do traço seja atendida. Diferente de \textcite{freitas_growth_2015}, tal condição não é uma das hipóteses iniciais do modelo e, portanto, requer que a taxa própria presente na equação \ref{g_Z_own} obedeça a seguinte desigualdade:

\begin{equation}
own < \frac{\phi_0}{\phi_1}
\end{equation}

Além disso, se o sistema é estável em torno do equilíbrio de longo prazo, o grau de utilização deve convergir. Partindo da Eq. \ref{Aux}, é preciso que a seguinte condição seja atentida:
$$
\frac{\partial g}{\partial u} < \frac{\partial g_K}{\partial u}
$$

$$
- \frac{g_Z \gamma_{u} v}{\omega u_N + g_Z v - u_N} < \frac{g_Z}{u_N}
$$
Portanto, além do termo em parênteses da Eq. \ref{AnaliticaNivel} ser o supermultplicador, fornece as condições necessárias e suficientes para que o modelo seja estável: 
$$
\omega + \frac{g_Z v}{u_N} + \gamma_u\cdot v < 1
%\omega + \frac{g_Z v}{u_N} < 1
$$
\begin{equation}
\omega + h^* < 1
\end{equation}
A condição anterior, como em \textcite{freitas_growth_2015}, significa que a propensão marginal a gastar (consumir e investir) seja menor que a unidade, caso contrário, 
%vigora-se a lei de Say.
o modelo não seria compatível com o PDE, estendido à uma economia em crescimento.


%%%%%%%%%%%%%%%%%% Ponte para apêndice

Tal exposição, no entanto, não se distingue da apresentada por \textcite{freitas_growth_2015} uma vez que formalização é semelhante em que o investimento residencial assume o papel dos gastos autônomos. A principal diferença é que este gasto autônomo também forma o estoque de capital da economia que, diferentemente do capital das firmas, não cria capacidade produtiva. Resta, portanto, explicitar a dinâmica entre estes dois estoques de capital distintos, captados por $k$. A equação que define o grau de utilização da capacidade pode ser reescrita como

$$
u = \frac{Y\cdot v}{K \cdot (1-k)}
$$
Dessa forma, dividir o produto pelo estoque de imóveis é o mesmo que
$$
\frac{Y}{k\cdot K}
$$
multiplicando pela relação técnica ($v$)
$$
\frac{Y}{k\cdot K}\cdot v = \frac{Y\cdot v}{K}\cdot \left(\frac{1}{k}\right)
$$
e multiplicando e dividindo por $1-k$ obtém-se a seguinte relação com o grau de utilização:
$$
\frac{Y\cdot v}{K\cdot (1-k)}\cdot \left(\frac{1-k}{k}\right) = u \cdot \left(\frac{1-k}{k}\right)
$$
Portanto,
$$
Y\frac{v}{K_h} =  u \cdot \left(\frac{1-k}{k}\right)
$$
$$
u = Y\frac{v}{K_h} \cdot \left(\frac{k}{1-k}\right)
$$
\begin{equation}
u = \left(\frac{1}{1-R}\right)\cdot\left(\frac{g_Z\cdot v}{1-\omega- h}\right)\cdot\left(\frac{k}{1-k}\right)
\end{equation}
Substituindo as variáveis endógenas de modo a explicitar apenas em termos de parâmetros e variáveis exógenas e resolvendo para o longo prazo e em termos de $k$
\begin{equation}
\label{kAnali}
k = 1 - \frac{v \left(\phi_{0} - \phi_{1} \left(-1 + \frac{rm\cdot(1+\sigma_{mo})}{\pi + 1}\right)\right)}{u_N \left(- \omega + 1\right)}
\end{equation}
cuja forma simplificada é\footnote{Etapas para obtenção de $k^*$
	$$
	u_t = - \frac{k v \left(\phi_{0} - \phi_{1} \left(-1 + \frac{rm\cdot(1+\sigma_{mo})}{\pi + 1}\right)\right)}{\left(1 - k\right) \left(\omega - 1 + \frac{v \left(\phi_{0} - \phi_{1} \left(-1 + \frac{rm\cdot(1+\sigma_{mo})}{\pi + 1}\right)\right)}{u_N}\right)}
	$$
	$$
	\frac{k}{1 - k} = \frac{u_N \left(- \omega \left(\pi + 1\right) + \pi + 1\right) - v \left(\phi_{0} \left(\pi + 1\right) - \phi_{1} \left(- \pi + rm + \sigma_{mo}\right)\right)}{v \left(\phi_{0} \left(\pi + 1\right) - \phi_{1} \left(- \pi + rm + \sigma_{mo}\right)\right)}
	$$
	Por fim, resolvendo para $k$:
	$$
	k = \frac{u_N \left(- \omega \left(\pi + 1\right) + \pi + 1\right) - v \left(\phi_{0} \left(\pi + 1\right) - \phi_{1} \left(- \pi + rm + \sigma_{mo}\right)\right)(1-R)}{u_N \left(- \omega \left(\pi + 1\right) + \pi + 1\right)}
	$$
simplificando
$$
k = 1 - \frac{v \left(\phi_{0} - \phi_{1} \left(-1 + \frac{rm\cdot(1+\sigma_{mo})}{\pi + 1}\right)\right)}{u_N \left(- \omega + 1\right)}
$$
}:
\begin{equation}
\label{_k_star}
k^* = 1 - \frac{h^*}{(1 - \omega)}
\end{equation}
a equação \ref{_k_star} indica que a participação plenamente ajustada dos imóveis no estoque de capital total da economia depende positivamente da participação dos salários na renda e diminui como decorrência de uma maior taxa de crescimento dos gastos autônomos.

Uma vez que as relações entre crescimento e distribuição no supermultiplicador sraffiano está bem estabelecido pela literatura, avança-se em direção da análise das particularidade do presente modelo: presença de dois tipos distintosde estoque de capital.
A equação \ref{kAnali} mostra que a participação dos imóveis no estoque de capital total depende 
positivamente da taxa de juros (Eq. \ref{partial_rm}) e 
negativamente do componente autônomo de $g_z$ (Eq \ref{partial_phi0}), da inflação de ativos (Eq \ref{partial_pi}) e da distribuição dos salários na renda (Eq \ref{partial_omega}). Formalmente:
\begin{equation}
%TODO Checar solução analítica
\label{partial_rm}
\frac{\partial k}{\partial rm} = - \frac{\phi_{1} v \left(\sigma_{mo} + 1\right)}{u_N \left(\pi + 1\right) \left(\omega - 1\right)} > 0
\end{equation}
\begin{equation}
\label{partial_phi0}
\frac{\partial k}{\partial \phi_0} = \frac{v}{u_N \left(\omega - 1\right)} < 0
\end{equation}
\begin{equation}
\label{partial_pi}
\frac{\partial k}{\partial \pi} = \frac{\phi_{1} v \left(rm\cdot(1+\sigma_{mo}) + 1\right)}{u_N \left(\pi + 1\right)^{2} \left(\omega - 1\right)} < 0
\end{equation}
\begin{equation}
\label{partial_omega}
\frac{\partial k}{\partial \omega} = - \frac{v \left(\phi_{0} \left(\pi + 1\right) - \phi_{1} \left(- \pi + rm\cdot(1 + \sigma_{mo})\right)\right)}{u_N \left(\pi + 1\right) \left(\omega - 1\right)^{2}} < 0
\end{equation}
Tais resultados, apesar de contra intuitivos, estão em linha com o supermultiplicador sraffiano uma vez que quanto maior a taxa de crescimento do investimento residencial, menor será a participação dos gastos autônomos na renda e que mudanças na distribuição (isto é, alterações no multiplicador) terão efeito nível, alterando apenas a composição dos diferentes tipos de estoque de capital. 


%NORMAS ESTOQUE FLUXO

\begin{comment}
Etapas da solução analítica
$$
u_t = - \frac{k v \left(\phi_{0} - \phi_{1} \left(-1 + \frac{rm\cdot(1+\sigma_{mo})}{\pi + 1}\right)\right)}{\left(1 - k\right) \left(\omega - 1 + \frac{v \left(\phi_{0} - \phi_{1} \left(-1 + \frac{rm\cdot(1+\sigma_{mo})}{\pi + 1}\right)\right)}{u_N}\right)}
$$
$$
\frac{k}{1 - k} = \frac{u_N \left(- \omega \left(\pi + 1\right) + \pi + 1\right) - v \left(\phi_{0} \left(\pi + 1\right) - \phi_{1} \left(- \pi + rm + \sigma_{mo}\right)\right)}{v \left(\phi_{0} \left(\pi + 1\right) - \phi_{1} \left(- \pi + rm + \sigma_{mo}\right)\right)}
$$
Por fim, resolvendo para $k$:
$$
k = \frac{u_N \left(- \omega \left(\pi + 1\right) + \pi + 1\right) - v \left(\phi_{0} \left(\pi + 1\right) - \phi_{1} \left(- \pi + rm + \sigma_{mo}\right)\right)(1-R)}{u_N \left(- \omega \left(\pi + 1\right) + \pi + 1\right)}
$$
\end{comment}