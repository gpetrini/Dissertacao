\titleformat{\chapter}[display]{\normalfont\huge}{\appendixname{} \thechapter}{20pt}{\bfseries\huge}
\chapter{Apêndice estatístico}
\label{Append_Stat}

\section{Testes de hipótese}


\input{./Fatos_Estilizados/H0_Residuos.tex}

\section{VECM Alternativo: $g_Z$, inflação e juros exógeno}

Para explicitar a robustez da inflação de ativos para a taxa de crescimento do investimento residencial, estima-se outro modelo vetor correção de erro (VECM).
Uma vez que já foi realizada a inspeção das variáveis (quebra estrutural, testes de raíz unitária e cointegração) bem como a contextualização das defasagens, prossegue-se para a estimação do modelo propriamente dita cuja relação de longo prazo a ser testada é:

$$
g_{Z_t} = \phi_{0} - \phi_1\cdot own_t
$$
que é decomposta nos seguintes termos:
$$
g_{Z_t} = \phi_{0} - \phi_1\cdot \left(\frac{1+\overline r_{mo_t}}{1+\dot p_{h_t}} -1\right)
$$
que pode ser aproximado para
\begin{equation}
\label{tx_alternativa}
g_{Z_t} = \phi_{0} + \phi_1\cdot \dot p_{h_t}
\end{equation}
em que $\overline r_{mo}$ indica a taxa de juros das hipotecas definido exogenamente, $\dot p_h$ é a inflação de imóveis e $g_Z$ é a taxa de crescimento do investimento residencial.
Dito isso, é possível reescrever a equação \ref{tx_alternativa} como um sistema de equações:

%TODO Equation
\begin{equation}
\begin{cases}
\Delta \dot p_{h_t} = \delta_{1} + \alpha_1(g_{Z_{t-1}} - \phi_0 - \phi_1\cdot \dot p_{h_{t-1}}) + \sum^{N=4}_{i=1}\beta_{1,i}\cdot \Delta g_{Z_{t-i}} +
\sum^{N=4}_{i=1}\gamma_{1,i}\cdot \Delta \dot p_{h_{t-i}} + \rho_1\cdot\overline r_{mo} + \varepsilon_{t,1}
\\
\Delta g_{Z_{t}} = \delta_{2} + \alpha_2(g_{Z_{t-1}} - \phi_0 - \phi_1\cdot \dot p_{h_{t-1}}) + \sum^{N=4}_{i=1}\beta_{2,i}\cdot \Delta g_{Z_{t-i}} +
\sum^{N=4}_{i=1}\gamma_{2,i}\cdot \Delta \dot p_{h_{t-i}} + \rho_2\cdot\overline r_{mo} + \varepsilon_{t,2}
\end{cases}
\end{equation}
em que $\delta$ indica tendência linear na relação de cointegração;
$\alpha_{is}$ são os coeficientes de correção de erro; 
$\beta_s$ e $\gamma_s$ são coeficientes associados as defasagens de  $g_Z$ e $\dot ph$ respectivamente e; $\varepsilon_s$ são os resíduos.
Seguindo a literatura do supermultiplicador sraffiano, os resultados esperados a serem testados são:
\begin{enumerate}
	\item $\varepsilon \sim I(0)$: Estacionariedade dos resíduos indica que inflação e $g_Z$ são cointegrados, ou seja, apresentam uma dinâmica de longo prazo em comum;
	\item $\alpha_1 = 0$: $\dot ph$ exogenamente fraca em relação ao $g_Z$;
	\item $\alpha_2 < 0$: Inflação causa (no sentido de Granger) investimento residencial;
	%TODO Checar sinal de alpha2
	\item $\phi_1 < 0$: $gZ$ e Inflação apresentam uma dinâmica positiva no longo prazo;
	\item $\phi_0 < 0$: Demanda por imóveis por motivos não-especulativos e associados a especificidades institucionais é estatisticamente significante e não-negativo;
	\item $\gamma_{2,is} > 0$: Inflação afeta o investimento residencial positivamente no curto prazo;
	\item $\beta_{1,is}$ = 0: Efeito do investimento de $g_Z$ sobre a Inflação não é estatisticamente significante.
\end{enumerate}


Sendo assim, estima-se um VEC com quatro defasagens cujos resíduos (gráfico \ref{residuos_infla}) não apresenta autocorrelação serial e heterocedasticidade.
Os gráficos da função resposta ao impulso e decomposição da variância (bla e ble, respectivamente) apresentam resultados semelhantes ao modelo presente no corpo do texto, qual sejam:

\begin{table}[H]
	\centering
	\caption{Parâmetros da estimação (VECM Alternativo)}
	\input{./Fatos_Estilizados/Estimacao_VECM_Infla.tex}
	\caption*{\textbf{Fonte:} Elaboração própria}
\end{table}

\begin{figure}[htb]
	\centering
	\caption{Função impulso resposta ortogonalizada}
	\label{fevd}
	\includegraphics[width=\textwidth]{../../Modelo/SeriesTemporais/figs/Impulso_VECM_Infla.png}
	\caption*{\textbf{Fonte:} Elaboração própria}
\end{figure}


\begin{figure}[htb]
	\centering
	\caption{Decomposição da variância da previsão}
	\label{fevd}
	\includegraphics[width = \textwidth]{../../Modelo/SeriesTemporais/figs/FEVD_VECMpython_Infla.png}
	\caption*{\textbf{Fonte:} Elaboração própria}
\end{figure}

\input{./Fatos_Estilizados/H0_Residuos_Infla.tex}

\begin{figure}[htb]
	\centering
	\caption{Inspeção dos resíduos da estimação}
	\label{residuos_infla}
	\includegraphics[height=.4\textheight]{../../Modelo/SeriesTemporais/figs/Residuos_4VECM_Infla.png}
	\caption*{\textbf{Fonte:} Elaboração própria}
\end{figure}


%TODO Ljung box