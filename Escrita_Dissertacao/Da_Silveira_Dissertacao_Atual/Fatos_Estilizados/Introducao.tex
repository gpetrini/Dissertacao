Uma das fronteiras da pesquisa empírica acerca da literatura de crescimento liderado pela demanda é aquela que enfatiza a importância dos gastos autônomos não criadores de capacidade produtiva ao setor privado. \textcite{freitas_pattern_2013}, por exemplo, decompõem o crescimento da economia brasileira mostrando o papel desses gastos para os anos de  1970 a 2005. \textcite{braga_investment_2018} conclui que os gastos improdutivos lideram o crescimento e que o investimento produtivo acompanha a tendência desses gastos, ao analisar o Brasil no período 1962-2015. Para o caso norte-americano, \textcite{girardi_long-run_2016} encontram evidências de que os gastos autônomos causam efeitos de longo prazo na taxa de crescimento enquanto \textcite{girardi_autonomous_2018} complementam com 20 países da OCDE. No entanto, por mais que exista uma literatura crescente sobre o papel dos gastos autônomos no crescimento econômico, ainda há poucos trabalhos que enfatizam a importância do investimento residencial em particular. 
%Com a notória exceção de \textcite{green_follow_1997} e \textcite{leamer_housing_2007}, a maioria desses trabalhos foi publicada após a crise dos \textit{subprime} de 2008 --- que evidenciou a relevância deste gasto para a dinâmica da economia norte-americana.

Desse modo, enquanto o capítulo anterior elegeu o modelo teórico mais adequado para atender os objetivos desta pesquisa, o presente capítulo pretende fornecer a base empírica dessa discussão. Portanto, busca-se uma forma de modelar a taxa de crescimento do investimento residencial que será utilizada nas simulações do capítulo seguinte. 
Cabe frisar que essa análise se restringe ao caso norte-americano no pós-desregulamentação bancária dos anos 80, especialmente após 1991. A razão deste recorte temporal decorre tanto da estagnação salarial observada desde a década passada \cites{mian_house_2011}{teixeira_uma_2011} que implicou na intensa substituição de salário por empréstimos \cite{barba_rising_2009} quanto da crescente participação das hipotecas no balanço patrimonial dos bancos \cite{jorda_great_2014} bem como mudanças regulatórias que reduziram as restrições ao acesso de crédito no mercado imobiliário no pós-crise dos \textit{savings and loans} \cites{linneman_impacts_1989}{duca_empirical_1991}{federal_deposit_insurance_corporation_savings_1997}. 

Compreendidos os objetivos deste capítulo, a seção seguinte irá avaliar os estudos empíricos que incorporam gastos autônomos não criadores de capacidade dando especial ênfase aqueles que utilizam o modelo do supermultiplicador sraffiano (SSM) e, portanto, complementar a discussão teórica realizada no capítulo anterior. 
Em seguida, cabe a seção \ref{Secao_Residencial} destacar a importância do investimento residencial para a dinâmica norte-americana. Além disso, nessa mesma seção são pontuadas as formas com que a literatura trata do tema bem como selecionar a proposta mais adequada e compatível com o SSM. 
Adiante, na seção \ref{Modelo_empirico}, é estimado um modelo  para averiguar os determinantes do investimento residencial para então comparar com os resultados obtidos com os da literatura. Por fim, a seção \ref{Conclucao_Empirica} apresenta as conclusões.