Uma das fronteiras da pesquisa empírica acerca da literatura de crescimento liderado pela demanda é aquela que enfatiza a importância dos gastos autônomos não criadores de capacidade produtiva ao setor privado. \textcite{freitas_pattern_2013}, por exemplo, decompõem o crescimento da economia brasileira mostrando o papel desses gastos para os anos de  1970 a 2005. \textcite{braga_investment_2018} conclui que o os gastos improdutivos lideram o crescimento e que o investimento produtivo acompanha a tendência desses gastos, ao analisar o Brasil no período 1962-2015. Para o caso norte-americano, \textcite{girardi_long-run_2016} encontram evidências de que os gastos autônomos causam efeitos de longo prazo na taxa de crescimento enquanto \textcite{girardi_autonomous_2018} complementam com 20 países da OCDE. 

No entanto, por mais que exista uma literatura crescente sobre o papel dos gastos autônomos no crescimento econômico, ainda há poucos trabalhos que enfatizam a importância do investimento residencial em particular. Com a notória exceção de \textcite{green_follow_1997} e \textcite{leamer_housing_2007}, a maioria desses trabalhos foi publicada após a crise dos \textit{subprime} de 2008 --- que evidenciou a relevância deste gasto para a dinâmica da economia norte-americana.

Desse modo, enquanto o capítulo anterior elegeu o modelo teórico mais adequado para atender os objetivos desta pesquisa, o presente capítulo pretende fornecer a base empírica dessa discussão. Portanto, busca-se uma forma de modelar a taxa de crescimento dos imóveis que será utilizada nas simulações do capítulo seguinte. 
Cabe frisar que essa análise se restringe ao caso norte-americano no pós década de 80. A razão deste recorte temporal decorre tanto da estagnação salarial observada \cite{teixeira_uma_2011} quanto da crescente participação das hipotecas no balanço patrimonial dos bancos \cite{jorda_great_2014} bem como ausência de restrição de crédito no mercado imobiliário \cites{linneman_impacts_1989}{duca_empirical_1991}. 

%TODO Referência restrição de crédito

Compreendido os objetivos deste capítulo, a seção seguinte irá avaliar os estudos que incorporam gastos autônomos não criadores de capacidade dando especial ênfase aqueles que utilizam o modelo do supermultiplicador sraffiano (SSM) e, portanto, complementar a discussão teórica realizada no capítulo anterior. Em seguida, cabe a seção \ref{Secao_Residencial} destacar a importância do investimento residencial para a dinâmica norte-americana. Além disso, nessa mesma seção são pontuadas as formas com que a literatura, seja ela ortodoxa ou heterodoxa, trata do tema bem como selecionar a proposta mais adequada e compatível com o SSM: taxa própria de juros dos imóveis. Adiante, na seção \ref{Modelo_empirico}, é estimado um VECM  para averiguar as relações de longo prazo entre a taxa de crescimento dos imóveis e referida taxa própria e compara-se o resultados obtidos com a literatura. Por fim, a seção \ref{Conclucao_Empirica} apresenta as conclusões bem como as lacunas a serem enfrentadas no capítulo seguinte.