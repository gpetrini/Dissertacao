Uma das fronteiras da pesquisa empírica acerca da literatura de crescimento liderado pela demanda é aquela que enfatiza a importância dos gastos autônomos não criadores de capacidade produtiva ao setor privado. \textcite{freitas_pattern_2013}, por exemplo, decompõem o crescimento da economia brasileira mostrando o papel desses gastos para os anos de  1970 a 2005. \textcite{braga_investment_2018} conclui que os gastos improdutivos lideram o crescimento e que o investimento produtivo acompanha a tendência desses gastos, ao analisar o Brasil no período 1962-2015. Para o caso norte-americano, \textcite{girardi_long-run_2016} encontram evidências de que os gastos autônomos causam efeitos de longo prazo na taxa de crescimento enquanto \textcite{girardi_autonomous_2018} complementam com 20 países da OCDE. No entanto, por mais que exista uma literatura crescente sobre o papel dos gastos autônomos no crescimento econômico, ainda há poucos trabalhos que enfatizam a importância do investimento residencial em particular. 
%Com a notória exceção de \textcite{green_follow_1997} e \textcite{leamer_housing_2007}, a maioria desses trabalhos foi publicada após a crise dos \textit{subprime} de 2008 --- que evidenciou a relevância deste gasto para a dinâmica da economia norte-americana.

Enquanto o capítulo anterior elegeu o modelo teórico mais adequado para atender os objetivos desta pesquisa, o presente capítulo fornece a base empírica dessa discussão. Portanto, busca-se uma forma de especificar os determinantes do investimento residencial que será utilizada nas simulações do capítulo seguinte. 
Cabe frisar que essa análise se restringe ao caso norte-americano no pós-desregulamentação financeira dos anos 80, especialmente após 1991. 
A razão deste recorte temporal decorre tanto da crescente participação das hipotecas no balanço patrimonial dos bancos \cite{jorda_great_2014} quanto das mudanças regulatórias que reduziram as restrições ao acesso de crédito no mercado imobiliário no pós-crise das \textit{savings and loans} (1982-1989) \cites{linneman_impacts_1989}{duca_empirical_1991}{federal_deposit_insurance_corporation_savings_1997}. 


Compreendidos os objetivos deste capítulo, a seção seguinte irá avaliar os estudos empíricos que incorporam gastos autônomos não criadores de capacidade dando especial ênfase aqueles que utilizam o modelo do supermultiplicador sraffiano (SSM). Sendo assim, ao analisar os trabalhos empíricos que incluem gastos autônomos complementa-se a discussão teórica realizada no capítulo anterior. 
Em seguida, cabe a seção \ref{Secao_Residencial} destacar a importância do investimento residencial para a dinâmica norte-americana. 
Uma vez indicado que este gasto é relevante para compreender a especificidade do ciclo econômico norte-americano, são analisadas, nesta mesma seção, os seus determinantes de acordo com a literatura e, assim, selecionar a proposta mais adequada e compatível com o SSM.
Adiante, na seção \ref{Modelo_empirico}, é estimado um modelo empírico para averiguar os determinantes do investimento residencial para a economia norte-americana de 1992 a 2019 e então comparar com os resultados obtidos com os da literatura. Por fim, a seção \ref{Conclucao_Empirica} apresenta as conclusões.