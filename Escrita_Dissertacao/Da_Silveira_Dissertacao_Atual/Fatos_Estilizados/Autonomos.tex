\section{Modelos de crescimento e os gastos autônomos: uma revisão empírica}
\label{RevF}

O objetivo desta seção é analisar os trabalhos empíricos que analisam a relação entre os gastos autônomos não criadores de capacidade produtiva ao setor privado e crescimento econômico e, assim, complementar a discussão teórica realizada na seção \ref{SecAutonomos}. Mais uma vez, seguindo a categorização de \textcite{cesaratto_technical_2003}, os referidos gastos são:
(i) consumo financiado por crédito ou riqueza acumulada;
(ii) gastos do governo;
(iii) investimento residencial;
(iv) exportações e;
(v) gastos com P\&D\footnote{
	%TODO Nota de rodapé sobre gastos com P\&D
	Nota de rodapé sobre gastos com P\&D?
}. 
Da revisão da literatura empírica, verificou-se três principais preocupações: 
\begin{description}
	\item[(a)] testar a importância dos gastos autônomos sobre a taxa de crescimento de longo prazo; 
	\item[(b)] avaliar a relação entre taxa de investimento e produto;
	\item[(c)] investigar a dinâmica de cada um dos gastos autônomos referidos anteriormente.
\end{description}
Cada um desses pontos será analisado adiante. Por fim, vale pontuar que, dados os objetivos desta investigação, serão privilegiados os trabalhos que tenham o supermultiplicador sraffiano desenvolvido por \textcite{serrano_long_1995} e \textcite{bortis_institutions_1996} como forma de análise.

No que diz respeito ao tema (a), o trabalho de \textcite{girardi_long-run_2016} se destaca por analisar os efeitos de longo prazo dos gastos autônomos sobre o produto bem como por apresentar uma forma de se calcular o supermultiplicador para a economia norte-americana. Para tanto, estimam um VECM e obtém os resultados esperados de acordo com a teoria\footnote{Mais precisamente, tais resultados se sustentam uma vez desconsiderado o consumo financiado por crédito. Como justificativa para tal medida, \textcite[p.~13]{girardi_long-run_2016} argumentam que tal gasto está associado a algumas fases do ciclo econômico e, portanto, apresenta uma parcela consideravelmente induzida.}: (i) gastos autônomos e o produto apresentam uma tendência de longo prazo (são cointegradas); (ii) relação de causalidade parte dos gastos autônomos para o produto e (iii) relação positiva entre taxa de crescimento dos gastos autônomos e taxa de investimento. Já no artigo de \textcite{girardi_autonomous_2018}, o mesmo é feito para alguns países da zona do euro com a diferença que foram utilizadas variáveis instrumentais como \textit{proxy} de alguns gastos autônomos e foram obtidos resultados semelhantes ao do estudo anterior. Por fim, o trabalho de \textcite{goes_supermultiplier_2018} possui semelhanças com o de \textcite{girardi_long-run_2016}, mas se distingue por extendê-lo para mais países e por adotar critérios para agrupá-los bem como por reportar a convergência do grau de utilização ao nível normal.

Os trabalhos empíricos envolvendo o supermultiplicador, no entanto, não estão restringidos aos EUA ou países da OCDE. \textcite{freitas_pattern_2013}, por exemplo, analisam o caso brasileiro para os anos de 1970 a 2005 e concluem que diferentes gastos autônomos (em ordem, gastos do governo e consumo financiado por crédito) lideraram o crescimento em momentos distintos. Paralelamente, \textcite{braga_investment_2018} investiga o efeito acelerador para o caso brasileiro de 1996 a 2017 e conclui que o investimento criador de capacidade produtiva é causado (no sentido de Granger) pelo produto, ou seja, é induzido.  

Enquanto \textcite{freitas_pattern_2013} e \textcite{girardi_autonomous_2015} abordam a importância dos gastos autônomos para o crescimento, \textcite{braga_investment_2018} avalia a relação entre taxa de investimento e crescimento. Assim, estão abarcadas as preocupações (a) e (b) elencadas anteriormente. Resta, portanto, evidenciar os trabalhos que destacam a importância de alguns gastos autônomos em específico. Um deles é o de \textcite{medici_cointegration_2011} em que avalia o caso argentino para os anos de 1980 a 2007 e encontra evidências de cointegração entre renda, consumo do governo e o consumo privado autônomo (\textit{i.e.} não assalariado) em que os últimos granger-causam o primeiro. O modelo apresentado por \textcite{deleidi_mission-oriented_2019}, por sua vez, também analisa a importância dos gastos do governo investigando se o tipo de política fiscal adotada tem impactos sobre o crescimento. Em linhas gerais, os autores concluem que gastos orientados em setores mais intensivos em P\&D e em mudanças estrutuais (correspondente ao gasto v) possuem efeitos maiores do que uma política centrada apenas em incentivos fiscais. Já no que diz respeito às exportações (gasto iv), destaca-se a literatura de restrição por balanço de pagamentos seguindo a lei de \textcite{mccombie_balance--payments_1994} 
%TODO Orig year Thirwall
em que as exportações são os determinantes do crescimento de longo prazo \cites{atesoglu_balance--payments-constrained_1993}{mccombie_empirics_1997}{moreno-brid_mexicos_1999}{bertola_balance--payments-constrained_2002}\footnote{Por mais que tal abordagem não lance mão explicitamente do modelo do supermultiplicador sraffiano, as conclusões são compatíveis uma vez que estão presentes gastos autônomos não criadores de capacidade e a especificação da função investimento pode seguir o princípio do ajuste do estoque de capital.}.

%======================== Investimento residencial: Arestis e gasto induzido

Por fim, no que tange o investimento residencial, verifica-se uma lacuna na literatura empírica heterodoxa de crescimento liderado pela demanda. Vale retomar a compatibilidade deste componente da demanda com o modelo do supermultiplicador uma vez que (i) não cria   capacidade produtiva ao setor privado e (ii) pelas hipotecas serem a principal forma de financiamento (e não salários) de acordo com o \textit{Survey of Construction} \cite{us_census_bureau_characteristics_2017}. Dito isso, caberá a seção seguinte examinar as formas que a literatura econométrica encontrou para incorporar o investimento residencial para então eleger uma alternativa compatível com o supermultiplicador sraffiano.