\section{Metodologia, materiais e análise}\label{Metodo}

A pesquisa proposta será dividida em três frentes cada qual com seu respectivo capítulo.
A primeira delas trata da relação entre distribuição de renda e crescimento. A segunda, por sua vez, irá abordar os nexos entre distribuição pessoal e funcional da renda e crédito tendo em vista as mudanças distributivas verificadas na economia brasileira. Por fim, serão estudadas as relações entre crédito e crescimento. 
Dessa forma, a dissertação será composta por três capítulos além da introdução e das conclusões.  


Compreendidos os objetos e objetivos de cada um dos capítulos, são explicitadas as formas em que serão realizados. O capítulo primeiro tem aspectos teóricos que servirão de base para a análise desempenhada no capítulo seguinte.
Dessa forma, esses conceitos são fundamentais por descrever e situar o tema desta pesquisa em um campo mais geral em que serão evidenciadas as discussões da literatura especializada assim como suas limitações. 

Sendo assim, este capítulo irá rever as teorias heterodoxas de crescimento dando ênfase aos elementos referentes à distribuição de renda. Para isso, serão apresentados os seguintes modelos: (i) Neo-Keynesiano; (ii) Pós-Keynesiano; (iii) supermultiplicador sraffiano\footnote{Partindo de \textcite[Capítulo 6]{lavoie_post-keynesian_2014}, considera-se os termos neo-Keynesianos e modelo de Cambridge como sinônimos assim como Pós-Keynesianos e modelo Neo-Kaleckiano. Além disso, dados os objetivos desta pesquisa, o SSM será apresentado em maior detalhe.}. Dito isso, segue uma breve apresentação destes modelos.

%=================================================================================
%								Cambridge
%=================================================================================

A família de modelos Neo-Keynesianos desenvolvida por Kaldor, Robinson e Passinetti surge de uma tentativa de estender o princípio da demanda efetiva (PDE) para o longo prazo assim como para ser uma alternativa às teorias marginalistas. No entanto, esses modelos supõem que a economia opera à plena utilização da capacidade no longo prazo e, como consequência, o sistema econômico está na fronteira salário-lucro. Além disso, nesses modelos o investimento é tratado exogenamente e, como resultado, é a taxa de lucros ($r$) que determina a taxa de acumulação ($g$) \cite[Capítulo 6]{lavoie_post-keynesian_2014}. Em outras palavras, o padrão de crescimento é \textit{profit-led} em que restrições do lado da oferta persistem inclusive no longo prazo.

Dito isso, a Eq \ref{Cambridge} apresenta este modelo em que a taxa de crescimento da poupança ($g^s$) é determinada pela propensão a poupar sobre os lucros ($s_p$) e pela taxa de lucro ($r$). A taxa de crescimento do investimento, por sua vez, depende tanto de parâmetros comportamentais que expressam os \textit{animal spirits} ($\gamma$) quanto da sensibilidade  da taxa de crescimento em relação aos lucros esperados ($\gamma_rr^e$).

\begin{equation}
\label{Cambridge}
\begin{cases}
g^s = s_pr\\
g^i = \gamma + \gamma_rr^e
\end{cases} \Rightarrow g^* = \frac{s_p\gamma}{(s_p-\gamma_r)} \Rightarrow r^* = \frac{\gamma}{(s_p - \gamma_r)}
\end{equation}

Em equilíbrio de longo prazo (\textit{i.e.} $r = r^e$), existe uma taxa de crescimento de equilíbrio ($g^*$) compatível com uma taxa de lucro desejada ($r^*$). A estabilidade deste modelo, por sua vez, depende que a função investimento seja menos inclinada que a função poupanças, ou seja, $\gamma_r < s_p$. Resta, no entanto, explicitar como essa família de modelos se relaciona com distribuição de renda.

Analisando as contribuições de Kaldor, \textcite{hein_distribution_2014} destaca que a propensão a poupar que determina a taxa de lucro é resultada de uma média ponderada das propensões dos capitalistas e trabalhadores. Sendo assim, mudanças na distribuição funcional da renda alteram a taxa de lucro que, por sua vez, influencia a taxa de crescimento. Nesses termos, há uma simultaneidade entre distribuição e acumulação, ou seja, a distribuição de renda é endógena. Não apenas isso, mas dada a hipótese de que a economia opera ao pleno-emprego, há uma relação negativa entre \textit{wage-share} e \textit{profit-share}. Considerando que a rigidez dos salários é maior que a dos preços, a taxa de lucro torna-se residual. Portanto, o fechamento econômico deste modelo decorre da distribuição de renda que ajusta a poupança ao investimento. 

%=================================================================================
%								Neo-Kaleckiano: 2 parágrafos
%=================================================================================

Isto posto, cabe destacar que os modelos neo-Keynesianos foram alvo de diversas críticas que, em grande medida, realçam as inconsistências lógicas entre o PDE e as conclusões do modelo. Os autores Neo-Kaleckianos, por sua vez, dirigiram suas críticas à ausência de uma estrutura de mercado oligopolizada e à convergência do grau de utilização da capacidade ao nível de pleno-emprego. Em resposta, surgem os modelos Pós-Keynesianos em que o grau de utilização da capacidade é endogeneizado e a determinação das parcelas da renda dependem do \textit{mark-up}.

Apesar das diferentes versões desta famílias de modelos, \textcite[p.~360]{lavoie_post-keynesian_2014} afirma que existem quatro elementos comuns: (i) investimento depende do nível de utilização da capacidade ($u$); (ii) preços são determinados via \textit{cost-plus pricing} e não como decorrência das forças de mercado; (iii) propensão marginal a poupar dos trabalhadores é menor do que dos capitalistas e normalmente nula e; (iv) não há convergência à plena-capacidade e oferta de trabalho não é uma restrição.

Dito isso, é possível apresentar o modelo neo-kaleckiano básico em que a taxa de lucro ($r$) é determinada pelo grau de utilização ($u$), pelo \textit{profit-share} e pelo inverso da relação capital-produto ($v$). A taxa de crescimento da poupança ($g^s$) é definida tal como no modelo de Cambridge. Por fim, a taxa de crescimento do investimento depende tanto de parâmetros comportamentais (\textit{e.g. animal spirits}, $\gamma$) quanto da sensibilidade do investimento em relação aos desvios do grau de utilização do nível esperado ($\gamma_u(u^e-u_n)$):
 
\begin{equation}
\begin{array}{ll}
r = \pi u / v\\
g^s = s_p\pi u/v\\
g^i = \gamma + \gamma_u(u^e-u_i)
\end{array}
\end{equation}

Neste modelo, o equilíbrio de longo prazo é atingido quando a taxa de crescimento da poupança ($g^s$) se iguala à do investimento ($g^i$), mas para isso depende que a poupança seja mais sensível do que as mudanças no grau de utilização da capacidade ($s_p\pi > v\gamma_u$). Neste ponto, a taxa de utilização esperada ($u^e$) é igual à corrente ($u_i$), ou seja, o equilíbrio ocorre quando as expectativas em relação ao grau de utilização são realizadas. Portanto, a poupança se ajusta ao investimento quando $u_i = u^e$, ou seja, o grau de utilização da capacidade é o fechamento deste modelo.

No entanto, dados os objetivos desta pesquisa, cabe pontuar quais são os determinantes da distribuição de renda nestes modelos. Como destaca \textcite[Capítulo 5]{hein_distribution_2014}, a parcela nos lucros na renda ($\pi$) é determinado pelo \textit{mark-up} ($\theta$) que, por sua vez, depende da estrutura de mercado. Sendo assim, a distribuição da renda é macroeconomicamente exógena, mas microfundamentada.

Isto posto e considerando a grande aderência deste modelo na tradição heterodoxa,  é esperado que seja alvo de críticas e adaptações. Dentre elas, cabe ressaltar a de \textcite{bhaduri_unemployment_1990} em que os autores colocaram em questão a não capacidade desses modelos em explicar o porquê do aumento do grau de utilização da capacidade mesmo quando o \textit{profit-share} é constante. Dito isso, os autores modificam a função investimento do modelo Kaleckiano canônico e concluem que o regime de acumulação pode ser tanto \textit{wage} quanto \textit{profit-led}, mas em ambos o investimento é a variável que determina crescimento econômico. 


%=================================================================================
%								SSM
%=================================================================================
Outro conjunto de críticas, por sua vez, decorre da hipótese de endogeinização do grau de utilização da capacidade. Seguindo as contribuições de Garegnani, alguns autores sraffianos questionam o porquê desta variável não convergir ao normal no longo prazo. Grosso modo, essa linha argumentativa defende que tanto a subutilização quanto a sobreutilização da capacidade são prejudiciais dada a concorrência capitalista. Em resposta tanto ao modelo de Cambridge quanto a família de modelos neo-kaleckianos, \textcite{serrano_sraffian_1995} elabora o modelo do supermultiplicador sraffiano  (adiante, SSM).

Em linhas gerais, o SSM descreve um padrão de crescimento liderado pela demanda em que os gastos não criadores de capacidade produtiva (ditos improdutivos) determinam a taxa de crescimento de longo prazo. Além disso, neste modelo, o grau de utilização da capacidade produtiva ($u_t$) tende, via concorrência, ao normal ($\mu$) no longo prazo\footnote{\textcite{nikiforos_comments_2018} argumenta que a convergência do grau de utilização da capacidade ao nível desejado tem contribuições para as teorias heterodoxas de crescimento que podem ser verificadas pelos esforços de autores neo-kaleckianos em incluí-la sem perder a essência do modelo, ou seja, ajuste endógeno de $u$.}.
Dito isso, seja $Z_t$ o componente autônomo da demanda agregada financiado por crédito em $t$; $h_t$ a propensão marginal a investir e; $s$ a propensão marginal à poupar:

\begin{equation}
Y_t = \left( \frac{1}{s-h_t} \right)\cdot Z_t
\label{SSM}
\end{equation}
A equação \ref{SSM} indica que os efeitos dos gastos improdutivos sobre o produto agregado ($Y_t$) são capturados pelo termo em parênteses denominado de supermultiplicador sraffiano. Seguindo a exposição de \textcite{serrano_sraffian_2017}, a Eq \ref{SSM_g} mostra a dinâmica da taxa de crescimento da economia ($g_t$) para uma dada taxa dos componentes autônomos da demanda mencionados ($g_z$) em que o ajuste do estoque de capital fixo em relação à capacidade produtiva é feito de forma tênue pelo parâmetro $\gamma$

\begin{equation}
g_t = g_z + \frac{h_t\gamma(u_t-\mu)}{s-h_t}
\label{SSM_g}
\end{equation}
No longo prazo, portanto, com a taxa de utilização da capacidade tendendo ao nível desejado (\textit{i.e.} $u_t = \mu$) implica que é a taxa de crescimento da economia é dada por $g_z$.

No entanto, resta especificar como este mecanismo ocorre. \textcite{serrano_sraffian_1995} demonstra que a possibilidade de ajuste endógeno da razão entre a propensão média ($S{Me}$)  e marginal à poupar ($s$). Essa endogeneidade, por sua vez, advém da existência de gastos autônomos que não criam capacidade \cite{serrano_sraffian_2017}. 
Em outras palavras, para uma dada propensão marginal à poupar ($s$), a poupança média se ajusta ao investimento que, por sua vez, é induzido pela necessidade de adaptar o grau de utilização da capacidade ao nível normal por conta da concorrência capitalista.  A forma com que este investimento induzido garante o nível adequado da propensão média à poupar se dá por meio do supermultiplicador. 

Dessa forma, tal como aventado pelo princípio da demanda efetiva, o supermultiplicador sraffiano possibilita que a propensão marginal à investir determine a poupança.
Com isso, restaura-se um regime de acumulação liderado pela demanda em que a distribuição de renda é determinada pela teoria sraffiana e o nível de utilização da capacidade tende ao normal \cite{nikiforos_comments_2018}. 

%=================================================================================
%								Tabela: modelos de crescimento
%=================================================================================
\begin{table}[htb]
	\centering
	\caption{Teorias do crescimento e distribuição de renda}
	\label{crescimento}
	\resizebox{\textwidth}{!}{%
		\begin{tabular}{l|c|c|c|c|c|l}
			\hline \hline
			Modelo & \begin{tabular}[c]{@{}c@{}} Padrão de \\crescimento \end{tabular} & \begin{tabular}[c]{@{}c@{}} Condição de\\ estabilidade \end{tabular} & \begin{tabular}[c]{@{}c@{}} Distribuição \\de renda \end{tabular} & \begin{tabular}[c]{@{}c@{}}Grau de utilização \\ da capacidade\end{tabular} & \begin{tabular}[c]{@{}c@{}} Capacidade  \\ produtiva \end{tabular} & \begin{tabular}[l]{@{}l@{}} Hipótese Keynesiana \\ (Ajuste S-I) \end{tabular} \\ \hline
			Cambridge & \begin{tabular}[c]{@{}c@{}} \textit{Profit-led} \\ (Restrições de oferta) \end{tabular} & $\gamma_r < s_p$ & Endógena & \begin{tabular}[c]{@{}c@{}} Exógena \\(pleno-emprego) \end{tabular} & Exógena& \begin{tabular}[l]{@{}l@{}}Via distribuição \\funcional da renda\end{tabular}\\  \hline
			Neo-Kaleckiano & \begin{tabular}[c]{@{}c@{}} Wage/Profit-led \\(via investimento)\end{tabular} & $s_p\pi>v\gamma_u$ & \begin{tabular}[c]{@{}c@{}} Exógena \\ (\textit{Mark-up}) \end{tabular} & Endógena   & Exógena &  \\ \hline
			\begin{tabular}[l]{@{}l@{}}Supermultiplicador \\Sraffiano \end{tabular} & \begin{tabular}[c]{@{}c@{}} Demand-led \\(via consumo)\end{tabular} & $\Downarrow \gamma$& \begin{tabular}[c]{@{}c@{}} Exógena \\ (Teoria Sraffiana)  \end{tabular} & \begin{tabular}[c]{@{}c@{}} Exógena \\(Tende ao normal) \end{tabular}  & Endógena & \begin{tabular}[c]{@{}l@{}} Via \textit{fraction} ($f$)\\ $f = s/S_{Me}$ \end{tabular} \\ \hline \hline
		\end{tabular}%
	}
\caption*{\textbf{Fonte:} Elaboração própria}
\end{table}


%=================================================================================
%								Pivetti
%=================================================================================
Dito isso, a Tabela \ref{crescimento} resume os modelos de crescimento anteriormente discutidos. Adiante,
serão avaliadas algumas teorias da distribuição de renda, em especial a teoria monetária da distribuição desenvolvida por \textcite{pivetti_essay_1992}. Partindo das contribuições de \textcite{sraffa_producao_1985}, o autor argumenta que, no longo prazo, é a taxa de juros que regula a taxa de lucro e não o oposto\footnote{Esta constatação é inspirada em autores como Marx e Keynes.}. Dada essa inversão causal, propõe que a taxa de lucro do investimento ($r_a$) é determinada tanto pela taxa de juros,  cuja autoridade monetária tem influência, relevante no longo prazo ($i_{\Delta LP}$) quanto pelo lucro normal do empreendimento ($npe$):

\begin{equation}
r_a = i_{\Delta LP} + npe
\label{pivetti}
\end{equation}

A Eq \ref{pivetti} mostra que taxa de juros e de lucros possuem uma dinâmica semelhante no longo prazo em que a relação causal vai da primeira para a última.
Com isso, dado o grau de liberdade existente na teoria clássica/sraffiana da distribuição de renda, Pivetti propõe que a taxa de juros relevante no longo prazo intermedeia a relação entre preços e salários nominais. 

Grosso modo, nesta abordagem, a
barganha salarial reflete características político-institucionais relevantes para a distribuição de
renda. Tais especificidades impossibilitam a determinação de uma teoria geral para a distribuição. Apesar de relevante, a negociação salarial tem efeitos indiretos sobre a
determinação das parcelas distributivas. Por fim, os efeitos permanentes decorrem de mudanças persistentes na taxa monetária de juros (\textit{i.e.} taxa de juros relevante no longo prazo).
Dessa forma, a política monetária pode ter menor autonomia a depender do poder de
determinadas classes político-econômicas na correlação de forças. 

Portanto, a determinação das parcelas de renda via conflito distributivo é internalizada na especificação da taxa de juros, ou seja, na política monetária. Partindo de um referencial distinto, \textcite{singer_cutucando_2015} avalia como as disputas no governo Dilma foram expressas na redução deliberada da taxa de juros. Sendo assim, fica evidente o potencial explicativo de uma teoria tal como a de \textcite{pivetti_essay_1992} para o caso brasileiro recente. Sendo assim, com esses elementos em mãos, serão destacadas algumas das variáveis macroeconômicas relevantes que, dadas as devidas mediações, auxiliarão a narrativa construída no capítulo seguinte.

No capítulo descritivo, portanto, serão articuladas algumas interpretações das mudanças redistributivas ocorridas no Brasil em que se combinou crescimento, distribuição de renda e inclusão social. Para isso, serão analisadas tanto as políticas econômicas adotadas como seus impactos. Em relação às medidas praticadas, serão examinadas as valorizações reais do salário mínimo, crédito direcionado ao consumo assim como mudanças em algumas taxas de juros selecionadas. Já em relação aos impactos, serão avaliados a participação dos salários na renda, endividamento e consumo das famílias e, especialmente, mudanças distributivas a partir de alguns critérios de riqueza (\textit{i.e.} participação na renda por decis e classe sócio-econômica) assim como dados tributários que forem pertinentes tal como o IRPF. Com isso, objetiva-se destacar os componentes responsáveis pela dinâmica da economia brasileira no período averiguado (2000-14) em termos da distribuição de renda. 

%=================================================================================
%								Brasil
%=================================================================================

%=================================================================================
%								Brasil: Corrêa e dos Santos (2013)
%=================================================================================
{\color{blue} Parágrafo sobre 2000-2004?}


No que refere aos anos de 2004-2010, \textcite{correa_notas_2013} argumentam que houve um processo de crescimento econômico elevado acompanhado tanto de inclusão social quanto de uma maior importância da demanda doméstica. Adiante, alegam que o cenário externo favorável permitiu que tal arranjo fosse possível. No entanto, destacam que a conjuntura internacional foi uma condição necessária mas não suficiente para permitir esta dinâmica. Em especial, as transferências sociais assim como aumento do salário mínimo desempenharam papel fundamental na melhora da distribuição de renda. Em paralelo, como aponta \textcite{dos_santos_notas_2013}, o Estado tomou para si a função de induzir o investimento privado neste período.

%=================================================================================
%								Brasil: Serrano & Summa (2018)
%=================================================================================
Os impactos das mudanças redistributivas elencadas não ficaram restritas à esfera econômica. Avaliando o mesmo período, \textcite{serrano_conflito_2018} pontuam que o arranjo composto de crescimento e inclusão social (denominado pelos autores de ``Breve Era de Ouro'') gerou mudanças significativas no mercado de trabalho\footnote{Em relação o mercado de trabalho, vale destacar o estudo de \textcite{carneiro_politica_2018-1} em que os autores destacam a redução da dispersão salarial não acompanhada de uma mudanças na composição do emprego. Dentre os fatores que melhor explicam essa mudança na distribuição de renda, pontuam elementos institucionais como a formalização do emprego quanto e as políticas salariais adotadas.} ao ponto de causas efeitos indesejados. A redução do desemprego aberto alinhado com aumentos reais do salário mínimo fizeram com que o poder de barganha dos trabalhadores ampliasse. Como consequência, as margens e taxas de lucro reduziram. Desse modo, os efeitos diretos dessa chamada ``revolução indesejada'' são os aumentos consistentes da participação dos salários na renda. Os efeitos indiretos, por sua vez, decorrem da guinada da política econômica iniciada em 2011 de aposta no investimento privado e melhor representada pelo ajuste fiscal de 2015. Argumenta-se que essa alternância de política impactou negativamente a demanda agregada e, consequentemente, provocando a desaceleração (rudimentar) do crescimento \cite{serrano_demanda_2015}.

%=================================================================================
%								Tabela
%=================================================================================

\begin{table}[htb]
\caption{Taxa de juros nominal e real (IPCA), crédito em relação ao PIB e endividamento das famílias em relação à Massa Salarial Ampliada Disponível (2005-2014, média anual em \%)}
	\begin{center}
	\label{tabela_resumo}
	\resizebox{\textwidth}{!}{%
		\begin{tabular}{ccc|ccc|ccc}
			\cline{2-9}
			 &\multicolumn{2}{c}{Selic} &  \multicolumn{3}{c}{Crédito$^a$} & \multicolumn{3}{c}{Endividamento$^b$} \\ \hline \hline
			Ano & Nominal & Real &   Pessoa Física & Pessoa Jurídica & Total & Habitacional & Não habitacional & Total \\
			%2000 &17.59 & 11.62 & - & - & 26.6 & - & - & - \\
			%2001 &17.47 & 9.80 & - & - & 26.67 & - & - & - \\
			%2002 &19.11 & 6.58 & - & - & 25.72 & - & - & - \\
			%2003 &23.37 & 14.07 & - & - & 24.54 & - & - & - \\
			%2004 &16.24 & 8.64 & - & - & 24.75 & - & - & - \\
			2005 & 19.12 & 13.43 & - & - & 26.46 & 3.11 & 17.31 & 20.42 \\
			2006 & 15.28 & 12.14 & - & - & 28.99 & 3.36 & 20.34 & 23.7 \\
			2007 & 11.98 & 7.53 & 14.92 & 17.37 & 31.97 & 3.83 & 23.4 & 27.23 \\
			2008 & 12.36 & 6.46& 16.73 & 20.49 & 37.23 & 4.54 & 27.08 & 31.62 \\
			2009 & 10.06 & 5.75 & 18.00 & 22.84 & 40.84 & 5.68 & 28.29 & 33.97 \\
			2010 & 9.80  & 3.90 & 19.22 & 23.61 & 42.84 & 7.71 & 29.95 & 37.66 \\
			2011 & 11.66 & 5.16 & 20.35 & 24.27 & 44.62 & 9.9 & 31.16 & 41.06 \\
			2012 & 8.53 & 2.70 & 21.65 & 25.63 & 47.28 & 11.93 & 31.13 & 43.06 \\
			2013 & 8.18 & 2.27 & 22.85 & 26.81 & 49.66 & 14.45 & 30.28 & 44.73 \\
			2014 & 10.86 & 4.46& 23.74 & 27.01 & 50.75 & 16.87 & 28.8 & 45.67 \\ \hline \hline
		\end{tabular}%
	}
\end{center}
\footnotesize{$^a$ Por conta de mudanças metodológicas, optou-se por não incluir dados referentes ao crédito à pessoa física e jurídica para os anos de 2005 e 2006.$^b$ Dados referentes ao endividamento das famílias disponíveis a partir de 2005.}\\
\caption*{\textbf{Fonte:} Bacen}
\end{table}


%=================================================================================
%								Brasil: Dos Santos (2013)
%=================================================================================
Em relação à demanda agregada, \textcite{dos_santos_notas_2013} investiga a dinâmica do consumo para os anos 2004-2012. Conclui que há uma relativa estabilidade, mesmo que por construção, do consumo privado em termos do PIB. Além disso, destaca que há uma mudança da importância relativa de bens de consumo duráveis tais como automóveis e eletrodomésticos. A explicação desta dinâmica, argumenta, decorre em grande medida pelo aumento da concessão de crédito e de aproximações da renda disponível assim como pela redução na taxa de juros real. Como consequência, observa-se um aumento do endividamento em relação à Massa Salarial Ampliada Disponível (MSAD).

Os dados apresentados na Tabela \ref{tabela_resumo} ilustram essa trajetória. De um lado, constatam-se diminuições na taxa de juros real (deflacionada pelo IPCA) média de 2005 à 2010 (redução de aproximadamente 50,05\%). De outro, observa-se um aumento tanto no crédito em termos do PIB quanto dos endividamento das famílias em relação à MSAD no mesmo período\footnote{Além disso, vale destacar que enquanto há um aumento simétrico do crédito à pessoa física e jurídica, o endividamento das famílias apresenta um comportamento distinto. Apesar de majoritário, o comprometimento da renda das famílias com serviço da dívida não-habitacional perdeu participação relativa com o crescimento do crédito habitacional decorrentes do Programa Minha Casa Minha Vida. }. 

%=================================================================================
%								Brasil: Fontenele (2016)
%=================================================================================
Vale destacar que outros efeitos da inclusão social podem ser captados pela análise das mudanças no padrão de consumo elencadas anteriormente. \textcite{fontenelle_alcances_2016} parte das jornadas de junho de 2013 para pontuar como movimentos sociais até então focalizados se tornaram uma massa heterogênea que tinham a crítica à oferta pública como ponto em comum. Grosso modo, a autora argumenta que houve uma transformação na reivindicação por cidadania para um embricamento da democracia com o ato de consumir. Em outras palavras, a forma de pensar do consumidor é trasladada ao cidadão \cite{streeck_citizens_2012}. 
A relevância desta discussão para esta pesquisa decorre das transformações do padrão de consumo em que o crédito desempenhou um papel de destaque \cite{schettini_novas_2011}. Em resumo, observa-se tanto uma democratização do consumo quanto um consumo democratizante, ou seja, o consumo mediou a ascensão social. A implicação macroeconômica desta constatação é o já apontado aumento do endividamento das famílias. 



É digno de nota que, tal como \textcite{dos_santos_notas_2013}, argumenta-se que esse maior endividamento não é necessariamente negativo, mas sim, expressa as mudanças redistributivas observadas. Em outras palavras, é esperado que em países neste estágio de desenvolvimento apresentem conjuntamente maior inclusão social seguido de aumento do endividamento.
Tendo em vista esse movimento, foram adotadas em 2010 medidas macroprudenciais para a redução de crédito \cite{ribeiro_o_2016}. O argumento aqui defendido é que essa alternância de política fez com que a demanda agregada e a taxa de crescimento se reduzissem. Como consequência, o endividamento das famílias continuou a aumentar, mas a taxas decrescentes.



%=================================================================================
%								Brasil: Serrano & Summa (2018) 2 - Investimento
%=================================================================================

Além disso, por mais que o consumo doméstico tenha desempenhado um papel importante na dinâmica deste período, o investimento foi o componente que apresentou maiores taxas de crescimento entre 2004-2010 \cite{dos_santos_notas_2013}. Nesse período, portanto, verifica-se a indução do investimento privado decorrente de aumento dos componentes autônomos da demanda tal como o crédito ao consumo.
Como destacam \textcite{serrano_conflito_2018}, a maior participação dos salários na renda e subsequente redução das margens e taxas de lucro não implicaram (nem implicam) em diminuição do investimento. Seguindo o SSM, argumentam que a decisão de investir decorre das perspectivas de demanda futura e não a recomposição das margens de lucro. 


Isto posto, cabe destacar que a
relevância teórica desta constatação é a negação de um \textit{trade-off} entre gasto improdutivo e investimento tão comum na literatura \cite{serrano_acumulacao_2001}. 
Nesses termos, verifica-se que a guinada a favor do investimento privado nos anos 2011-2014 não só foi incapaz de retomar o padrão de crescimento anterior como também fundamentou as bases de sua derrocada \cite{serrano_demanda_2015}. Com isso, revela-se a justificativa e pertinência de se analisar o Brasil à luz do supermultiplicador sraffiano.


%=================================================================================
%								Recorte temporal
%=================================================================================
{\color{blue} Atenção: 2000}

Dito isso, cabe destacar o porquê do recorte temporal adotado, qual seja, 2000-2014. Os anos se referem aos dois mandatos do então presidente Lula e ao primeiro governo Dilma. Por mais que estes governos tiveram uma orientação deliberadamente redistributiva, serão realçadas as devidas mudanças entre eles. Para isso, o período em questão será dividido em: (i) antecedentes da ``Breve era de ouro'' (2000-2004); (ii) crescimento inclusivo (2004-2010) e; desaceleração e reversão macroprudencial  (2010-2014).
Vale notar que a escolha de encerrar esta pesquisa no ano de 2014 foi feita para não comprometer a análise com mudanças cujos impactos estão em curso tal como o ajuste fiscal de 2015 e o fim do processo de \textit{impeachment} da presenta Dilma. 
Em outras palavras, esta investigação tem um caráter estrutural e, dessa forma, serão evitadas as transformações de ordem conjuntural. 





%=================================================================================
%								Simulação
%=================================================================================


O capítulo seguinte, por fim, será analítico e serão utilizadas ferramentas computacionais para atingir os objetivos pretendidos. Mais especificamente, serão realizadas simulações inspiradas na descrição da economia brasileira feitas no capítulo precedente tendo como base no SSM. 
Sendo assim, evidencia-se a consistência teórica desta pesquisa. De um lado, analisa-se a economia brasileira por uma teoria sraffiana da distribuição como a de \textcite{pivetti_essay_1992}. De outro, utiliza-se um modelo de crescimento em que a distribuição de renda é exógena e compatível com os programas de pesquisa sraffianos como descrito em \textcite{aspromourgos_sraffian_2004}. Posto isso, dispomos tantos dos princípios teóricos que fundamentam esta investigação quanto dos fatores relevantes que descrevem a trajetória da economia brasileira no período recente. Sendo assim, torna-se possível, com o uso de simulações computacionais, retratar esta dinâmica a partir do SSM.  Argumenta-se que este modelo, por ser capaz de incorporar o crédito como um dos componentes autônomos da demanda (\textit{i.e.} $Z$), destaca-se como um modelo adequado para tratar deste episódio.

As simulações computacionais tal como pretendidas neste projeto não constam na grade regular das disciplinas recomendadas e disponíveis ao Instituto de Economia. Sendo assim, foi explicitada na tabela \ref{crono} uma linha referente ao tempo destinado ao aprendizado de linguagem de programação para obtenção dos instrumentos necessários. Dessa forma, dada a versatilidade e aceitação na academia, serão estudadas rotinas escritas em python\footnote{No momento em que este projeto está sendo elaborado, e tal como sugerido pela tabela \ref{crono}, as pesquisas em linguagem de programação estão em andamento. Neste caso, dada a familiaridade do requerente com a linguagem R, estão sendo cursados aulas de Python específicas para usuários de R disponíveis na plataforma DATACAMP. Mais informações em \url{https://www.datacamp.com/courses/python-for-r-users}, acessado em 5 de julho de 2018}. A escolha desta linguagem em particular se justifica pela estrutura gramatical de alto nível que facilita o aprendizado de seu usuário\footnote{Site oficial da linguagem python: \url{https://www.python.org}, acessado em 5 de julho de 2018 {\color{blue} Precisa?}}.

Ademais, é digno de nota que o uso de tal ferramenta permite não apenas a verificação das discussões apresentadas pela literatura como também a reprodutibilidade dos resultados. Tendo em vista essas possibilidades, o presente projeto irá disponibilizar os dados e as rotinas de programação utilizadas na plataforma OSF \cite{center_for_open_science_osfhome_nodate}. Com isso, é facilitada tanto a revisão por pares quanto a divulgação dos métodos utilizados. Além disso, a distribuição dos dados e códigos permite que o avanço científico não fique restrito às instituições de pesquisas com maior aporte financeiro. 


%=================================================================================
%								Análise dos resultados
%=================================================================================

Por fim, resta explicitar a forma de análise dos resultados obtidos. Os dados referentes ao Brasil serão comparados com a bibliografia apresentada e, como destacado, a interpretação das variáveis utilizadas é proveniente do debate teórico realizado. Já os resultados das simulações serão averiguados de acordo com a literatura do supermultiplicador sraffiano para então verificar se há compatibilidade das conclusões previstas assim como a presença de inconsistências teóricas. Além disso, os resultados serão confrontados com a discussão dos capítulos precedentes.
Compreendidas as etapas a serem desempenhadas, a seção \ref{cronograma} explicita o plano de trabalho desta investigação, adequando-o tanto às exigências institucionais do programa de mestrado quanto aos procedimentos necessários para viabilizá-la.


\begin{comment}
=================================================================================
					Temporariamente descartado
=================================================================================

Como esperado, essa crise está sendo alvo das mais diferentes 
interpretações.
Grosso modo, boa parte da literatura alveja as políticas econômicas como fonte desta desaceleração dinâmica, seja por serem elas austeras \cite{serrano_demanda_2015}, intervencionistas \cite{barbosa_filho_crise_2017}, estruturais \cite{bacha_saida_2017} ou até mesmo decorrentes das limitações da ossatura do Estado desenvolvimentista \cite{carneiro_economia_2017}. Com isso, indicam-se as fragilidades do padrão de crescimento brasileiro decorrentes das medidas inadequadas de política econômica, mas argumenta-se aqui que existem fatores estruturais que devem ser considerados.

\end{comment}