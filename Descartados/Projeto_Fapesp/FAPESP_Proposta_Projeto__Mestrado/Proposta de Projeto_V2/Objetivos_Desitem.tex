\section{Objetivos}\label{OBJ}

\begin{description}
	\item[Objetivo geral] Analisar a dinâmica da economia brasileira em termos de crescimento nos anos de 2000-2014 com ênfase nas mudanças redistributivas observadas assim como identificar os fatores que explicam esta trajetória; {\color{blue} Muito geral?}
	\item[Objetivos específicos:] (i) Investigar as diferentes teorias de crescimento heterodoxas e suas respectivas relações com distribuição de renda; (ii) Apresentar a teoria monetária da distribuição de \textcite{pivetti_essay_1992} assim como suas limitações e adequar este arcabouço teórico ao Brasil; (iii) Explorar as mudanças na distribuição pessoal e funcional da renda no caso brasileiro; (iv) Dialogar com a literatura assim como expor suas respectivas limitações e  diferenças argumentativas em relação ao objetivo geral apresentado; (v) Explicitar os impactos da ampliação do crédito ao consumidor, valorização do salário mínimo e determinação da taxa de juros à luz da teoria monetária da distribuição e; (vi) Examinar a economia brasileira com base no modelo do supermultiplicador sraffiano a partir de simulações computacionais. {\color{blue} Especificar mais?}
\end{description}



