\section{Metodologia}\label{Metodo}

A presente seção tem por objetivo apresentar a estrutura de capítulos e os métodos adotados na dissertação. Os objetos de cada capítulo são identificados na seção \ref{estrut}.
A descrição da metodologia a ser utilizada para tornar esta investigação possível fica à cargo da seção \ref{passos}.\todo{Repeticao}

\subsection{Estrutura da dissertação}\label{estrut}
A pesquisa proposta será dividida em três frentes cada qual com seu respectivo capítulo.
A primeira delas trata da relação entre distribuição de renda e crescimento. A segunda, por sua vez, irá abordar os nexos entre distribuição pessoal e funcional da renda e crédito tendo em vista as mudanças distributivas verificadas na economia brasileira. Por fim, serão estudadas as relações entre crédito e crescimento. 
Dessa forma, a dissertação será composta por três capítulos além da introdução e das conclusões.  

O capítulo primeiro possuirá um cunho teórico e abordará as teorias heterodoxas de crescimento com ênfase na discussão da distribuição de renda. O capítulo seguinte, de teor descritivo, analisará o desempenho recente da economia brasileira tendo em vista os elementos teóricos levantados no capítulo anterior. A abordagem adotada segue as contribuições de \textcite{pivetti_essay_1992} denominadas como teoria monetária da distribuição. É também nesse capítulo que serão expressas as razões pela escolha do recorte temporal aqui adotado (2003-14). Por fim, o terceiro capítulo será analítico e nele serão utilizadas ferramentas computacionais para atingir os objetivos pretendidos. Mais especificamente, serão realizadas simulações inspiradas na descrição da economia brasileira feita no capítulo precedente tendo como base o modelo do supermultiplicador sraffiano.

\subsection{Passos metodológicos}\label{passos}
Compreendidos os objetos e objetivos de cada um dos capítulos, esta seção tem por função explicitar a forma em que serão realizados. O capítulo primeiro tem aspectos teóricos que servirão de base para a análise desempenhada no capítulo seguinte. Dessa forma, esse embasamento teórico é fundamental por descrever e situar o tema desta pesquisa em um campo mais geral em que serão evidenciadas as discussões da literatura especializada assim como suas limitações. 

Sendo assim, este capítulo irá rever as teorias heterodoxas de crescimento dando ênfase aos elementos referentes à distribuição de renda. Para isso, serão apresentados os seguintes modelos: (i) Cambridge; (ii) neo-kaleckiano; (iii) supermultiplicador sraffiano. Com isso, propõe-se uma alternativa às teorias marginalistas sem excluir por completo as contribuições que possam ser pertinente à discussão proposta.

Em paralelo, serão avaliadas algumas teorias da distribuição de renda, em especial a teoria monetária da distribuição desenvolvida por \textcite{pivetti_essay_1992}. Com esses elementos em mãos, serão destacadas algumas das variáveis macroeconômicas relevantes que, dadas as devidas mediações, auxiliarão a narrativa construída no capítulo seguinte.

No capítulo descritivo, portanto, serão articuladas algumas interpretações das mudanças redistributivas ocorridas no Brasil em que se combinou crescimento e distribuição de renda. Para isso, serão analisadas tanto as políticas econômicas adotadas como seus impactos. Em relação às medidas praticadas, serão examinadas as valorizações reais do salário mínimo, crédito direcionado ao consumidor assim como mudanças em algumas taxas de juros selecionadas. Já em relação aos impactos, serão avaliados a participação dos salários na renda, endividamento e consumo das famílias e, especialmente, mudanças distributivas a partir de alguns critérios de riqueza (\textit{i.e.} participação na renda por decis e classe sócio-econômica) assim como dados tributários que forem pertinentes tal como o IRPF. Com isso, objetiva-se destacar os componentes responsáveis pela dinâmica da economia brasileira no período averiguado (2003-14). 

Cabe destacar o porquê do recorte temporal adotado. Os anos se referem aos dois mandatos do então presidente Lula e ao primeiro governo Dilma em que verifica-se uma orientação deliberadamente redistributiva. No entanto, ambos os governos não devem ser tratados como iguais e, por conta disso, serão explicitadas as devidas diferenças e rupturas. Além disso, optou-se por encerrar esta pesquisa no ano de 2014 para não comprometer a análise com mudanças que estão em curso. Em outras palavras, esta investigação tem um caráter estrutural e, dessa forma, serão evitadas as transformações de ordem conjuntural. 
% TODO Justificativa recorte temporal

Posto isso, dispomos tantos dos princípios teóricos que fundamentam esta pesquisa quanto dos fatores relevantes que descrevem a trajetória da economia brasileira no período recente. Sendo assim, torna-se possível, com o uso de simulações computacionais, retratar esta dinâmica a partir do supermultiplicador sraffiano. Neste modelo, como abordado na seção \ref{Rev},  a distribuição de renda é determinada exogenamente e, neste caso, será examinada a partir da teoria monetária da distribuição mencionada acima. Além disso, este modelo é capaz de incorporar o crédito ao consumidor em suas predições e, assim, destaca-se como um modelo adequado para tratar deste episódio.

Em resumo, os passos metodológicos desta dissertação dividem-se em: teoria, descrição e análise. Compreendidas as etapas a serem realizadas, a seção \ref{cronograma} explicita o plano de trabalho desta investigação, adequando-a tanto com as exigências institucionais do plano de mestrado quanto os procedimentos necessários para viabilizá-la.