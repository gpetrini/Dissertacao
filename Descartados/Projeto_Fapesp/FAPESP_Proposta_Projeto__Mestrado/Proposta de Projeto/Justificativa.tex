\section{Justificativa}\label{Just}
A discussão em torno da distribuição de renda tem ganhado destaque na literatura econômica e apresentado resultados relevantes no que diz respeito às teorias de crescimento. Estudos recentes analisando a economia norte-americana reportam a importância da distribuição de renda na determinação da dinâmica econômica. \textcite{grossmann-wirth_role_2018}, por exemplo, explicam a lenta recuperação dos EUA à partir da redução do consumo das famílias no pós Grande Recessão. Partindo da análise dos fluxos das dívidas familiares, os autores concluem  que o consumo privado não tem a capacidade de se basear no endividamento tal como antes.\todo[color=gray!40]{Revisto}

O endividamento das famílias norte-americanas mencionado acima pode ser entendido à partir da piora na distribuição de renda, ou ainda, da redistribuição funcional da renda à favor dos lucros. \textcite{barba_rising_2009} argumentam que o arranjo composto por manutenção do padrão de consumo relativo\footnote{Padrão de consumo relativo no sentido aspiracional, ou seja, manter um nível de vida em linha com as famílias em situação financeira semelhante e, ao mesmo tempo, modernizar os bens consumidos tal como as classes melhor remuneradas. \textcite{barba_rising_2009} lançam mão desta noção a partir do conceito de consumo conspícuo de \textcite{veblen_teoria_1965}.} e estagnação salarial fez com que as famílias dos estratos de renda mais baixos se endividassem\todo[]{Reformular Passividade familias}. Com isso, houve um processo de substituição das rendas do trabalho por empréstimos, permitindo que o crescimento econômico se baseasse no consumo privado. Como contrapartida, verificou-se uma redução significativa da poupança privada, ou em outros termos, uma diminuição dos saldos financeiros líquidos do setor privado \cite{godley_seven_1999}. 

Esta interpretação mostra, portanto, que o endividamento crescente das famílias é resultado tanto de mudanças persistentes na distribuição de renda quanto da elevada desigualdade. Dessa forma, houve um descolamento entre dispêndio e salários de tal modo que foram necessárias outras fontes de recursos para prover esta sofisticação do consumo.
Assim, fica evidente a importância dinâmica do crédito privado que, ao permitir um padrão de crescimento pautado no consumo, tornou possível que  os trabalhadores gastassem muito além dos seus rendimentos, ou melhor, aquilo que não ganham \cite{serrano_trabajadores_2008}. \todo[color=blue!40]{Non-sectur?}

 
Dessa forma, a Grande Recessão mostrou como o aumento do serviço da dívida privada em termos da renda disponível pode gerar processos dinamicamente insustentáveis quando acompanhado de uma piora da distribuição de renda.
Nesses termos, a experiência norte-americana recente sugere que o endividamento das famílias pode ter resultados macroeconômicos distintos no curto, médio e longo prazo. \todo[color=blue!40]{Non-sectur?}
Sendo assim, fica mais do que evidente a importância de se discutir as relações entre distribuição de renda e crescimento.

No entanto, apesar da relevância dos resultados apresentados anteriormente, há muito o que ser explorado e com isso assinala-se a relevância deste projeto\footnote{\textcite{szymborska_household_2018}, por exemplo, afirma que pouco se sabe os tipos de riqueza responsáveis pela melhora (ou piora) da distribuição de renda das famílias. Partindo dos dados do \textit{Survey of Consumer Finances}, conclui que diferentes formas de riqueza afetam a distribuição de renda de maneira distinta. Nesses termos, fica explicito que por mais que este tema seja bastante debatido, existem muitas lacunas a serem preenchidas.}. 
Além disso, por mais distinto que seja o objeto de análise em questão, há muito do se que incorporar dos estudos referentes à outros países. \todo[color=blue!40]{Frase jogada?} 

As diferenças, por sua vez, também podem ser fontes adicionais para inspiração de pesquisas futuras. Não é preciso adentrar nas especificidades institucionais para evidenciar as distinções entre o caso norte-americano e o brasileiro. Neste caso em particular, observa-se que o aumento do endividamento das famílias norte-americanas esteve concentrado nos menores estratos de renda. Partindo desta constatação, \textcite{stockhammer_rising_2015} conclui que a Grande Recessão é resultado tanto da desregulamentação financeira quanto dos efeitos macroeconômicos da desigualdade. O caso brasileiro, em seu turno, possui nuances significativas em que as políticas macroeconômicas redistributivas possibilitaram uma melhora nos menores estratos enquanto pesquisas recentes sugerem um acirramento da concentração entre os mais ricos no mesmo período \cite{medeiros_upper_2015}.


Diante disso, propõe-se investigar como a modernização do padrão de consumo das famílias acompanhada da presença crescente do crédito ao consumidor teve implicações relvantes sobre o crescimento. \todo{Reformular}
Desta forma, a principal justificativa desta pesquisa é a importância dos efeitos e especificidades das mudanças relativas nas parcelas de renda no período recente (2003-2014) para a dinâmica econômica brasileira. Em especial, destaca-se o aumento do endividamento privado \cite{ribeiro_o_2016} junto  da ascensão tanto de uma cultura \textit{política} do consumo quanto uma democratização pelo consumo \todo{Rep Consumo} \cite{fontenelle_alcances_2016}. 

É digno de nota que, com a publicação da portaria \todo[color=green!40]{Ver portaria},
% TODO encontrar número da portaria
serão divulgados relatórios anuais (à partir de 2014) referentes aos dados provenientes do Imposto sobre a Renda da Pessoa Física (IRPF) que trarão não apenas fontes adicionais para se estudar distribuição pessoal da renda como também uma base de comparação entre diferentes levantamentos domiciliares (\textit{i.e.} PNAD, Censo e POF\footnote{Em \textcite{souza_distribuicao_2015}, são apresentadas as diferenças entre essas pesquisa em termos da distribuição de renda. O autor conclui que existe um certo padrão entre as discrepâncias mesmo após uma harmonização \textit{ex post} das séries. A PNAD, em especial, apresenta um teor mais igualitário em que a renda dos mais pobres é sobrestimada enquanto a dos mais ricos é subestimada.}). Por mais que tais publicações fujam do recorte temporal deste projeto, foram divulgados dados referentes aos anos de 2007 à 2013 que precisam ser melhor analisados.
Portanto, outra justificativa desta pesquisa se dá pela relevância que tais estudos virão a ter no futuro\footnote{No momento em que este projeto está sendo elaborado, muito se discute sobre a subestimação da renda dos mais ricos em que os dados tributários referentes ao IRPF mencionados possibilitaram melhor esclarecimento \cites{afonso_irpf_2014}{medeiros_upper_2015}.}. 
\todo[color=blue!40]{Faz sentido?}
Compreendidos os objetivos e a relevância desta pesquisa, a seção \ref{Rev} fará uma revisão bibliográfica. Adiante, na seção \ref{Metodo}, são apresentados os métodos para torna-la possível.

\begin{comment}
%=====================================================
%				TEMPORARIAMENTE DESCARTADO
%=====================================================
Deste episódio como um todo, verificou-se um 


Além disso, é importante frisar que este 
\end{comment}




