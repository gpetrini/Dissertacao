\section{Objetivos}\label{OBJ}

Esta seção irá apresentar os objetivos desta pesquisa divididos em dois grupos: geral e específicos. Isto posto, a seção \ref{Just} irá realças as justificativas para esta investigação.\todo{Precisa?}

\begin{description}
	\item[Objetivo geral] Analisar a dinâmica da economia brasileira em termos de crescimento nos anos de 2003-2014 com ênfase nas mudanças redistributivas observadas assim como identificar os fatores que explicam esta trajetória;
	\item[Objetivos específicos] {\color{white}são eles}
% TODO Primeiro item
	\begin{itemize}
		\item Investigar as diferentes teorias de crescimento heterodoxas e suas respectivas relações com distribuição de renda;
		\item Apresentar a teoria monetária da distribuição de \textcite{pivetti_essay_1992} assim como suas limitações e adequar este arcabouço teórico ao Brasil;
		\item Explorar as mudanças na distribuição pessoal e funcional da renda no caso brasileiro;
		\item Dialogar com a literatura assim como expor suas respectivas limitações e  diferenças argumentativas em relação ao objetivo geral apresentado;
		\item Explicitar as políticas econômicas adotadas no período assim como seus impactos à luz da teoria monetária da distribuição, tais como:
		\begin{itemize}
			\item Ampliação do crédito ao consumidor e endividamento das famílias;
			\item Determinação da taxa de juros e distribuição de renda;
			\item Valorização real do salário mínimo e participação dos salários na renda;
		\end{itemize}
		\item Examinar a economia brasileira à luz do modelo do supermultiplicador sraffiano a partir de simulações computacionais.
	\end{itemize}
\end{description}


