\section{Resultados esperados}\label{Result}
Realizada esta pesquisa, esperam-se os seguintes resultados:
\begin{itemize}
	\item As mudanças redistributivas observadas são relevantes para explicar a dinâmica da economia brasileira no período em questão;
	\item O crédito ao consumidor teve efeitos significativos tanto sobre o consumo de bens duráveis quanto no aumento do endividamento das famílias;
	\item O maior acesso ao crédito decorre tanto da maior participação dos salários na renda viabilizada pelas valorizações reais do salário mínimo (aumento do colateral) quanto medidas deliberadas de política econômica;
	\item Encontrar uma taxa de juros relevante ao longo prazo tal como argumentado por \textcite{pivetti_essay_1992};
	\item Espera-se destacar o conflito distributivo por meio de mudanças na taxa de juros mencionada acima para o caso brasileiro;
	\item Os componentes que explicam a dinâmica econômica do Brasil podem ser captados pelo modelo do supermultiplicador sraffiano.
\end{itemize}