\newcommand{\HRule}[1]{\rule{\linewidth}{#1}}
\onehalfspacing
\setcounter{tocdepth}{3}
\setcounter{secnumdepth}{3}

\newcommand{\titulo}[1]{\def\meuTitulo{#1}}
\newcommand{\tituloIngles}[1]{\def\meuTituloIngles{#1}}
\newcommand{\numFAPESP}[1]{\def\numFAP{#1}}
\newcommand{\tipoRelatorio}[1]{\def\tipoRelat{#1 }} %o espaço depois do #1 é importante
\newcommand{\autor}[1]{\def\nomeAutor{#1}}
\newcommand{\cidade}[1]{\def\nomeCidade{#1}}
\newcommand{\universidade}[1]{\def\nomeUniversidade{#1}}
\newcommand{\faculdade}[1]{\def\nomeFaculdade{#1}}
\newcommand{\periodoVigencia}[1]{\def\periodVig{#1}}
\newcommand{\periodoRelatorio}[1]{\def\periodRelat{#1}}
\author{}
\date{}

\newcommand{\Figure}[1]{Figura~\ref{fig:#1}}
\newcommand{\Table}[1] {Tabela~\ref{#1}}
\newcommand{\Equation}[1] {Equa\c{c}\~ao~\ref{#1}}
\newcommand{\addFigure}[3] { %Parametros scale, fig_name, caption 
    \begin{figure}[!hbt]
      \centering
      \includegraphics[scale=#1]{figures/#2}
      \caption{#3}\label{fig:#2}
    \end{figure}
}

\newcommand{\geraTitulo}{
\clearpage
\begin{titlepage}
  \begin{center}
      \vspace*{-3cm}
      { \setstretch{.5} 
        \textsc{\nomeUniversidade} \\
        \HRule{.2pt}\\
        \textsc{\nomeFaculdade}
      }

      \vspace{5.5cm}

      \Large \textbf{\textsc{\meuTitulo}}
	  \HRule{1.5pt} \\ [0.5cm]
      \linespread{1}
      \large Projeto de Pesquisa para Solicitação  de Auxílio à Pesquisa Regular na modalidade Mestrado, fomentado pela Fundação de Amparo à Pesquisa do Estado de São Paulo. \\ 
  	   \HRule{1.5pt} \\ [0.5cm]
%       Projeto FAPESP \texttt{\#\numFAP}
       
       Candidato: \nomeAutor
         \\ [0.1cm]
       Orientador: Antônio Carlos Macedo e Silva 
       \\ [0.1cm]
       Co-orientador: Em aberto
       \vfill
       {\normalsize  \nomeCidade, \today}
\end{center}
\end{titlepage}
}

\usepackage{titlesec}
\titleformat{\chapter}{\normalfont\LARGE\bfseries}{\thechapter}{1em}{}
\titlespacing*{\chapter}{0pt}{3.5ex plus 1ex minus .2ex}{2.3ex plus .2ex}

%----------------------------------------------------------------------
% HEADER & FOOTER
%----------------------------------------------------------------------
\pagestyle{fancy}
\fancyhf{} % Limpa todos os campos de header and footer fields
%\setlength\headheight{15pt}
\renewcommand{\headrulewidth}{0pt}
%\fancyhead[R]{Anglia Ruskin University} 
\fancyfoot[R]{\thepage}%of \pageref{LastPage}}

\addto\captionsportuguese{\renewcommand{\contentsname}{Sumário}}
\addto\captionsportuguese{\renewcommand{\bibname}{Referências bibliográficas}}

%------
% Resumo e Abstract
%------
\newcommand{\Resumo}[1]{
   \begin{otherlanguage}{portuguese}
       \addcontentsline{toc}{chapter}{Resumo}
       \begin{abstract} \thispagestyle{plain} \setcounter{page}{2}
          #1
        \end{abstract}
   \end{otherlanguage} 
} %end \Resumo


\newcommand{\Abstract}[1]{
   \begin{otherlanguage}{english}
      \addcontentsline{toc}{chapter}{Abstract}
      \begin{abstract} \thispagestyle{plain} \setcounter{page}{3}
       #1
      \end{abstract}    
    \end{otherlanguage} 
} %end \abstract

%------
% Folha de rosto
%------
\begin{comment}

INFORMAÇÕES GERAIS DO PROJETO. Não é necessário para uma proposta de projeto

\newcommand{\folhaDeRosto}{
   \chapter*{Informações Gerais do Projeto}
   \addcontentsline{toc}{chapter}{Informações Gerais do Projeto}
   \begin{itemize}
      \item Título do projeto: 
            \begin{itemize}\item[] \textbf{\meuTitulo} \end{itemize}
      \item Nome do pesquisador responsável: 
            \begin{itemize}\item[]\textbf{\nomeAutor}\end{itemize}
      \item Instituição sede do projeto: 
            \begin{itemize}
               \item[]\textbf{\nomeFaculdade \ da \nomeUniversidade} 
            \end{itemize}
      \item Equipe de pesquisa:
            \begin{itemize}
               \item[]\textbf{\nomeAutor} 
            \end{itemize}
       \item Número do projeto de pesquisa:
            \begin{itemize}
               \item[]\textbf{\numFAP} 
            \end{itemize}
       \item Período de vigência:
            \begin{itemize}
               \item[]\textbf{\periodRelat} 
            \end{itemize}
       \item Período coberto por este relatório científico:
            \begin{itemize}
               \item[]\textbf{\periodVig} 
            \end{itemize}
   \end{itemize}
   \clearpage
}
\end{comment}