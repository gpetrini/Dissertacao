\documentclass[12pt]{report}
\usepackage[a4paper]{geometry}
\usepackage[utf8]{inputenc}
\usepackage[english,portuguese]{babel}
\usepackage[myheadings]{fullpage}
\usepackage[T1]{fontenc}
\usepackage{fancyhdr}
\usepackage{graphicx, setspace}
\usepackage{sectsty}
\usepackage{url}
\usepackage{mathptmx} %% Para times
\usepackage{comment}
\usepackage{multirow}
\usepackage{graphicx}
\usepackage[table,xcdraw]{xcolor}
\usepackage{enumitem}
\usepackage{blindtext}
\usepackage{float}
\usepackage[bottom]{footmisc}
\usepackage{url}




\usepackage[alf, versalete,  abnt-emphasize = bf, % destaca o titulo em negrito;
abnt-etal-list = 3, % trabalhos com mais de 3 autores recebem et al.,;
abnt-etal-text=it, % escreve o et al., em italico;
abnt-and-type = &, % usa o carater '&' no lugar de 'e' para mais de um autor;
abnt-last-names = abnt, % trata sobrenomes 'estritamente' conforme a ABNT; e
abnt-repeated-author-omit = yes % autores com + de uma entrada recebem '____.'
]{abntex2cite}

\newcommand{\HRule}[1]{\rule{\linewidth}{#1}}
\onehalfspacing
\setcounter{tocdepth}{3}
\setcounter{secnumdepth}{3}

\newcommand{\titulo}[1]{\def\meuTitulo{#1}}
\newcommand{\tituloIngles}[1]{\def\meuTituloIngles{#1}}
\newcommand{\numFAPESP}[1]{\def\numFAP{#1}}
\newcommand{\tipoRelatorio}[1]{\def\tipoRelat{#1 }} %o espaço depois do #1 é importante
\newcommand{\autor}[1]{\def\nomeAutor{#1}}
\newcommand{\cidade}[1]{\def\nomeCidade{#1}}
\newcommand{\universidade}[1]{\def\nomeUniversidade{#1}}
\newcommand{\faculdade}[1]{\def\nomeFaculdade{#1}}
\newcommand{\periodoVigencia}[1]{\def\periodVig{#1}}
\newcommand{\periodoRelatorio}[1]{\def\periodRelat{#1}}
\author{}
\date{}

\newcommand{\Figure}[1]{Figura~\ref{fig:#1}}
\newcommand{\Table}[1] {Tabela~\ref{#1}}
\newcommand{\Equation}[1] {Equa\c{c}\~ao~\ref{#1}}
\newcommand{\addFigure}[3] { %Parametros scale, fig_name, caption 
    \begin{figure}[!hbt]
      \centering
      \includegraphics[scale=#1]{figures/#2}
      \caption{#3}\label{fig:#2}
    \end{figure}
}

\newcommand{\geraTitulo}{
\clearpage
\begin{titlepage}
  \begin{center}
      \vspace*{-3cm}
      { \setstretch{.5} 
        \textsc{\nomeUniversidade} \\
        \HRule{.2pt}\\
        \textsc{\nomeFaculdade}
      }

      \vspace{5.5cm}

      \Large \textbf{\textsc{\meuTitulo}}
	  \HRule{1.5pt} \\ [0.5cm]
      \linespread{1}
      \large Relatório Científico \tipoRelat do Projeto de Auxílio à Pesquisa Regular, fomentado pela Fundação de Amparo à Pesquisa do Estado de São Paulo. \\ 
  	   \HRule{1.5pt} \\ [0.5cm]
       Projeto FAPESP \texttt{\#\numFAP}
       \\ [0.1cm]
       Pesquisador Responsável: \nomeAutor
       
       \vfill
       
       {\normalsize  \nomeCidade, \today}
\end{center}
\end{titlepage}
}

\usepackage{titlesec}
\titleformat{\chapter}{\normalfont\LARGE\bfseries}{\thechapter}{1em}{}
\titlespacing*{\chapter}{0pt}{3.5ex plus 1ex minus .2ex}{2.3ex plus .2ex}

%----------------------------------------------------------------------
% HEADER & FOOTER
%----------------------------------------------------------------------
\pagestyle{fancy}
\fancyhf{} % Limpa todos os campos de header and footer fields
%\setlength\headheight{15pt}
\renewcommand{\headrulewidth}{0pt}
%\fancyhead[R]{Anglia Ruskin University} 
\fancyfoot[R]{\thepage}%of \pageref{LastPage}}

\addto\captionsportuguese{\renewcommand{\contentsname}{Sumário}}
\addto\captionsportuguese{\renewcommand{\bibname}{Referências bibliográficas}}

%------
% Resumo e Abstract
%------
\newcommand{\Resumo}[1]{
   \begin{otherlanguage}{portuguese}
       \addcontentsline{toc}{chapter}{Resumo}
       \begin{abstract} \thispagestyle{plain} \setcounter{page}{2}
          #1
        \end{abstract}
   \end{otherlanguage} 
} %end \Resumo

\newcommand{\Abstract}[1]{
   \begin{otherlanguage}{english}
      \addcontentsline{toc}{chapter}{Abstract}
      \begin{abstract} \thispagestyle{plain} \setcounter{page}{3}
       #1
      \end{abstract}    
    \end{otherlanguage} 
} %end \abstract

%------
% Folha de rosto
%------
\newcommand{\folhaDeRosto}{
   \chapter*{Informações Gerais do Projeto}
   \addcontentsline{toc}{chapter}{Informações Gerais do Projeto}
   \begin{itemize}
      \item Título do projeto: 
            \begin{itemize}\item[] \textbf{\meuTitulo} \end{itemize}
      \item Nome do pesquisador responsável: 
            \begin{itemize}\item[]\textbf{\nomeAutor}\end{itemize}
      \item Instituição sede do projeto: 
            \begin{itemize}
               \item[]\textbf{\nomeFaculdade \ da \nomeUniversidade} 
            \end{itemize}
      \item Equipe de pesquisa:
            \begin{itemize}
               \item[]\textbf{\nomeAutor} 
            \end{itemize}
       \item Número do projeto de pesquisa:
            \begin{itemize}
               \item[]\textbf{\numFAP} 
            \end{itemize}
       \item Período de vigência:
            \begin{itemize}
               \item[]\textbf{\periodRelat} 
            \end{itemize}
       \item Período coberto por este relatório científico:
            \begin{itemize}
               \item[]\textbf{\periodVig} 
            \end{itemize}
   \end{itemize}
   \clearpage
}

\begin{document}
	\pagenumbering{gobble}
	
{\let\clearpage\relax \chapter{Bibliografia fundamental por categoria}}

Estes dados bibliográficos serão inseridos no arquivo final do projeto FAPESP:

\section{Padrão de crescimento e/ou desenvolvimento}

Os textos dessa seção se referem às características do padrão de crescimento denominado de desenvolvimentista e suas relações com o desenvolvimento econômico.

\citeonline{bielschowsky2013estrategia}

\citeonline{biancarelli2014era}

\citeonline{VERGNHANINI2013}

\citeonline{serrano2012desaceleraccao}

\citeonline{Morais2010}

\section{Características institucionais}

Esta seção reúne textos sobre as características institucionais da economia brasileira e seus respectivos impactos sobre a dinâmica:

\begin{itemize}
\item Regime de metas para a inflação
\item Lei de responsabilidade fiscal
\item Controle de capitais
\item Elementos que inviabilizam a atuação do Estado na economia ou deterioram a ossatura do Estado
\end{itemize}

\citeonline{FREITAS2010}


\citeonline{NASSIF2015}

\citeonline{9788539304608}

\citeonline{dos2016novo}



\citeonline{carneiro2017navegando}

\section{Indústria}

Referências que tratam de:

\begin{itemize}
\item Debate sobre desindustrialização
\item Discussão sobre produtividade
\item Mudanças estruturais e organizacionais
\item Limites do padrão de crescimento desenvolvimentista (lado da oferta)
\end{itemize}

 \citeonline{Sarti2017}

 \citeonline{DoencaBrDiegues}
 
 \citeonline{Bacha2015} $\Rightarrow$ Ortodoxo
 
 \citeonline{Hiratuka2015}


\citeonline{Bacha2013} $\Rightarrow$ Ortodoxo

\citeonline{Silva2013}

\citeonline{Almeida2013}

\citeonline{Mazzucato2013}



\section{Setor externo}

Estes textos serão referências para as mudanças no setor externo chamando atenção para o "fator China" e crise financeira internacional.

\citeonline{CINTRA2015}

\citeonline{Bacha2013} $\Rightarrow$ Ortodoxo

\citeonline{BIANCARELLI2012}

\section{Mercado de trabalho}

Esta seção apresenta textos que serão utilizados como referência para analisar o comportamento do mercado de trabalho com foco nos ganhos salariais reais durante o padrão de crescimento desenvolvimentista.

 \citeonline{Baltar2015}


\section{Economia política}

Seção que reúne textos que parte de uma discussão de economia política ou tratam do papel do Estado na economia e suas relações com o desenvolvimento econômico.

\citeonline{singer2015cutucando}

\citeonline{Mazzucato2013}

\citeonline{PRADO2001}

\citeonline{lerner1943functional}

\citeonline{forstater1998toward}


\citeonline{wren2016general2}

\section{"Análise do discurso"}

Seção que trata de planos e documentos oficiais do governo. Também serão incluídas entrevistas e notícias envolvendo principais personagens do período "desenvolvimentista".

\citeonline{MAIOR2015} $\Rightarrow$ Plano de governo


\section{Escola Pós-Keynesiana}

\citeonline{KEYNES1988}

\citeonline{9781847204837}

\citeonline{de1988fundamentos}

\citeonline{chick1995there}

\citeonline{Dequech2007}

\citeonline{macedo2008keynesian}



\subsection{Metodologia SFC}

\citeonline{dos2006keynesian}

\citeonline{DosSantos2009}

\citeonline{caverzasi2013stock}

\citeonline{godley2012monetary}

\citeonline{martins2016desalavancagem}

\section{Debate na literatura internacional}

Seção que reúne textos que discutem temas e conceitos tratados na literatura internacional. Não necessariamente discutem Brasil ou temas específicos, mas são considerados textos "clássicos" e fundamentais.

\citeonline{Stiglitz2013}

\citeonline{Lin2011}

\citeonline{Aghion2011}

\citeonline{Rodrik2008}

\citeonline{Nelson2008}

\citeonline{Crotty2002a}

\citeonline{Reinhart2010}

\citeonline{Arestis2009a}

\citeonline{Arestis2007a}

\citeonline{Arestis2009}

\citeonline{Arestis_1999}

\citeonline{Arestis2008}


\citeonline{blanchard1991sustainability}

%-----
% Referências bibliográficas
%-----

\renewcommand{\bibname}{}
%\bibliographystyle{abntex2}
\bibliography{Bibliografia_Mestrado_FAPESP}



\end{document}